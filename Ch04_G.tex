\documentclass[leqno]{book}
\usepackage[small,nohug,heads=vee]{diagrams}
\diagramstyle[labelstyle=\scriptstyle]
\usepackage{amsmath}
\usepackage{amssymb}
\usepackage{amsthm}
\usepackage[pdftex]{graphicx}
\usepackage{mathrsfs}
%\usepackage{mathabx}
\usepackage{enumitem}
%\usepackage{multicol}
%\usepackage[utf8]{inputenc}

\makeatletter
\newcommand*\bcd{\mathpalette\bcd@{.5}}
\newcommand*\bcd@[2]{\mathbin{\vcenter{\hbox{\scalebox{#2}{$\m@th#1\bullet$}}}}}
\makeatother

\begin{document}

\chapter{Hyperbolic Geometry}

For many centuries in the realm of the mathematicians, people tried to prove the parallel postulate (Axiom 2.5) from the other axioms, but all attempts failed.  Later, Carl Friedrich Gauss and Ferdinand Karl Schweikart discovered non-Euclidean types of geometry,\footnote{``Non-Euclidean geometry.'' \emph{Wikipedia}, Wikimedia Foundation, 28 October 2003, 10:55, en.wikipedia.org/wiki/Non-Euclidean\_geometry.} where most of the axioms in Section 2.1 \emph{except the parallel postulate} hold.  In this chapter, hyperbolic geometry will be covered.  The term was named by Felix Klein.  The chapter starts with basic geometry of the hyperbolic plane, then presents the mathematics behind polygonal tilings in the hyperbolic plane.  Then it introduces higher-dimensional hyperbolic spaces, and covers many different models by which the spaces can be represented, comparing them to one another.

One of the uses of the hyperbolic plane was due to M.C.~Escher.  He has made many pieces of hyperbolic-tiling art.  How to make one of them will be in the last exercise of Section 4.5.  There is also a downloadable computer game \emph{HyperRogue} developed by Zeno Rogue.\footnote{``HyperRogue.'' Retrieved from: https://zenorogue.itch.io/hyperrogue}  This game takes place in a truncated order-7 triangular tiling of the hyperbolic plane, where the player explores a two-dimensional world in many ways that are not feasible in a Euclidean-plane setting.  All in all, the hyperbolic plane has properties worth learning about.

\subsection*{4.1. Stereographic Projection.  Central Inversion}
\addcontentsline{toc}{section}{4.1. Stereographic Projection.  Central Inversion}
The aim of this section will be to introduce the fundamental concept of stereographic projection.  It is similar to a perspective projection, but it maps from a sphere to (mostly) a plane.  It will serve an important role in both this chapter and the next.

Consider the unit sphere $x^2+y^2+z^2=1$.  We let $N=(0,0,1)$ be the north pole, and we let $\Pi$ be a plane below the sphere, parallel to the $xy$-plane.  [$\Pi$ does not actually need to be below the sphere; we are assuming it is just this once, so that the diagram is easy to understand.]  For each point $p\ne N$ on the sphere, we can take the line passing through $N$ and $p$ and see where it intersects the plane $\Pi$:
\begin{center}\includegraphics[scale=.25]{StereoProjection.png}\end{center}
[We know the line is not parallel to $\Pi$; if it were, it would be tangent to the sphere.]  This maps each point on the sphere other than $N$ to a point on the plane.

Conversely, connecting $N$ to any point in the plane gives another point on the sphere, as the reader can readily see.  Thus, we appear to have a bijection between the points on the sphere and those on the plane.  There is, however, one problem: the north pole $N$ has no place to go, because there are numerous lines passing through just $N$, some of which are parallel to the plane and the rest of which meet all the points of the plane.

We go around this caveat by adjoining a symbolic point $\infty$ to the plane $\Pi$.  We let $\overline{\Pi}=\Pi\sqcup\{\infty\}$.  If we extend our previously described projection to the entire sphere by sending $N$ to $\infty$, we finally have a bijection from the sphere to $\overline{\Pi}$.  Moreover, it is a topological homeomorphism when $\overline{\Pi}$ is given the topology of the Alexandroff one-point compactification.  It is clear that we can regard $\overline{\Pi}$ as an abstract ``extended plane,'' $\overline{\Pi}=\mathbb R^2\sqcup\{\infty\}$, by merely thinking of $\Pi$ as a plane through the origin, taking an orthonormal basis, and using it to establish an isometry with $\mathbb R^2$.  If we regard $\Pi$ as the complex plane $\mathbb C$ (which will be done in most of the next sections), $\overline{\Pi}$ is referred to as the \textbf{Riemann sphere} or \textbf{extended complex plane}.\\

\noindent\emph{Warning}: Do not confuse $\mathbb R^2\sqcup\{\infty\}$ with the projective plane $P^2(\mathbb R)$.  $P^2(\mathbb R)$ has many points at infinity (which form a line), whereas $\mathbb R^2\sqcup\{\infty\}$ has only \emph{one} such point.  They are different concepts; however, the next section shows a way to think of $\mathbb R^2\sqcup\{\infty\}$ in projective terms.\\

\noindent The mapping just described has a special name:\\

\noindent\textbf{Definition.} \emph{Let $S$ be a sphere in $\mathbb R^3$, let $N$ be a point on $S$, and let $\Pi$ be a plane perpendicular to the line through $N$ and the center, such that $N\notin\Pi$.  The \textbf{stereographic projection} from $S$ to $\overline{\Pi}=\Pi\sqcup\{\infty\}$ is the function $\varphi:S\to\overline{\Pi}$, such that $\varphi(N)=\infty$, and for each $p\ne N$ in $S$, $\varphi(p)$ is the intersection of the line through $N$ and $p$ with the plane $\Pi$.}\\

\noindent The plane $\Pi$ must be perpendicular to the radius in order for the stereographic projection to be well-defined.  (Otherwise, there would be points $p\ne N$ in $S$, such that the line through $N$ and $p$ is parallel to $\Pi$, so that there would be no intersection.)  If $N\in\Pi$, (in this case $\Pi$ is tangent to the sphere at $N$), then $\varphi(p)$ would be $N$ for \emph{all} $p\in S$ (other than $N$), so $\varphi$ would not really be a bona fide projection.

For the remainder of this section, we will take $S^2$ to be the unit sphere $x^2+y^2+z^2=1$, $N$ to be $(0,0,1)$ and $\Pi$ to be the $xy$-plane, $z=0$.  %This may seem rather misleading, because earlier we said $\Pi$ needed to be \emph{below} the sphere \---- and in this case, the $xy$-plane passes through it.  However, there is no actual need for $\Pi$ to be below the sphere; we only declared it to be at the beginning, so that the concept would be easy to illustrate clearly.
Note that $\Pi$ passes through the sphere in this case, so that certain points $p\in S^2$ will map to plane points lying in \emph{between} $N$ and $p$.

We shall find a general formula for the stereographic projection from a point $p\ne N$ of $S^2$, as follows.  Suppose $p=(x,y,z)$.  Then since $\varphi(p)$ is in the $xy$-plane, it is of the form $(x',y',0)$.  Moreover, the following three points are collinear:
$$N=(0,0,1),~~~~p=(x,y,z),~~~~\varphi(p)=(x',y',0)$$
This means that the vector cross product $(\varphi(p)-N)\times(p-N)=(x',y',-1)\times(x,y,z-1)$ is the zero vector.\footnote{The cross product of two vectors in $\mathbb R^3$ is zero if and only if the vectors lie on a line.}  The reader can then work out the expressions and conclude that $x'=\frac x{1-z}$ and $y'=\frac y{1-z}$.  Therefore we have
\begin{center}
\textbf{If $\varphi:S^2\to\overline{\Pi}$ is the stereographic projection from the unit sphere to the extended $xy$-plane, and $N\ne p=(x,y,z)$ is a point of $S^2$, then:}
\begin{equation}\tag{F1}\varphi(p)=\left(\frac x{1-z},\frac y{1-z},0\right)\end{equation}
\end{center}
Once we have a formula for this projection, we would like a formula for its inverse, from $\overline{\Pi}$ back to $S^2$.  Suppose $(x,y,0)$ is a point of the $xy$-plane, and we wish to find the the corresponding point $(x_0,y_0,z_0)\in S^2$.  In this case, we have two equations:
$$x_0^2+y_0^2+z_0^2=1\text{ (because the point is on the sphere);}$$
$$N=(0,0,1),~~~~(x_0,y_0,z_0),~~~~(x,y,0)\text{ are collinear.}$$
The second equation can be rephrased by saying $(x_0,y_0,z_0-1)=\lambda(x,y,-1)$ for some $\lambda\in\mathbb R$.  After all, those vectors are obtained by connected pairs of the above points, hence the points are collinear if and only if one of the vectors is a scalar multiple of another.  [We note that $(x,y,-1)$ cannot be the zero vector, since $-1\ne 0$.]  We can compute the value of $\lambda$ by taking the magnitudes of the vectors $(x_0,y_0,z_0-1)$ and $(x,y,-1)$:
$$\|(x_0,y_0,z_0-1)\|^2=x_0^2+y_0^2+(z_0-1)^2=(x_0^2+y_0^2+z_0^2)-2z_0+1=2-2z_0=2(1-z_0)$$
$$\|(x,y,-1)\|^2=x^2+y^2+(-1)^2=x^2+y^2+1\implies\|\lambda(x,y,-1)\|^2=\lambda^2(x^2+y^2+1)$$
Therefore, since $(x_0,y_0,z_0-1)=\lambda(x,y,-1)$ and their magnitudes are equal, we get $\lambda^2(x^2+y^2+1)=2(1-z_0)$.

Yet, looking in the $z$-component of the equation $(x_0,y_0,z_0-1)=\lambda(x,y,-1)$, we immediately get $\lambda=1-z_0$.  Hence we have an equation
$$(1-z_0)^2(x^2+y^2+1)=2(1-z_0)$$
Since $(x_0,y_0,z_0)\ne N$, $z_0\ne 1$, and we may divide by $1-z_0$:
$$(1-z_0)(x^2+y^2+1)=2$$
We instantly get $1-z_0=\frac 2{x^2+y^2+1}$ as a result, so that $z_0=\frac{x^2+y^2-1}{x^2+y^2+1}$.  Furthermore, since $\lambda=1-z_0$, looking at the first two components of $(x_0,y_0,z_0-1)=\lambda(x,y,-1)$ entails $x_0=x(1-z_0)$ and $y_0=y(1-z_0)$.  Therefore,
\begin{equation}\tag{F2}(x_0,y_0,z_0)=\left(\frac{2x}{x^2+y^2+1},\frac{2y}{x^2+y^2+1},\frac{x^2+y^2-1}{x^2+y^2+1}\right)\end{equation}
is the point of $S^2$ that maps to $(x,y,0)\in\Pi$ via the stereographic projection $\varphi$.

The reader is left to verify the following basic facts about formulas (F1) and (F2):
\begin{itemize}
\item The point of (F2) is really on the sphere $S^2$.

\item The two formulas give inverse mappings both ways.  [When going from the sphere to the plane and then back to the sphere, one must be aware that the original point was on the sphere.]
\end{itemize}
The formulas will make it easier to study the stereographic projection.\\

\noindent\textbf{THE CENTRAL INVERSION}\\

\noindent In the remainder of this section, we shall introduce the concept of the central inversion.  [This does not refer to the negation of the identity linear operator $-I$; it is a different kind of ``central inversion.'']

The central inversion (a.k.a., circle inversion) is a transformation of $\overline{\mathbb R^2}=\mathbb R^2\sqcup\{\infty\}$.  As we will see, it fixes every point on the unit circle $x^2+y^2=1$, and it exchanges points inside the circle with points outside.  It is conformal (Exercise 4 below) but orientation-reversing.  It also maps circles and lines to circles and lines.

Here is how we intuitively obtain it.  Start with the reflection $R$ of the unit sphere $S^2$ over the $xy$-plane; this is given by $R(x,y,z)=(x,y,-z)$, and it is clear that this is an involution sending $S^2$ to itself.  The question remains: if we apply stereographic projection to both the starting and ending points of this transformation, how do the results relate?  This can be answered by \emph{conjugating} $R$ by the stereographic projection.  Indeed, if $R(p)=p'$, then $\varphi\circ R\circ\varphi^{-1}$ is a transformation of $\overline{\mathbb R^2}$ sending $\varphi(p)\mapsto\varphi(p')$.

Thus, we may take any point $(x,y)\in\overline{\mathbb R^2}$ and ask ourselves what its image is under $\varphi\circ R\circ\varphi^{-1}$.

First, by (F2) above, $\varphi^{-1}(x,y)=\left(\frac{2x}{x^2+y^2+1},\frac{2y}{x^2+y^2+1},\frac{x^2+y^2-1}{x^2+y^2+1}\right)$.  Applying $R$ to this vector merely negates the last component, so that gives us $\left(\frac{2x}{x^2+y^2+1},\frac{2y}{x^2+y^2+1},\frac{-x^2-y^2+1}{x^2+y^2+1}\right)$.
Finally, we must apply $\varphi$ to that vector.  We recall (F1) that if $N\ne(x',y',z')\in S^2$ then $\varphi(x',y',z')=\left(\frac{x'}{1-z'},\frac{y'}{1-z'}\right)$.  Taking $x'=\frac{2x}{x^2+y^2+1},y'=\frac{2y}{x^2+y^2+1},z'=\frac{-x^2-y^2+1}{x^2+y^2+1}$, it can be observed that
$$1-z'=\frac{2x^2+2y^2}{x^2+y^2+1}=\frac{2(x^2+y^2)}{x^2+y^2+1}$$
from which we have $\frac{x'}{1-z'}=\frac x{x^2+y^2}$ and $\frac{y'}{1-z'}=\frac y{x^2+y^2}$.  Thus, $\varphi\circ R\circ\varphi^{-1}$ sends $(x,y)$ [$x,y$ not both zero] to the point $\left(\frac x{x^2+y^2},\frac y{x^2+y^2}\right)$.  Moreover, it sends $(0,0)\mapsto\infty$ and $\infty\mapsto(0,0)$, since the stereographic projection makes $\infty\leftrightarrow(0,0,1)$ and $(0,0)\leftrightarrow(0,0,-1)$ correspond.\\

\noindent\textbf{Definition.} \emph{On $\overline{\mathbb R^2}=\mathbb R^2\sqcup\{\infty\}$, the \textbf{central inversion} (or \textbf{circle inversion}) is the map $\psi$ given by $\psi(0,0)=\infty,\psi(\infty)=(0,0)$ and $\psi(x,y)=\left(\frac x{x^2+y^2},\frac y{x^2+y^2}\right)$ for $x,y$ not both zero.}\\

\noindent The reader can readily see that $\psi$ is an involution, and that $\psi(p)=p$ for all points $p$ on the unit circle.  Also, if $p$ is regarded as a nonzero vector then $\psi(\vec p)=\frac{\vec p}{\|\vec p\|^2}$; moreover, $\|\psi(\vec p)\|=\frac 1{\|\vec p\|}$, so that $\vec p$ is inside the circle if and only if $\psi(\vec p)$ is outside.  Useful properties of the central inversion will now be stated and shown.\\

\noindent\textbf{Proposition 4.1.} (i) \emph{The central inversion of a line through the origin is the line itself.}

(ii) \emph{The central inversion of a line not through the origin, is a circle whose arc passes through the origin, and vice versa.}

(iii) \emph{The central inversion of a circle whose arc does not pass through the origin is another such circle.}

\begin{proof}
The proofs use basic direct substitution into the formula.  The idea is that if any subset of $\overline{\mathbb R^2}$ is given by an equation $f(p)=0$, its central inversion is given by $f(\psi(p))=0$.  We will prove the first half of (ii) as an example to illustrate this principle; the rest will be left to the reader.

Let $\ell$ be a line not through the origin.  Then $\ell$ is given by an equation $ax+by=c$, where $a,b$ are not both zero, and $c\ne 0$.  [If $c$ were zero, the line would go through the origin.]  Then by the principle described in the paragraph above, $\ell$'s central inversion is given by
$$a\frac x{x^2+y^2}+b\frac y{x^2+y^2}=c;$$
basic algebra then entails $x^2+y^2-\frac acx-\frac bcy=0$, which is the equation of a circle [see Exercise 9 of Section 2.6].  The equation can be rewritten as:
$$\left(x-\frac a{2c}\right)^2+\left(y-\frac b{2c}\right)^2=\frac{a^2+b^2}{4c^2}$$
and so the central inversion of $\ell$ is the circle centered at $\left(\frac a{2c},\frac b{2c}\right)$ with radius $\frac{\sqrt{a^2+b^2}}{2|c|}$.  Direct substitution into the equation shows that the origin $(0,0)$ is on the circle.
\end{proof}

\noindent The following diagram illustrates the central inversion.  It takes a Euclidean triangle $\triangle ABC$, and passes it to the green figure, where $A$ gets mapped to $A'$, $B$ to $B'$ and $C$ to $C'$ (which is $C$ itself because $C$ is on the unit circle):
\begin{center}\includegraphics[scale=.35]{CentInversion.png}\end{center}

\subsection*{Exercises 4.1. (Stereographic Projection.  Central Inversion)} % Introduce stereographic projection (first noting how important it'll be for this and the next chapter)
% from S^2\to\mathbb R^2\sqcup\{\infty\}.  Find formulas.  Then note that conjugating the equatorial reflection of the sphere yields central inversion, so go over that.
% POTENTIAL EXERCISE: stereographic projection is conformal
\begin{enumerate}
\item Show that stereographic projection sends every circle on $S^2$ that does not have $N$ on its arc to a circle.  [A circle in $S^2$ can be realized as the intersection of $S^2$ with a plane in $\mathbb R^3$; now use (F2) to equate which points of the $xy$-plane have corresponding points on this plane.]

\item What is the stereographic projection of a circle that has $N$ on its arc?

\item Use the previous two exercises to give an alternate proof that the central inversion preserves circles and lines.  [Remember that the central inversion is the conjugation of the sphere's equatorial reflection $(x,y,z)\mapsto(x,y,-z)$ by stereographic projection.]

\item\emph{(Conformality of stereographic projection.)} \---- Let $U\subset\mathbb R^2$ be an open set\footnote{This means that for each $x\in U$, there exists $\varepsilon>0$ such that $\{y\in\mathbb R^2:\|y-x\|<\varepsilon\}\subset U$.} and $f:U\to\mathbb R^3$ a differentiable map.  $f$ is said to be \textbf{conformal} if it preserves angles between curves.

(a) Show that the following statements are equivalent:

~~~~(i) $f$ is conformal.

~~~~(ii) For each point $p\in U$, and nonzero vectors $\vec v_1,\vec v_2$ of $\mathbb R^2$, the angle between $\vec v_1$ and $\vec v_2$ equals the angle between $df_p(\vec v_1)$ and $df_p(\vec v_2)$.

~~~~(iii) For each point $p\in U$, there is a constant $\lambda>0$ (which depends only on $p$), such that $df_p(\vec v_1)\cdot df_p(\vec v_2)=\lambda^2\vec v_1\cdot\vec v_2$ for all $\vec v_1,\vec v_2\in\mathbb R^2$.

~~~~(iv) For each point $p\in U$, there is a constant $\lambda>0$ (which depends only on $p$), such that $\|df_p(\vec v)\|=\lambda\|\vec v\|$ for all $\vec v\in\mathbb R^2$.

~~~~(v) For each point $p\in U$, $df_p$ has orthogonal columns with the same magnitude. % 3\times 2, because U is 2-dimensional and \mathbb R^3 is 3-dimensional.  If df_p were 2\times 3, then the only way it could have orthogonal columns with the same magnitude is for it to be zero, as you can't have an orthonormal set with more vectors than the dimension of the space

~~~~(vi) For each point $p\in U$, $(df_p)^T(df_p)$ is a scalar multiple of the identity matrix.

[(i) $\iff$ (ii) because the Chain Rule implies that the differential sends a tangent vector to a curve in $U$ to a tangent vector of the output curve.  (ii) $\iff$ (iii): Recall that the angle between $\vec v_1$ and $\vec v_2$ is $\cos^{-1}\frac{\vec v_1\cdot\vec v_2}{\|\vec v_1\|\|\vec v_2\|}$.  (iii) $\iff$ (iv): Use the fact that $\vec v\cdot\vec w=\frac 12(\|\vec v+\vec w\|^2-\|\vec v\|^2-\|\vec w\|^2)$.  (iii) $\implies$ (v): Take the $\vec v_j$ to be the standard basis vectors.  (v) $\iff$ (vi) because for any matrix $A$, the entries of $A^TA$ are the dot products of $A$'s columns with each other.  (vi) $\implies$ (iii): $(df_p)^T(df_p)=\lambda^2I_2$ for some $\lambda$.]

(b) Show that the inverse of the stereographic projection, given by (F2), is conformal.  [Find a formula for its differential, then use part (a).]

(c) Show that the composition of conformal maps is conformal.

(d) Use this to show that the central inversion is conformal.

\item The central inversion of a point $p\ne(0,0)$ is the point whose distance from the origin is the multiplicative inverse of the distance from $p$ to the origin, and is on the same ray from the origin as $p$.

\item If $\overline{\mathbb R^2}$ is regarded as $\mathbb C\sqcup\{\infty\}$, then the central inversion of $z\ne 0$ in $\mathbb C$ is $1/\overline z$, where $\overline z$ is the complex conjugate of $z$.

\item Here is a geometric interpretation of the central inversion. %http://mathworld.wolfram.com/Inversion.html

(a) Suppose $P$ is a point inside the unit circle, but is not equal to the center $O$ of the circle.  Let $\ell$ be the line through $P$ perpendicular to $\overset{\longleftrightarrow}{OP}$.  Suppose $Q$ is an intersection point of the unit circle with $\ell$.  Let $\ell'$ be the line through $Q$ perpendicular to $\overset{\longleftrightarrow}{OQ}$, and let $P'=\overset{\longleftrightarrow}{OP}\cap\ell'$.  Show that $P'$ is the central inversion of $P$. [Use Exercise 5 and similar triangles (covered in Section 2.1).]

(b) On the other hand, if $P$ is outside the circle, let $\ell$ be a line through $P$ tangent to the circle.  Suppose it meets the circle at a point $Q$, and let the perpendicular from point $Q$ to line $\overset{\longleftrightarrow}{OP}$ meet the line at $P'$.  Then $P'$ is the central inversion of $P$.
\end{enumerate}

\subsection*{4.2. Generalized Circles and M\"obius Transforms}
\addcontentsline{toc}{section}{4.2. Generalized Circles and M\"obius Transforms}
The fundamental setup for hyperbolic geometry will start from the concept of the M\"obius transform of $\mathbb R^2\sqcup\{\infty\}$.  We will first define the notion of a generalized circle, so we can later use it to study M\"obius transforms in depth.

We shall start with a fundamental lemma.  It is assumed that every line in $\mathbb R^2\sqcup\{\infty\}$ contains $\infty$ along with the Euclidean points of the line.\\

\noindent\textbf{Lemma 4.2.} \emph{Stereographic projection sends circles not containing $N$ to circles, and circles containing $N$ to lines.  Conversely, every line or circle in the plane corresponds to a circle in the sphere.}\\

\noindent Incidentally, this lemma was stated in Exercises 1-2 in the previous section.
\begin{proof}
A circle in $S^2$ can be realized as the intersection with $S^2$ of a plane, say $Ax+By+Cz=D$, where $A,B,C$ are not all zero.  We wish to find an equation for points in the $xy$-plane whose corresponding points on $S^2$ are part of this plane; for, these points are then the image of the circle through the stereographic projection.  Note that the circle contains $N$ $\iff$ the plane contains $N=(0,0,1)\iff C=D$.

By equation (F2) from the previous section, this equation is
$$A\frac{2x}{x^2+y^2+1}+B\frac{2y}{x^2+y^2+1}+C\frac{x^2+y^2-1}{x^2+y^2+1}=D$$
which equates to:
$$(C-D)(x^2+y^2)+2Ax+2By-C-D=0.$$
If the original circle on $S^2$ does not contain $N$, which means $C\ne D$, then the equation is quadratic in $x,y$, where the $xy$ coefficient is zero, and the $x^2$ and $y^2$ coefficients are nonzero and equal.  Hence this is the equation of a circle, by Exercise 9 of Section 2.6.  If the original circle on $S^2$ contains $N$, which means $C=D$, the equation becomes $2Ax+2By=C+D$, which is a line ($A,B$ can't both be zero, because that would make the plane $Ax+By+Cz=D$ tangent to $S^2$, so it wouldn't intersect in a circle).  Of course, the image also contains $\infty$ in this case.

Conversely, if $\omega$ is a line or a circle in the $xy$-plane, let $a,b,c\in\omega$ be three distinct points.  Then observe that \emph{$\omega$ is determined by $a,b,c$}: if $\omega$ is a line then $a,b,c$ are collinear, and hence, any two of these points determines $\omega$ (unless one of them is $\infty$), and if $a,b,c$ are not collinear, $\omega$ is a circle, and is determined by them.\footnote{The perpendicular bisectors of $\overline{a~b}$ and $\overline{a~c}$ intersect in a point which must be the center of $\omega$.  Since both the center and one point on the arc are determined, the circle is.}  Let $a',b',c'$ be the corresponding points $S^2$: then these points cannot be collinear (since the sphere intersects any line in at most two points), which means that they determine a circle $\omega_1$.  Thus $\omega_1$ must be contained in $S^2$ (because they meet in at least three points $a',b',c'$); and by the preceding part of the argument, the stereographic projection sends $\omega_1$ to either a line or a circle containing $a,b,c$.  This figure must be $\omega$.
\end{proof}

\noindent In view of Lemma 4.2, the images under stereographic projection of the circles are the circles and the unions of the lines with $\infty$.  These subsets of $\mathbb R^2\sqcup\{\infty\}$ have a special name.\\

\noindent\textbf{Definition.} \emph{In $\overline{\mathbb R}=\mathbb R^2\sqcup\{\infty\}$, a \textbf{generalized circle} is defined to be either a circle, or a set of the form $\ell\cup\{\infty\}$ where $\ell\subset\mathbb R^2$ is a line.}\\

\noindent Several remarks are in order.  First, by Proposition 4.1, central inversion sends generalized circles to generalized circles.  Clearly, isometries and scalings of the plane also do; the M\"obius transform will be a more general kind of transformation which does this.  We shall define these transformations, by first changing the setting.  Instead of thinking of $\mathbb R^2$ as a two-dimensional plane, we think of it as the complex plane $\mathbb C$.  Thus the extended plane we are dealing with is the Riemann sphere $\overline{\mathbb C}=\mathbb C\sqcup\{\infty\}$.

A clever way to look at $\overline{\mathbb C}$ is by adapting the definition of projective $n$-space.  We recall that projective $n$-space is obtained by taking $\mathbb R^{n+1}-\{\vec 0\}$, and then quotienting out by an equivalence relation which identifies $\vec v,\vec w$ if $\vec v=\lambda\vec w$ for some $\lambda\ne 0$ in $\mathbb R$.  We can easily do the same thing over the complex numbers $\mathbb C$: we define an equivalence relation on $\mathbb C^{n+1}-\{\vec 0\}$ by identifying vectors $\vec v,\vec w$ if there exists $\lambda\ne 0$ in $\mathbb C$ such that $\vec v=\lambda\vec w$, then take the quotient by the relation to get $P^n(\mathbb C)$.\footnote{$P^n(\mathbb C)$ is referred to as \textbf{complex projective space}.  The notion was introduced by Karl Georg Christian von Staudt in 1860.}

Just as in the case for projective space over the real numbers, we may define a projective transformation $P^m(\mathbb C)\to P^n(\mathbb C)$ by taking an injective linear transformation $\mathbb C^{m+1}\to\mathbb C^{n+1}$ and viewing it on equivalence classes of vectors.  In particular, the group of projective transformations of $P^n(\mathbb C)$ is $PGL_{n+1}(\mathbb C)$, obtained by taking the group of nonsingular $(n+1)\times(n+1)$ matrices over $\mathbb C$, modulo the normal subgroup $\{\lambda I_{n+1}:\lambda\in\mathbb C_{\ne 0}\}$.

We will only be interested in the case where $n=1$: in this case, $P^1(\mathbb C)$ is the Riemann sphere $\mathbb C\sqcup\{\infty\}$, for essentially the same reason $P^1(\mathbb R)=\mathbb R\sqcup\{\infty\}$.  In this case, a projective transformation $\varphi$ from $\mathbb C\sqcup\{\infty\}$ to itself takes on the form
$$\varphi(z)=\frac{az+b}{cz+d}$$
with $a,b,c,d\in\mathbb C$ constant, $ad-bc\ne 0$.  Furthermore, if $c=0$ then $\varphi(\infty)=\infty$, and if $c\ne 0$, then $\varphi(-d/c)=\infty$ and $\varphi(\infty)=a/c$.  Such a map is called a \textbf{M\"obius transformation}.

We shall also define an \textbf{anti-M\"obius transform} to be the composition of a M\"obius transform with complex conjugation.  It can be interpreted with the complex conjugation on either side; more generally, an anti-M\"obius transform is of the form $z\mapsto\frac{a\overline z+b}{c\overline z+d}$ with $a,b,c,d\in\mathbb C,ad-bc\ne 0$.  For instance, central inversion is an anti-M\"obius transform by Exercise 6 of the previous section.  Our main result of the chapter is:\\

\noindent\textbf{Proposition 4.3.} \emph{M\"obius transforms and anti-M\"obius transforms are conformal, and preserve generalized circles.  Conversely, every bijection of $\overline{\mathbb C}$ which is conformal and preserves generalized circles is either a M\"obius or an anti-M\"obius transform.}
\begin{proof}
The conformality of M\"obius transforms follows from a more general fact that holomorphic functions are conformal on the complex plane.  Indeed, let $U\subset\mathbb C$ be an open set, and let $f:U\to\mathbb C$ be holomorphic.  Then the limit $f'(a)=\lim_{h\to 0}\frac{f(a+h)-f(a)}h$ is well-defined, and independent of what path in $\mathbb C$ is taken by $h$: if this derivative is $u+vi$, then equating the limits as $h\to 0$ from the positive real axis, and $h\to 0$ from the imaginary axis where the imaginary part is positive, we get $df_a=\begin{bmatrix}u&-v\\v&u\end{bmatrix}$.  This means that at each point $a\in U$, the differential $df_a$, when $f$ is regarded as a map from $U\subset\mathbb R^2$ to $\mathbb R^2$, is of the form $\begin{bmatrix}u&-v\\v&u\end{bmatrix},u,v\in\mathbb C$.  Yet $\begin{bmatrix}u&-v\\v&u\end{bmatrix}$ has orthogonal columns with the same magnitude; it follows from Exercise 4(a) of the previous section that $f$ is conformal.

The anti-M\"obius transforms are not analytic, but they are complex conjugates of analytic functions; thus they preserve angles, but reverse their orientation.

For a proof of conformality which does not use analysis, note that, in view of Exercise 1(b) below, it suffices to show that plane scaling, isometries of $\mathbb R^2$, and the central inversion are conformal [because conformal maps are closed under composition].  This is clear for plane scaling and isometries; for the central inversion, it is Exercise 4(d) of the previous section.

As for the proof that M\"obius/anti-M\"obius transforms preserve generalized circles, again we only need to show that plane scaling, isometries of $\mathbb R^2$, and the central inversion preserve generalized circles.  For plane scaling and isometries, this is clear; for the central inversion, it follows from Proposition 4.1.

As for the converse statement of this proposition, we will let $G$ be the group of bijections of $\overline{\mathbb C}$ which are conformal and preserve generalized circles; and $H$ the group of M\"obius and anti-M\"obius transformations.  By the first statement in this proposition, which we just proved, $H$ is a subgroup of $G$.  We will use Proposition 1.9 to show straightforwardly that $H=G$.

Let $f\in G$.  If $f(\infty)=a\ne\infty$, we let $T_a$ be the translation $z\mapsto z+a$ and $\psi$ the central inversion.  Then $T_a,\psi\in H$, and $\psi\circ T_a^{-1}\circ f$ sends $\infty\mapsto\infty$:
$$\infty\overset{f}\longrightarrow a\overset{T_a^{-1}}\longrightarrow 0\overset{\psi}\longrightarrow\infty$$
By Proposition 1.9, $f\in H$ if and only if $\psi\circ T_a^{-1}\circ f\in H$.  Thus, we need only prove $\psi\circ T_a^{-1}\circ f\in H$; in other words, in showing $f\in G\implies f\in H$, we may assume $f(\infty)=\infty$.

Furthermore, if $f(0)=b$, then again by Proposition 1.9, $f\in H$ if and only if $T_b^{-1}\circ f\in H$.  Yet, $T_b^{-1}\circ f(0)=T_b^{-1}(b)=0$, so we may likewise assume $f(0)=0$.  Finally, by multiplying $f$ by a nonzero complex number, we may assume $f(1)=1$.  Note that this last operation scales and rotates $f$; if the complex number is $re^{i\theta}$ in polar form, it rotates $f$ by an angle of $\theta$ and scales it by $r$.

Hence, in showing $f\in G\implies f\in H$, we may assume $f(0)=0$, $f(1)=1$ and $f(\infty)=\infty$.  In this case, $f$ preserves lines (because lines are precisely the generalized circles containing $\infty$), and so $f$ must fix the line determined by $0,1$; i.e., $f$ must fix the $x$-axis.  If $\ell$ is a horizontal line other than the $x$-axis, then $\ell$ does not meet the $x$-axis at any points $\ne\infty$, hence $f(\ell)$ doesn't meet the $x$-axis away from $\infty$ either, because $f$ is bijective.  This means that $f(\ell)$ is also parallel to the $x$-axis and hence a horizontal line, and $f$ sends all horizontal lines to horizontal lines.  Since $f$ is conformal, it follows that $f$ maps all vertical lines (lines parallel to the $y$-axis) to vertical lines as well.

Therefore, $f$ can be written in the form $(x,y)\mapsto(g(x),h(y))$ from $\mathbb R^2\to\mathbb R^2$.  [Since $f$ sends horizontal lines to horizontal lines, $f(x_1,y)$ and $f(x_2,y)$ are on the same horizontal line, and hence $h$ does not depend on the left argument $x$; similarly for $g$.]  Its differential is then of the form
$$df=\begin{bmatrix}\frac{\partial g}{\partial x}&0\\0&\frac{\partial h}{\partial y}\end{bmatrix}$$
which means $\frac{\partial g}{\partial x}=\pm\frac{\partial h}{\partial y}$ [for all $x,y$] by Exercise 4(a) of the previous section.  Since $\frac{\partial g}{\partial x}$ is a function in $x$ alone, and $\frac{\partial h}{\partial y}$ is a function in $y$ alone, our conclusion is that both of them are actually constant.  If $a=\frac{\partial g}{\partial x}\in\mathbb R$, then $g(x)=ax$ and $h(y)=\pm ay$ for all $x,y$.  Since $f(1)=1$ as complex numbers, $f(1,0)=(1,0)$, hence $g(1)=1$ and $a=1$.  Therefore, $f$ is either the identity or complex conjugation, both of which are in $H$.
\end{proof}

\noindent As we will see in this chapter and the next, M\"obius transforms are essentially the isometries of the hyperbolic and spherical planes.  Section 4.6 will cover a higher-dimensional analogue of the M\"obius transform, which is not so trivial (because there aren't genuine number systems of arbitrary dimensions over $\mathbb R$ which generalize $\mathbb C$), and will go over higher-dimensional spaces. % I admit "spherical plane" is weird.  But does "hyperbolic plane" really make any more sense?

\subsection*{Exercises 4.2. (Generalized Circles and M\"obius Transforms)} % Prove that stereographic projection preserves circles, except the ones that go through the north pole
% which land on lines.  Then define a generalized circle.  Carry everything over to the Riemann sphere P^1(\mathbb C)=\mathbb C\sqcup\{\infty\}, and define the concept of a
% M\"obius transform, showing that they are conformal, and preserve generalized circles.
\begin{enumerate}
\item (a) The group of all M\"obius transforms of $\overline{\mathbb C}$ is generated by the transforms $z\mapsto z+b$, $z\mapsto az$ ($a\ne 0$) and $z\mapsto\frac 1z$.

(b) Every M\"obius or anti-M\"obius transform of $\overline{\mathbb C}$ is an composition of a scaling ($z\mapsto rz,r\in\mathbb R_{>0}$), an isometry of $\mathbb R^2$, and maybe or maybe not the central inversion.

\item The group of M\"obius transforms acts transitively on the set of generalized circles.

\item If $a,b,c\in\overline{\mathbb C}$ are distinct points, show that there is a unique generalized circle passing through them.  Conclude that any two generalized circles intersect in at most two points.

\item If $[a:a_0],[b:b_0],[c:c_0],[d:d_0]\in P^1(\mathbb C)$, define their \textbf{cross ratio} as follows:
$$[\![[a:a_0],[b:b_0],[c:c_0],[d:d_0]]\!]=[(a_0c-c_0a)(b_0d-d_0b):(a_0d-d_0a)(b_0c-c_0b)].$$
Verify that this is well-defined (if and only if the inputs don't consist of three equal points), and Propositions 3.4 and 3.5 hold just as in the case of cross ratios on the real projective line.

Show that $[\![a,b,c,d]\!]\in\mathbb R\cup\{\infty\}$ if and only if $a,b,c,d$ lie on a common generalized circle.  Use this to show that if $a,b,c$ are distinct points, then the generalized circle they determine is $\{x\in\overline{\mathbb C}:[\![a,b,c,x]\!]\in\mathbb R\cup\{\infty\}\}$.  [Since cross ratios are invariant under M\"obius transforms (Proposition 3.5(iv) can be adapted), this provides an alternate proof that M\"obius transforms preserve generalized circles.]

\item If $T$ is an anti-M\"obius transform, show that $[\![T(a),T(b),T(c),T(d)]\!]=\overline{[\![a,b,c,d]\!]}$, i.e., the complex conjugate of $[\![a,b,c,d]\!]$.  [Note that this provides an alternate proof that generalized circles are preserved by $T$.]

\item Let $\omega$ be a generalized circle.  Show that the following are equivalent:

~~~~(i) The central inversion sends $\omega$ to itself;

~~~~(ii) $\omega$ meets the unit circle orthogonally;

~~~~(iii) Either $\omega$ is a line through the origin, or else it is a circle with a radius $r$ and a center $a\in\mathbb C$ such that $|a|^2-r^2=1$.

[(i) $\iff$ (ii) by Exercise 4(d) of the previous section.  (ii) $\iff$ (iii): Explain why (ii) is equivalent to saying that, if you take a radius of the unit circle which lands at an intersection point of the unit circle and $\omega$, then it is tangent to $\omega$.]

\item Let $\omega$ be a generalized circle.  Show that the following are equivalent:

~~~~(i) The central inversion sends $\omega$ to its own negation;

~~~~(ii) The corresponding circle on the sphere (via stereographic projection) is a great circle;

~~~~(iii) The corresponding circle on the sphere contains at least one point and its antipode;

~~~~(iv) The corresponding circle on the sphere is closed under taking antipodes;

~~~~(v) Either $\omega$ is a line through the origin, or else it is a circle with a radius $r$ and a center $a\in\mathbb C$ such that $r^2-|a|^2=1$.

[Show that the negation of the central inversion, when conjugated by stereographic projection, is the map of the sphere sending every point to its antipode.]

\item Suppose $u,a,b,c,v\in\overline{\mathbb C}$ are points with $u\ne v,u,v\notin\{a,b,c\}$.  (We are not requiring $a,b,c$ to be distinct from each other.) %[In other words, there is a segment of the generalized circle with endpoints $u,v$, which contains $a,b,c$, such that $a$ is between $u,b$ and $c$ is between $b,v$.  See diagram.]
Show that $[\![b,a,u,v]\!][\![c,b,u,v]\!]=[\![c,a,u,v]\!]$.  [By applying a M\"obius transform, one may assume $u=0,a=1,v=\infty$.]

\item (a) The group of M\"obius transforms which fix the real line $\mathbb R\cup\{\infty\}$ is isomorphic to $PGL_2(\mathbb R)$.  In other words, a M\"obius transform fixes $\mathbb R\cup\{\infty\}$ if and only if, when considered as an element of $PGL_2(\mathbb C)$, it can be represented with a real matrix.

(b) Use this to show that the group of M\"obius/anti-M\"obius transforms which fix the upper half-plane $\{x+yi\in\mathbb C:y>0\}$ is isomorphic to $PGL_2(\mathbb R)$.  [A M\"obius transformation which fixes the real line, but sends the upper half-plane to the lower-half, can be composed with complex conjugation.]

(c) Conclude that the group of M\"obius/anti-M\"obius transforms which fix the interior of the unit disk is isomorphic to $PGL_2(\mathbb R)$.  [Conjugate the group in part (b) using a M\"obius transform that sends the real line to the unit circle.]

\item (a) Let $a,b,c\in\overline{\mathbb C}$ be distinct points.  Let $a',b',c'\in\overline{\mathbb C}$ be another triple of distinct points.  Show that there is a unique M\"obius transform sending $a\mapsto a',b\mapsto b',c\mapsto c'$. [Adapt Exercise 8 of Section 3.2.]  Conclude that a M\"obius transform which is not the identity fixes at most two points.

(b) Show by example that part (a) may be false for anti-M\"obius transforms.

\item Suppose $a,b,c,d,a',b',c',d'\in\overline{\mathbb C}$.  Assuming the cross ratios are defined, there exists a M\"obius transform sending $a\mapsto a',b\mapsto b',c\mapsto c',d\mapsto d'$ if and only if $[\![a,b,c,d]\!]=[\![a',b',c',d']\!]$.
\end{enumerate}

\subsection*{4.3. The Hyperbolic Plane: Poincar\'e Disk and Half-Plane Models}
\addcontentsline{toc}{section}{4.3. The Hyperbolic Plane: Poincar\'e Disk and Half-Plane Models}
The Poincar\'e disk model is the most common model in which the hyperbolic plane is pictured.  After all, it is of finite structure, its lines are easily depictable, and its circles are also circles to the Euclidean eye.  It was originally proposed by Eugenio Beltrami, yet it is named after Henri Poincar\'e because his discoveries in this matter are better known. % To quote the Wikipedia page on the Poincar\'e disk model: "Along with the Klein model and the Poincaré half-space model, it was proposed by Eugenio Beltrami who used these models to show that hyperbolic geometry was equiconsistent with Euclidean geometry. It is named after Henri Poincaré, because his rediscovery of this representation fourteen years later became better known than the original work of Beltrami."  And isn't it bad to cite Wikipedia?

We shall start by considering the unit circle $x^2+y^2=1$, which is a generalized circle, and showing how to make its interior into a clever kind of plane.  Let $H^2(\mathbb R)$ be the open unit disk $\{(x,y)\in\mathbb R^2:x^2+y^2<1\}$.  Moreover, let $\overline{H^2}(\mathbb R)$ be the closed unit disk, $\{(x,y)\in\mathbb R^2:x^2+y^2\leqslant 1\}$.  Finally, let $H^2_\infty(\mathbb R)=\overline{H^2}(\mathbb R)-H^2(\mathbb R)$, which is the unit circle.

$H^2(\mathbb R)$ is referred to as the \textbf{hyperbolic plane} (rather, the \textbf{Poincar\'e disk model} of the hyperbolic plane).  We will call $H^2_\infty(\mathbb R)$ the \textbf{rim} / \textbf{boundary at infinity}, and its elements are to be referred to as \textbf{ideal points}.  These points are significantly different from the points in $H^2(\mathbb R)$, even from the point of view of intrinsic geometry (the constructions invariant under isometries), as we will see later.  We will call $\overline{H^2}(\mathbb R)$ the \textbf{extended hyperbolic plane}.

Clasiccal hyperbolic geometry deals only with $H^2(\mathbb R)$, just as classical Euclidean geometry deals with the affine plane without the ideal points that give us $P^2(\mathbb R)$.

In $\overline{H^2}(\mathbb R)$, a \textbf{line} (or \textbf{geodesic}) is defined to be a set of the form $\omega\cap\overline{H^2}(\mathbb R)$, where $\omega$ is a generalized circle orthogonal to the unit circle.  In other words, lines are generalized circles which are perpendicular to the rim.  It is clear that $\omega$ can be recovered from the line.
\begin{center}\includegraphics[scale=.25]{PoincareDisk.png}\end{center}

We could build another model as follows: instead of the unit circle, take the $x$-axis, $y=0$.  Then define $H^2(\mathbb R)=\{(x,y)\in\mathbb R^2:y>0\}$, $H^2_\infty(\mathbb R)$ as the $x$-axis (in particular, it contains $\infty$), and $\overline{H^2}(\mathbb R)=H^2(\mathbb R)\cup H^2_\infty(\mathbb R)$.  In this model, a line refers to a set of the form $\omega\cap\overline{H^2}(\mathbb R)$ where $\omega$ is a generalized circle orthogonal to the $x$-axis; thus the lines are vertical lines, and semicircle arcs centered on the $x$-axis.  This is called the \textbf{Poincar\'e (upper) half-plane model} of the hyperbolic plane.  This model is often used in the study of modular forms and elliptic curves.
\begin{center}\includegraphics[scale=.3]{PoincareHalfPlane.png}\end{center}
For the sake of definiteness, $H^2(\mathbb R),\overline{H^2}(\mathbb R),H^2_\infty(\mathbb R)$ will henceforth refer specifically to the Poincar\'e disk model.

Note that a line has exactly two ideal points.  Thus in hyperbolic geometry, lines appear to have endpoints just like line segments; but the fact of the matter is that they continue indefinitely, and have no endpoints in $H^2(\mathbb R)$, even though in $\overline{H^2}(\mathbb R)$, they may be considered to touch the rim.  A portion of a line between two points is called a \textbf{ray} if exactly one of the points is ideal, and a \textbf{line segment} if neither point is ideal.  It is clear that a line segment/ray can be uniquely extended to a line.

If $a,b$ are points on a line, we let $p,q$ be the ideal points of the line, with $p$ on $a$'s side:
\begin{center}\includegraphics[scale=.3]{HDistance.png}\end{center}
We define the \textbf{distance} between $a$ and $b$, denoted as $ab$ or $\rho(a,b)$, to be $\ln[\![b,a,p,q]\!]$, the natural logarithm of the cross ratio.  This is a real number by Exercise 4 of the previous section.  This can be shown to be positive (Exercise 3).  Later we will see that this distance depends on only $a,b$, because the two points determine a line.  By Exercise 8 of the previous section, and well-known properties of logarithms, we have the segment addition postulate: if $a,b,c$ are on the line with $b$ between $a$ and $c$, then $\rho(a,c)=\rho(a,b)+\rho(b,c)$. % You say I need to prove the triangle inequality?  I don't remember using it anywhere.  Plus, the triangle inequality is equivalent to saying that, if \overline{AB} is drawn and circles are drawn centered at A and B, the circles only intersect if their interiors overlap and neither circle goes around the other.  [Because if C is an intersection point of the circles, you can conclude about \triangle ABC.]  This is easy to prove for hyperbolic geometry once you know that circles are Euclidean circles in these models.

If $\ell_1$ and $\ell_2$ are lines, they intersect in at most one point (since two points determine a line).  However, such a point may or may not be an ideal point.  We say that lines are \textbf{divergent-parallel} / \textbf{ultraparallel} if they do not intersect at all; \textbf{convergent-parallel} / \textbf{limiting parallel} if they intersect at an ideal point; and \textbf{intersecting} if they intersect at a regular point of $H^2(\mathbb R)$:
\begin{center}\includegraphics[scale=.2]{HLinePairs.png}\end{center}
If two lines intersect, we define the \textbf{angle measure} between them to be the Euclidean angle measure between them (namely, the angle between the Euclidean tangent lines to the curves at their intersection point).\footnote{A model of a kind of geometry where angle measures coincide with those of the Euclidean angles is called a \textbf{conformal model}.}  If the intersection point is ideal, the angle is always zero (since both lines meet the rim perpendicularly), whereas if the intersection point is regular, the angle is only zero when the lines coincide (Exercise 5(a)).  The angle addition postulate is clear, but we will later see that if we construct two parallel lines and a transversal, we do \emph{not} have congruence of corresponding angles like we did in the Euclidean plane. % A transversal through two parallel lines does not generally have the corresponding angles congruent.  I suppose I should have specified that.

It is now natural to try to show that two points determine a line, etc.  Surprisingly, the best way to carry on right now is to cover what the isometries are.\\

\noindent\textbf{ISOMETRIES}\\

\noindent In the Poincar\'e disk (resp., half-plane) model, isometries are taken precisely to be the M\"obius and anti-M\"obius transforms of $\overline{\mathbb C}$ which fix the unit disk (resp., the upper half of the plane).  Since such transforms are conformal, it is clear that they preserve generalized circles orthogonal to the rim; i.e., they preserve hyperbolic lines.  They also preserve distances, because M\"obius transforms preserve cross ratios, and anti-M\"obius transforms conjugate cross ratios (thus preserving the real cross ratios) by Exercise 5 of the previous section.  Finally, they preserve angle measures, since they are conformal by Proposition 4.3.

The M\"obius transforms are called \textbf{orientation-preserving isometries}, while the anti-M\"obius transforms are called \textbf{orientation-reversing isometries}.

We note that if $a,b,c\in H^2_\infty(\mathbb R)$ are distinct points, as are $a',b',c'\in H^2_\infty(\mathbb R)$, there is a unique isometry sending $a\mapsto a',b\mapsto b',c\mapsto c'$.    Indeed, by Exercise 10(a) of the previous section, there is a unique M\"obius transform $T$ of $\overline{\mathbb C}$ for which $T(a)=a',T(b)=b',T(c)=c'$.  In this case, $T$ must fix the unit circle $H^2_\infty(\mathbb R)$ [because, by Exercise 3 of the previous section, $T$ sends the unique generalized circle through $a,b,c$ to that through $a',b',c'$].  \emph{However}, $T$ may send the interior of the circle to the exterior, and thus it is not technically a symmetry of the plane, since the plane is only the \emph{interior} of the circle.  In this case, however, we compose $T$ with the central inversion $\psi:z\mapsto 1/\overline z$, and we get the desired isometry sending $a\mapsto a',b\mapsto b',c\mapsto c'$, which is incidentally orientation-reversing.

We can do the same for the half-plane model: this time, take $a,b,c,a',b',c'$ on the (extended) $x$-axis.  In this case, $T$ may send the upper half of the plane to the lower half of the plane, but then you can compose it with complex conjugation to get an orientation-reversing isometry.

We shall now obtain a general formula for isometries of the hyperbolic plane.  Let us stick to the half-plane model to start; suppose $T$ is an \emph{orientation-preserving} isometry of the half-plane model.  Then one can write $T(z)=\frac{az+b}{cz+d},ad-bc\ne 0$.  With that, $c,d$ can't both be zero.  Thus by multiplying $a,b,c,d$ by a nonzero constant complex number (which results in the same isometry), we may assume either $c=1$, or $c=0$ and $d=1$.

If $c=1$, then $a=T(\infty)\in\mathbb R$ (because $T$ preserves the half-plane's rim $\mathbb R\cup\{\infty\}$), and similarly, $-d=T^{-1}(\infty)\in\mathbb R$.  Therefore $a,d$ are both real numbers.  Since $T(0)=\frac bd$ and is in $\mathbb R\cup\{\infty\}$, we conclude that $b\in\mathbb R$ as well; thus, $a,b,c,d$ are all real numbers.  If $c=0$ and $d=1$, then $T(z)=az+b$; in this case it is clear that $a,b\in\mathbb R$.

Thus $T$ can be written as $T(z)=\frac{az+b}{cz+d}$ with $a,b,c,d\in\mathbb R,ad-bc\ne 0$.  [Obviously, not \emph{every} rational-function expression of $T$ has real coefficients, though.  We are just saying that there exists such a rational-function expression.]  Moreover, since $T$ fixes the upper half of the plane, it sends $i$ (a point in the upper half) to a point in the upper half.  Computing,
$$T(i)=\frac{ai+b}{ci+d}=\frac{b+ai}{d+ci}=\frac{(b+ai)(d-ci)}{c^2+d^2}=\frac{bd+ac}{c^2+d^2}+i\frac{ad-bc}{c^2+d^2}$$
Since this point is in the upper half of the plane, its imaginary part is positive; therefore $ad-bc>0$.  The reader can readily verify that, conversely, every $z\mapsto\frac{az+b}{cz+d}$ with $a,b,c,d\in\mathbb R,ad-bc>0$ is an orientation-preserving isometry, thus we have:
\begin{center}
\textbf{The orientation-preserving isometries of the Poincar\'e half-plane model are the rational functions of the form $\frac{az+b}{cz+d}$, where $a,b,c,d\in\mathbb R$ and $ad-bc>0$.}
\end{center}
Clearly, $\begin{bmatrix}a&b\\c&d\end{bmatrix}\mapsto\frac{az+b}{cz+d}$ is a surjective homomorphism from $GL^+_2(\mathbb R)$ to the orientation-preserving isometry group $\operatorname{Isom}^+(H^2(\mathbb R))$ of the half-plane model, whose kernel is $\mathbb R^*=\{aI_2:a\ne 0\text{ in }\mathbb R\}$.  By Theorem 1.17, we get that the orientation-preserving isometry group of the half-plane model is isomorphic to $PGL^+_2(\mathbb R)$.  By Exercise 8(c) of Section 1.5, it is hence also isomorphic to $PSL_2(\mathbb R)$.

The following facts can be likewise shown, and the arguments are left to the reader:
\begin{itemize}
\item The orientation-reversing isometries of the Poincar\'e half-plane model are the maps of the form $z\mapsto\frac{a\overline z+b}{c\overline z+d}$, where $a,b,c,d\in\mathbb R$ and $ad-bc<0$.

\item The orientation-preserving isometries of the Poincar\'e disk model are the rational functions of the form $\frac{az+b}{\overline bz+\overline a}$, where $a,b\in\mathbb C$ and $|a|^2>|b|^2$. [\emph{Note}: Instead of proving this from scratch, one could conjugate the half-plane model's isometries via the transforms in Exercise 6.]

\item The orientation-reversing isometries of the Poincar\'e disk model are the functions $z\mapsto\frac{a\overline z+b}{\overline b\overline z+\overline a}$, where $a,b\in\mathbb C$ and $|a|^2>|b|^2$.
\end{itemize}
\noindent Exercises 1 and 2 show that the isometries of the hyperbolic plane act transitively on the regular points, and that the stabilizer of a regular point consists of rotations and reflections over the point.  Thus, \emph{locally} speaking, this plane behaves identically to the Euclidean plane.\\

\noindent\textbf{BASIC GEOMETRIC RESULTS}\\

\noindent At this point, we shall look back at Section 2.1's results on Euclidean plane geometry.  We will show that certain statements, but not others, also hold for the hyperbolic plane.  However, we will not be deducing theorems from axioms, because the setting we are dealing with is concrete.  We shall often use the isometries (and their transitivity \---- Exercise 1) to our advantage, as they will enable us to arrange for points/lines/etc.~to be situated conveniently.  We are also free to switch between the disk and half-plane models (Exercise 6).\\

\noindent\textbf{Proposition 4.4.} \emph{In the hyperbolic plane, two points determine a line (regardless of whether the points are ideal).  If at most one is ideal, they also determine a ray.  If both points are regular, they determine a line segment and two rays.}
\begin{proof}
Let $p_1,p_2\in\overline{H^2}(\mathbb R)$ be distinct points.

If both points are ideal, we may assume we are in the half-plane model, and that $p_2=\infty$.  Being an ideal point different from $p_2$, $p_1$ is some real number $r$.  Since the only lines containing $\infty$ are the vertical lines, it follows that the unique line passing through $p_1,p_2$ is the vertical line $x=r$.

Now let us suppose that $p_1$ is \emph{not} an ideal point.  We may assume that we are in the Poincar\'e disk model, and (by applying an isometry), that $p_1=0$.  Then $p_2$ is some nonzero complex number with absolute value $\leqslant 1$; moreover, the Euclidean line through $p_1,p_2$ is a diameter of the unit circle, hence is a hyperbolic line.  To show that this line is unique, we let $\ell$ be a line through $p_1,p_2$, then we show that it must be a Euclidean line (which uniquely determines it).  If $\ell$ is not a Euclidean line, then it is (by Exercise 6 of the previous section), a circle with a radius $r$ and a center $a\in\mathbb C$ such that $|a|^2-r^2=1$.  Moreover, this circle is given by an equation $|z-a|=r$.  Since $0=p_1\in\ell$, we get $|0-a|=r$, hence $|a|=r$ and so $|a|^2-r^2=0\ne 1$, a contradiction.

This proves that there is a unique line through $p_1$ and $p_2$.  If $p_2$ is an ideal point, the Euclidean line segment from $p_1$ to $p_2$ is a ray (it has only one \emph{regular} endpoint), and is the unique ray determined by them.  If $p_2$ is not an ideal point, the verification of the line segment and the two rays is left to the reader.
\end{proof}

\noindent We continue to run through our list of Euclidean axioms. % I said it explicitly before Proposition 4.4, but I guess that's insufficient

Proposition 2.2's analogue holds as well; a hyperbolic line segment has well-defined endpoints.  Rays and lines have endpoints too, except some of them are ideal points.  We leave the verifications to the reader.  Axiom 2.4 can be carried over as well; the addition postulate has been covered when distances were defined at the beginning of the section, and Axiom 2.4(i) can easily be proven for the hyperbolic plane by assuming $A$ is the center of the Poincar\'e disk and using Exercise 4.  Incidentally, points of a given distance are closer to the Euclidean eye when they are closer to the rim, as arbitrary distances can be measured from any regular points, and nothing will go outside the bounds.

However, the parallel postulate (Axiom 2.5) does \emph{not} hold in the hyperbolic plane; instead, we have\\

\noindent\textbf{Proposition 4.5.} \emph{Given a line $\ell$ and a point $a\notin\ell$, there are infinitely many lines through $a$ parallel to $\ell$.  Exactly two of them are convergent-parallel.}
\begin{center}\includegraphics[scale=.3]{HypParallels.png}\end{center}
\begin{proof}
The above diagram, where $\ell$ is the horizontal Euclidean line, should give an intuitive picture why the proposition is true.  Here is a formal proof.

We may assume that we are in the half-plane model, and that $\ell$ is the $y$-axis, $x=0$.  Then $a$ is some point $u+vi$, such that $u\ne 0$ (because $a\notin\ell$).  We may further assume $u>0$.

Fix any $0\leqslant r\leqslant u$, and consider the line $\ell'$ through $r$ (as an ideal point) and $a$.  We claim it is parallel to $\ell$.  If $r=u$, then $\ell'$ is the vertical line $x=u$, hence is convergent-parallel to $\ell$, meeting at $\infty$.  If $r<u$, then $\ell'$ is a semicircle arc, where $r$ is one of the endpoints of the diameter.  Clearly, $r$ is the left endpoint, because the semicircle contains the point $a$ with a larger real coordinate.  This implies in particular that the real coordinate of any point on the semicircle is $\geqslant r\geqslant 0$, hence $\ell'$ is parallel to $\ell$ (and convergent-parallel if and only if $r=0$).  Thus, $\ell'$ is parallel to $\ell$; it does not intersect at any \emph{regular} point.  Clearly these lines are all distinct for the different values of $x$, hence there are infinitely many lines through $a$ parallel to $\ell$.

The reader can readily verify that if $p$ and $q$ are the ideal points of $\ell$, then the lines $\overset{\longleftrightarrow}{p~a}$ and $\overset{\longleftrightarrow}{q~a}$ are the only two lines through $a$ that are convergent-parallel to $\ell$.
\end{proof}
\noindent Next up is Axiom 2.6, concerning angles.  Since angle measures in Poincar\'e disk/half-plane models are just the Euclidean angle measures, the statements of Axiom 2.6 are easy to prove, using either one of the following two strategies: (1) use the disk model, and apply an isometry to assume the angles' endpoint $A$ is the center, so that all lines through $A$ are Euclidean lines; or (2) use Exercise 5(a) to prove the hyperbolic analogue of Axiom 2.6(i).

The concept of complementary, supplementary and right angles are carried over from Euclidean geometry, as are the concepts of linear pairs and vertical angles.  It is clear, just as in the Euclidean case, that angles in a linear pair are supplementary, and vertical angles are congruent.  We also have the perpendicular postulate:\\

\noindent\textbf{Proposition 4.6.} \textsc{(Perpendicular postulate)} \emph{Given a line $\ell$ and a point $a$ which is not equal to either of ideal points of $\ell$, there exists a unique line through $a$ perpendicular to $\ell$.}\\

\noindent Note that the proposition works if $a$ is an ideal point, just not if $a$ is one of the ideal points of $\ell$.
\begin{proof}
We may assume we are in the half-plane model and $\ell$ is the $y$-axis.  Note that the lines perpendicular to $\ell$ are precisely the semicircle arcs centered at $0$ (why?).  Under the hypothesis, $a$ is some complex number $u+vi\ne 0$ (since $0$ and $\infty$ are the ideal points of $\ell$, $a$ is not equal to either of them).  It is then clear that the unique line through $a$ perpendicular to $\ell$ is the semicircle arc of radius $\sqrt{u^2+v^2}$ centered at $0$.
\end{proof}

\noindent Since the parallel postulate fails, we do not have the nice properties of parallel lines and their transversals that Euclidean geometry possesses (Axiom 2.9 and Proposition 2.10).  However, given two divergent-parallel lines, there is one particular point which holds all the ``nice'' transversals, as we will eventually see.  The first main result is\\

\noindent\textbf{Theorem 4.7.} \textsc{(Ultraparallel Theorem)} \emph{If $\ell_1$ and $\ell_2$ are divergent-parallel lines, there is a unique line $\ell$ simultaneously perpendicular to both $\ell_1$ and $\ell_2$.}\\

\noindent Note that the line would not be unique in Euclidean geometry: given two parallel lines, there are infinitely many lines perpendicular to one of them, and all those lines are perpendicular to the other as well.
\begin{proof}
Again we may assume we are in the half-plane, and that $\ell_1$ is the $y$-axis.  Since $\ell_1$ and $\ell_2$ are divergent-parallel, $\ell_2$ is a semicircle arc whose endpoints are both (strictly) positive or both negative.  We may assume the endpoints are both positive.
\begin{center}\includegraphics[scale=.5]{UltraparallelTheorem.png}\end{center}
In this case, if $a$ and $r$ are the center and radius of $\ell_2$ respectively, then $a>r>0$ and $\ell_2$ is given by an equation $|z-a|=r$.  If $\ell$ is a line perpendicular to $\ell_1$ which intersects $\ell_2$, then $\ell$ is a semicircle arc centered at $0$; let $u$ be its radius.  Then since a radius of a circle is perpendicular to the circle's arc, we get that $\ell\perp\ell_2\iff$ the radii of $\ell,\ell_2$ going to their intersection $\ell\cap\ell_2$ are perpendicular $\iff$ the triangle with vertices $0,\ell\cap\ell_2,a$ has a right angle at $\ell\cap\ell_2\iff$ the squares of the Euclidean radii of $\ell,\ell_2$ add to $a^2$ (by the Pythagorean Theorem) $\iff u^2+r^2=a^2\iff u=\sqrt{a^2-r^2}$.

Furthermore, the unique line $\ell$ perpendicular to both lines is the semicircle arc centered at $0$ with radius $\sqrt{a^2-r^2}$.
\end{proof}

\noindent $\ell$ is referred to as the \textbf{common perpendicular} to $\ell_1,\ell_2$.  Note that if $\ell_1,\ell_2$ intersect at any point, even if it is an ideal point, then no such line $\ell$ exists; this would contradict the uniqueness part of Proposition 4.6, because if $p=\ell_1\cap\ell_2$ then $\ell_1$ and $\ell_2$ would both be lines through $p$ perpendicular to $\ell$.  Thus Theorem 4.7 is false if $\ell_1,\ell_2$ are convergent-parallel.

If $\ell$ is the common perpendicular to $\ell_1,\ell_2$, the midpoint of the line segment from $\ell\cap\ell_1$ to $\ell\cap\ell_2$ will be referred to as the \textbf{attention point} of $\ell_1$ and $\ell_2$.  Later we will see that a transversal has congruent corresponding angles if and only if it passes through the attention point.\\%for lack of better terminology

\noindent The best way to do that is to cover triangle theorems first.  The definition of a triangle is identical to that for the Euclidean plane; it uses three noncollinear points, and the \emph{hyperbolic} line segments connecting them.  Note that the vertices of a triangle may be ideal points; clearly a vertex is an ideal point if and only if the angle measure is zero.  A triangle is called \textbf{regular} if it has no ideal vertices; an \textbf{omega triangle} if it has exactly one ideal vertex; and an \textbf{ideal triangle} if all its vertices are ideal vertices.

We recall (Proposition 2.11) that the angles of a Euclidean triangle always add to $\pi$, or $180^\circ$.  Contrariwise, the sum of the angles of a hyperbolic triangle is always \emph{less} than $180^\circ$:\\

\noindent\textbf{Proposition 4.8.} \emph{The measures of the angles of a hyperbolic triangle add to $<180^\circ$.}
\begin{proof}
This proposition is trivial for ideal triangles, so we may assume that the triangle is not an ideal triangle.  We let $a,b,c$ be the vertices of the triangle.  Assume $a$ is a regular point, and (by applying an isometry) that it is the center of the Poincar\'e disk:
\begin{center}\includegraphics[scale=.25]{HypTriangle.png}\end{center}
The \emph{Euclidean triangle} $\triangle abc$ shares the same line segments $\overline{ab},\overline{ac}$.  However, the Euclidean line segment from $b$ to $c$ stretches outside the triangle, and the hyperbolic triangle's side curves inwards.  To see why this is so, note that the hyperbolic $\overline{bc}$ must be from a circle with a center $u$ and radius $r$ such that $|u|^2-r^2=1$, and that such a circle has $0$ as an exterior point, and hence, inside the unit disk, this circle necessarily curves toward the origin.

Due to this, it is easy to see that the angles at $b$ and $c$ are smaller than the corresponding angles of the Euclidean triangle.  Yet the vertex $a$ has the exact same angle.  Thus the sum of the three angles is smaller than the sum of the three angles of the Euclidean triangle, which is $180^\circ$.
\end{proof}

\noindent The concept of congruence of triangles is taken exactly from the Euclidean case: $\triangle abc\cong\triangle a'b'c'$ if $\overline{ab}\cong\overline{a'b'},\angle bac\cong\angle b'a'c'$, etc.  It so happens that in the hyperbolic plane, triangles with congruent angles are always congruent.  Thus, there is no irredundant notion of ``similar'' triangles in hyperbolic geometry.\\

\noindent\textbf{Proposition 4.9.} (i) \textsc{(Side-side-side / SSS)} \emph{If $\overline{ab}\cong\overline{a'b'},\overline{bc}\cong\overline{b'c'},\overline{ca}\cong\overline{c'a'}$, and all vertices of the triangles are regular, then $\triangle abc\cong\triangle a'b'c'$.}

(ii) \textsc{(Side-angle-side / SAS)} \emph{If $\overline{ab}\cong\overline{a'b'}$, $\overline{ac}\cong\overline{a'c'}$, $\angle a\cong\angle a'$ and $a,a'$ are regular, then $\triangle abc\cong\triangle a'b'c'$.}

(iii) \textsc{(Hypotenuse-leg / HL / RHS)} \emph{If $\angle b,\angle b'$ are right angles, $\overline{ab}\cong\overline{a'b'}$ and $\overline{ac}\cong\overline{a'c'}$, then $\triangle abc\cong\triangle a'b'c'$.}

(iv) \textsc{(Angle-side-angle / ASA)} \emph{If $\angle a\cong\angle a'$, $\overline{ab}\cong\overline{a'b'}$, $\angle b\cong\angle b'$, and both $a,b$ are regular, then $\triangle abc\cong\triangle a'b'c'$.}

(v) \textsc{(Angle-angle-side / AAS)} \emph{If $\angle a\cong\angle a'$, $\angle b\cong\angle b'$, $\overline{bc}\cong\overline{b'c'}$, and both $a,b$ are regular, then $\triangle abc\cong\triangle a'b'c'$.}

(vi) \textsc{(Angle-angle-angle / AAA)} \emph{If $\angle a\cong\angle a'$, $\angle b\cong\angle b'$, and $\angle c\cong\angle c'$, then $\triangle abc\cong\triangle a'b'c'$.}\\

\noindent Several remarks are in order.  First, side-side-angle is once again invalid; the argument of Exercise 5 of Section 2.1 applies.  Also, in the Euclidean case, (iv) automatically implied (v), due to two angle measures of a triangle determining the measure of the third angle.  This is not true for hyperbolic triangles (the angle measures can have any sum $<180^\circ$), and hence, (iv) and (v) must be given separately.
\begin{proof}
Each of these can be proved by applying convenient isometries to the triangles, which cause certain corresponding parts to coincide.  We will prove (ii) here to illustrate the idea.

By using the Poincar\'e disk, and applying an isometry to $\triangle abc$, we may assume $a=0$, $b$ is a positive real number, and $c$ has a positive imaginary part.  We may do the same thing to $\triangle a'b'c'$, assuming $a'=0$, $b'$ is a positive real number and $c'$ has a positive imaginary part.  [Of course, there generally isn't an isometry which does this to both triangles at once, but we do not need such an isometry.]

First, we are given that $\overline{ab}\cong\overline{a'b'}$; comparing their segment lengths and using Problem 4, $\ln\frac{1+b}{1-b}=\ln\frac{1+b'}{1-b'}$, and therefore $b=b'$.  Since $\angle bac\cong\angle b'a'c'=\angle bac'$, we conclude that $c,c'$ lie on the same ray from the origin $a$.  Finally, from $\overline{ac}\cong\overline{a'c'}$ we conclude $c=c'$.  Therefore, we certainly have $\triangle abc\cong\triangle a'b'c'$, because $\triangle abc$ \emph{is} the triangle $\triangle a'b'c'$ [after the application of the isometries].
\end{proof}

\noindent\textbf{Proposition 4.10.} \textsc{(Isosceles Triangle Theorem)} \emph{If $\triangle abc$ is a triangle, then $\overline{ab}\cong\overline{ac}$ if and only if $\angle b\cong\angle c$.}
\begin{proof}
Copy the proof of Proposition 2.15, using Proposition 4.9 in place of Axiom 2.13.
\end{proof}

\noindent Hyperbolic triangles will be studied in more detail in Section 4.5.

With the aid of triangle congruence theorems, we may finally prove\\

\noindent\textbf{Proposition 4.11.} \emph{Let $\ell_1$ and $\ell_2$ be two divergent-parallel lines, $\ell_3$ a transversal, and $p$ the attention point of $\ell_1,\ell_2$.  Then the following are equivalent:}

(i) \emph{$p\in\ell_3$.}

(ii) \emph{Corresponding angles are congruent.}

(iii) \emph{Corresponding exterior angles are supplementary.}

(iv) \emph{Alternate exterior angles are congruent.}

(v) \emph{Corresponding interior angles are supplementary.}

(vi) \emph{Alternate interior angles are congruent.}
\begin{proof}
(i) $\implies$ (vi). Let $\ell$ be the common perpendicular to $\ell_1$ and $\ell_2$; let $a_1=\ell\cap\ell_1,a_2=\ell\cap\ell_2$.  Then $p$ is the midpoint of $\overline{a_1a_2}$ by definition.  Moreover, let $b_1=\ell_3\cap\ell_1,b_2=\ell_3\cap\ell_2$.  Then $\angle a_1pb_1\cong\angle a_2pb_2$ because they are vertical angles; $\angle pa_1b_1\cong\angle pa_2b_2$, because $\ell$ is perpendicular to both $\ell_j$, so those are both right angles; and $\overline{a_1p}\cong\overline{a_2p}$ since $p$ is the midpoint of $\overline{a_1a_2}$.  By ASA congruence (Proposition 4.9(iv)), $\triangle a_1b_1p\cong\triangle a_2b_2p$.  Hence $\angle a_1b_1p\cong\angle a_2b_2p$, which proves (vi).

(vi) $\implies$ (i). Let $b_1=\ell_3\cap\ell_1,b_2=\ell_3\cap\ell_2$, and let $m$ be the midpoint of $\overline{b_1b_2}$.  By Proposition 4.6, there is a unique line $\ell$ through $m$ which is perpendicular to $\ell_1$.  (We do not yet know if $\ell$ is perpendicular to $\ell_2$.)  Set $a_1=\ell\cap\ell_1,a_2=\ell\cap\ell_2$.  Then $\angle a_1mb_1\cong\angle a_2mb_2$ (vertical angles), $\overline{mb_1}\cong\overline{mb_2}$ (virtue of a midpoint), and $\angle a_1b_1m\cong\angle a_2b_2m$, by the hypothesis (vi).  Therefore by ASA, $\triangle a_1b_1m\cong\triangle a_2b_2m$.  Hence $\overline{a_1m}\cong\overline{a_2m}$, so $m$ is the midpoint of $\overline{a_1a_2}$.  Yet also, $\angle ma_2b_2\cong\angle ma_1b_1$, which is a right angle, and hence $\ell$ is perpendicular to $\ell_2$.  Thus $\ell$ is the common perpendicular of $\ell_1,\ell_2$ and $m$ is its midpoint; in other words, $m=p$ and so $p\in\ell_3$.

As in Proposition 2.10, all of (ii) - (vi) are equivalent because angles in linear pairs are supplementary.
\end{proof}

\noindent Circles shall now be introduced: as in the Euclidean case, if $r$ is a positive real number and $o\in H^2(\mathbb R)$ is a regular point, the set of points $a$ whose distance to $o$ equals $r$ is called a circle.  It is clear that $o$ needs to be regular; if $o$ were an ideal point, its distances to other points would be infinite.  Nevertheless, there is a notion of a ``circle centered at an ideal point''; it will be covered in the next section.

For circles in the hyperbolic plane, the length of a diameter is exactly twice the radius.  Two different circles have at most two intersection points.  [All of this follows exactly as in the Euclidean case.]  Moreover, an interesting result about the Poincar\'e disk and half-plane models is this:\\

\noindent\textbf{Proposition 4.12.} \emph{In the Poincar\'e disk and half-plane models of the hyperbolic plane, the circles are precisely the Euclidean circles consisting entirely of regular points.  Moreover, Axiom 2.19 is satisfied in the hyperbolic plane.}\\

\noindent Each circle is a Euclidean circle, but unless it is centered at the origin in the Poincar\'e disk model, the center of the circle is a different point from the center of the Euclidean circle; see Exercise 12.
\begin{proof}
In the Poincar\'e disk model, it is clear that circles centered at $0$ are Euclidean circles.  However, any regular point can be transferred to $0$ via an isometry (Exercise 1(a)), and the isometries, being (anti)-M\"obius transforms, preserve generalized circles.  Thus, every circle is a generalized circle of $\overline{\mathbb C}$.  Since the disk and half-plane models are related via a M\"obius transform (Exercise 6), this is also true for circles in the half-plane model.  Since $\infty$ is not a regular point of either model, we conclude that every circle is a Euclidean circle (since it does not contain $\infty$, it is not a Euclidean line).

This proves that every circle is a Euclidean circle consisting of regular points.  Conversely, if $\omega$ is a Euclidean circle consisting of regular points, let $a,b,c\in\omega$ be three distinct points, and take perpendicular bisectors of the line segments connecting them, as in Exercise 17 of Section 2.1.  If $p$ is the intersection point of two of these perpendicular bisectors, then $p$ is equidistant from the points $a,b,c$; by applying an isometry, we may assume we are in the Poincar\'e disk model and $p=0$.  By Exercise 4, $a,b,c$ are then equidistant from the origin in the Euclidean sense, which implies that $\omega$ \---- the generalized circle determined by them \---- is a Euclidean circle centered at the origin.  As such, it is also a circle in the hyperbolic plane centered at the origin, and thus every Euclidean circle consisting of regular points is a hyperbolic circle.

Axiom 2.19 for the hyperbolic plane can readily be proven, by assuming $O$ is the origin of the Poincar\'e disk, and $O'$ is on the positive $x$-axis.  The argument is left to the reader.  Note Exercise 4 and the segment addition postulate.
\end{proof}

\noindent Observe that Proposition 2.20 cannot be applied to hyperbolic geometry because angles of triangles do not add to $180^\circ$ (they add to less than that).  Proposition 2.20 does not exactly hold in hyperbolic geometry; it does have an analogue, but we will not need it.\\ % https://www.maths.gla.ac.uk/wws/cabripages/hyperbolic/hybrid2.html

\noindent\textbf{CATEGORIZING ISOMETRIES OF $H^2(\mathbb R)$}\\

\noindent We shall conclude this section by naming different kinds of isometries of the hyperbolic plane, just like we did for the Euclidean plane in Section 2.6.  In the long run, these isometries are similar to the Euclidean ones.  We recall that every isometry of the Euclidean plane $\mathbb R^2$ is either the identity, a translation, a rotation, a reflection, or a glide reflection.  All of these isometries exist for the hyperbolic plane too.

A \textbf{translation} along a line $\ell$ is an orientation-preserving isometry which fixes $\ell$ and moves all the points in $\ell$ in the same direction along the line.  For example, if $\ell$ is the $y$-axis in the half-plane model, a translation is a Euclidean scaling map, $z\mapsto rz$ with $r>0$.  [Certainly these are not Euclidean isometries, but being M\"obius transforms which fix the rim, they are hyperbolic isometries.]  A \textbf{rotation} is an orientation-preserving isometry which fixes a regular point; for example, rotations around $0$ in the Poincar\'e disk model are Euclidean rotations (Exercise 2).  A \textbf{reflection} is an orientation-reversing isometry which fixes a line pointwise, and a \textbf{glide reflection} is a composition of a reflection over a line and a translation along that line.  The reader is encouraged to work these out for the $y$-axis in the half-plane model.  However, there is one kind of isometry which differs from the rest and doesn't exist in Euclidean geometry: it is called a \textbf{horolation}, and it intuitively rotates around an ideal point.  Just as we didn't define circles centered at ideal points, horolations have a different state of mind.

If $p$ is a regular point, the rotations around $p$ form a group isomorphic to the circle group $SO(2)$ (or $\mathbb R/\mathbb Z$), because one can use an isometry to assume $p=0$ in the disk, then apply Exercise 2.  If $p$ is an ideal point, however, there are many orientation-preserving isometries which fix $p$.  Examples are translations along a line containing $p$; such an isometry fixes one other ideal point (namely the other end of the line).  Horolations, contrariwise, fix the ideal point $p$ and \emph{no other points of the plane, whether ideal or not}; they form a group isomorphic to $\mathbb R$.

An example of a horolation can be given in the half-plane model: $z\mapsto z+r$, with $r\in\mathbb R$ fixed.\footnote{Do not get fooled: this is \emph{not} a translation, because horizontal lines are \emph{not} hyperbolic lines in this model.}  The only fixed point of this map is $\infty$, which is an ideal point.  Hence it ``rotates around $\infty$.''  An example of a horolation of the Poincar\'e disk model is
$$z\mapsto\frac{(2+ir)z-ir}{irz+(2-ir)}$$
because, as the reader can verify, the only fixed point of this in $\overline{H^2}(\mathbb R)$ is $1$.  It turns out that the trajectory of a regular point under the above map in the disk model (for various $r$) is a Euclidean circle tangent to the unit circle at $1$, as we will see in the next section.

Horolations are the only ``new'' kinds of isometries of the hyperbolic plane, as we will now show.\\

\noindent We start by letting $T$ be an orientation-preserving isometry of the half-plane (although it doesn't matter which model we use).  Then $T$ can be expressed as $T(z)=\frac{az+b}{cz+d}$ with $a,b,c,d\in\mathbb R,ad-bc>0$ (see the discussion under ``\textsc{Isometries}'').  What we can advantageously do is conjugate $A=\begin{bmatrix}a&b\\c&d\end{bmatrix}$ by an invertible real-valued matrix, to get a conjugate of $T$ in the group of isometries.  [If we conjugate by a matrix with negative determinant, the M\"obius transform corresponding to what we conjugated by would send the upper half of the plane to the lower half, but then we can compose it with complex conjugation, which commutes with the other transforms involved because they use real coefficients: in this case, we would be conjugating $T$ by an orientation-reversing isometry.] % I suppose I should have clarified what I meant.  If we conjugate by a matrix with the negative determinant, the *conjugate* will still fix the upper half, but the matrix we conjugated *by* sends the upper to the lower.

Thus, it is tempting to put the matrix $A$ in Jordan normal form.  However, we can only conjugate $A$ by \emph{real}-valued matrices, hence this cannot be done if $A$ has nonreal eigenvalues.  In this case, $A$'s eigenvalues are complex conjugates $u\pm vi$, and then $A$ is conjugate (by a real matrix) to $\begin{bmatrix}u&-v\\v&u\end{bmatrix}$.  Thus, depending on whether the eigenvalues are real, and whether their geometric multiplicities coincide with their algebraic multiplicities, $A$ can be conjugated to a real matrix of one of these forms:
$$\text{(i)} \begin{bmatrix}u&0\\0&v\end{bmatrix},u,v\ne 0;~~~~\text{(ii)} \begin{bmatrix}u&1\\0&u\end{bmatrix},u\ne 0;~~~~\text{(iii)} \begin{bmatrix}u&-v\\v&u\end{bmatrix},v\ne 0.$$
In case (i), $T$ gets conjugated to the map $U:z\mapsto\frac{uz}v=\frac uvz$.  The map $U$ multiplies elements of the complex plane by a fixed real scalar $\frac uv$, which must be positive (otherwise $U$ would send the upper half of the plane to the lower half).  In other words, $U$ is a dilation of the complex plane centered at the origin.  As a hyperbolic isometry, (unless $U$ is the identity), the only line which gets fixed by $U$ is the $y$-axis, and in fact, $U$ is a translation of the plane along the $y$-axis.  This means that $T$ is a translation (which could be along any line; it's just that whatever isometry we conjugated by must send this line to the $y$-axis).

In case (ii), $T$ gets conjugated to the map $U:z\mapsto\frac{uz+1}u=z+\frac 1u$.  This is a Euclidean horizontal translation, and its only fixed point is the ideal point $\infty$, which means that it is a horolation around the point $\infty$.  Consequently, $T$ is a horolation (which again could be around any ideal point).

In case (iii), $T$ gets conjugated to the map $U:z\mapsto\frac{uz-v}{vz+u}$.  Basic algebra shows that the only points in the extended complex plane fixed by $U$ are $\pm i$, as follows.  First, since $v\ne 0$, $U$ does not fix $\infty$.  If $z\in\mathbb C$, then
$$U(z)=z\iff\frac{uz-v}{vz+u}=z\iff uz-v=z(vz+u)=vz^2+uz$$
$$\iff vz^2+v=0\iff v(z^2+1)=0\iff z^2+1=0\iff z=\pm i$$
Since $-i$ is not in the upper half-plane, but $i$ is, we get that $i$ is the only fixed point of $U$, and is a regular point; therefore, $U$ rotates around the point $i$, and in this case $T$ is a rotation.

Hence the only orientation-preserving isometries of the hyperbolic plane are the identity, translations, horolations and rotations.

Now we let $T$ be an orientation-\emph{reversing} isometry.  This time, one can express it as $T(z)=\frac{a\overline z+b}{c\overline z+d}$ where $a,b,c,d\in\mathbb R,ad-bc<0$.  Again let $A=\begin{bmatrix}a&b\\c&d\end{bmatrix}$.  When we conjugate $A$ by an invertible real-valued matrix, we get a matrix which gives rise to a conjugate of $T$ in the isometry group.  [Commuting of the complex conjugation map is needed to verify this, but this map always commutes with the rational functions with real coefficients, i.e., if $a,b,c,d\in\mathbb  R$ then $\overline{\left(\frac{az+b}{cz+d}\right)}=\frac{a\overline z+b}{c\overline z+d}$.]  This time, since $\det A=ad-bc<0$, $A$ must have two distinct real eigenvalues, one positive and one negative (because, for instance, if its eigenvalues were nonreal, they would be complex conjugates, hence their product \---- the determinant of $A$ \---- would be positive).

Write the conjugate of $A$ as $\begin{bmatrix}u&0\\0&v\end{bmatrix}$ where $u>0,v<0$ are real numbers.  Then $T$ is conjugated to the map $U:z\mapsto\frac uv\overline z$.  If $\frac uv=-1$, then $U(z)=-\overline z$, which means $U$ is the Euclidean reflection over the imaginary axis (the $y$-axis).  Since the $y$-axis is a hyperbolic line, $U$ (hence also $T$) is a reflection over a hyperbolic line in this case.  We leave it to the reader to verify that if $\frac uv\ne -1$, $T$ is a glide reflection.  This means that, just like in the Euclidean case, every orientation-reversing isometry of the plane is either a reflection or a glide reflection.

\subsection*{Exercises 4.3. (The Hyperbolic Plane: Poincar\'e Disk and Half-Plane Models)} % Outline:
% Introduce points and lines in the two Poincar\'e models.  Now, define isometries, and show that they preserve lines, and use the notion to define
% circles, and Section 2.1's results (or analogues of the results).
\begin{enumerate}
\item Isometries of the hyperbolic plane act transitively on each of the following sets:

(a) The set of regular points, $H^2(\mathbb R)$.

(b) The set of ideal points $H^2_\infty(\mathbb R)$.

(c) The set of lines.

\item Let $T$ be an isometry of the Poincar\'e disk which fixes the origin, i.e., $T(0)=0$.  Show that $T$ is a Euclidean rotation or reflection around the origin.

\item (a) If $y_2>y_1>0$ are real numbers, then the distance between $iy_1,iy_2\in\overline{\mathbb C}$ in the Poincar\'e half-plane model is $\ln\frac{y_2}{y_1}$.  [Compute it as $\ln[\![iy_2,iy_1,0,\infty]\!]$.] % My fault

(b) Conclude that the distance between any two distinct non-ideal points in the hyperbolic plane is positive.  [Use a M\"obius transform.]

\item Let $z$ be a complex number with $|z|<1$.  Show the distance from $z$ (as a point in the Poincar\'e disk model) to $0$ is equal to $\ln\frac{1+|z|}{1-|z|}$.

\item (a) If $p\in H^2(\mathbb R)$, and $\vec v$ is a nonzero vector at $p$, show that there is a unique (hyperbolic) line through $p$ tangent to $\vec v$.

(b) Explain why this is false if $p$ is an ideal point.

\item Let $T:z\mapsto\frac{z-i}{z+i}$ and $U:w\mapsto i\frac{1+w}{1-w}$.  Then $T$ and $U$ are M\"obius transforms, with $T$ sending the upper half-plane model of the hyperbolic plane to the Poincar\'e disk model, and $U$ being its inverse.  This provides a way to convert between the two models given in this section.

\item (a) Show that every isometry of $H^2(\mathbb R)$ is the composition of finitely many reflections.  [Use the categorization of isometries.]

(b) If $0\ne a\in H^2(\mathbb R)$, then the unique reflection of $H^2(\mathbb R)$ sending $0$ to $a$ is given by $\psi(z)=\frac{a(\overline z-\overline a)}{\overline a(a\overline z-1)}$.  What line does it reflect across?

\item The isometry group of the upper half-plane model is generated by $z\mapsto z+r$ for $r\in\mathbb R$; $z\mapsto sz$ for $s\in\mathbb R_{>0}$; and the isometries $z\mapsto -\overline z$ and $z\mapsto 1/\overline z$.

\item If $y_1,y_2>0$, consider the regular points $z_1=x_1+y_1i,z_2=x_2+y_2i$ of the upper half-plane model of the hyperbolic plane.  The aim of this exercise is to show that their distance is
$$\rho(z_1,z_2)=\cosh^{-1}\left(1+\frac{(x_1-x_2)^2+(y_1-y_2)^2}{2y_1y_2}\right).$$
Here $\cosh^{-1}r=\ln(r+\sqrt{r^2-1})$, a partial inverse (for real numbers $\geqslant 1$) of the hyperbolic cosine $\cosh x=\frac{e^x+e^{-x}}2$.

(a) Show that the formula is satisfied if $x_1=x_2=0$, i.e., the points are on the $y$-axis.  [Use Exercise 3(a).]

(b) Show that $\rho$ is invariant under isometries: i.e., if $T$ is an isometry of the upper half-plane, then $\rho(T(z_1),T(z_2))=\rho(z_1,z_2)$.  [If the group of isometries acts on the set of ordered pairs of regular points, this is equivalent to saying $\rho$ is a function on orbits of said set.  By Proposition 1.22 and Exercise 8, it thus suffices to show the statement whenever $T$ is one of the isometries given in Exercise 8.]

(c) Conclude that the formula gives the distance for any two points.  [Use Exercise 1(c) and Proposition 1.20.]

\item Let $z_1,z_2$ be points in the Poincar\'e disk model of the hyperbolic plane (viewed as complex numbers).  Show that the distance between the points is
$$\rho(z_1,z_2)=\ln\frac{|\overline{z_1}z_2-1|+|z_1-z_2|}{|\overline{z_1}z_2-1|-|z_1-z_2|}.$$
[Show that $z\mapsto\frac{z_1-z}{\overline{z_1}z-1}$ is an isometry of the Poincar\'e disk sending $z_1\mapsto 0$; then use Exercise 4.]

\item Let $a,b,c,d$ be points in sequence around the rim $H^2_\infty(\mathbb R)$ of the hyperbolic plane.  Then $\overline{ab}$ and $\overline{cd}$ are divergent parallel lines, and their attention point is $\overline{ac}\cap\overline{bd}$.

\item In the Poincar\'e disk model of the hyperbolic plane, consider the circle with center $\frac 13\in H^2(\mathbb R)$ and radius $1$.  Find its Euclidean center and radius.  [Use symmetry considerations to show that the real axis is a diameter of both the hyperbolic circle and the Euclidean circle.]

\item Show that all nontrivial horolations of the hyperbolic plane are conjugate.

\item (a) If $T$ is a nontrivial translation of the hyperbolic plane, then the line along which it translates is unique.  [Consider ideal points which are fixed by $T$.]

(b) Explain why part (a) is false if $T$ is a translation of the Euclidean plane.

\item Let $\ell$ be the line in the upper half-plane model given by $|z|=1$.  (This is a semicircle arc of unit radius.)  Show that the reflection over $\ell$ coincides with the central inversion on this half-plane.

\item\emph{(Cross ratio formula for angle.)} \---- Let $\ell_1,\ell_2$ be intersecting lines in either the Poincar\'e disk or half-plane model (i.e., they intersect at a regular point).  Let $a,c$ be the ideal points of $\ell_1$ and $b,d$ the ideal points of $\ell_2$.  If $\theta$ is the angle between the lines, as seen in the sector between $a,b$, then $[\![a,b,c,d]\!]=\frac 2{1+\cos\theta}$.  [Use an isometry to assume that the lines are in the disk model and $\ell_1\cap\ell_2=0$.]

\item (a) Prove the inequalities in Exercise 11 of Section 2.1 for hyperbolic triangles.

(b) Show that for each point on a circle, there is a unique tangent line, which is perpendicular to the radius at that point.

\item\emph{(Angle bisectors.)} \---- If $\angle bac$ is an angle in the hyperbolic plane, define its \textbf{angle bisector} by extending the rays $\overset{\longrightarrow}{a~b}$ and $\overset{\longrightarrow}{a~c}$ to meet at ideal points $p,q$, then taking the line through $a$ perpendicular to $\overset{\longleftrightarrow}{p~q}$. 

If $a$ is a regular point, this bisects the angle in the usual sense.  However, if $a$ is an ideal point, \emph{every} ray from $a$ technically bisects the angle (because angles at $a$ are zero); yet the one whose construction is shown in this exercise is the most ``canonical.''
\end{enumerate}

\subsection*{4.4. Hypercycles and Horocycles}
\addcontentsline{toc}{section}{4.4. Hypercycles and Horocycles}
Hypercycles and horocycles are two special types of curves in the hyperbolic plane.  A hypercycle is a curve with a constant perpendicular distance to a line.  A horocycle is similar to a circle but it is centered at an ideal point.  Later we will see that both of these are, once again, generalized circles in the Poincar\'e disk and half-plane models.

To understand the concept of a hypercycle, we start by letting $\ell$ be a line in the \emph{Euclidean} plane.  Let $p$ be a point outside $\ell$, and consider all points that this point maps to via translations along $\ell$.  By basic algebra, these points form the line $\ell'$ through $p$ parallel to $\ell$.  Moreover, every point on $\ell'$ has the same distance to $\ell$ (Exercise 3 of Section 2.1), and every line perpendicular to $\ell$ is also perpendicular to $\ell'$ (by Proposition 2.10).

In the hyperbolic plane, however, no \emph{line} satisfies the above statements.  After all, suppose the trajectory of $p$ via the translations along $\ell$ formed a line $\ell'$.  Then if $\ell_1$ is the perpendicular to $\ell$ at any point, then (by symmetry considerations), $\ell_1$ meets $\ell'$ in a linear pair of two congruent angles, hence they are perpendicular.  This means that every perpendicular line to $\ell$ is also perpendicular to $\ell'$.  But this is impossible if $\ell'$ is a line: if $\ell$ and $\ell'$ intersect in either a regular or ideal point, we cannot have any line perpendicular to both of them, because that would entail a triangle with two right angles, contradicting Proposition 4.8; and if $\ell$ and $\ell'$ are divergent-parallel, only \emph{one} line is perpendicular to both of the lines by Theorem 4.7.

Thus, the points met by $p$ via the translations along $\ell$ do not form a line.  We then ask the question of what set $\ell'$ they \emph{do} form.  We may answer it easily by assuming that we are in the half-plane model, and that $\ell$ is the $y$-axis.  In this case, $p$ takes on the form $x+yi$ with $x\ne 0,y>0$ (because $p\notin\ell$).  The translations along $\ell$ (the $y$-axis) are the Euclidean dilations $z\mapsto rz,r>0$.  From there, we conclude that $\ell'=\{r(x+yi):r>0\}$, which is a slanted Euclidean line meeting the origin.
\begin{center}\includegraphics[scale=.16]{Hypercycle3.png}\end{center}
The following properties can be easily shown for the half-plane model with $\ell$ the $y$-axis.  Hence by isometry considerations, they follow in the general case (in either of the two models).
\begin{itemize}
\item $\ell'$ is a generalized circle, meeting the rim (the set of ideal points) at the same points where $\ell$ meets it.  $\ell'$ is not a geodesic unless $\ell'=\ell$. % At first I thought you were telling me "\ell' is not a geodesic unless \ell'=\ell", and I was like "I said it was a generalized circle, I didn't say it was a geodesic?"  Then I realized you were suggesting that I write this sentence

\item Every line perpendicular to $\ell$ is also perpendicular to $\ell'$, and vice versa.

\item Every point on $\ell'$ has the same (perpendicular) distance to $\ell$.

\item The reflection over any line perpendicular to $\ell$ (or $\ell'$) sends $\ell'$ to itself.

\item The translations along $\ell$ are precisely the orientation-preserving isometries that fix $\ell'$.  [Note that they are not the only orientation-preserving isometries which fix the line $\ell$; for that, there are also $180$-degree rotations around points on $\ell$.]
\end{itemize}
Examples of this in the Poincar\'e disk model are shown below.
\begin{center}\includegraphics[scale=.25]{Hypercycle1.png}\includegraphics[scale=.25]{Hypercycle2.png}\end{center}
The curve $\ell'$ in the hyperbolic plane is called a \textbf{hypercycle} (or \textbf{equidistant curve}).  The line $\ell$ is called its \textbf{axis}, \textbf{center} or \textbf{base line}.  The line segments from points on $\ell$ to points on $\ell'$ which are perpendicular to them are called \textbf{radii}, and the common length of these line segments is called the \textbf{radius} of the hypercycle.

As for horocycles, we recall that a circle centered at a regular point $p$ is the set of all points with a fixed given distance from $p$.  We wish to carry the idea over to the case where $p$ is an ideal point.  However, since finite distances do not exist for ideal points, we must think of a different strategy.  For that, we note that if $p$ is a regular point, then the circles centered at $p$ are precisely the curves to which all lines through $p$ are perpendicular.  [This is easy to see by assuming we are in the Poincar\'e disk model and $p=0$.]

Thus, if $p$ is an ideal point, we define a \textbf{horocycle} (centered at $p$) to be a curve $\omega$ to which all lines meeting $p$ are perpendicular.  In this situation, the reader can readily verify that, (conversely), every line perpendicular to $\omega$ meets the ideal point $p$.

To understand the basic structure of a horocycle, we shall (again) assume we are in the Poincar\'e half-plane model, and that $p=\infty$.  In this case, the lines passing through $p$ are precisely the (Euclidean) vertical lines.  Thus we want $\omega$ to be perpendicular to all of the vertical lines.  In this case, every tangent vector to $\omega$ is perpendicular to the vertical line through the point, hence is horizontal (parallel to the $x$-axis).  Elementary calculus shows that $\omega$ is then a Euclidean horizontal line.
\begin{center}\includegraphics[scale=.25]{Horocycle1.png}\end{center}
As in the case of hypercycles, there are many clear consequences for general horocycles $\omega$:
\begin{itemize}
\item $\omega$ is a generalized circle, meeting the rim at only the point $p$.

\item A line through $\omega$ is perpendicular to $\omega$ if and only if it meets $p$.

\item The reflection over any line through $p$ (or perpendicular to $\omega$) sends $\omega$ to itself.

\item Every horolation centered at $p$ fixes the horocycle $\omega$.  [Because if $p=\infty$ in the half-plane, the horolations centered at $p$ are the Euclidean horizontal translations.]  Conversely, every orientation-preserving isometry which fixes $\omega$ is a horolation centered at $p$.
\end{itemize}
\begin{center}\includegraphics[scale=.25]{Horocycle2.png}\includegraphics[scale=.2]{Horocycle3.png}\end{center}
We will not make much use of hypercycles, but horocycles will have a role in the next section.

\subsection*{Exercises 4.4. (Hypercycles and Horocycles)} % Motivate hypercycles by first showing that the locus of points with a given distance from a line is not itself a line.
% Motivate horocycles by thinking of them as "circles centered at infinity."  Cover basic structure, e.g., the fact that in the Poincar\'e disk model, horocycles are Euclidean circles.
\begin{enumerate}
\item Explain why three distinct points in the hyperbolic plane (some of which may be ideal points) determine exactly one of the following things: a line; a hypercycle; a horocycle; or a circle.

\item Given an example of two hypercycles intersecting in two points.  Explain why they cannot intersect in three or more.

\item Show that when two (Euclidean) circles intersect in two points, the intersection points share the same angle measures.  Use this to give an alternative proof that a circle tangent to the rim in the Poincar\'e disk model is a horocycle centered at the point of tangency.

\item Show that the group of isometries acts transitively on each of the following sets:

(a) The set of hypercycles of radius $r$, where $r$ is a fixed positive real number;

(b) The set of all horocycles.

\item Let $p$ be an ideal point, and $p_1,p_2$ regular points.  Show that the following are equivalent:

~~~~(i) $p_1$ and $p_2$ lie on the same horocycle centered at $p$;

~~~~(ii) The triangle with vertices $p,p_1,p_2$ has congruent angles at $p_1,p_2$;

~~~~(iii) $p$ is on the perpendicular bisector of the line segment $\overline{p_1~p_2}$.

[Assume you are in the half-plane model and $p=\infty$.  Show that each statement is equivalent to saying that, as complex numbers, $p_1,p_2$ have the same imaginary part.]

\item If $p_1,p_2\in H^2(\mathbb R)$ are distinct regular points, then there are exactly two horocycles containing them.  [Use the previous exercise.]

\item (a) If $\ell$ is a line and $p$ is an ideal point not on the line, then there is a unique horocycle centered at $p$ tangent to $\ell$.  [Assume $p=\infty$ in the half-plane model.]

(b) If $\ell$ is a line and $p$ is a regular point on $\ell$, then there are two horocycles tangent to $\ell$ at $p$, on either side of $\ell$.

\item Suppose $\ell$ is a line, $p$ is a point on $\ell$.  For each $r>0$, let $\omega_r$ be the circle of radius $r$ tangent to $\ell$ at the point $p$.  In the Euclidean plane, $\lim_{r\to\infty}\omega_r$ is the line $\ell$.  In the hyperbolic plane, $\lim_{r\to\infty}\omega_r$ is the horocycle of part (b) of the previous exercise.

\item\emph{(Apeirogons.)} \---- The concept of an apeirogon in the hyperbolic plane will be introduced in this exercise.

Let $a,b>0$ be fixed real numbers.  Let $S$ be the set of points $\{nb+ia:n\in\mathbb Z\}$ in the half-plane model of the hyperbolic plane.  Connect each pair $(nb+ia,(n+1)b+ia),n\in\mathbb Z$ with a (hyperbolic) line segment.  The resulting figure is illustrated below.
\begin{center}\includegraphics[scale=.3]{Apeirogon1.png}\end{center}
(a) The figure is a graph with infinitely many edges and vertices in both directions.  Moreover, the vertices are part of a horocycle, and all of the angles are congruent.

(b) In terms of $a$ and $b$, find the common angle measure.  Conclude that this angle measure can be arranged to be any number strictly between $0$ and $\pi$.

(c) The orientation-preserving isometries of the figure consist of a discrete group of horolations centered at $\infty$, which is isomorphic to $\mathbb Z$.  The orientation-reversing isometries of the figure are reflections over lines containing $\infty$.

(After possibly applying an isometry), the figure above is called an \textbf{apeirogon}.  It is intuitively a polygon with infinitely many sides.  There are clever ways of tiling $H^2(\mathbb R)$ with it, as will be seen in the next section.
\begin{center}\includegraphics[scale=.25]{Apeirogon2.png}\end{center} % Fun fact: a *pseudogon* is obtained by connecting line segments along a hypercycle.  Fix a,b>0,r>1, and let \ell' be the line y=(a/b)x; then take the sequence of points (br^n,ar^n),n\in\mathbb Z and connect them by segments.  Since isometries fixing $\ell'$ are Euclidean dilations, they fix the points when the dilation is by a power of r.  I didn't bother covering this.
\end{enumerate}

\subsection*{4.5. Triangles, Polygons and Tilings}
\addcontentsline{toc}{section}{4.5. Triangles, Polygons and Tilings}
We recall (Exercise 4 of Section 2.2) the laws of sines and cosines for triangles in the Euclidean plane.  In this section, we shall find similar laws for hyperbolic triangles.  These laws will be noticeably trickier because, for example, there are no such things as similar triangles; any two triangles with identical angles are congruent by Proposition 4.9(vi).  Once we establish them, we will cover basic ways to construct various polygons and tilings in the hyperbolic plane.

The main concept needed to study hyperbolic triangles are the hyperbolic functions.  [One of them is the hyperbolic cosine, which has already been covered in Exercise 9 of Section 4.3.]  They are similar to the trigonometric functions, but instead of going round in circles, they (intuitively) shoot off on a hyperbola.  To motivate the concept, we will start by mentioning Euler's discovery about exponentials of imaginary numbers: $e^{ix}=\cos x+i\sin x$.\footnote{Taking $x=\pi$ entails $e^{i\pi}=-1$, since $\cos\pi=-1$ and $\sin\pi=0$.  Adding $1$ to both sides, one gets Euler's famous statement, $e^{i\pi}+1=0$.}  There are many ways to see this:

\begin{itemize}
\item One way is by taking the Taylor series (centered at $0$) of each function, and then summing things together.  Since
$$e^x=1+x+\frac{x^2}{2!}+\frac{x^3}{3!}+\frac{x^4}{4!}+\dots$$
$$\cos x=1-\frac{x^2}{2!}+\frac{x^4}{4!}-\frac{x^6}{6!}+\dots$$
$$\sin x=x-\frac{x^3}{3!}+\frac{x^5}{5!}-\frac{x^7}{7!}+\dots$$
\begin{align*}
\text{We have }e^{ix}
&=1+ix+\frac{(ix)^2}{2!}+\frac{(ix)^3}{3!}+\frac{(ix)^4}{4!}+\dots\\
&=1+ix-\frac{x^2}{2!}-\frac{ix^3}{3!}+\frac{x^4}{4!}+\dots\\
&=\left(1-\frac{x^2}{2!}+\frac{x^4}{4!}-\frac{x^6}{6!}+\dots\right)+i\left(x-\frac{x^3}{3!}+\frac{x^5}{5!}-\frac{x^7}{7!}+\dots\right)\\
&=\cos x+i\sin x.
\end{align*}
\item Alternatively, define $f(x)=\cos x+i\sin x$.  Then, since $\frac d{dx}\cos x=-\sin x$ and $\frac d{dx}\sin x=\cos x$, we get
$$\frac d{dx}f(x)=-\sin x+i\cos x=i(\cos x+i\sin x)=if(x).$$
Since $f'(x)=kf(x)\implies f(x)=Ae^{kx}$ for some constant $A$, we conclude that we can write $f(x)=Ae^{ix}$ for some $A$.  Moreover, $A=f(0)=\cos 0+i\sin 0=1$.  Therefore, $f(x)=e^{ix}$, which proves that $e^{ix}=\cos x+i\sin x$.
\end{itemize}
\noindent Since $\cos x$ is an even function and $\sin x$ is an odd function, the reader can readily verify that $e^{-ix}=\cos x-i\sin x$.  Basic algebraic manipulation thus enables us to formulate the trigonometric functions in terms of complex numbers and the exponential function:
$$\cos x=\frac{e^{ix}+e^{-ix}}2~~~~~~\sin x=\frac{e^{ix}-e^{-ix}}{2i}$$
[These are analytic interpretations of the trigonometric functions.]  Once sine and cosine are derived, it is easy to get the rest of the trigonometric functions, because, for example, $\tan x=\frac{\sin x}{\cos x}$ and $\sec x=\frac 1{\cos x}$:
$$\tan x=\frac{-i(e^{ix}-e^{-ix})}{e^{ix}+e^{-ix}}~~~~~~\cot x=\frac{i(e^{ix}+e^{-ix})}{e^{ix}-e^{-ix}}$$
$$\sec x=\frac 2{e^{ix}+e^{-ix}}~~~~~~\csc x=\frac{2i}{e^{ix}-e^{-ix}}$$
We will not make much use of these formulas; however, we get the hyperbolic functions by imitating the formulas without using the imaginary unit $i$ as a factor.  Thus, we define them as follows:
$$\cosh x=\frac{e^x+e^{-x}}2~~~~~~\sinh x=\frac{e^x-e^{-x}}2$$
$$\tanh x=\frac{e^x-e^{-x}}{e^x+e^{-x}}~~~~~~\coth x=\frac{e^x+e^{-x}}{e^x-e^{-x}}$$
$$\operatorname{sech}x=\frac 2{e^x+e^{-x}}~~~~~~\operatorname{csch}x=\frac 2{e^x-e^{-x}}$$
$\cosh x$ is called the \textbf{hyperbolic cosine} of $x$, $\sinh x$ is called the \textbf{hyperbolic sine}, and so on.  In mathematical expressions, these functions are denoted by adding an ``h'' at the end of the trigonometric function name.

The reader is encouraged to verify the following identities for the hyperbolic functions:
\begin{itemize}
\item $\cosh x$ and $\operatorname{sech}x$ are even functions; the other four are all odd functions.

\item $\cosh^2x-\sinh^2x=1$.  [This, of course, is analogous to the Pythagorean identity $\sin^2x+\cos^2x=1$.  It indicates that, just as $t\mapsto(\cos t,\sin t)$ parametrizes a circle in the plane, $t\mapsto(\cosh t,\sinh t)$ parametrizes a branch of the hyperbola $x^2-y^2=1$.]

\item $\tanh x=\frac{\sinh x}{\cosh x}$.

\item $\coth x=\frac 1{\tanh x}$; $\operatorname{sech}x=\frac 1{\cosh x}$; $\operatorname{csch}x=\frac 1{\sinh x}$.

\item $\cosh(x\pm y)=\cosh x\cosh y\pm\sinh x\sinh y$, and $\sinh(x\pm y)=\sinh x\cosh y\pm\cosh x\sinh y$.  (Therefore, taking $x=y$, we get $\cosh(2x)=\cosh^2x+\sinh^2x=2\cosh^2x-1=2\sinh^2x+1$, and $\sinh(2x)=2\sinh x\cosh x$.)
\end{itemize}
Note that the corresponding sum formulas for the trigonometric functions have a swapped sign: $\cos(x\pm y)=\cos x\cos y\mp\sin x\sin y$.  The hyperbolic functions do not.
\begin{itemize}
\item $\tanh(x\pm y)=\frac{\tanh x\pm\tanh y}{1\pm\tanh x\tanh y}$.  In particular, taking $x=y$, $\tanh(2x)=\frac{2\tanh x}{1+\tanh^2x}$.

\item $1-\tanh^2x=\operatorname{sech}^2x$, and $\coth^2x-1=\operatorname{csch}^2x$.

\item $e^x=\cosh x+\sinh x$.  [This is analogous to $e^{ix}=\cos x+i\sin x$.]\\
\end{itemize}
\noindent Now that we are familiar with the hyperbolic functions, we are ready to see how they relate to the hyperbolic triangles.  Throughout this section, a triangle is assumed to have three regular vertices (triangles with ideal vertices are covered in Exercise 2).  A triangle is given as $\triangle ABC$ where the vertices are capital letters; $A,B,C$ denote the angle measures, and $a,b,c$ denote the side lengths opposite those respective vertices:
\begin{center}\includegraphics[scale=.25]{HypTriangleSample.png}\end{center}
\noindent\textbf{Proposition 4.13.} \emph{For the above triangle,} %https://www.maths.gla.ac.uk/wws/cabripages/hyperbolic/hypertrig.html

(i) \emph{The \textbf{hyperbolic law of cosines} holds: $\cosh c=\cosh a\cosh b-\sinh a\sinh b\cos C$.}

(ii) \emph{The \textbf{hyperbolic law of sines} holds: $\frac{\sinh a}{\sin A}=\frac{\sinh b}{\sin B}=\frac{\sinh c}{\sin C}$.}

(iii) \emph{The \textbf{second hyperbolic law of cosines} holds: $\cos C=-\cos A\cos B+\sin A\sin B\cosh c$.}\\

\noindent Several remarks are in order.  To begin with, (i) derives the angle measures from the side lengths, and (iii) derives the side lengths from the angle measures.  The special case of (i) where $C=\pi/2$ shows that if $a,b,c$ are the sides of a hyperbolic right triangle with $c$ the hypotenuse, then $\cosh c=\cosh a\cosh b$.

Recall the Euclidean law of cosines: by Exercise 4(c) of Section 2.2, if $\triangle ABC$ is a Euclidean triangle then $c^2=a^2+b^2-2ab\cos C$.  For both types of geometry, you can see that if $C=\pi$, (which means $\cos C=-1$), then $c=a+b$ (for the Euclidean case it follows from the binomial theorem; for hyperbolic triangles it follows from the sum formula for the hyperbolic cosine mentioned earlier).  In this case, $\angle C$ is a striaght angle and $A,B,C$ are collinear with $C$ in between, so this is a manifestation of the segment addition postulate.  Similarly, if $C=0$ then $c=|a-b|$.
\begin{proof}
(i) By applying an isometry, we may assume we are in the Poincar\'e disk model, vertex $C$ is at $0$ and vertex $A$ is at a positive real number $r$.  Write vertex $B$'s location as $se^{iC}$ with $s$ a positive real number (this is plausible because $C$ is the Euclidean measure of $\angle ACB$).  By Exercise 1(c), we have $b=2\tanh^{-1}r$, and $a=2\tanh^{-1}|se^{iC}|=2\tanh^{-1}s$.  Moreover, by Exercise 10 of Section 4.3,
$$c=\ln\frac{|rse^{iC}-1|+|r-se^{iC}|}{|rse^{iC}-1|-|r-se^{iC}|}=2\tanh^{-1}\frac{|r-se^{iC}|}{|rse^{iC}-1|}.$$
Therefore, $\tanh(c/2)=\frac{|r-se^{iC}|}{|rse^{iC}-1|}$, from which $\tanh^2(c/2)=\frac{|r-se^{iC}|^2}{|rse^{iC}-1|^2}$ follows.  Yet
$$\cosh c=\frac{\cosh c}1=\frac{\cosh^2(c/2)+\sinh^2(c/2)}{\cosh^2(c/2)-\sinh^2(c/2)}$$
$$=\frac{[\cosh^2(c/2)+\sinh^2(c/2)]/\cosh^2(c/2)}{[\cosh^2(c/2)-\sinh^2(c/2)]/\cosh^2(c/2)}=\frac{1+\tanh^2(c/2)}{1-\tanh^2(c/2)}$$
$$=\frac{|rse^{iC}-1|^2+|r-se^{iC}|^2}{|rse^{iC}-1|^2-|r-se^{iC}|^2}$$
As $|z|^2=z\overline z$ for $z\in\mathbb C$, we may rewrite $|rse^{iC}-1|^2=(rse^{iC}-1)(rse^{-iC}-1)=r^2s^2-rs(e^{iC}+e^{-iC})+1=r^2s^2-2rs\cos C+1$, and $|r-se^{iC}|^2=(r-se^{iC})(r-se^{-iC})=r^2-rs(e^{iC}+e^{-iC})+s^2=r^2-2rs\cos C+s^2$.\footnote{We recall that $\cos C=\frac{e^{iC}+e^{-iC}}2$.}  Hence,
$$\frac{|rse^{iC}-1|^2+|r-se^{iC}|^2}{|rse^{iC}-1|^2-|r-se^{iC}|^2}=\frac{(r^2s^2+1-2rs\cos C)+(r^2+s^2-2rs\cos C)}{(r^2s^2+1-2rs\cos C)-(r^2+s^2-2rs\cos C)}$$
$$=\frac{r^2s^2+r^2+s^2+1-4rs\cos C}{r^2s^2-r^2-s^2+1}=\frac{(1+r^2)(1+s^2)-4rs\cos C}{(1-r^2)(1-s^2)}$$

By Exercise 1(d), $\cosh b=\frac{1+r^2}{1-r^2}$ and $\sinh b=\frac{2r}{1-r^2}$.  Similarly, $\cosh a=\frac{1+s^2}{1-s^2}$ and $\sinh a=\frac{2s}{1-s^2}$.  By basic algebraic manipulation, we get
$$\frac{(1+r^2)(1+s^2)-4rs\cos C}{(1-r^2)(1-s^2)}=\cosh a\cosh b-\sinh a\sinh b\cos C,$$
proving (i) as desired.

(ii) By symmetry considerations, it suffices to show that $\frac{\sinh a}{\sin A}=\frac{\sinh b}{\sin B}$.  Let $\alpha=\cosh a,\beta=\cosh b,\gamma=\cosh c$.  Then by the identity $\cosh^2x-\sinh^2x=1$, we get $\sinh^2a=\alpha^2-1$ and $\sinh^2b=\beta^2-1$.  By part (i), $\cosh c=\cosh a\cosh b-\sinh a\sinh b\cos C$; in other words, $\gamma=\alpha\beta-\sinh a\sinh b\cos C$.  Consequently, $\sinh a\sinh b\cos C=\alpha\beta-\gamma$.  Squaring throughout,
$$(\alpha\beta-\gamma)^2=\sinh^2a\sinh^2b\cos^2C$$
and hence,
$$\sinh^2a\sinh^2b\sin^2C=\sinh^2a\sinh^2b(1-\cos^2C)=\sinh^2a\sinh^2b-\sinh^2a\sinh^2b\cos^2C$$
$$=(\alpha^2-1)(\beta^2-1)-(\alpha\beta-\gamma)^2=1-\alpha^2-\beta^2-\gamma^2+2\alpha\beta\gamma$$
Observe that this expression is symmetric in the three variables $\alpha,\beta,\gamma$.  Repeating the argument with $a,b,c$ permuted gives
$$\sinh^2a\sinh^2c\sin^2B=1-\alpha^2-\beta^2-\gamma^2+2\alpha\beta\gamma$$
$$\sinh^2b\sinh^2c\sin^2A=1-\alpha^2-\beta^2-\gamma^2+2\alpha\beta\gamma$$
from which $\sinh^2a\sinh^2c\sin^2B=\sinh^2b\sinh^2c\sin^2A$.  Dividing by $\sinh^2c$ throughout, $\sinh^2a\sin^2B=\sinh^2b\sin^2A$.  Taking square roots, and noting that $a,b$ are positive and $0<A,B<\pi$ (so that everything will be positive), $\sinh a\sin B=\sinh b\sin A$.  We divide by $\sinh a\sinh b$ to get the desired statement $\frac{\sinh a}{\sin A}=\frac{\sinh b}{\sin B}$.

(iii) Let $\alpha=\cosh a,\beta=\cosh b,\gamma=\cosh c$ as in part (ii).  In that part, we have (effectively) shown that
$$\sinh a\sinh b\sin C=\sinh a\sinh c\sin B=\sinh b\sinh c\sin A=\Delta,$$
where $\Delta=\sqrt{1-\alpha^2-\beta^2-\gamma^2+2\alpha\beta\gamma}$.  Part (i) entails:
$$\begin{array}{c l}\sinh a\sinh b\cos C=\alpha\beta-\gamma\\\sinh a\sinh c\cos B=\alpha\gamma-\beta\\\sinh b\sinh c\cos A=\beta\gamma-\alpha\end{array}$$
and as in part (ii), $\sinh^2c=\gamma^2-1$.  Furthermore,
$$\frac{\cos C+\cos A\cos B}{\sin A\sin B}=\frac{(\alpha\beta-\gamma)(\gamma^2-1)+(\beta\gamma-\alpha)(\alpha\gamma-\beta)}{\Delta^2}$$
as can be seen by multiplying the numerator and denominator of the left-hand side by $\sinh^2c\sinh a\sinh b$.  Yet the right-hand side readily simplifies to $\gamma=\cosh c$.  The statement (iii) thus follows.
\end{proof}

\noindent Many other properties of hyperbolic triangles can be easily proven using Proposition 4.13.  For starters, $\triangle ABC$ is equilateral (i.e., has three equal sides) if and only if it has three equal angles, as a consequence of the Isosceles Triangle Theorem (4.10).  Thus if $\triangle ABC$ is an equilateral triangle, it has some side length $s$ and some angle $\theta$.  We observe that by Proposition 4.13(i), % The Isosceles Triangle Theorem was Proposition 4.10.  I'll remind the reader :)
$$\cosh s=\cosh^2s-\sinh^2s\cos\theta$$
This, along with the fact that $\sinh^2s=\cosh^2s-1$, entails 
$$(1-\cos\theta)\cosh^2s-\cosh s+\cos\theta=0$$
When $\theta$ is fixed, this is a quadratic equation in the variable $u=\cosh s$, and it is readily seen that the roots are $u=1,\frac{\cos\theta}{1-\cos\theta}$.  [$u=1$ is manifestly a root, and Vieta's formulas can easily be used to find the other root.]  Yet $u=1$ entails $s=0$, which is not possible.  Therefore, $\cosh s=u=\frac{\cos\theta}{1-\cos\theta}$.  Incidentally, since $\cosh s>1$, we get $\cos\theta>\frac 12$ in the result; i.e., $\theta<60^\circ=\frac{\pi}3$, which follows anyway from Proposition 4.8.

Either using Proposition 4.13(iii), or directly deriving from $\cosh s=\frac{\cos\theta}{1-\cos\theta}$, one gets $\cos\theta=\frac{\cosh s}{\cosh s+1}$.  Thus:\\

\noindent\textbf{Proposition 4.14.} \emph{If a hyperbolic equilateral triangle has side length $s$ and angle measure $\theta$, then $\cosh s=\frac{\cos\theta}{1-\cos\theta}$ and $\cos\theta=\frac{\cosh s}{\cosh s+1}$.}\\

\noindent Now we turn our attention to right triangles.  We assume $\triangle ABC$ is a right triangle with $\angle C$ the right angle, so that $a,b$ are the lengths of the legs and $c$ is that of the hypotenuse.  As previously stated, Proposition 4.13(i) entails $\cosh c=\cosh a\cosh b$.  Furthermore, by Proposition 4.13(iii), (along with the fact that $C=90^\circ$, so that $\cos C=0$ and $\sin C=1$),
$$\cos B=-\cos A\cos C+\sin A\sin C\cosh b=\sin A\cosh b$$
from which we conclude that $\cosh b=\frac{\cos B}{\sin A}$.  We immediately have a formula for one of the lengths of the legs.  By symmetry considerations, $\cosh a=\frac{\cos A}{\sin B}$.  Hence, multiplying them, $\cosh c=\cosh a\cosh b=\frac{\cos A\cos B}{\sin A\sin B}=\cot A\cot B$.  [Note that for a \emph{Euclidean} right triangle, one would have $\cot A\cot B=1$ because $A$ and $B$ are complementary angles.  This is not so for hyperbolic triangles!]

Furthermore, Proposition 4.13(ii) implies $\frac{\sinh a}{\sin A}=\frac{\sinh c}{\sin C}=\sinh c$, from which we get $\sin A=\frac{\sinh a}{\sinh c}$.  Thus, for a hyperbolic right triangle, the sine of an acute angle is the hyperbolic sine of the opposite leg divided by the hyperbolic sine of the hypotenuse.  You probably recall that for Euclidean triangles, the sine of an acute angle is the opposite leg divided by the hypotenuse; now we know a simlar statement for hyperbolic triangles.

Furthermore,
$$\cos^2A=1-\sin^2A=1-\left(\frac{\sinh a}{\sinh c}\right)^2=1-\frac{\sinh^2a}{\sinh^2c}$$
$$=\frac{\sinh^2c-\sinh^2a}{\sinh^2c}=\frac{(\sinh^2c+1)-(\sinh^2a+1)}{\sinh^2c}=\frac{\cosh^2c-\cosh^2a}{\sinh^2c}$$
$$\overset{(*)}=\frac{\cosh^2a\cosh^2b-\cosh^2a}{\sinh^2c}=\frac{\cosh^2a(\cosh^2b-1)}{\sinh^2c}=\frac{\cosh^2a\sinh^2b}{\sinh^2c}$$
$$=\frac{\cosh^2a\sinh^2b\cosh^2b}{\sinh^2c\cosh^2b}\overset{(*)}=\frac{\cosh^2c\sinh^2b}{\sinh^2c\cosh^2b}=\frac{\tanh^2b}{\tanh^2c}$$
where the parts marked $(*)$ make use of the fact that $\cosh c=\cosh a\cosh b$.  Since $\cos A>0$ ($A$ is acute), we conclude that $\cos A=\frac{\tanh b}{\tanh c}$.  This is similar to the identity $\cos A=\frac bc$ for Euclidean right triangles, but also quite different.

Finally, we compute
$$\tan A=\frac{\sin A}{\cos A}=\frac{\sinh a/\sinh c}{\tanh b/\tanh c}=\frac{\sinh a\tanh c}{\tanh b\sinh c}=\frac{\sinh a}{\tanh b}\frac{\tanh c}{\sinh c}$$
and since $\frac{\sinh}{\cosh}=\tanh$,
$$\frac{\sinh a}{\tanh b}\frac{\tanh c}{\sinh c}=\frac{\sinh a}{\tanh b}\frac 1{\cosh c}=\frac{\sinh a\cosh b}{\sinh b}\frac 1{\cosh c}$$
$$=\frac{\sinh a\cosh b}{\sinh b}\frac 1{\cosh a\cosh b}=\frac{\sinh a}{\sinh b\cosh a}=\frac{\tanh a}{\sinh b}.$$
We have just shown\\

\noindent\textbf{Proposition 4.15.} \emph{Let $\triangle ABC$ be a hyperbolic right triangle with $\angle C$ the right angle, and let $a,b,c$ be the side lengths opposite $A,B,C$ respectively.  Then:}

(i) \emph{$\cosh c=\cosh a\cosh b$.}

(ii) \emph{$\cosh b=\frac{\cos B}{\sin A}$, $\cosh a=\frac{\cos A}{\sin B}$ and $\cosh c=\cot A\cot B$.}

(iii) \emph{$\sin A=\frac{\sinh a}{\sinh c}$, $\cos A=\frac{\tanh b}{\tanh c}$ and $\tan A=\frac{\tanh a}{\sinh b}$.}\\

\noindent Just as in the Euclidean plane, one can define more general polygons. % "New paragraph" - I always use \noindent right after a proposition.
An $n$-gon consists of an ordered $n$-tuple of points $A_1,\dots,A_n$ (called \textbf{vertices}) and the $n$ line segments $\overline{A_1A_2},\dots,\overline{A_{n-1}A_n},\overline{A_nA_1}$ (called \textbf{edges} or \textbf{sides}), such that no two segments intersect each other unless they share a vertex.  As usual, we will restrict ourselves to convex polygons (there is a point not touching the sides, which is simultaneously inside all of the angles).  By adapting Proposition 2.30, and using Proposition 4.8 in place of Proposition 2.11, it is easy to show that the measures of the angles of a regular $n$-gon add to $<180(n-2)^\circ$; i.e., less than $\pi(n-2)$.  The definitions of an equilateral/equiangular/regular polygon are identical to those of Section 2.3.

It can be shown (by mirroring the argument of Section 2.3) that if $A_1\dots A_n$ is a regular $n$-gon, then there is a unique circle passing through all its vertices (called the \textbf{circumscribed circle}); the center of this circle is called the center of the polygon.  There is a unique circle tangent to all the sides (called the \textbf{inscribed circle}),\footnote{The situation is actually trickier for $n$-gons with ideal vertices; they do not generally have full dihedral symmetry (unless $n=3$).  We will not delve into details here; the polygon in this paragraph is assumed to have regular vertices.} which has the same center.  Also, for every circle there are regular $n$-gons inscribed in the circle and circumscribed around it.\\

\noindent\textbf{TILINGS}\\

\noindent We conclude this section by showing how polygons can tile the hyperbolic plane.  These tilings are constructed in an unfamiliar fashion because, unlike in the Euclidean plane, we are not free to choose the side length of a regular polygon.  It is not actually hard to find the side length of a regular $n$-gon when given one of its angle measures (Exercise 6); however, for semiregular tilings which use multiple types of polygons with the same side length, we will not know the angle measures so easily.  The ideal helping hand for this is the \emph{triangle pattern}.

Triangle patterns apply equally to Euclidean geometry (Chapter 2), hyperbolic geometry (this chapter) and spherical geometry (Chapter 5).  However, they were not introduced in Chapter 2 because Euclidean tilings were extremely easy to establish without them.

The main idea is this: let $a,b,c$ be integers $\geqslant 2$.  We wish to start with a triangle whose angle measures are exactly $\pi/a,\pi/b,\pi/c$.  Depending on how the sum of these angle measures compares with $\pi$, the triangle exists in exactly one of the three aforementioned types of geometry:
\begin{itemize}
\item If the angles add to $\pi$ (or what is the same thing, $\frac 1a+\frac 1b+\frac 1c=1$), the triangle is in the Euclidean plane, in view of Proposition 2.11.

\item If the angles add to $<\pi$ (i.e., $\frac 1a+\frac 1b+\frac 1c<1$), the triangle is in the hyperbolic plane, in view of Proposition 4.8.  Note: in this case, we may allow $a,b,c$ to be $\infty$, so that the corresponding vertices of the triangle may be ideal points.

\item If the angles add to $>\pi$ (i.e., $\frac 1a+\frac 1b+\frac 1c>1$), the triangle is on the sphere, as will be seen in the next chapter.
\end{itemize}
Elementary arithmetic shows that if $a,b,c$ are integers $\geqslant 2$ such that $\frac 1a+\frac 1b+\frac 1c=1$, then the unordered triple $[a,b,c]$ has only three possibilities: $[2,4,4]$, $[2,3,6]$ and $[3,3,3]$.  These give, in the Euclidean plane, the isosceles right triangle, the 30-60-90 right triangle (which is half the equilateral triangle), and the equilateral triangle.

However, if $\frac 1a+\frac 1b+\frac 1c<1$, there are infinitely many possibilities for $[a,b,c]$, and hyperbolic triangles exist for all of them (Exercise 5).  This gives rise to many triangles exclusive to the hyperbolic plane.  The best-known example is $[2,3,7]$ ($\frac 12+\frac 13+\frac 17=\frac{41}{42}$), which yields many common tilings, such as the triheptagonal tiling, and the truncated order-7 triangular tiling (the setting of the Zeno Rogue game \emph{HyperRogue}).  By allowing $a$, $b$ or $c$ to be infinity, we will also be able to establish tilings which use regular apeirogons (see Exercise 9 of Section 4.4).

Once the triangle is constructed in any of the geometries, here's what to do.  Start by placing the triangle anywhere in the plane.  Then, construct the image of that triangle when reflected over one of the sides.  Now do this to the rest of the edges of the original triangle, without erasing any triangles that have already been constructed.  This is illustrated below, with the blue triangle the original one we started with:
\begin{center}\includegraphics[scale=.4]{TriangleReflecting.png}\end{center}
Now for the three triangles just constructed, reflect them over their own edges.  [Each triangle will already have one of its edge reflections in the picture, though.]  Continue reflecting triangles over their own edges until they take up all the space.  This involves infinitely many triangles (except in the spherical case).  Since each vertex of the triangle fits around a pivot an exact, even number of times (e.g., the $\pi/a$-angle vertex fits $2a$ times), it can be proven that the triangles tile perfectly, with any two triangles either disjoint, sharing a single vertex, or sharing an entire edge; the rigorous argument will be omitted here.

The final results, for the $[2,3,6]$ case of the Euclidean plane, and the $[2,3,7]$ case of the (Poincar\'e disk model of the) hyperbolic plane, are shown below.
\begin{center}
\includegraphics[scale=.25]{EucTrianglePattern.png}~~~~
\includegraphics[scale=.25]{HypTrianglePattern.png}
\end{center} % Unfortunately, we are not yet ready to include the spherical [2,3,5] with the family...
Effectively, constructing this tiling only requires a basic knowledge of isometries in the geometry, and the properties of triangles.  Yet once you have triangle tilings, it is easy to get pretty much any uniform tiling, by doing a simple, specific construction using one of the triangles, passing it over to the others by reflecting, and then erasing the original triangles.

For instance, given the Euclidean tiling of 30-60-90 triangles displayed above, suppose only the shortest leg is drawn for each triangle, and then the triangles are erased.  Then what you get is the hexagonal tiling.  If, on the other hand, only the altitudes to the hypotenuses of the original triangles are visible, a tiling emerges which is made up of hexagons and triangles, with two of each kind of polygon at each vertex.
\begin{center}
\includegraphics[scale=.25]{HexTilingForm1.png}
\includegraphics[scale=.25]{HexTilingForm2.png}
\end{center}
The same strategy can be used to obtain hyperbolic tilings.
\begin{itemize}
\item Take the $[2,3,7]$ triangle pattern up above.  Draw only the short leg of each triangle, and you get the heptagonal tiling, shown below.  It consists of heptagons with $2\pi/3=120^\circ$ angles.  By Exercise 6(a), the side length of these heptagons is $\cosh^{-1}\left(\frac{4\cos(2\pi/7)+1}3\right)\approx 0.566256$.
\begin{center}
\includegraphics[scale=.3]{HeptagonalTiling.png}
\end{center}
Here is a sample of the same tiling in the Poincar\'e half-plane model:
\begin{center}
\includegraphics[scale=.3]{HeptagonalTiling_HP.png}
\end{center}

\item Start again with the $[2,3,7]$ pattern.  When the altitude to the hypotenuse of each triangle is drawn, you get the triheptagonal tiling, made of heptagons and triangles, shown on the left.  It is noticeably difficult to find the angles of these polygons without the use of triangle patterns, as the angle of the triangle must be supplementary to that of the heptagon, but they must also have the same side length.  On the other hand, in the $[2,3,7]$ pattern, suppose you construct the following for each triangle:
\begin{center}
Start by \emph{considering} the angle bisector of the $2\pi/3$-angle vertex.  (Don't construct it.)  At the point where it meets the long leg of the right triangle, construct perpendicular line segments to the short leg and the hypotenuse.  [The line segment to the short leg is actually contained in the long leg, because it's a right triangle.]  Thus the construction consists of two identical-length line segments.
\end{center}
Then, you get the truncated order-7 triangular tiling, or the setting of \emph{HyperRogue}, shown on the right.
\begin{center}
\includegraphics[scale=.3]{TriheptagonalTiling.png}~~~~
\includegraphics[scale=.3]{HyperrogueTiling.png}
\end{center}

\item Start with the $[3,3,4]$ (resp., $[3,3,5]$) triangle pattern.  This pattern consists of isosceles triangles, with angle measures $60^\circ$, $60^\circ$ and $45^\circ$ (resp., $36^\circ$).  Take one of the triangles, and draw an altitude from one of the $60^\circ$ vertices (i.e., not the one that goes to the midpoint of the base).  Drawing just these will give you a tiling of triangles and squares (resp., pentagons), with three of each kind of polygon at each vertex, as shown below.
\begin{center}
\includegraphics[scale=.3]{HypExoticTiling1.png}~~~~
\includegraphics[scale=.3]{HypExoticTiling2.png}
\end{center}
\emph{Caution}: Since the triangles in the original pattern are isosceles, and actually have some symmetry, you must make sure that \emph{only reflections over the edges} are used to transfer the constructed altitude throughout the plane.  A $45^\circ$ (or $36^\circ$) rotation around the smallest-angle vertex of the triangle will not give the desired tiling.

\item The $[2,3,\infty]$ triangle pattern consists of triangles with two vertices measuring $90^\circ$ and $60^\circ$, and one ideal vertex.  By Exercise 2(c), the length of the side opposite the ideal vertex is $\ln\sqrt 3=\frac 12\ln 3$.

When a triangle has ideal vertices, its triangle pattern can be used to construct uniform tilings with regular apeirogons.  For example, drawing only the finite side opposite the ideal vertex for each triangle entails the regular order-3 apeirogonal tiling, seen on the left.  On the other hand, drawing the line segments from the incenter of the triangle perpendicular to the sides, one gets the semi-regular tiling on the right.
\begin{center}
\includegraphics[scale=.3]{ApeirogonRegularTiling.png}~~~~
\includegraphics[scale=.3]{ApeirogonTiling.png}
\end{center}
You could alternatively get the first tiling by considering the $[\infty,\infty,\infty]$ triangle pattern, which consists entirely of ideal triangles.  Connecting the incenters of the triangles with line segments will give you the regular tiling.
\end{itemize}
\noindent All in all, you can get (almost) any uniform tiling in one of the kinds of geometry, by starting with the triangle pattern, doing a specific construction in one of the triangles, then passing it over to the others.  This is known as the \textbf{Wythoff construction}.

\subsection*{Exercises 4.5. (Triangles, Polygons and Tilings)} % Introduce triangles, establish all relations between them (https://en.wikipedia.org/wiki/Hyperbolic_triangle) then go over
% triangle groups.  Show how to mathematically construct certain (semi)-regular tilings (e.g., the one with vertex config 3-5-3-5-3-5).  Don't bother classifying (semi)-regular tilings; we
% didn't do that in Section 2.3.
% POTENTIAL: Mention Escher's art.  I said I would in the chapter intro.
\begin{enumerate}
\item The inverse hyperbolic functions are defined on the real numbers as follows:
$$\cosh^{-1}x=\ln(x+\sqrt{x^2-1})~~~~~~\sinh^{-1}x=\ln(x+\sqrt{x^2+1})$$
$$\tanh^{-1}x=\frac 12\ln\frac{1+x}{1-x}~~~~~~\coth^{-1}x=\frac 12\ln\frac{x+1}{x-1}$$
$$\operatorname{sech}^{-1}x=\ln\frac{1+\sqrt{1-x^2}}x~~~~~~\operatorname{csch}^{-1}x=\ln\left(\frac 1x+\frac{\sqrt{1+x^2}}{|x|}\right)$$

(a) Show that each of these functions is a \emph{right} inverse of the corresponding hyperbolic function (e.g., $\cosh(\cosh^{-1}x)=x$ when $\cosh^{-1}x$ is a well-defined real number).

(b) Explain why they need not be left inverses of the hyperbolic functions.  Find domains for the hyperbolic functions which would make these two-sided inverses, just as the arc-sine would be an inverse of the sine if the sine's domain were only $[-\pi/2,\pi/2]$.

(c) If $z$ is a point in the Poincar\'e disk model of the hyperbolic plane (viewed as a complex number with absolute value $<1$), then the distance from $0$ to $z$ is $2\tanh^{-1}|z|$.  [Exercise 4 of Section 4.3.]

(d) For $-1<x<1$, we have $\cosh(\tanh^{-1}x)=\frac 1{\sqrt{1-x^2}}$, and $\cosh(2\tanh^{-1}x)=\frac{1+x^2}{1-x^2}$.

(e) $\sinh(\tanh^{-1}x)=\frac x{\sqrt{1-x^2}}$, and $\sinh(2\tanh^{-1}x)=\frac{2x}{1-x^2}$.

\item (a) Given any two ideal triangles, there exists an isometry sending one to the other; in other words, the isometries act transitively on the ideal triangles.  [Recall that an ideal triangle is a triangle all of whose vertices are ideal points.]

(b) Let $\triangle ABC$ and $\triangle A'B'C'$ be triangles with $A,B,A',B'$ ideal points and $C,C'$ regular points.  These are triangles with exactly two ideal vertices.  Show that there is an isometry sending one to the other $\iff\angle C\cong\angle C'$.

(c) Let $\triangle ABC$ be an omega triangle with $A$ the ideal vertex.  If $\alpha,\beta$ are the angles at the regular vertices $B,C$, show that the length of side $\overline{BC}$ is $\ln(\cot(\alpha/2)\cot(\beta/2))$.  [Assume $A=\infty$ in the half-plane model.]  Conclude that the angle measures at $B,C$ and side length $\overline{BC}$ come in only two degrees of freedom.

\item The \textbf{area} of a hyperbolic triangle is defined to be $180^\circ=\pi$ minus the sum of its angles. % You have given the integral definition of the area of a triangle, which the book does not intend to cover until Chapter 6!

(a) Prove that this notion of area satisfies Lemma 2.29. Conclude that the area of a hyperbolic $n$-gon can be defined the same way the area of a Euclidean polygon was in Section 2.3.

(b) The area of a (hyperbolic) $n$-gon is $\pi(n-2)$ minus the sum of its angles.

\item If $s$ and $\theta$ are the side length and angle measure of a hyperbolic equaliteral triangle, show that

(a) $\cos\theta=\frac{\tanh(s/2)}{\tanh s}$;

(b) $\cosh(s/2)=\frac{\cos(\theta/2)}{\sin\theta}=\frac 1{2\sin(\theta/2)}$.

\item The objective of this exercise is to show that if $\alpha,\beta,\gamma$ are positive real numbers such that $\alpha+\beta+\gamma<\pi=180^\circ$, a hyperbolic triangle with angles $\alpha,\beta,\gamma$ exists.

(a) Let $c$ be the side length opposite $\gamma$.  Show that $0<c<\cosh^{-1}\left(\frac{1+\cos\alpha\cos\beta}{\sin\alpha\sin\beta}\right)$.  [Use Proposition 4.13(iii).]

(b) Now suppose $\alpha,\beta$ are fixed, but $c$ can vary, making the opposite angle $\gamma$ vary.  [With $\alpha,\beta,c$ given, the triangle is easy to construct.]  As $c\to 0$, $\cos\gamma\to-\cos(\alpha+\beta)$, and hence $\gamma\to\pi-\alpha-\beta$.  As $c\to\cosh^{-1}\left(\frac{1+\cos\alpha\cos\beta}{\sin\alpha\sin\beta}\right)$, $\gamma\to 0$.  [Note that it can be shown that $\cosh^{-1}\left(\frac{1+\cos\alpha\cos\beta}{\sin\alpha\sin\beta}\right)=\ln(\cot(\alpha/2)\cot(\beta/2))$, which makes sense in view of Exercise 2(c).]

(c) Conclude, using the Intermediate Value Theorem, that any third angle $\gamma\in(0,\pi-\alpha-\beta)$ can be obtained for a particular positive real number $c$.

\item Suppose a regular $n$-gon in the hyperbolic plane has side length $s$ and angle measure $\theta$.

(a) Show that $\cosh s=\frac{2\cos(2\pi/n)+1+\cos\theta}{1-\cos\theta}$.  [Connecting two consecutive vertices to the center of the polygon gives a triangle with angles $2\pi/n,\theta/2,\theta/2$ (why?).  Now use Proposition 4.13(iii).]

(b) Determine the radii of the circumcircle and incircle of the polygon.  [A right triangle can be constructed by connecting the center of the polygon to a vertex, and dropping a perpendicular from the center of the polygon to an adjacent edge.  One of the legs is an inradius and the hypotenuse is a circumradius (why?).  The remaining leg is half the side of the polygon.  Use Proposition 4.15 and part (a).]

\item Explain how to construct each of the following tilings (presumably using the Wythoff construction).  All polygons are regular.
\begin{center}
\includegraphics[scale=.3]{TruncatedHeptagonalTiling.png}~~~~
\includegraphics[scale=.3]{RhombitriheptagonalTiling.png}\\
Truncated Heptagonal Tiling~~~~~~~~~Rhombitriheptagonal Tiling
\end{center}
[For the latter one, start by considering the angle bisector of the right angle in the triangle pattern, without constructing it.]

\item For each of the triangle patterns $[2,3,6]$, $[2,3,7]$, what tiling emerges when, for each triangle, you draw the perpendicular line segments from the incenter to each side?  [The incenter of a hyperbolic triangle is defined like how Exercise 17 of Section 2.1 defined the incenter of a Euclidean triangle.  All of the necessary facts hold in the hyperbolic case as well.  Note that the aforementioned perpendicular line segments must all have the same length.]

\item M.C.~Escher (1898-1972) was a talented mathematical artist who was known to challenge people's sense of space.\footnote{A.~Redhunt. ``Tessellations of the Hyperbolic Plane and M.C. Escher.''  \emph{The Geometric Viewpoint}, Colby.}  He has made many kinds of uniform art for the Euclidean plane, the surfaces of polyhedra, and the hyperbolic plane.  Below is his hyperbolic-tiling art, \emph{Circle Limit IV}.
\begin{center}
\includegraphics[scale=.45]{LW436-MC-Escher-Circle-Limit-IV-19601.jpg}
M.C.~Escher: Circle Limit IV
\end{center}
(a) Find a uniform (polygonal) tiling off of which this art could be based.  [Consider connecting angel feet to devil feet.]

(b) What kinds of symmetry does it have?
\end{enumerate}

\subsection*{4.6. Generalized Spheres and M\"obius Transforms in Higher Dimensions}
\addcontentsline{toc}{section}{4.6. Generalized Spheres and M\"obius Transforms in Higher Dimensions}
Just like the previous two kinds of geometry, we would like to generalize hyperbolic geometry to any number of dimensions.  However, it is not so clear how to do this, because the hyperbolic plane has been based off the complex plane, which cannot be generalized to arbitrary dimensions.  [In fact, $\mathbb R^3$ can't be seen as an extension of the complex numbers, but Hamilton discovered $\mathbb R^4$ as such an extension, which is what we now know as quaternions.  For most $n$, $\mathbb R^n$ is not a number system with all the desirable properties.] % OFC \mathbb R^n$ is a number system if you don't mind zero divisors, just add and multiply coordinatewise.

This brings a handicap to the concept of M\"obius transforms.  However, we will still see that the concept works fine, with the aid of Chapter 1's results.  If $n$ is a positive integer, we let $\overline{\mathbb R^n}=\mathbb R^n\sqcup\{\infty\}$, where $\infty$ is a symbolic point.  [Remember, this is not projective $n$-space unless $n=1$: it has only \emph{one} infinity point.]  We let $S^n$ be the hypersphere $\{\vec v\in\mathbb R^{n+1}:\|\vec v\|=1\}$, and we identify $\overline{\mathbb R^n}$ with the hyperplane in $\overline{\mathbb R^{n+1}}$ given by $x_{n+1}=0$.

As in Section 4.1, we define \textbf{stereographic projection} to be the map $\varphi:S^n\to\overline{\mathbb R^n}$, which sends $N=\vec e_{n+1}$ to $\infty$, and each other point $p\in S^n$ to the intersection with $\overline{\mathbb R^n}$ of the line through $N$ and $p$.  By imitating the arguments in Section 4.1 for the case $n=2$, one can verify the following formulas (the first one assumes $x_{n+1}\ne 1$):
\begin{equation}\tag{F1}
\varphi(x_1,\dots,x_n,x_{n+1})=\left(\frac{x_1}{1-x_{n+1}},\frac{x_2}{1-x_{n+1}},\dots,\frac{x_n}{1-x_{n+1}}\right)
\end{equation}
\begin{equation}\tag{F2}
\varphi^{-1}(x_1,\dots,x_n)=\left(\frac{2x_1}{x_1^2+\dots+x_n^2+1},\dots,\frac{2x_n}{x_1^2+\dots+x_n^2+1},\frac{x_1^2+\dots+x_n^2-1}{x_1^2+\dots+x_n^2+1}\right)
\end{equation}
We furthermore define the \textbf{central inversion} (or \textbf{hypersphere inversion}; sphere inversion in the case $n=3$) to be the map $\psi=\varphi\circ R\circ\varphi^{-1}:\overline{\mathbb R^n}\to\overline{\mathbb R^n}$ where $R$ is the reflection over $\overline{\mathbb R^n}$; i.e., $R(x_1,\dots,x_n,x_{n+1})=(x_1,\dots,x_n,-x_{n+1})$.  Imitating Section 4.1, the reader is encouraged to establish this formula:
$$\psi(x_1,\dots,x_n)=\left(\frac{x_1}{x_1^2+\dots+x_n^2},\dots,\frac{x_n}{x_1^2+\dots+x_n^2}\right)$$
In other words, whenever $\vec v\in\mathbb R^n$ is a nonzero vector, $\psi(\vec v)=\frac 1{\|\vec v\|^2}\vec v$.  Of course, we have $\psi(\vec 0)=\infty$ and $\psi(\infty)=\vec 0$.  In addition, $\psi$ fixes the unit hypersphere $S^{n-1}$.

The following properties can then be established, in pretty much the same way they were the first time around.  The verifications are left to the reader.  [$k$-planes are assumed to include the point $\infty$.]
\begin{itemize}
\item For $0\leqslant k<n$, stereographic projection sends $k$-spheres not containing $N$ to $k$-spheres, and $k$-spheres containing $N$ to $k$-planes.  Conversely, every $k$-sphere or $k$-plane in $\overline{\mathbb R^n}$ is the image of a $k$-sphere on $S^n$.

\item Central inversion fixes $k$-planes containing the origin.  It sends $k$-planes not through the origin to $k$-spheres through the origin, and vice versa.  It sends $k$-spheres not through the origin to other such $k$-spheres.

\item Stereographic projection and central inversion are conformal.  [A hint for proving this is to find a formula for the Jacobian of (F2), and then show that its columns are orthogonal and have the same magnitude.]
\end{itemize}
Influenced by these ideas, we define a \textbf{generalized $k$-sphere} in $\overline{\mathbb R^n}$ to be either a $k$-sphere in $\mathbb R^n$, or a set of the form $P\cup\{\infty\}$ where $P$ is a $k$-plane.  If $k=n-1$, a generalized $k$-sphere is called a \textbf{generalized hypersphere}.  A generalized $1$-sphere is called a \textbf{generalized circle}.  If $n=3$, then we shall call a generalized $2$-sphere a \textbf{generalized sphere} for simplicity.

In defining a M\"obius transform for arbitrary dimensions, our main fundamental result is this.\\

\noindent\textbf{Proposition 4.16 and Definition.} \emph{If $n\geqslant 2$ and $T:\overline{\mathbb R^n}\to\overline{\mathbb R^n}$ is a bijection, the following are equivalent:}

(i) \emph{$T$ is of one of the following forms:}
$$T(\vec v)=\alpha A\vec v+\vec\beta\text{~~~~or~~~~}T(\vec v)=\alpha A(\psi(\vec v-\vec\gamma))+\vec\beta$$
\emph{where $0\ne\alpha\in\mathbb R$, $\vec\beta,\vec\gamma\in\mathbb R^n$, $A\in O(n)$ and $\psi$ is the central inversion.}

(ii) \emph{$T$ is a conformal map which preserves generalized $k$-spheres for any $0\leqslant k<n$.}

(iii) \emph{$T$ is a conformal map which preserves generalized hyperspheres.}

\emph{Such a transformation $T$ is called a \textbf{M\"obius transform} of $\overline{\mathbb R^n}$.}\\

\noindent There is an unsatisfying inconsistency; according to Section 4.2, M\"obius transforms of the Riemann sphere $\mathbb C\sqcup\{\infty\}$ must be orientation-preserving; the orientation-reversing ones are called anti-M\"obius transforms.  However, when dealing with the general case, we shall use the term ``M\"obius transform'' regardless of what happens to the orientation.
\begin{proof}
(i) $\implies$ (ii). Bijections satisfying (ii) are clearly closed under function composition; yet they include dilations, orthogonal transformations, translations, and the central inversion (as previously mentioned).

(ii) $\implies$ (iii) is clear.

(iii) $\implies$ (i). Either $T(\infty)$ is equal to $\infty$ or it isn't.  Let us first suppose $T(\infty)=\infty$.  Then $T$ preserves (Euclidean) hyperplanes, because they are precisely the generalized hyperspheres containing $\infty$.  Let $\vec\beta=T(\vec 0)$.  Each coordinate axis of $\mathbb R^n$ gets mapped by $T$ to a line (through $\vec\beta$), because the coordinate axis is an intersection of $n-1$ hyperplanes, and the bijection $T$ commutes with this intersection.  Since $T$ is conformal, these output lines are perpendicular to each other.  We may thus let $\vec v_j=\frac{T(\vec e_j)-\vec\beta}{\|T(\vec e_j)-\vec\beta\|}$ for each $j$; the $\vec v_j$ are orthonormal vectors and hence
$$A=\begin{bmatrix}\uparrow&\uparrow&\dots&\uparrow\\\vec v_1&\vec v_2&\dots&\vec v_n\\\downarrow&\downarrow&\dots&\downarrow\end{bmatrix}$$
is an orthogonal matrix, i.e., $A\in O(n)$.  Define $S(\vec v)=A^{-1}(T(\vec v)-\vec\beta)$.  Then construction shows that $S$ satisfies (iii), $S(\vec 0)=\vec 0$ and for each $j$, $S(\vec e_j)=\lambda_j\vec e_j$ for some $\lambda_j\ne 0$ in $\mathbb R$.

As $S(\infty)=\infty$, $S$ preserves Euclidean hyperplanes.  Thus it must preserve the hyperplane $x_1=0$ (because this is the hyperplane spanned by $\vec 0,\vec e_2,\dots,\vec e_n$, whose images under $S$ are scalar multiples).  Consequently, it preserves the hyperplanes parallel to $x_1=0$, since a bijection preserves disjoint sets by elementary set theory; this means that it sends every hyperplane $x_1=c$ to a hyperplane of the form $x_1=d$, and thus the first coordinate $x_1$ of $S(\vec v)$ is completely determined by the first coordinate of $\vec v$.  The same argument holds for the rest of the coordinates.  Thus, we may write $S(x_1,\dots,x_n)=(f_1(x_1),\dots,f_n(x_n))$; the Jacobian $dS$ can then be written as
$$dS_{(x_1,\dots,x_n)}=\begin{bmatrix}\frac{\partial f_1}{\partial x_1}\\&\frac{\partial f_2}{\partial x_2}\\&&\ddots\\&&&\frac{\partial f_n}{\partial x_n}\end{bmatrix}$$
Since $S$ is conformal, this matrix has orthogonal columns with the same magnitude.  Yet the magnitude of the $j$-th column is a function of $x_j$ alone \---- this implies that this common magnitude is constant.  If the columns have a magnitude of $\alpha^2$ ($\alpha>0$), then each diagonal entry is either $a$ or $-a$; by multiplying $A$ by a suitable diagonal matrix with $\pm 1$'s on the diagonal, we may assume that the diagonal entries are all equal to $\alpha$.  In this case, $dS=\alpha I_n$, so that (since $S(\vec 0)=\vec 0$), $S(\vec v)=\alpha\vec v$, and $T(\vec v)=A(S(\vec v))+\vec\beta=\alpha A\vec v+\vec\beta$.

This completes the proof where $T(\infty)=\infty$.  Now suppose $T(\infty)\ne\infty$.  Let $\vec\gamma=T^{-1}(\infty)$, and define $S(\vec v)=T(\psi(\vec v)+\vec\gamma)$.  Then $S$ satisfies (iii) and $S(\infty)=\infty$.  Therefore, by the previous two paragraphs, $S(\vec v)=\alpha A\vec v+\vec\beta$ with $0\ne\alpha\in\mathbb R$, $\vec\beta\in\mathbb R^n$ and $A\in O(n)$.  Consequently, $T(\vec v)=S(\psi(\vec v-\vec\gamma))=\alpha A(\psi(\vec v-\vec\gamma))+\vec\beta$.
\end{proof}

\noindent\emph{Remark.} There is actually a deep theorem (due to Liouville) that states that if $n\geqslant 3$, then just $T$ being conformal guarantees that it is a M\"obius transform.  [This theorem is false for $n=2$; in fact, by the Riemann mapping theorem, there are conformal mappings from the disk to any homeomorphic region in $\mathbb R^2$ whatsoever.]  We will omit the details here.\\

\noindent By characterizing M\"obius transforms via property (ii) or (iii) in Proposition 4.16, it is clear that they form a group under function composition.  This group will be important to us later on.

We will depend heavily on M\"obius transforms in studying higher-dimensional hyperbolic spaces in the next section.

\subsection*{Exercises 4.6. (Generalized Spheres and M\"obius Transforms in Higher Dimensions)} % Introduce stereographic projection in higher dimensions, then go over generalized
% spheres and M\"obius transforms.  (Mention Liouville's theorem without trying to prove it)
\begin{enumerate}
\item Let $S\subset\mathbb R^n$ be a hypersphere which contains the origin (on its rim).  Show that when the central inversion in $\mathbb R^n$ is restricted to $S$, the result is stereographic projection from $S$ to a hyperplane, and the projection is taken from the origin.

\item Let $T$ be a M\"obius transform of $\overline{\mathbb R^n}$.

(a) If $T(p)=p$ for every $p\in\overline{\mathbb R^{n-1}}$, show that $T$ is either the identity or the reflection over $\overline{\mathbb R^{n-1}}$.  [According to statement (i) of Proposition 4.16, M\"obius transforms that fix $\infty$ are particularly easy to deal with.]

(b) If $T(p)=p$ for every $p\in S^{n-1}$, show that $T$ is either the identity or the central inversion.  [Establish a M\"obius transform sending $S^{n-1}\to\overline{\mathbb R^{n-1}}$, and then use it to conjugate $T$.]

\item Let $S$ be the hypersphere centered at $\vec c$ with radius $r>0$.  If $T(\vec v)=r^2\psi(\vec v-\vec c)+\vec c$, then $T$ is the unique M\"obius transform other than the identity which fixes every point of $S$.  [$T$ is called the \textbf{inversion via the hypersphere $S$}.]

\item Let $\varphi:S^n\to\overline{\mathbb R^n}$ be stereographic projection.

(a) If $A\in O(n+1)$ is viewed as a bijection of $S^n$, then conjugating it by stereographic projection ($\varphi\circ A\circ\varphi^{-1}:\overline{\mathbb R^n}\to\overline{\mathbb R^n}$) gives a M\"obius transform.  [Use statement (iii) in Proposition 4.16.]

(b) Show by example that a M\"obius transform of $\overline{\mathbb R^n}$ need not be conjugated to an orthogonal transformation.

\item Define $S:\overline{\mathbb R^n}\to\overline{\mathbb R^n}$ via $S(\vec 0)=\infty$, $S(\infty)=\vec 0$ and $S(\vec v)=-\frac 1{\|\vec v\|^2}\vec v$.

(a) $S$ is a M\"obius transform.

(b) When $S$ is conjugated via stereographic projection to a map $S^n\to S^n$, this map is the antipodal map $\vec p\mapsto -\vec p$ of the hypersphere. % For a second I thought you were saying the conjugate wasn't the antipodal map. kek

(c) If $T$ is a M\"obius transform of $\overline{\mathbb R^n}$, then $S\circ T=T\circ S$ if and only if $T$ is conjugated via stereographic projection to an orthogonal transformation of the hypersphere.  [If $S\circ T=T\circ S$, first show that the conjugation of $T$ preserves the great $(n-1)$-spheres of $S^n$, i.e., those which have the same Euclidean center as $S^n$.  Then use Proposition 1.9 to assume the transformation fixes certain ones.]  Use this exercise and the previous one to conclude that $O(n+1)$ is isomorphic to the centralizer of $S$ in the group of M\"obius transforms of $\overline{\mathbb R^n}$ [Exercise 11(b) of Section 1.2].

\item If $T$ is a M\"obius transform of $\overline{\mathbb R^n}$ and $\psi$ is the central inversion, then $T\circ\psi=\psi\circ T$ if and only if $T$ fixes the sphere $S^{n-1}$ (not necessarily pointwise).  [Use Exercise 2(b).]  Note that such a transformation $T$ need not fix the closed ball $\{\vec v\in\mathbb R^n:\|\vec v\|\leqslant 1\}$, because it may send the interior of the ball to the exterior (e.g., if $T=\psi$).

\item Let $\omega$ be a generalized $k$-sphere, $0\leqslant k<n$.  Show that the following are equivalent:

~~~~(i) The central inversion sends $\omega$ to itself;

~~~~(ii) $\omega$ meets the unit hypersphere orthogonally;

~~~~(iii) Either $\omega$ is a $k$-plane through the origin, or else it is the intersection of a $(k+1)$-plane\footnote{If $k=n-1$, we take this ``$(k+1)$-plane'' to be all of $\overline{\mathbb R^n}$.} through the origin with a hypersphere of radius $r$ and center $\vec a\in\mathbb R^n$ such that $\|\vec a\|^2-r^2=1$.

Furthermore, if these conditions hold and $\omega$ is a $k$-sphere, then the radius $r$ and center $\vec a$ of $\omega$ satisfy $\|\vec a\|^2-r^2=1$.  

\item Let $\omega$ be a generalized $k$-sphere, $0\leqslant k<n$.  Show that the following are equivalent:

~~~~(i) The central inversion sends $\omega$ to its own negation (i.e., $\omega$ is fixed by the transformation in Exercise 5);

~~~~(ii) The corresponding $k$-sphere on the $n$-sphere (via stereographic projection) shares the same Euclidean center as the $n$-sphere;

~~~~(iii) The corresponding $k$-sphere on the $n$-sphere contains at least one point and its antipode;

~~~~(iv) The corresponding $k$-sphere on the $n$-sphere is closed under taking antipodes;

~~~~(v) Either $\omega$ is a $k$-plane through the origin, or else it is the intersection of a $(k+1)$-plane through the origin with a hypersphere of radius $r$ and center $\vec a\in\mathbb R^n$ such that $r^2-\|\vec a\|^2=1$.

Furthermore, if these conditions hold and $\omega$ is a $k$-sphere, then the radius $r$ and center $\vec a$ of $\omega$ satisfy $r^2-\|\vec a\|^2=1$.

[Adapt Exercises 6-7 of Section 4.2.]

\item (a) Show that the dihedral angle between generalized spheres in $\overline{\mathbb R^3}$ is constant throughout their intersection.  The dihedral angle between surfaces at an intersection point is the angle between the tangent vectors to the surfaces at that point which are perpendicular to the intersection.

(b) Show that the dihedral angle between a generalized sphere and a generalized circle (when they intersect in two points) is the same at the two points.
\end{enumerate}

\subsection*{4.7. Hyperbolic $n$-space}
\addcontentsline{toc}{section}{4.7. Hyperbolic $n$-space}
Given the material from the previous section, we can readily study (or introduce) hyperbolic space in an arbitrary number of dimensions.  We do this merely by generalizing the material of Section 4.2, using the previous section in place of Section 4.1.  As before, the isometries are the M\"obius transforms, and will play a major role in the study.

We let $H^n(\mathbb R)$ be the open unit ball $\{\vec v\in\mathbb R^n:\|\vec v\|<1\}$, and $\overline{H^n}(\mathbb R)$ the closed unit ball $\{\vec v\in\mathbb R^n:\|\vec v\|\leqslant 1\}$.  We let $H^n_\infty(\mathbb R)=\overline{H^n}(\mathbb R)-H^n(\mathbb R)=\{\vec v\in\mathbb R^n:\|\vec v\|=1\}$.  $H^n(\mathbb R)$ is referred to as the \textbf{Poincar\'e ball model} of hyperbolic $n$-space, $H^n_\infty(\mathbb R)$ is referred to as the \textbf{rim} / \textbf{boundary at infinity}, and its elements are called \textbf{ideal points}.  Observe that $H^n_\infty(\mathbb R)$ is a generalized hypersphere.

For $0\leqslant k<n$, a \textbf{$k$-plane} is defined to be a set of the form $\omega\cap\overline{H^n(\mathbb R)}$, where $\omega$ is a generalized $k$-sphere which meets the rim orthogonally.  Generally, a $1$-plane is called a \textbf{line}, a $2$-plane is called a \textbf{plane}, and an $(n-1)$-plane is called a \textbf{hyperplane}.

Alternatively, if we take $H^n(\mathbb R)$ to be the upper half of $\mathbb R^n$ given by $\{(x_1,\dots,x_n)\in\mathbb R^n:x_n>0\}$, $H^n_\infty(\mathbb R)$ to be the generalized hypersphere $x_n=0$ (in particular, it contains $\infty$), and $\overline{H^n}(\mathbb R)$ to be $H^n(\mathbb R)\cup H^n_\infty(\mathbb R)$, then we get the \textbf{Poincar\'e half-space model} of hyperbolic $n$-space.  Its $k$-planes, as before, are sets of the form $\omega\cap\overline{H^n}(\mathbb R)$ where $\omega$ is a generalized $k$-sphere which meets the rim orthogonally; in particular, hyperplanes are Euclidean hyperplanes parallel to the $x_n$-axis and portions of hyperspheres centered on $x_n=0$.

For definiteness, $H^n(\mathbb R),\overline{H^n}(\mathbb R),H^n_\infty(\mathbb R)$ will refer to the ball model, unless otherwise specified.\\

\noindent\textbf{ISOMETRIES AND DISTANCE}\\

\noindent As before, the easiest step for now is to cover what the isometries are.  In either the Poincar\'e ball or half-space model, we would like an isometry to be a M\"obius transform which fixes the subset $H^n(\mathbb R)$.  It is clear that such a map fixes the generalized hypersphere $H^n_\infty(\mathbb R)$, and (since they are conformal), they therefore preserve $k$-spheres orthogonal to $H^n_\infty(\mathbb R)$: in other words, they preserve $k$-planes.

In Proposition 4.18, we will obtain a distance metric which is invariant under these transformations.  Since we cannot define cross ratios for $\mathbb R^n$ the way we did for the complex plane, we must define the notion of distance rather differently.  Our trick is to use Proposition 1.21, which tells us we need only define it on certain kinds of points, provided certain conditions are satisfied.\\

\noindent\textbf{Lemma 4.17.} \emph{Two points $p,q\in H^n(\mathbb R)$ determine a line.}
\begin{proof}
As in Section 4.3, it is routine to show that the group of M\"obius transforms acts transitively on $H^n(\mathbb R)$.  Thus, by applying a M\"obius transform, we may assume $p=0$, the Euclidean origin of the ball.  In that case, the Euclidean line $\ell$ through $0$ and $q$ is a diameter of the hypersphere, hence is orthogonal to the ball, i.e., a hyperbolic line.  If $\ell'$ is another line going through $0$ and $q$, then (assuming $\ell'\ne\ell$), $\ell'$ cannot be a Euclidean line, because $\ell$ is the \emph{unique} Euclidean line through $0$ and $q$.  Hence $\ell'$ is a Euclidean circle, and its radius $r$ and center $\vec a$ satisfy $\|\vec a\|^2-r^2=1$ by Exercise 7 of the previous section.  Yet, since $0\in\ell'$, we must have $\|\vec a\|=r$, which leads to a clear contradiction.
\end{proof}
\noindent\textbf{Proposition 4.18 and Definition.} \emph{Let $H^n(\mathbb R)$ be the half-space model for hyperbolic $n$-space.  There is a unique function $\rho:H^n(\mathbb R)\times H^n(\mathbb R)\to\mathbb R$ such that:}

(i) \emph{$\rho(T(p),T(q))=\rho(p,q)$ whenever $T$ is a M\"obius transform of $H^n(\mathbb R)$; in other words, $\rho$ is invariant under these M\"obius transforms.s}

(ii) \emph{$\rho(a\vec e_n,b\vec e_n)=|\ln(b/a)|$, where $a,b>0$ and $\vec e_n$ is the vector $(0,\dots,0,1)$.} % Re. your comment about \sqrt{dx^2+dy^2}/y, the calculussy metric is for Chapter 6!

\emph{This function gives the \textbf{distance} between two regular points in hyperbolic $n$-space.}\\

\noindent It is worth remarking that we are temporarily using the half-space model instead of the ball model to define distances.  However, this distance metric easily carries over to the ball model, via the conversion in Exercise 4.  (Similar, but tougher, reasoning can be used to prove Proposition 4.18 directly for the ball model, but that will be needless.)

\begin{proof}
Let $X=H^n(\mathbb R)\times H^n(\mathbb R)$, and let the group $G$ of M\"obius transforms act on $X$ via $g\cdot(p,q)=(g(p),g(q))$.  Observe that condition (i) states that, in terms of Section 1.6, $\rho$ is a function on orbits from the set $X$ on which the group acts.

Let $X'=\{(a\vec e_n,b\vec e_n):a,b>0\}\subset X$, and define $f:X'\to\mathbb R$ via $f(a\vec e_n,b\vec e_n)=|\ln(b/a)|$.  Then all that is left is to prove that conditions (i), (ii) of Proposition 1.21 are satisfied; it will follow that $f$ extends to a unique function on orbits $\varphi$ on $X$, completing the proof of this proposition.

To show (i), let $(p,q)\in X$.  By Lemma 4.17, we may let $\ell$ be the line determined by them, and let $c$ be an ideal point of $\ell$.  We let $\psi$ be the inversion via the unit hypersphere centered at $c$ [see Exercise 3 of the previous section]; then $\psi\in G$.  Moreover, $\psi$ sends $\ell$ to a generalized circle $\ell'$ containing $\infty$ (because $c\in\ell$ and $\psi(c)=\infty$), hence $\ell'$ is a Euclidean line, and is incidentally parallel to the $x_n$-axis.  Thus, $\ell'$ is of the form $\{(c_1,\dots,c_{n-1},t):t\geqslant 0\}$ with the $c_j$ fixed.  If $p'=\psi(p)$ and $q'=\psi(q)$, then $p',q'\in\ell'$.  Finally, the map $T:\vec v\mapsto\vec v-(c_1,\dots,c_{n-1},0)$ is clearly an M\"obius transform, And $T(\ell')$ is the $x_n$-axis.  Thus, $T(p'),T(q')$ are on the $x_n$-axis.  Therefore, $T\circ\varphi$ is an M\"obius transform which sends $(p,q)$ to two points on the $x_n$-axis, i.e., an element of $X'$, proving condition (i) of Proposition 1.21.

As for (ii), suppose that $(p,q),(p',q')\in X'$, say $p=a\vec e_n,q=b\vec e_n,p'=a'\vec e_n,q'=b'\vec e_n$, and let $T\in G$ be an M\"obius transform such that $T(p)=p',T(q)=q'$.  Then $f(p,q)=|\ln(b/a)|$ and $f(p',q')=|\ln(b'/a')|$; we wish to show that they are equal.  Let $\ell$ be the line determined by $p,q$ (Lemma 4.17); then $\ell$ is the $x_n$-axis, and since $p',q'\in\ell$, we have $T(\ell)=\ell$.  Moreover, $T$ either fixes $0,\infty$ or swaps them (as $0,\infty$ are the ideal points of $\ell$).

Suppose $T$ fixes $0,\infty$.  Then since $T(\infty)=\infty$, Proposition 4.16 shows that $T$ is of the form $T(\vec v)=\alpha A\vec v+\vec\beta$, with $\alpha\ne 0$ in $\mathbb R$, $A\in O(n)$, and $\vec\beta\in\mathbb R^n$.  Since $T(\vec 0)=\vec 0$, $\vec\beta=\vec 0$.  Since $T(\ell)=\ell$, it is clear that $A\vec e_n=\pm\vec e_n$; we may assume $A\vec e_n=\vec e_n$ by changing $\alpha$ and $A$ to their negatives otherwise.  Consequently, $T(p)=\alpha p$ and $T(q)=\alpha q$, since $p,q$ are scalar multiples of $\vec e_n$ which is fixed by $A$.  Thus, we have
$$\left|\ln\frac{b'}{a'}\right|=\left|\ln\frac{\alpha b}{\alpha a}\right|=\left|\ln\frac ba\right|$$
and so $f(p,q)=f(p',q')$ in this case.

Now suppose $T$ swaps $0,\infty$.  Then $\psi\circ T$ (were $\psi$ is the origin-centered central inversion) is another M\"obius transform which fixes the line $\ell$; let $p^*=\psi(p')=\frac 1{a'}\vec e_n$ $[=(\psi\circ T)(p)]$, and $q^*=\psi(q')=\frac 1{b'}\vec e_n$.  By construction, $\psi\circ T$ fixes $0$ and $\infty$.  Therefore, by the argument of the preceding paragraph, $\left|\ln\frac{1/b'}{1/a'}\right|=\left|\ln\frac ba\right|$.  Furthermore,
$$\left|\ln\frac ba\right|=\left|\ln\frac{1/b'}{1/a'}\right|=\left|-\ln\frac{1/a'}{1/b'}\right|=\left|\ln\frac{1/a'}{1/b'}\right|=\left|\ln\frac{b'}{a'}\right|$$
hence $f(p,q)=f(p',q')$ again.
\end{proof}

\noindent The M\"obius transforms that fix $H^n(\mathbb R)$ are hence called \textbf{isometries}.  By applying them and using Proposition 4.18(ii), it is clear that the segment addition postulate holds, and that points of a given distance are closer to the Euclidean eye when closer to the rim.

Though the M\"obius transforms preserve distances by Proposition 4.18, it is natural to ask whether they are the only global functions which do.  The answer is yes, but the proof will be omitted here, as it would digress from the main point of the matter.\\

\noindent\textbf{BASIC FACTS ABOUT HYPERBOLIC $3$-SPACE}\\

\noindent We conclude this section with a few basic properties about hyperbolic $3$-space.  First, we have already shown that two points determine a line.  We also have:\\

\noindent\textbf{Proposition 4.19.} \emph{In $H^3(\mathbb R)$:}

(i) \emph{Any two distinct planes are either disjoint in $\overline{H^3}(\mathbb R)$ (in which case we say they are \textbf{divergent-parallel}), intersect in one ideal point (in which case we say they are \textbf{convergent-parallel}), or intersect in a line.}

(ii) \emph{Three points that are not collinear determine a plane.}

(iii) \textsc{(Ultraparallel Theorem)} \emph{Given two divergent-parallel planes, there is a unique line simultaneously perpendicular to both of them.}\\

\noindent These results have analogues in higher-dimensional spaces; e.g., in $H^4(\mathbb R)$, four points that are not coplanar determine a hyperplane.
\begin{proof}
(i) If the planes meet at a regular point $p$, we may assume $p$ is the origin of the ball model.  By imitating the proof of Lemma 4.17, one sees that the planes are then Euclidean planes through the origin, hence they clearly intersect in a line.  If the planes meet at an ideal point $q$ but at no regular point, we may assume we are in the half-space model and $q=\infty$.  Then the planes are Euclidean planes parallel to the $z$-axis (as those are the only planes containing $\infty$), and since they do not intersect at any regular points, they must be parallel as Euclidean planes.  In this case, their intersection consists solely of the ideal point $\infty$.  Finally, if the planes do not intersect at all, there is nothing to prove.

(ii) Let $a,b,c$ be the points.  If all of them are ideal, we may assume we are in the half-space model and $a=\infty$.  Then the points $b,c$ in the $xy$-plane determine a line $\ell$, and then the set $\{(x,y,z):(x,y)\in\ell,z\geqslant 0\}\cup\{\infty\}$ is a (hyperbolic) plane containing $a,b,c$.  Now suppose at least one of $a,b,c$ is regular.  Without loss of generality, assume $a$ is regular; further assume we are in the ball model and $a=0$.  Then $a,b,c$ are contained in a Euclidean plane, and this plane is a hyperbolic plane as well (because it contains the origin).  This proves that a plane containing $a,b,c$ exists.

As for the uniqueness of the plane, copy the last paragraph of the proof of Proposition 2.43, using Proposition 4.19(i) in place of Proposition 2.43(i).

(iii) The proof is similar to that of Theorem 4.7.  Let $P_1$ and $P_2$ be the planes.  By applying isometries, we may assume that we are in the half-space model, $P_1$ is the $yz$-plane, and $P_2$ is a Euclidean hemisphere centered on the positive $x$-axis; say it is centered at $(a,0,0)$ and its Euclidean radius is $r$.  Since the planes are divergent-parallel, $0<r<a$.  Then the reader can readily verify through algebra that:
\begin{itemize}
\item A line $\ell$ is perpendicular to $P_1$ if and only if it is a semicircle arc centered on the $y$-axis, contained in a plane perpendicular to the $y$-axis; i.e., it can be parametrized via $t\mapsto(s\sin t,c,s\cos t),t\in[-\pi/2,\pi/2]$ (with $s>0,c$ constant).

\item If $\ell$ satisfies those conditions, $\ell\perp P_2$ if and only if $c=0$ and $s=\sqrt{a^2-r^2}$.
\end{itemize}
[Remember, if a curve is perpendicular to a surface, then it must go in the direction of the normal vector to the surface.]

From this, the statement is immediate.
\end{proof}

\subsection*{Exercises 4.7. (Hyperbolic $n$-space)}
\begin{enumerate}
\item Show that if $0\leqslant k<n$, then in $H^n(\mathbb R)$, a $k$-plane $P$ is isometric to $H^k(\mathbb R)$; i.e., there is a bijection $P\cong H^k(\mathbb R)$ which preserves distances, as well as lines and angles.  [Assume $P$ is a Euclidean $k$-plane in the half-space model.  Note that this statement also makes sense when ideal points are included in the situation.]  Conclude (by taking $k=2$) that the triangle theorems from Section 4.3 hold in arbitrary hyperbolic $n$-space.

\item In hyperbolic $3$-space, show that a plane and a line not contained in the plane intersect in at most one point.  [The point could be ideal; in this case we say the plane and line are convergent-parallel.]

\item Suppose $p=(a_1,\dots,a_n)$ and $q=(b_1,\dots,b_n)$ are regular points of the half-space model of hyperbolic $n$-space.  [Thus $a_n>0$ and $b_n>0$.]  Adapt Exercise 9 of Section 4.3 to show that their distance is
$$\rho(p,q)=\cosh^{-1}\left(1+\frac{(a_1-b_1)^2+(a_2-b_2)^2+\dots+(a_n-b_n)^2}{2a_nb_n}\right)$$

\item Define $f,g:\overline{\mathbb R^n}\to\overline{\mathbb R^n}$ as follows:
$$f(u_1,\dots,u_n)=\frac 1{u_1^2+\dots+u_{n-1}^2+(u_n+1)^2}\left(2u_1,\dots,2u_{n-1},u_1^2+\dots+u_n^2-1\right)$$
$$f(\infty)=(0,\dots,0,1),~~~~f(0,\dots,0,-1)=\infty$$
$$g(v_1,\dots,v_n)=\frac 1{v_1^2+\dots+v_{n-1}^2+(v_n-1)^2}\left(2v_1,\dots,2v_{n-1},1-v_1^2-\dots-v_n^2\right)$$
$$g(\infty)=(0,\dots,0,-1),~~~~g(0,\dots,0,1)=\infty$$
(a) $f$ and $g$ are M\"obius transforms, and $f\circ g=1_{\overline{\mathbb R^n}}=g\circ f$.  [To avoid hassle, left and right compose $f,g$ with translations to get maps sending $0\mapsto\infty\mapsto 0$.]

(b) $f$ sends the half-space model of hyperbolic $n$-space to the ball model, and $g$ sends the ball model to the half-space model.

(c) Conclude that $f$ and $g$ are isometries which convert between the two models of hyperbolic $n$-space.

\item Let $\vec v$ be a regular point in the ball model of hyperbolic $n$-space; i.e., $\|\vec v\|<1$.  Show that the hyperbolic distance from the origin to $\vec v$ is $\ln\frac{1+\|\vec v\|}{1-\|\vec v\|}$, or what is the same thing, $2\tanh^{-1}\|\vec v\|$.

\item\emph{(Categorization of isometries of $H^3(\mathbb R)$.)} \---- The aim of this exercise is to categorize isometries of $H^3(\mathbb R)$, the way isometries of $H^2(\mathbb R)$ were categorized at the end of Section 4.3.

(a) Suppose $\operatorname{Isom}(H^3(\mathbb R))$ is the group of isometries of the half-space model, and $G$ is the group of M\"obius and anti-M\"obius transforms of $\overline{\mathbb R^2}$.  Define $\varphi:\operatorname{Isom}(H^3(\mathbb R))\to G$ by taking an isometry of the space and restricting it to the rim at infinity.  Show that $\varphi$ is a group isomorphism. [Proposition 4.16 applies to both dimensions.]

(b) An isometry of $H^3(\mathbb R)$ is orientation-preserving if and only if its restriction to the rim is a M\"obius transform of $\mathbb C\sqcup\{\infty\}$; i.e., $z\mapsto\frac{az+b}{cz+d}$ with $ad-bc\ne 0$.

(c) Classify the M\"obius transforms of $\mathbb C\sqcup\{\infty\}$ up to conjugacy.  [The group of M\"obius transforms is isomorphic to $PGL_2(\mathbb C)\cong GL_2(\mathbb C)/\mathbb C^*$; consider writing matrices in Jordan normal form.]  Use this and parts (a)-(b) to classify the orientation-preserving isometries of $H^3(\mathbb R)$.

(d) Show that every orientation-reversing isometry is a composition of a reflection and an orientation-preserving isometry.  Use this and part (c) to classify the orientation-reversing isometries of $H^3(\mathbb R)$.  Some of them are \textbf{horolary reflections}, obtained by performing a horolation on a plane and then reflecting space over the plane.  [The tricky part is to use casework on how the reflection and the orientation-preserving isometry relate; e.g., if the orientation-preserving isometry is a rotation, one must look at its axis' intersection with the plane of reflection.]

\item (a) If $T$ is an isometry of $H^n(\mathbb R)$ and $T(\vec 0)=\vec 0$, show that $T$ is Euclidean orthogonal transformation in $O(n)$.  [Use Proposition 1.9 to make certain convenient assumptions.]

(b) Show that every isometry of $H^n(\mathbb R)$ is a finite composition of reflections.  [In the ball model, start by composing the isometry with a reflection so that the resulting isometry fixes the origin.  Now the isometry is in $O(n)$ by part (a), and the rest uses linear algebra.]

\item\emph{(Hypersheets, horosheets and space fillings.)} \---- (a) Let $P$ be a plane in $H^3(\mathbb R)$, $p$ a regular point outside $P$, and $Q$ the trajectory of $p$ along all translations that fix $P$.  Show that (after completion), $Q$ is a generalized sphere in the ball and half-space models, and that it has the same ideal points as $P$.  [We call $Q$ a \textbf{hypersheet}; it is a two-dimensional analogue of a hypercycle.]

(b) Explain why the isometry group of a hypersheet is isomorphic to that of a plane.

(c) Let $p$ be an ideal point, and let $S$ be a surface to which all lines through $p$ are perpendicular.  Show that $S$ is a generalized sphere which meets the rim $H^3_\infty(\mathbb R)$ at exactly one point.  [We call $S$ a \textbf{horosheet}.]

(d) Show that the isometry group of a horosheet is isomorphic to $\operatorname{Isom}(\mathbb R^2)$, the isometry group of the Euclidean plane. [Assume its ideal point is $\infty$ in the half-space model.] % Thus hyperbolic 3-space has models of all three metric geometries!  A hyperplane (also a hypersheet) has the geometry of the hyperbolic plane; a horosheet has the geometry of the Euclidean plane; and a sphere has the geometry of a spherical plane.

(e) Recall the five Platonic solids from Section 2.7.  Suppose you start with only the vertices of a Platonic solid centered at the origin, and require the vertices to be inside the (closed) Poincar\'e ball.  Show that if you connect them with hyperbolic line segments and plane portions, you get a hyperbolic-space model of the solid with the same isometry group as the original solid. % Section 2.7 gave algebraic coordinates for the Platonic solids centered at the origin.  Since isometries fixing the origin are the same as Euclidean isometries fixing the origin, they act transitively on the vertices, edges and faces, and the stabilizer of each n-gonal face is isomorphic to D_n.  So the faces are regular and the dihedral angles are constant, this doesn't need to be assumed.

(f) Let $P$ be a Platonic solid from part (e).  If the vertices of $P$ are ideal, then the angles of each face are zero, and the dihedral angles of the solid measure $\pi-\frac{2\pi}n$ where $n$ is the number of faces to a vertex.  [Think of the spheres as Euclidean spheres in $\mathbb R^3$.]

(g) Show that a regular ideal tetrahedron, cube, octahedron or dodecahedron may be used to fill the $3$-space without gaps or overlaps.  [Keep reflecting them over their own faces.]  Then by taking the centers of the cells and connecting two of them by an edge if and only if the cells meet by a face, you get a space-filling made up of uniform Euclidean tilings on horosheets [this makes sense in view of part (d)].

(h) Explain why a regular ideal icosahedron cannot fill the $3$-space.
\end{enumerate}

\subsection*{4.8. Beltrami-Klein Model}
\addcontentsline{toc}{section}{4.8. Beltrami-Klein Model}
The Beltrami-Klein model of $H^2(\mathbb R)$ is a disk-shaped model for which lines are Euclidean chords of the circle.  It is also known as the \textbf{Klein disk model}, \textbf{Cayley-Klein model}, or \textbf{projective model} (as we will eventually see that isometries are projective transformations).

% https://en.wikipedia.org/wiki/Beltrami%E2%80%93Klein_model#History
It was first discovered in two of Eugenio Beltrami's papers in 1868, first for the plane, then for hyperbolic space in an arbitrary number of dimensions.  Those papers were unnoticed for a while, until the model was named after Felix Klein.  This happened because in 1859, when Arthur Cayley found a way to derive Euclidean geometry from projective geometry, Klein became acquainted with his work.  Klein then realized that Cayley's ideas entailed a way to derive hyperbolic geometry from projective geometry.

We will start by studying this model for the hyperbolic plane (two dimensions).  We could do this from scratch, but then it would not be clear how to transform figures between this model and the Poincar\'e disk and half-plane models, so we wish to find the conversion first.  If $D$ is the closed disk $\{\vec v\in\mathbb R^2:\|\vec v\|\leqslant 1\}$, then our intuition tells us that we seek a function $\chi:D\to D$ satisfying the following two conditions:
\begin{itemize}
\item $\chi$ fixes every point on the unit circle; i.e., $\|\vec v\|=1\implies\chi(\vec v)=\vec v$.

\item $\chi$ sends every line in the Poincar\'e disk model (i.e., generalized circle peprpendicular to the circle) to a Euclidean chord of the circle.
\end{itemize}
\noindent These conditions determine the function rather easily.  To begin with, every diameter $\ell$ of the circle is a line in the Poincar\'e disk model, and so we have $\chi(\ell)=\ell$ \---- because the conditions imply that $\chi(\ell)$ is a Euclidean chord which shares the endpoints of $\ell$.  Therefore $\chi$ fixes every diameter of the circle; by intersecting two diameters, we get $\chi(\vec 0)=\vec 0$.

If $0<c<1$ is a real number, we next claim that $\chi((c,0))$ is uniquely determined.  After all, $(c,0)$ can be realized as the intersection of two Poincar\'e-disk lines: the $x$-axis, and the line $\ell^\perp$ through $(c,0)$ perpendicular to the $x$-axis, which is in the Euclidean sense a circle centered on the $x$-axis.  Suppose $\ell^\perp$ is centered at $(a,0),a>0$ and has radius $r$.  Since it is a line in the Poincar\'e disk, Exercise 6 of Section 4.2 implies $a^2-r^2=1$.  Yet $(c,0)$ is on the circle; hence, $a-r=c$.\footnote{We know it is not $a+r$ which is equal to $c$, because that would imply $a-r=\frac 1c>c$.}  Therefore, $a+r=\frac{a^2-r^2}{a-r}=\frac 1c$.  Algebraic manipulation shows $a=\frac{1+c^2}{2c}$ and $r=\frac{1-c^2}{2c}$; hence $\ell^\perp$ is given by an equation
$$\left(x-\frac{1+c^2}{2c}\right)^2+y^2=\left(\frac{1-c^2}{2c}\right)^2$$
or equivalently,
$$(x^2+y^2)-\frac{1+c^2}cx+1=0$$
One can check that this equation is satisfied by $(c,0)$.  Moreover, we can easily find the ideal points of $\ell^\perp$, by taking the conjunction of this equation with the unit circle's equation $x^2+y^2=1$.  With that done, we get $1-\frac{1+c^2}cx+1=0$; hence, $x=\frac{2c}{1+c^2}$.  Since $y^2=1-x^2$, we conclude $y=\pm\frac{1-c^2}{1+c^2}$, so that the ideal points of $\ell^\perp$ are $\left(\frac{2c}{1+c^2},\pm\frac{1-c^2}{1+c^2}\right)$.

Since $\chi$ fixes every ideal point, we conclude that $\chi(\ell^\perp)$ is the Euclidean line segment between $\left(\frac{2c}{1+c^2},\pm\frac{1-c^2}{1+c^2}\right)$.  Also, $\chi$ clearly maps the $x$-axis to itself.  Taking the intersection of these yields $\chi((c,0))=\left(\frac{2c}{1+c^2},0\right)$.  This argument can essentially be repeated for any point in the disk, by consideration of a Euclidean rotation; we get that $\chi$ must be given by:
$$\chi(\vec v)=\frac{2\vec v}{1+\|\vec v\|^2}\text{ for }\vec v\in D.$$
Note that $\chi$ is bijective, and its inverse is given by $\chi^{-1}(\vec u)=\frac{\vec u}{1+\sqrt{1-\|\vec u\|^2}}$; this can be directly verified.  We claim that $\chi$ successfully does satisfy the bullet points at the beginning of the section.

That $\|\vec v\|=1\implies\chi(\vec v)=\vec v$ is a direct computation.  To see where $\chi$ sends a line $\ell$ in the Poincar\'e disk model, we recall that if $\ell$ is a diameter of the circle, then $\chi(\ell)=\ell$ because $\chi$ sends vectors to scalar multiples of themselves.  Thus, $\chi(\ell)$ is a Euclidean chord of the circle.

We may henceforth assume $\ell$ is not a diameter of the circle, and that it is a Euclidean circle perpendicular to the circle.  Observe that in this case, $\ell$ can be given an equation $\|\vec v\|^2-\vec c\cdot\vec v+1=0$, where $\vec c$ is a fixed vector.  After all, if $\vec a$ is its center and $r$ is its radius, then $\|\vec a\|^2-r^2=1$, and $\ell$ is given by $\|\vec v-\vec a\|^2=r^2$: with that, we may take $\vec c=2\vec a$.

Moreover, the set $\chi(\ell)$ is equal to $\{\vec u\in D:\chi^{-1}(\vec u)\in\ell\}$.  The formula $\chi^{-1}(\vec u)=\frac{\vec u}{1+\sqrt{1-\|\vec u\|^2}}$ and the aforementioned equation for $\ell$ entails
$$\vec u\in\chi(\ell)\iff\frac{\vec u}{1+\sqrt{1-\|\vec u\|^2}}\in\ell\iff\big\|\frac{\vec u}{1+\sqrt{1-\|\vec u\|^2}}\big\|^2-\vec c\cdot\frac{\vec u}{1+\sqrt{1-\|\vec u\|^2}}+1=0$$
$$\iff\left(\frac{\|\vec u\|}{1+\sqrt{1-\|\vec u\|^2}}\right)^2-\vec c\cdot\frac{\vec u}{1+\sqrt{1-\|\vec u\|^2}}+1=0$$
$$\iff\frac{\|\vec u\|^2}{(1+\sqrt{1-\|\vec u\|^2})^2}-\frac{\vec c\cdot\vec u}{1+\sqrt{1-\|\vec u\|^2}}+1=0$$
$$\iff\|\vec u\|^2-(1+\sqrt{1-\|\vec u\|^2})(\vec c\cdot\vec u)+(1+\sqrt{1-\|\vec u\|^2})^2=0$$
$$\iff-(1+\sqrt{1-\|\vec u\|^2})(\vec c\cdot\vec u)+2+2\sqrt{1-\|\vec u\|^2}=0$$
$$\iff-(1+\sqrt{1-\|\vec u\|^2})(\vec c\cdot\vec u)+2(1+\sqrt{1-\|\vec u\|^2})=0$$
$$\iff-(\vec c\cdot\vec u)+2=0\iff\vec c\cdot\vec u=2,$$
and that last equation describes a Euclidean line perpendicular to $\vec c$.  Hence $\chi(\ell)$ is a Euclidean chord of the circle as desired.  We have thus established:\\

\noindent\textbf{Proposition 4.20 and Definition.} \emph{If $D$ is the closed unit disk, and $\chi:D\to D$ is defined by $\chi(\vec v)=\frac{2\vec v}{1+\|\vec v\|^2}$, then $\chi$ is the unique bijection such that}

(i) \emph{$\chi(\vec v)=\vec v$ for all $\|\vec v\|=1$;}

(ii) \emph{$\chi$ sends lines in the Poincar\'e disk model of the hyperbolic plane to Euclidean chords of the circle.}

\emph{This map $\chi$ is called the \textbf{conversion from the Poincar\'e disk to the Klein disk}.  Its inverse $\chi^{-1}:\vec u\mapsto\frac{\vec u}{1+\sqrt{1-\|\vec u\|^2}}$ is called the \textbf{conversion from the Klein disk to the Poincar\'e disk}.}

\emph{The \textbf{Beltrami-Klein model} of the hyperbolic plane is the model in $D$ where lines are Euclidean chords of the circle, for which $\chi^{-1}$ sends constructions to the corresponding constructions in the Poincar\'e disk model.}\\

\noindent We thus have, by moving the points around as above, a new way to display hyperbolic-plane constructions.  As an example, consider the picture equipped with Proposition 4.5, consisting of a line and several lines through a point parallel to this line.  The right-hand side illustrates what it would look like in the Beltrami-Klein model:\footnote{To distinguish the models, we will specifically use a fuschia color on the Beltrami-Klein model constructions.}
\begin{center}
\includegraphics[scale=.3]{HypParallels.png}~~~~
\includegraphics[scale=.3]{HypParallels_BK.png}\\
Poincar\'e Disk Model~~~~~~~~~~~~~~~~~~~~~Beltrami-Klein Model
\end{center}

Observe that in the Beltrami-Klein model, the Euclidean angle between two lines meeting at the rim is never zero (even though the hyperbolic angle is zero).  This is an immediate consequence of the lines being Euclidean straight lines.  In fact, the model is not conformal: \emph{angle measures between lines in the Beltrami-Klein model do not generally coincide with the Euclidean angle measures}.  For example, the lines below are perpendicular, but certainly don't look like they are to the Euclidean eye:
\begin{center}
\includegraphics[scale=.3]{Perp_BK.png}
\end{center}
In view of Proposition 4.8, the model can't be conformal anyway: if the Beltrami-Klein model were conformal, then (since triangles look exactly like Euclidean triangles), the angles of a triangle would add to $180^\circ=\pi$, which would contradict Proposition 4.8.

However, we claim we have a basic way of being able to tell when two lines are perpendicular.  We recall (see the end of Section 3.6) that if $\ell$ is a chord of a circle $\omega$, then the \textbf{pole point} of $\ell$ is defined by taking the points where the chord meets $\omega$, forming the tangent lines to $\omega$ at these points, and then taking their intersection.  The particularly interesting thing about the Beltrami-Klein model is that lines are perpendicular if and only if they contain each other's pole points:\\

\noindent\textbf{Proposition 4.21.} \emph{If $\ell_1$ and $\ell_2$ are lines in the Beltrami-Klein model of the hyperbolic plane, the following are equivalent:}

(i) \emph{The lines are perpendicular.}

(ii) \emph{$[\![a,b,c,d]\!]=2$, where $a,c$ are the ideal points of $\ell_1$, and $b,d$ are the ideal points of $\ell_2$, when viewed as complex numbers.}

(iii) \emph{$\ell_1$ contains the pole point of $\ell_2$, when everything is viewed in the projective plane $P^2(\mathbb R)$.}\\

\noindent There is a special reason that (iii) uses the projective plane.  First, as we will see later, the isometries of the Beltrami-Klein model are precisely the projective transformations which fix the circle.  Secondly, if $\ell_2$ is a diameter, its pole point is at infinity, and in this case (iii) states that $\ell_1\perp\ell_2$ in the Euclidean sense.
\begin{proof}
(i) $\iff$ (ii). The ideal points are transformed to themselves through the conversion map $\chi$; thus $a,b,c,d$ are also the ideal points of the lines in the Poincar\'e disk model.  Now use Exercise 16 of Section 4.3.

(ii) $\iff$ (iii). First suppose that $\ell_2$ is a diameter.  Then $b=-d$, and
$$[\![a,b,c,d]\!]=\frac{(c-a)(d-b)}{(d-a)(c-b)}=\frac{(c-a)(2d)}{(d-a)(c+d)}=\frac{2(c-a)d}{(d-a)(c+d)}$$
and hence
$$[\![a,b,c,d]\!]=2\iff\frac{2(c-a)d}{(d-a)(c+d)}=2\iff(c-a)d=(d-a)(c+d)$$
$$\iff cd-ad=cd+d^2-ac-ad\iff d^2=ac,$$
which is true if and only if $\ell_1$ is perpendicular to $\ell_2$ in the Euclidean sense (as one can see by arguing geometrically).  Since $\ell_2$ is a diameter, this is in turn equivalent to $\ell_1$ containing the pole point of $\ell_2$.  This proves (ii) $\iff$ (iii) when $\ell_2$ is a diameter.

Now suppose $\ell_2$ is not a diameter.  Then the pole point of $\ell_2$, as a complex number, is equal to $\frac{2bd}{b+d}$: this can be verified by subtracting $b$ from the expression, noting that the resulting complex number is perpendicular to $b$, concluding that $\frac{2bd}{b+d}$ is on the tangent line to the circle at $b$, then doing the same for $d$.  Also, a complex number $z$ is in the line extending $\ell_1$ if and only if $\frac{z-a}{c-a}\in\mathbb R$, because $z\mapsto\frac{z-a}{c-a}$ and sends $a\mapsto 0,c\mapsto 1$, hence $\ell_1$ to the real line.  Moreover, as the real numbers are precisely those equal to their own complex conjugates,
$$\frac{z-a}{c-a}\in\mathbb R\iff\frac{z-a}{c-a}=\overline{\left[\frac{z-a}{c-a}\right]}=\frac{\overline z-\overline a}{\overline c-\overline a}=\frac{ac(\overline z-\overline a)}{ac(\overline c-\overline a)}\overset{(*)}=\frac{ac\overline z-c}{a-c}=\frac{c-ac\overline z}{c-a}$$
where the equality $(*)$ uses the fact that $a,c$ are \emph{unit} complex numbers, so that $a\overline a=c\overline c=1$.  Taking $z$ to be the pole point $\frac{2bd}{b+d}$ of $\ell_2$,
$$\text{(iii)}\iff\frac{\frac{2bd}{b+d}-a}{c-a}\in\mathbb R\iff\frac{\frac{2bd}{b+d}-a}{c-a}=\frac{c-ac\overline{\left[\frac{2bd}{b+d}\right]}}{c-a}\iff$$
$$\frac{2bd}{b+d}-a=c-ac\overline{\left[\frac{2bd}{b+d}\right]}=c-ac\frac{2\overline b\overline d}{\overline b+\overline d}=c-ac\frac{bd(2\overline b\overline d)}{bd(\overline b+\overline d)}=c-ac\frac{2}{b+d}$$
where we have again used the fact that $b,d$ are unit complex numbers.  Multiplying by $b+d$ shows that last statement to be equivalent to
$$2bd-ab-ad=cb+cd-2ac,$$
and one readily verifies (by using the definition $[\![a,b,c,d]\!]=\frac{(c-a)(d-b)}{(d-a)(c-b)}$), that said statement is equivalent to $[\![a,b,c,d]\!]=2$, or (ii).  Thus, the proof of (ii) $\iff$ (iii) is complete.
\end{proof}

\noindent Given Proposition 4.21, we can readily establish formulas for the isometries in the Beltrami-Klein model.

We do this by first establishing a geometric construction for the reflection of the point $P$ over the line $\ell$, shown below.
\begin{center}
\includegraphics[scale=.3]{GeoReflect_BK.png}
\end{center}
The construction goes like this: Let $R$ be one ideal point of $\ell$.  Draw the line from $R$ that passes through $P$; let the other ideal point of this new line be $Q$.  Now draw the line through $Q$ and the pole point of $\ell$, and denote its other ideal point by $Q'$.  Draw the line from the $R$ to $Q'$.  Finally, draw the line through $P$ and the pole point of $\ell$, and let $P'$ be its intersection with $\overset{\longleftrightarrow}{RQ'}$.

Since $\overset{\longleftrightarrow}{PP'}$ and $\overset{\longleftrightarrow}{QQ'}$ contain the pole point of $\ell$, Proposition 4.21 shows that they are perpendicular to $\ell$ as lines in the Beltrami-Klein model of the hyperbolic plane.  Moreover, Exercise 18 of Section 4.3 shows that $\ell$ is the angle bisector of $\angle QRQ'=\angle PRP'$, and hence, (by transfering to the Poincar\'e disk model and assuming $\ell$ is the $x$-axis), we conclude that $P'$ is the reflection of $P$ over $\ell$.

Thus, the construction described in the paragraph following the above picture takes the point $P$ and reflects it over the line $\ell$ to get $P'$.  Yet, observe that it solely uses the intrinsic projective geometry, based off of the point $P$, the line $\ell$, and the conic which is the circle bounding the hyperbolic plane.  Thus, the entire construction commutes with projective transformations.  Yet by Exercise 2, there is a projective transformation $T$ sending $\ell$ to the $x$-axis, and applying $T$ to the constructions will conjugate the correspondence $P\mapsto P'$ by $T$.  The resulting correspondence reflects points over the $x$-axis in the Euclidean sense (why?), hence is a projective transformation.  Undoing the conjugation by $T$, we conclude that the original correspondence $P\mapsto P'$ was a projective transformation itself.

We have just shown that the reflection over any line, in the Beltrami-Klein model, is a projective transformation of $P^2(\mathbb R)$.  Thus we easily conclude\\

\noindent\textbf{Proposition 4.22.} \emph{In the Beltrami-Klein model of $H^2(\mathbb R)$, every isometry is a projective transformation of $P^2(\mathbb R)$ which fixes the unit circle.}\begin{proof}
We have proven this for reflections; now use Exercise 7(a) of Section 4.3.
\end{proof}

\noindent In fact, we also have the converse of Proposition 4.22; every projective transformation which fixes the unit circle is an isometry:\\

\noindent\textbf{Proposition 4.23.} \emph{A map from the Beltrami-Klein model to itself is an isometry if and only if it is a projective transformation of $P^2(\mathbb R)$ which fixes the unit circle.}\\

\noindent This is, in fact, why this model of hyperbolic geometry is sometimes called the ``projective model.''
\begin{proof}
Let $H$ be the group of isometries, and $G$ be the group projective transformations fixing the unit circle.  By Proposition 4.22, $H\subset G$, and hence $H$ is a subgroup of $G$.  We shall use Proposition 1.9 to show that $H=G$.

Let $T\in G$.  Then $T$ fixes all interior points of the circle (see Section 3.5), so that $T(\vec 0)$ is a regular point of the hyperbolic plane.  Exercise 1(a) of Section 4.3 further implies that there is an isometry $U\in H$ sending $T(\vec 0)\mapsto\vec 0$.  Then $U\circ T\in G$ and $(U\circ T)(\vec 0)=\vec 0$.  In view of Proposition 1.9, it suffices to show that $U\circ T$ is in $H$.  In other words, in showing $T\in G\implies T\in H$, we may assume $T(\vec 0)=\vec 0$.

Moreover, $T((1,0))$ is some point of the unit circle, say $(\cos\theta,\sin\theta)$.  If $R$ is a clockwise rotation by $\theta$ around the origin, then $R$ is also a rotation around the origin in the Euclidean sense (why?), and $R((\cos\theta,\sin\theta))=(1,0)$.  We further use Proposition 1.9 to reduce the proof to showing $R\circ T\in H$.  In other words, we may assume $T((1,0))=(1,0)$ as well.

Since $T$ fixes both $\vec 0=(0,0)$ and $(1,0)$, it fixes the line determined by them, i.e., the $x$-axis.  Consequently, since $T$ fixes the unit circle, $T$ must fix the points where the $x$-axis meets the circle; hence, $T$ fixes $(-1,0)$.  Using the intrinsic nature of the construction of a pole point, it is clear that projective transformations preserve pole points of lines via conics; therefore, since $T$ fixes the $x$-axis, it also fixes $[0:1:0]\in P^2(\mathbb R)$, which is the pole point of the $x$-axis via the unit circle.  Consequently, $T$ fixes the $y$-axis, because it is the line determined by $\vec 0$ and $[0:1:0]$ (and both points are fixed by $T$).

Furthermore, $T$ fixes intersection of the $y$-axis with the unit circle, which is $\{(0,1),(0,-1)\}$: but $T$ may either fix these points or swap them.  If $T$ fixes each of the points $(0,1),(0,-1)$, then $T$ must be the identity: otherwise, by consideration of the eigenspaces of the linear map $\mathbb R^3\to\mathbb R^3$ used to define $T$, the set of fixed points of $T$ would be contained in the union of a point and a line; this does not hold water if the five points $(0,0),(\pm 1,0),(0,\pm 1)$ are fixed.  If $T$ swaps $(0,1)$ and $(0,-1)$, then $T$ must be the (Euclidean) reflection over the $x$-axis, because composing $T$ with this reflection yields a projective transformation fixing $(0,0)$, $(\pm 1,0)$ and $(0,\pm 1)$ and so the previous argument applies to the new transformation.  In either case, $T$ is a hyperbolic isometry; i.e., $T\in H$.
\end{proof}

\noindent Along with the model not being conformal, we also claim that its circles are \emph{not} generally Euclidean circles.  We will now see what they really are.

If a circle is centered at the origin, it is clearly a Euclidean circle (for essentially the same reason it is in the Poincar\'e disk model).  Yet, any point $p$ can be sent to the origin through an isometry; such an isometry is a projective transformation by Proposition 4.22 (or 4.23), and hence preserves conics, as seen in Section 3.5.  Since this projective transformation sends a circle $\omega$ centered at $p$ to an origin-centered circle, the image of $\omega$ is a conic in $P^2(\mathbb R)$ (an origin-centered Euclidean circle), hence so is $\omega$ itself.  As $\omega$ is contained in the bounded unit disk, it clearly must be an ellipse (parabolas and hyperbolas do not fit inside the disk).

We can determine precisely what kind of ellipse it is.  First, the ellipse must be symmetric around the line through its center and the origin of the disk, because the (Euclidean) reflection over the line is a hyperbolic isometry, hence fixes any circle centered at one of its fixed points.  In fact, this line gives the minor axis of the ellipse.  This can be seen by starting in the Poincar\'e disk model, and drawing three lines tangent to the circle: two of them go through the origin, and the third embraces between the circle and the origin.  When passing over to the Beltrami-Klein model, the circle is still tangent to these lines, yet the first two lines remain the same and the third one becomes the Euclidean chord between the same ideal points:\footnote{Neither circle is tangent to the rim; if the one on the right appears to be, it is a graphical illusion because of how close they are to the Euclidean eye.}
\begin{center}
\includegraphics[scale=.3]{CircleAnalysis_PD.png}~~~~
\includegraphics[scale=.3]{CircleAnalysis_BK.png}\\
Poincar\'e Disk Model~~~~~~~~~~~~~~~~~~~~~Beltrami-Klein Model
\end{center}
This brushes the circle toward the rim, while the tangency to the first two lines implies that it gets wider.  Thus, the resulting ellipse is thinner in the direction of the diameter, and wider in the perpendicular dimension.  In other words, circles in the Beltrami-Klein model are ellipses which are ``flat'' towards the rim, and flatter when they are closer.  We will not give precise equations of these ellipses now, as the next section will introduce a few other models of the hyperbolic plane, along with their conversions; and from there, one can get a clearer picture.
\begin{center}
\includegraphics[scale=.3]{Circles_BK.png}\\Circles in the Beltrami-Klein model.
\end{center}

At this point it would be interesting to take some of the tilings from Section 4.5, and display them in the Beltrami-Klein model of the hyperbolic plane.
Here is the heptagonal tiling from Section 4.5, displayed side-by-side in both the Poincar\'e disk and Beltrami-Klein models.  Note that in the Klein model, most of the polygons near the rim look closer to flat ellipses; this is because that's what circles are in the Beltrami-Klein model.  Also, the tiling may appear to be on top of a hemisphere in birds-eye view: in the next section, we will see why this is so.
\begin{center}
\includegraphics[scale=.3]{HeptagonalTiling.png}~~~~
\includegraphics[scale=.3]{HeptagonalTiling_BK.png}\\
Poincar\'e Disk Model~~~~~~~~~~~~~~~~~~~~~Beltrami-Klein Model
\end{center}
Here are two other tilings from Section 4.5, the triheptagonal tiling and the truncated order-$7$ triangular tiling, each in the Beltrami-Klein model.
\begin{center}
\includegraphics[scale=.3]{TriheptagonalTiling_BK.png}~~~~
\includegraphics[scale=.3]{HyperrogueTiling_BK.png}\\
\end{center}

\noindent\textbf{HIGHER DIMENSIONS}\\

\noindent We conclude this section by discussing the Beltrami-Klein model of hyperbolic $n$-space.  It uses the unit $n$-ball $\{\vec v\in\mathbb R^n:\|\vec v\|\leqslant 1\}$, along with essentially the same conversion map from the Poincar\'e ball model, $\chi(\vec v)=\frac{2\vec v}{1+\|\vec v\|^2}$, but this time $\vec v$ is an $n$-dimensional vector.  By imitating the proofs in this section for the $2$-dimensional plane, the following can be verified; it is worth it for the reader to work these out.
\begin{itemize}
\item $\chi(\vec v)=\vec v$ for all $\|\vec v\|=1$.

\item $\chi$ sends $k$-planes ($0\leqslant k<n$) in the Poincar\'e ball model to the intersections of Euclidean $k$-planes with the unit ball (the generalization of chords).

\item Intersecting lines are perpendicular if and only if they contain each other's pole points in the $2$-dimensional cross section containing them.  [For this, the proof of (ii) $\iff$ (iii) in Proposition 4.21 need not be repeated: you just need basic manipulation of cross ratios of ideal points.]

\item A line and a hyperplane are perpendicular if and only if the line contains the pole point of the hyperplane.  [The \textbf{pole point} of the hyperplane $\Pi$ is obtained by taking the tangent hyperplanes to the hypersphere at all points where it meets $\Pi$, and then seeing where the hyperplanes meet.  The fact that they all meet in a common point can be visualized for $3$ dimensions, and then the intuition entails an easy proof for an arbitrary number of dimensions.]

\item A reflection across a hyperplane in this model is a projective transformation.  [This can be established through the same geometric construction given immediately after Proposition 4.21.]  Therefore, by Exercise 7(b) of the previous section, every isometry is a projective transformation.  Conversely, every projective transformation fixing the unit hypersphere is an isometry.

\item Hyperspheres in this model are hyper-ellipsoids, which possess all Euclidean symmetries that pointwise fix the line through the center and the origin.  [In particular, spheres in the Beltrami-Klein model of $3$-space are ellipsoids of revolution.]  They are ``flat'' towards the rim and flatter when they are closer.
\end{itemize}
\subsection*{Exercises 4.8. (Beltrami-Klein Model)} % Introduce the Beltrami-Klein model of the hyperbolic plane, by first noting that in the Poincar\'e disk, lines aren't Euclidean straight lines.
% Show the thing about perpendicular lines (and pole points); and reflections; use this to prove isometries are projective transformations.  Also say a thing about higher dimensions.
% POTENTIAL: Add the BK tilings that have been made in Turtle Graphics.
\begin{enumerate}
\item If $a$ and $b$ are points in the Beltrami-Klein model of the hyperbolic plane, construct the (Euclidean) chord through these points.  Let it meet the circle at points $p$ and $q$, with $p$ next to $a$.
\begin{center}\includegraphics[scale=.3]{HDistance_BK.png}\end{center}
Show that the distance between $a$ and $b$ is $\frac 12\ln[\![b,a,p,q]\!]$, where we are referring to the cross ratio of collinear points in $P^2(\mathbb R)$, defined in Section 3.3 right before Proposition 3.6.  [Explain why we may assume the chord is the $x$-axis and $a=0$.  Then use the formula for $\chi$ in Proposition 4.20.]

\item If $\omega$ is a circle in $P^2(\mathbb R)$, and $G$ is the group of projective transformations that fix $\omega$, then $G$ acts transitively on the chords of $\omega$.  [See Exercise 7 of Section 3.5; this is similar.] % By "nondegenerate" I meant the chord goes between two distinct points.  If a line segment from a point on the circle to itself counted as a chord, the action would not be transitive.

\item Use Proposition 4.21 to give an alternate proof of the Perpendicular Postulate (Proposition 4.6).  [Recall that in $P^2(\mathbb R)$, two points determine a line.]

\item Use Proposition 4.21 to give an alternate proof of the Ultraparallel Theorem (Theorem 4.7).

\item Let $\angle BAC$ be an angle in the Beltrami-Klein model.

(a) Explain why the angle bisector of $\angle BAC$ may not coincide with the Euclidean angle bisector.

(b) Let $B'$ and $C'$ be the respective ideal points of the rays $\overset{\longrightarrow}{AB}$ and $\overset{\longrightarrow}{AC}$.  Then, take the lines that are tangent to the unit circle at $B'$ and $C'$, and let $D$ be their intersection.  Draw the line from $D$ to $A$.  Show that this line is the angle bisector of $\angle BAC$.  [Use Proposition 4.21 and Exercise 18 of Section 4.3.]

\item Let $-1<a<1$ be a fixed real number.  Show that in the Beltrami-Klein model of $n$-space, the reflection over the hyperplane $x_n=a$ is given by
$$(x_1,\dots,x_n)\mapsto\left(\frac{(a^2-1)x_1}{2ax_n-a^2-1},\dots,\frac{(a^2-1)x_{n-1}}{2ax_n-a^2-1},\frac{(a^2+1)x_n-2a}{2ax_n-a^2-1}\right).$$
[Let $T$ be the reflection, viewed as a projective transformation.  Explain why $T$ must fix the pole point of the hyperplane $x_n=a$.  In particular, if $\varphi$ is a linear map of $\mathbb R^{n+1}$ inducing $T$, then the subspace $x_n=ax_{n+1}$ of $\mathbb R^{n+1}$ (which corresponds to the hyperplane) and the line corresponding to the pole point are eigenspaces for $\varphi$.  Also note that $T^2$ is the identity.  Now compute $T$, using a change-of-basis matrix.]

\item Use the previous exercise to find a formula for the translation along the $x_1$-axis by a (hyperbolic) distance of $d>0$, where the origin goes toward the positive $x_1$-axis.

\item (a) Show that a hypercycle with base line $\ell$ is an elliptical arc which is tangent to the unit circle at the ideal points of $\ell$.  [Assume $\ell$ is the $x$-axis, then use Exercise 1.]

(b) Since all these hypercycles are tangent to the unit circle, they make a nonzero constant Euclidean angle with $\ell$.  Yet, as the radius of the hypercycle approaches zero, the hypercycle approaches the line $\ell$ itself, so this Euclidean angle becomes zero.  Comment on how this does not actually make the model inconsistent.  [Compare, in both the Poincar\'e disk and Beltrami-Klein models, the interaction of the hypercycle with another line meeting one of $\ell$'s ideal points.]

(c) Show that a horocycle is an ellipse which ``osculates'' the unit circle; in other words, the unit circle and horocycle can be parametrized by functions $\varphi(t),\psi(t)$ respectively, such that
$$\varphi(0)=\psi(0),~~~~\varphi'(0)=\psi'(0),~~~~\varphi''(0)=\psi''(0).$$
[Express the horocycle as a converging limit of hypercycles.]
\end{enumerate}

\subsection*{4.9. Hyperboloid and Hemisphere Models: Model Conversion}
\addcontentsline{toc}{section}{4.9. Hyperboloid and Hemisphere Models: Model Conversion}
We know three models of hyperbolic $n$-space which can be viewed in Euclidean $n$-space: the Poincar\'e ball (or disk) model, the Poincar\'e half-space (or half-plane) model, and the Beltrami-Klein model.  Here, we shall introduce two more models that \emph{cannot} be viewed in Euclidean $n$-space, and then see a clever geometric way to relate the models.

The first such model is the hyperboloid model, also known as the Lorentz / Minkowski model.  It takes place on the upper branch of the hyperboloid in $\mathbb R^{n+1}$, given by
$$x_{n+1}=\sqrt{x_1^2+\dots+x_n^2+1},$$
and has the following properties.  [Note that the equation $x_{n+1}^2=x_1^2+\dots+x_n^2+1$ is \emph{not} quite right, as it gives both branches of the hyperboloid, and the model of hyperbolic space only takes place on the upper branch.]
\begin{itemize}
\item Its ideal points are \emph{not} directly visible.  However, the hyperboloid branch can be viewed as a subset of $P^{n+1}(\mathbb R)$ rather than $\mathbb R^{n+1}$.  In this case, the (topological) closure of the branch consists of an $(n-1)$-sphere of points at infinity, given by $\{[x_1:\dots:x_n:x_{n+1}:0]:x_{n+1}^2=x_1^2+\dots+x_n^2\}$.  In this case, these points at infinity are the ideal points of the $n$-space.  This makes perfect sense because ideal points of hyperbolic space are infinitely far away in the hyperbolic sense anyway.

\item $k$-planes in this model are intersections of the hyperboloid branch with $(k+1)$-planes in $\mathbb R^{n+1}$ through the origin.  In particular, if $n=2$, then lines in the model are intersections of the branch $z=\sqrt{x^2+y^2+1}$ with planes through the origin in $\mathbb R^3$.

\item $k$-spheres appear tall and skinny to the Euclidean eye when they are high up on the branch.

\item The model is not conformal; angle measures between lines do not generally coincide with the Euclidean angle measures.
\end{itemize}
To illustrate this model, the heptagonal tiling and the truncated order-$7$ triangular tiling are displayed below, on the hyperboloid branch $z=\sqrt{x^2+y^2+1}$ in the case $n=2$.
\begin{center}
\includegraphics[scale=.2]{HeptagonalTiling_Hyper.png}~~~~
\includegraphics[scale=.2]{HyperrogueTiling_Hyper.png}
\end{center}

The other model that cannot be viewed in Euclidean $n$-space is the hemisphere model.  It takes place on the upper half of the unit $n$-sphere, $x_{n+1}=\sqrt{1-x_1^2-\dots-x_n^2}$.  Significant properties of this model are:
\begin{itemize}
\item Its ideal points are on the equatorial $(n-1)$-sphere, given by $x_{n+1}=0$, $x_1^2+\dots+x_n^2=1$.

\item $k$-planes in this model are intersections of the hemisphere with $(k+1)$-planes in $\mathbb R^{n+1}$ parallel to the $x_{n+1}$ axis.  In particular, if $n=2$, then lines in the model are intersections of the hemisphere $z=\sqrt{1-x^2-y^2}$ with planes parallel to the $z$-axis; these are upright-standing semicircle arcs.

\item $k$-spheres are Euclidean $k$-spheres contained in the hemisphere, which do not touch the rim.

\item The model is conformal; every angle measure matches the Euclidean angle measure.
\end{itemize}
It is worth remarking that lines are \emph{not} geodesics in the hemisphere's own geometry.  If they were, then by the spherical nature of the hemisphere, the angles of a triangle would add to $>180^\circ$, contradicting Proposition 4.8.

To illustrate this model, the heptagonal tiling and the rhombitriheptagonal tiling are displayed below, on the hemisphere $z=\sqrt{1-x^2-y^2}$ in the case $n=2$.
\begin{center}
\includegraphics[scale=.3]{HeptagonalTiling_Hemis.png}~~~~
\includegraphics[scale=.3]{RhombitriheptagonalTiling_Hemis.png}
\end{center}
The hemisphere model is not usually used to construct things in hyperbolic geometry; however, it does provide a curious geometric link between the conversions.\\

\noindent Of course, the only way to identify these models with hyperbolic space as we know it is to construct the conversion maps.  Here's how it is done:

(1) Start with a point $(a_1,\dots,a_n,a_{n+1})$ in the hyperboloid model of hyperbolic $n$-space.  Thus, $a_{n+1}>0$ and $a_{n+1}^2=a_1^2+\dots+a_n^2+1$.

(2) Draw the line through that point and the origin, and see where it meets the hyperplane $x_{n+1}=1$.  This intersection point is $\left(\frac{a_1}{a_{n+1}},\dots,\frac{a_n}{a_{n+1}},1\right)$.  Observe that the sum of the squares of the first $n$ components is less than $1$, so that this point is in the unit $n$-ball of the hyperplane.  This is the corresponding point in the Beltrami-Klein model.

(3) Project this point vertically (parallel to the $x_{n+1}$-axis) to a point on the hemisphere $x_{n+1}=\sqrt{1-x_1^2-\dots-x_n^2}$.  In this case, the $x_{n+1}$-coordinate becomes
$$\sqrt{1-\left(\frac{a_1}{a_{n+1}}\right)^2-\dots-\left(\frac{a_n}{a_{n+1}}\right)^2}=\sqrt{\frac{a_{n+1}^2-a_1^2-\dots-a_n^2}{a_{n+1}^2}}=\sqrt{\frac 1{a_{n+1}^2}}=\frac 1{a_{n+1}},$$
and hence the resulting point is
\begin{equation}\tag{HS}\left(\frac{a_1}{a_{n+1}},\dots,\frac{a_n}{a_{n+1}},\frac 1{a_{n+1}}\right)\end{equation}
which is the corresponding point in the hemisphere model.

(4) Consider the point $S=(0,\dots,0,-1)$, which is the south pole of the $n$-sphere in which the hemisphere is contained.  [It is not actually in the hemisphere, as its last component is negative.]  Connect this point to the point (HS), and see where the resulting line meets the hyperplane $x_{n+1}=0$.  Basic algebra entails the resulting point is
$$\left(\frac{a_1}{a_{n+1}+1},\dots,\frac{a_n}{a_{n+1}+1},0\right).$$
If this point is regarded as a vector $\vec v$, then $\vec v\cdot\vec v=\frac{a_{n+1}-1}{a_{n+1}+1}$, and hence $\|\vec v\|<1$.  This means that the point is contained in the unit $n$-ball of the hyperplane $x_{n+1}=0$.  This is the corresponding point in the Poincar\'e ball model.

In other words, the Poincar\'e ball model is obtained by taking the stereographic projection from the hypersphere to the hyperplane $x_{n+1}=0$, via the point $S$.  It maps the hypersphere precisely to the unit $n$-ball.

It actually turns out that the line through $S$ and the point (HS) also meets the original point $(a_1,\dots,a_n,a_{n+1})$ on the hyperboloid.  The reader can readily check this.  This gives an alternative way to find the hemisphere and Poincar\'e ball points directly given the hyperboloid point.

(5) Finally, to get the Poincar\'e half-space model, we consider the point $P=(-1,0,\dots,0)$ which is on the rim of the hemisphere.  We stereographically project the hemisphere-model point $\left(\frac{a_1}{a_{n+1}},\dots,\frac{a_n}{a_{n+1}},\frac 1{a_{n+1}}\right)$ onto the hyperplane $x_1=0$ through this point.  The result is
$$\left(0,\frac{a_2}{a_1+a_{n+1}},\dots,\frac{a_n}{a_1+a_{n+1}},\frac 1{a_1+a_{n+1}}\right).$$
This is the only conversion which breaks free of the permutability of the points $x_1,\dots,x_n$, but it is rather important that it does.  Of course, one could get the half-space model by taking the stereographic projection from \emph{any} rim point of the hemisphere.\\

\noindent The reader can readily verify that all of the aforementioned conversions are consistent with what has been covered earlier in the chapter, along with the properties of the hyperboloid and hemisphere models that have been stated here.  For example, if $\vec v$ is a point in the Poincar\'e ball model, then $\frac{2\vec v}{1+\|\vec v\|^2}$ is the corresponding point in the Beltrami-Klein model.

The following diagram illustrates the conversions between the five models, by taking the cross section in the plane spanned by the $x_1$ and $x_{n+1}$ axes.  Note, however, the stereographic projection for the half-space model has \emph{not} been taken to the hyperplane $x_1=0$.  It has been taken to a different hyperplane off to the side, so that the reader can see all the models clearly.
\begin{center}
\includegraphics[scale=.5]{ModelConversion.png}
\end{center}
Note that these cross-sectional curves are lines in each of the hyperbolic-space models.  An interesting exercise is to parametrize one of them, and then convert the parametrization to the other models.  A typical parametrization of the hyperbola branch is $t\mapsto(\sinh t,\cosh t),t\in\mathbb R$.  Indeed, we have $\cosh t>0$ and $\cosh^2t=\sinh^2t+1$.

We leave it to the reader to verify the following facts, possibly using results from Section 4.5:
\begin{itemize}
\item The line on the hyperboloid model is parametrized via $t\mapsto(\sinh t,\cosh t)$.

\item The line on the Beltrami-Klein model [$y=1$] is parametrized via $t\mapsto(\tanh t,1)$.

\item The line on the hemisphere model is parametrized via $t\mapsto(\tanh t,\operatorname{sech}t)$.

\item The line on the Poincar\'e ball model is parametrized via $t\mapsto(\tanh(t/2),0)$.  [Hint: first show that $\tanh(t/2)=\frac{\sinh t}{1+\cosh t}$.]

\item The line on the half-space model is parametrized via $t\mapsto(0,e^{-t})$.
\end{itemize}
By examining these, one can see that $t$ measures arc length: the points corresponding to $t=a$ and $t=b$ have a hyperbolic distance of exactly $|a-b|$.

The exercises will cover two more models of hyperbolic $n$-space, which can be viewed in Euclidean $n$-space, but which are relatively rare.

\subsection*{Exercises 4.9. (Hyperboloid and Hemisphere Models: Model Conversion)} % Finally introduce these two models and explain how to convert between them.
% POTENTIAL EXERCISES: Band model, Gans model ( https://en.wikipedia.org/wiki/Hyperbolic_geometry#The_Gans_model )
% In the exercise re. the band model, mention the deep result that any (genuine) 2-dimensional region can be made into a conformal model of the hyperbolic plane.
\begin{enumerate}
\item (a) Show that there is a projective transformation $T:(x_1,\dots,x_n,x_{n+1})\mapsto\left(\frac{x_1}{x_{n+1}},\dots,\frac{x_n}{x_{n+1}},\frac 1{x_{n+1}}\right)$ of $P^{n+1}(\mathbb R)$, and that $T^2$ is the identity.

(b) Show that $T$ exchanges the hyperboloid branch and the hemisphere, and is in fact an isometry from the hyperboloid model of hyperbolic $n$-space to the hemisphere model.

(c) Use this to show that if $n=2$, then circles on the hyperboloid model are ellipses obtained by intersecting the branch with a (projective) plane.  Also, horocycles on the hyperboloid model are parabolas, and hypercycles on the hyperboloid model are hyperbolas obtained by intersecting the branch with planes that are not through the origin.

\item Let $T$ be a projective transformation of $P^{n+1}(\mathbb R)$ which fixes the upper hyperboloid branch.

(a) Show that $T$ fixes the set of points at infinity in the closure of the branch; i.e., the set $\{[x_1:\dots:x_n:x_{n+1}:0]:x_{n+1}^2=x_1^2+\dots+x_n^2\}$.  [It may help to think of the continuity of $T$.]

(b) Show that $T$ fixes the origin of $\mathbb R^{n+1}$.  [Explain why $T$ permutes the hyperplanes which are tangent to the hyperboloid at the infinity points in~(a).  Show that they all contain the origin.]

(c) Show that $T$ fixes the unit $n$-ball in the hyperplane $x_{n+1}=1$.

(d) Now show that $T$ is a hyperbolic isometry of the hyperboloid model, and conversely, every isometry is of this form.  [Because of the geometry behind the conversion between this model and the Beltrami-Klein model, it is clear that restrictions of $T$ to the hyperboloid and to the unit $n$-ball are conjugates under the conversion.  Now use Proposition 4.23, or more directly, its generalization to higher dimensions stated at the end of Section 4.8.]

(e) Show that a hyperbolic isometry of the hemisphere model is given by a projective transformation of $P^{n+1}(\mathbb R)$ which fixes the upper hemisphere.  [Exercise 1 may help.]

\item Explain why the argument of the previous exercise does \emph{not} imply that isometries of the Poincar\'e ball model are projective transformations.

\item\emph{(Band model.)} \---- The band model of the hyperbolic plane is a model which takes place on an infinite horizontal strip in $\mathbb R^2$.  Its regular points are of the form $(x,y)$ with $x,y\in\mathbb R$ and $|y|<\pi/2$.  Its rim consists of the points $(x,\pm\pi/2)$ for all $x\in\mathbb R$, along with \emph{two} symbolic infinity points, one on the left and one on the right.

The model can be obtained as follows: let $z$ be a point in the Poincar\'e half-plane model.  View $z$ as a complex number, such that $\operatorname{Im}(z)\geqslant 0$.  Rotate $z$ clockwise by $90$ degrees to get $-iz$; this point is in the right half of the complex plane where the real part is nonnegative.  Moreover, the argument of $-iz$ is a real number between $-\pi/2$ and $\pi/2$, and is the imaginary part of $\ln(-iz)$.  The point $\ln(-iz)$ in the complex plane is then the corresponding point of the band model.  The points corresponding to $z=0$ and $z=\infty$ are the symbolic infinity points of the band model; they will be referred to as $-\infty$ and $+\infty$ respectively.

Note that this model is conformal (because the conversion map $z\mapsto\ln(-iz)$ is holomorphic, and hence conformal).  Thus angle measures coincide with the Euclidean angle measures, and all lines meet the rim orthogonally.  Each line is either the $x$-axis (which connects the symbolic infinity points), or looks like one of these:
\begin{center}
\includegraphics[scale=.2]{BandModelLines.png}
\end{center}
(a) Show that every horizontal translation of the band by a real number is an isometry.  [Conjugate it via the conversion and see what map of the half-plane is entailed.]

(b) Show that for each $a\in\mathbb R$, the line from $(a,\pi/2)$ to $(a,-\pi/2)$ is simply the Euclidean line segment between these points.

(c) The line from $(0,\pi/2)$ to $+\infty$ is given by the equation $x=\ln\csc y$ ($y>0$).  [Do the construction in the half-plane model and then apply the conversion map.]

(d) The line from $(a,\pi/2)$ to $(-a,-\pi/2)$ is given by the equation $y=\sin^{-1}\frac{\sinh x}{\sinh a}$, with $x\in[-a,a]$.

(e) The line from $(a,\pi/2)$ to $(-a,\pi/2)$ is given by the equation $y=\sin^{-1}\frac{\cosh x}{\cosh a}$ with $x\in[-a,a]$.

(f) Explain how parts (a)-(e) can be used to derive the line between any two ideal points.

(g) If $|y_1|,|y_2|<\pi/2$, show that the hyperbolic distance between $(x_1,y_1)$ and $(x_2,y_2)$ is
$$\cosh^{-1}\left(\frac{e^{x_1-x_2}+e^{x_2-x_1}-2\sin y_1\sin y_2}{2\cos y_1\cos y_2} \right ).$$
[Pass them through the conversion, and then use Exercise 9 of Section 4.3.]

(h) Find equations for hypercycles, horocycles and circles.  [It would help to think of the preimage of any generalized circle under the exponential map $z\mapsto e^z$ from $\mathbb C\to\mathbb C$.]

Here is a sample of the triheptagonal tiling in the band model:
\begin{center}
\includegraphics[scale=.25]{TriheptagonalTiling_Band.png}
\end{center}
At this point it is worth mentioning a rather deep fact about conformal models of the hyperbolic plane.  You can take \emph{any} open region of $\mathbb R^2$ which is homeomorphic to the disk (such as an ellipse, a square, or even the shape of a pet labrador), and there will be a conformal model of the hyperbolic plane in that region.  This is due to the Riemann mapping theorem, which states that there is a conformal mapping from the open disk to any such region.  Such a conformal mapping can then be used as a conversion map between models.

\item\emph{(Gans model.)} \---- The Gans model is a nonconformal model of hyperbolic $n$-space, which uses the entire Euclidean space $\mathbb R^n$.  It was proposed in 1966 by David Gans, in the \emph{American Mathematical Monthly}.  This space does not contain the ideal points; however, they may be adjoined symbolically if one desires.

This model can easily be converted from the hemisphere model as follows: let $(x_1,x_2,\dots,x_{n+1})$ be a regular point in the hemisphere model; thus, $x_{n+1}>0$ and $x_1^2+\dots+x_n^2+x_{n+1}^2=1$.  Draw the line through the origin and this point and see where it meets the hyperplane $x_{n+1}=1$; this point of intersection is $\left(\frac{x_1}{x_{n+1}},\dots,\frac{x_n}{x_{n+1}},1\right)$, and (as a point of the hyperplane $x_{n+1}=1$), is the corresponding point of the Gans model.  [Note that this is not stereographic projection because the origin is not on the $n$-sphere.]

Alternatively, the model can be converted from the hyperboloid model through orthographic projection, i.e., $(x_1,\dots,x_n,x_{n+1})\mapsto(x_1,\dots,x_n,1)$ with the former point on the hyperboloid branch.  The results of this chapter readily show that this gives the same model.

In parts (a)-(b), assume $n=2$; in this case a point $(x,y,z)$ on the hemisphere corresponds to the point $(x/z,y/z)\in\mathbb R^2$.

(a) Every line is either a Euclidean line through the origin, or a branch of a hyperbola whose center is the origin.  In particular, lines that do not go through the origin are concave away from it.  [Consider the cone centered at the origin of $\mathbb R^3$, going through a line in the hemisphere model.]

(b) Hypercycles are Euclidean lines not through the origin and branches of hyperbolas not centered at the origin; horocycles are parabolas; and circles are (Euclidean) ellipses.

(c) A circle is tall in the direction of the Euclidean line through the origin and its center, and narrow in the perpendicular direction.

(d) More generally, in the Gans model of hyperbolic $n$-space, a $k$-plane is a Euclidean $k$-plane through the origin or a branch of a $k$-dimensional hyperboloid whose center is the origin.  Classify what a $k$-sphere is as well.

Here is a sample of the rhombitriheptagonal tiling in the Gans model of the hyperbolic plane:
\begin{center}
\includegraphics[scale=.25]{RhombitriheptagonalTiling_Gans.png}
\end{center}
\end{enumerate}

\end{document}