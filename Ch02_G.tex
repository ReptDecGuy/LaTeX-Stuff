\documentclass[leqno]{book}
\usepackage[small,nohug,heads=vee]{diagrams}
\diagramstyle[labelstyle=\scriptstyle]
\usepackage{amsmath}
\usepackage{amssymb}
\usepackage{amsthm}
\usepackage[pdftex]{graphicx}
\usepackage{mathrsfs}
\usepackage{mathabx}
\usepackage{enumitem}
\usepackage{multicol}
\usepackage[utf8]{inputenc}

\makeatletter
\newcommand*\bcd{\mathpalette\bcd@{.5}}
\newcommand*\bcd@[2]{\mathbin{\vcenter{\hbox{\scalebox{#2}{$\m@th#1\bullet$}}}}}
\makeatother

\begin{document}

\chapter{Euclidean Geometry}

Euclidean geometry is the first kind of geometry to have been discovered.  Geometric facts were discovered by people throughout the world over thousands of years.  The Alexandrian Greek mathematician Euclid collected some of these facts into an influential series of books around 300 B.C.  Euclid's method consisted of taking some axioms and using them to prove statements.  We will present the axioms in Section 2.1.  Much later, however, several other kinds of geometry were discovered.  Like Euclid's, they are logically self-consistent.  We will study all these geometries by using algebra, rather than merely taking axioms for them.  Thus from the second section of this chapter onwards, we will deal with Euclidean space algebraically.  It will be clear how this geometry connects with the axiomatic one described at first.

\subsection*{2.1. Axiomatic Plane Geometry}
\addcontentsline{toc}{section}{2.1. Axiomatic Plane Geometry}
When mathematicians mention the ``plane,'' we naturally think of the coordinate plane: two axes, $x$ and $y$, and a swarm of points around them, $\mathbb R^2$.  However, a true geometer does not actually use the concrete plane; in fact, Euclid used \emph{no} axes, no $x$ and $y$, and not even a fixed unit of length.  An axiomatic system is used.

Imagine picking up some whipped cream from the ice cream parlor, spreading it out on a table, and enforcing certain portions of the whipped cream to be ``lines.''  Suppose you are also the boss of the whipped cream and are giving the lines some rules to follow; for example, no two distinct lines can intersect more than once.  With enough rules, you are in the setting of axiomatic plane geometry, even though you did not actually start with a plane.

Thus we imagine a set where elements are called ``points,'' and certain subsets are called ``lines.''  Unlike usual mathematics, we shall denote points by the \emph{capital} letters $A$, $B$, $C$, etc.; this is a standard in plane geometry.  We shall list axioms and use them to prove basic propositions.

The following axioms revolve around lines, line segments and rays.  There are many ways to give axioms for Euclidean geometry.  In the late 1800s, several mathematicians proposed axioms that were more complete than Euclid's (such as Hilbert).  One system will be given here.\\

\noindent\textbf{Axiom 2.1.}

(i) \emph{Two points determine a line; i.e., given any two points $A\ne B$ there is a unique line passing through both $A$ and $B$.  We refer to this line as $\overset{\longleftrightarrow}{AB}$.}

(ii) \emph{If $A,B,C$ are points on a common line, either $C$ is \textbf{between} $A$ and $B$ or it isn't.  The set of points between $A$ and $B$ is denoted $\overline{AB}$.  Moreover, (a) $A,B\in\overline{AB}\subset\overset{\longleftrightarrow}{AB}$; (b) for any points $C$ and $D$ in $\overline{AB}$ we have $\overline{CD}\subset\overline{AB}$; (c) whenever $A,B,C$ are distinct points on the same line, exactly one of them is between the other two; (d) if $A$ and $B$ are in a line $\ell$ then there exists a point $C\in\ell$ such that $C\ne A$ and $A$ is between $C$ and $B$; (e) $\overline{AA}=\{A\}$; and (f) when $B$ is between $A$ and $C$, $\overline{AC}=\overline{AB}\cup\overline{BC}$.}

\emph{If $A\ne B$, then $\overline{AB}$ is called the \textbf{line segment} from $A$ to $B$.  Note that each line segment is contained in a unique line; by (i), there is a unique line containing $A$ and $B$, and by (ii) (a), it contains the whole segment.}

(iii) \emph{Two points $A,B$ also determine a \textbf{ray}, the set of points $C\in\overset{\longleftrightarrow}{AB}$ such that either $C$ is between $A$ and $B$, or else $B$ is between $A$ and $C$.  This ray is denoted $\overset{\longrightarrow}{AB}$.}\\

\noindent The three concepts are intuitively illustrated in the diagram below.  (The arrows indicate indefinite continuation of the stroke.)
\begin{center}\includegraphics[scale=.6]{LineSegmentRay.png}\end{center} % I guess I'll make it bigger, these revisions will wreck the paging anyway...
Note that if $A$ and $B$ are points, then $\overset{\longleftrightarrow}{AB}=\overset{\longleftrightarrow}{BA}$ and $\overline{AB}=\overline{BA}$; however, $\overset{\longrightarrow}{AB}\ne\overset{\longrightarrow}{BA}$.  After all, by (d) of Axiom 2.1(ii) there exists $C$ such that $B$ is strictly in between $A$ and $C$, and then $C\in\overset{\longrightarrow}{AB}$ but $C\notin\overset{\longrightarrow}{BA}$.

There are several important consequences of (ii).  First, if $A,B,C$ are points such that $A$ is between $B$ and $C$, and $B$ is between $A$ and $C$, then $A=B$.  Indeed, $A,B,C$ cannot all be distinct (if they were, then by part (c) of (ii), no more than one of $A,B,C$ would be between the other two), hence either $A=B$, $B=C$ or $A=C$.  If $B=C$, then $A$ would be between $C$ and $C$, hence $A\in\overline{CC}=\{C\}$ (by (e)) and hence $A=C=B$.  Similarly, $A=C$ implies $A=B$; thus we have $A=B$ in all three cases.

It further follows that if $A,B,C$ are points on a line with $B$ between $A$ and $C$, $\overline{AB}\cap\overline{BC}$ consists of only the point $B$.  After all if $D\in\overline{AB}\cap\overline{BC}$, then $D$ is between $A$ and $B$, but also $D$ is between $B$ and $C$.  Since $D\in\overline{AB}\subset\overline{AC}$ (by part (b) of (ii)), $D$ is between $A$ and $C$.  Moreover, $B\in\overline{AC}=\overline{AD}\cup\overline{DC}$ (by (f)), so either $B\in\overline{AD}$ or $B\in\overline{DC}$.  If $B\in\overline{AD}$, then $B$ is between $A$ and $D$, and $D$ is also between $A$ and $B$; therefore $B=D$ by the argument of the preceding paragraph.  Similarly if $B$ is between $D$ and $C$ then $B=D$.  Thus $D=B$ follows from the assumption that $D\in\overline{AB}\cap\overline{BC}$, which means that set consists of only $B$.

Finally, the points $A,B$ can be recovered from the line segment $\overline{AB}$.  After all, suppose $\overline{AB}=\overline{A'B'}$; then it is clear that $A,B,A',B'$ all lie on the same straight line.  Since $B'$ is between $A$ and $B$, we have by (f) of Axiom 2.1(ii) that $B'\in\overline{AA'}\cup\overline{A'B}$.  If $B'\in\overline{AA'}$ then $B$ is between $A$ and $A'$, but also $A'$ is between $A$ and $B$; therefore we have $A'=B$.  Moreover, $\overline{AB}=\overline{BB'}$; this jointly implies that $A$ is between $B$ and $B'$ and that $B'$ is between $B$ and $A$, and therefore $A=B'$.  Hence in this case, the (unordered) set $\{A,B\}$ of points used to define $\overline{AB}$ coincides with $\{A',B'\}$.  Similarly, one can show that $B'\in\overline{A'B}$ implies that $B=B'$ and $A=A'$.  Thus we have proven\\

\noindent\textbf{Proposition 2.2 and Definition.} \emph{If $A,B$ are points, the set $\{A,B\}$ is determined by the line segment $\overline{AB}$.  These points $A$ and $B$ are called the \textbf{endpoints} of $\overline{AB}$.  If $A=B$ (which implies $\overline{AB}=\{A\}$ by (e)), the segment is said to be \textbf{degenerate}; otherwise it is called \textbf{nondegenerate}.}\footnote{Unless otherwise specified we shall always assume line segments to be nondegenerate.}\\

\noindent Now we prove basic facts about rays: first we claim that if $B$ is between $A$ and $C$, $\overset{\longrightarrow}{AB}=\overset{\longrightarrow}{AC}$.  For suppose $D\in\overset{\longrightarrow}{AB}$.  Then either $D$ is between $A$ and $B$ (in which case $D\in\overline{AB}\subset\overline{AC}\subset\overset{\longrightarrow}{AC}$), or else $B$ is between $A$ and $D$ (in this case, if $A$ is between $C$ and $D$ then $B\in\overline{AD}\cap\overline{AC}=\{A\}$ a contradiction; hence either $C$ is between $A,D$ or $D$ is between $A,C$, implying $D\in\overset{\longrightarrow}{AC}$).  This proves that $\overset{\longrightarrow}{AB}\subset\overset{\longrightarrow}{AC}$.  Conversely, if $E\in\overset{\longrightarrow}{AC}$, then either $C$ is between $A$ and $E$ (in which case $B\in\overline{AC}\subset\overline{AE}$ and $B$ is between $A$ and $E$), or else $E$ is between $A$ and $C$ (in this case $E\in\overline{AB}\cup\overline{BC}\subset\overset{\longrightarrow}{AB}\cup\overline{BC}$ \---- and $E\in\overline{BC}$ implies $B\in\overline{AC}=\overline{AE}\cup\overline{EC}$ and hence $B\in\overline{AE}$ because $B\in\overline{EC}$ implies $B=E$); therefore, $E\in\overset{\longrightarrow}{AB}$.

Now if $\overset{\longrightarrow}{AB}=\overset{\longrightarrow}{A'B'}$ (with $A\ne B,A'\ne B'$), then we have $A=A'$, but (by the preceding paragraph) it does \emph{not} necessarily follow that $B=B'$.  For, suppose on the contrary that $A\ne A'$.  Then since $A'\in\overset{\longrightarrow}{AB}$, then the preceding paragraph entails $\overset{\longrightarrow}{AB}=\overset{\longrightarrow}{AA'}$ (because either $A'$ is between $A,B$ or $B$ is between $A,A'$, and the claim can be applied to either case).  Similarly, since $A\in\overset{\longrightarrow}{A'B'}$ we have $\overset{\longrightarrow}{A'B'}=\overset{\longrightarrow}{A'A}$.  Therefore $\overset{\longrightarrow}{AA'}=\overset{\longrightarrow}{A'A}$, which contradicts what we have established in the first paragraph after the statement of Axiom 2.1.  Hence\\

\noindent\textbf{Proposition 2.3 and Definition.} \emph{If $A\ne B$ are points, then the point $A$ is completely determined by the ray $\overset{\longrightarrow}{AB}$.  $A$ is called the \textbf{endpoint} of $\overset{\longrightarrow}{AB}$.}\\

\noindent Note that a ray has only one endpoint, even though a line segment has two (and a line has none).\\

\noindent\textbf{LENGTH OF LINE SEGMENTS}\\

\noindent For any line segment $\overline{AB}$ we shall associate a real number, which we denote $AB$ and call the \textbf{length} of $AB$.  We also call this real number the \textbf{distance} between the points $A$ and $B$.  The following axioms will be posed for this notion.\\

\noindent\textbf{Axiom 2.4.} (i) \emph{If $\rho$ is a ray with endpoint $A$, and $r$ is a positive real number, there exists a point $B$ on $\rho$ such that $AB=r$.}

(ii) \textsc{(Segment Addition Postulate)} \emph{If $B$ is between $A$ and $C$, then $AC=AB+BC$.}

(iii) \emph{If $A\ne B$ then $AB>0$.}\\

The first part guarantees that all real numbers are possible distances.  The second is a well-known axiom in plane geometry: your intuition most likely caught it at once, even though it does not logically follow from anything else we have stated.  The third does the basic job of saying that two distinct points have positive distance from one another.

Note that the point $B$ in part (i) is unique: if $B$ and $B'$ are both points on $\rho$ such that $AB=AB'=r$, then (since $B'\in\overset{\longrightarrow}{AB}$), either $B'$ is between $A$ and $B$, or else $B$ is between $A$ and $B'$.  In the former case, part (ii) entails $AB=AB'+B'B$; i.e., $r=r+B'B$, which implies $B'B=0$.  By part (iii) we conclude $B'=B$.  The same argument works if $B$ is between $A$ and $B'$.

Also, applying part (ii) with $A=B=C$ entails that $AA=0$ for any point $A$.  Combining this with (iii), we get that \emph{a line segment is degenerate if and only if its length is zero}.

We say that two line segments $\overline{AB}$ and $\overline{CD}$ are \textbf{congruent} (denoted $\overline{AB}\cong\overline{CD}$) if $AB=CD$.  This is manifestly an equivalence relation on the line segments.  Note, however, that lengths of rays and lines are \emph{not} defined (why?).

If $\overline{AB}$ is any (nondegenerate) line segment, let $r=\frac 12(AB)$.  Then by part (i), there is a point $M$ on $\overset{\longrightarrow}{AB}$ such that $AM=r$.  Furthermore, if $B$ were between $A$ and $M$, then we would have
$$r=AM=AB+BM\geqslant AB=2r$$
which is a contradiction because $r>0$.  Therefore, $M$ must be between $A$ and $B$ (by definition of the ray).  Furthermore, since $AB=2r$, the Segment Addition Postulate immediately entails that $MB=r$ as well.  Since $AM=MB=r$, we have $\overline{AM}\cong\overline{MB}$.  The reader can readily verify that $M$ is the \emph{unique} point on $\overline{AB}$ satisfying this condition.  This point is called the \textbf{midpoint} of the line segment $\overline{AB}$.\\

\noindent\textbf{PARALLEL POSTULATE}\\

\noindent Let $\ell_1$ and $\ell_2$ be two distinct lines.  Observe that Axiom 2.1(i) alone implies that $\ell_1$ and $\ell_2$ intersect in at \emph{most} one point: for if $A,B$ were two different points in $\ell_1\cap\ell_2$, then there would be multiple lines (such as $\ell_1,\ell_2$) passing through both $A$ and $B$, contradicting Axiom 2.1(i).  However, the lines may not intersect at all.

Lines $\ell_1$ and $\ell_2$ are said to be \textbf{parallel}, denoted $\ell_1\parallel\ell_2$, if they do not intersect.  Intuitively this implies that they go in exactly the same direction (due to their infinitude on both sides, if they went in even slightly different directions, they would eventually approach on one side).  The concept is illustrated below.
\begin{center}\includegraphics[scale=.4]{ParallelIntersecting.png}\end{center}
Thus, we find it believable that given any point not on a line $\ell$, there is exactly one line through the point that is parallel to $\ell$.\\

\noindent\textbf{Axiom 2.5.} \textsc{(Parallel Postulate)} \emph{Given a line $\ell$ and a point $A\notin\ell$, there exists a unique line through $A$ parallel to $\ell$.}\\

\noindent There is an interesting story regarding this axiom, where many mathematicians tried to prove it from the other axioms.  However, they failed to find any way to prove the parallel postulate without using any proposition that already depends on the parallel postulate.  After many years, their attempts to prove this statement led to the discovery of other types of geometry where this statement is not true.  Examples are projective, hyperbolic and spherical geometry, which will be covered in Chapters 3, 4 and 5 respectively. % I'm especially surprised this "revision" is in red: I gave examples, didn't need to give *all* of them.

If $\ell_1$ and $\ell_2$ are not parallel, however, their intersection consists of exactly one point $A$.  In this case, we have (set theoretically) $\ell_1\cap\ell_2=\{A\}$.  By abuse of notation, we shall allow ourselves to say $\ell_1\cap\ell_2=A$, and that $A$ is the intersection of the lines $\ell_1,\ell_2$.\\

\noindent\textbf{ANGLES}\\

\noindent Notice how despite a ray having a unique endpoint, there are generally many rays with a given endpoint (they can point in many directions).  We now introduce an important concept which revolves around this.\\

\noindent\textbf{Definition.} \emph{An \textbf{angle} is an unordered pair of two rays with the same endpoint.  If $A,B,C$ are points with $A\ne B,C$, the angle made up of $\overset{\longrightarrow}{AB}$ and $\overset{\longrightarrow}{AC}$ is denoted $\angle BAC$, or just $\angle A$ if the context is clear.}\\

\noindent Here is an illustration of $\angle BAC$:
\begin{center}\includegraphics[scale=.4]{AngleBAC.png}\end{center}
In the notation $\angle BAC$, it is crucial that the common endpoint (which in this case is $A$) is written in between the other two endpoints.  The angle $\angle ABC$ is \emph{not} the same as $\angle BAC$: instead, it is the angle made up of rays $\overset{\longrightarrow}{BA}$ and $\overset{\longrightarrow}{BC}$.  However, $\angle CAB=\angle BAC$ (why?).

A point $D$ is said to be \textbf{inside} $\angle BAC$ if there exists a line $\ell$ through $D$ such that: (a) $\ell$ intersects $\overset{\longrightarrow}{AB}$ in a point $E$, (b) $\ell$ intersects $\overset{\longrightarrow}{AC}$ in a point $F$, and (c) $D$ is between $E$ and $F$.  Note that this is impossible if $A,B,C$ all lie on one line $\ell_1$ and $D$ does not lie on this line: in this case, $E$ and $F$ would have two distinct lines going through them (namely, $\ell_1$ and $\ell$), contradicting Axiom 2.1(i).

Just like the length of a line segment, we shall associate with each angle a real number, which we call its \textbf{measure}.  We denote the measure of $\angle BAC$ as $m\angle BAC$, or just $m\angle A$ if the context is clear.  However, we will not usually write this as a real number.  After all, an angle is usually measured in radians in mathematics, but when studying pure geometry, people are usually more satisfied dealing with the \emph{degree} measures of the angles.  For example, there are $360$ degrees in a circle, but $2\pi$ radians.  Thus we equip ourselves with the following convention:
\begin{center}
\textbf{Throughout this book, for $\alpha\in\mathbb R$, $\alpha^\circ$ is defined to be $\frac{\alpha\pi}{180}$.  This is the (radian) measure of the angle which is $\alpha$ degrees.}
\end{center}
For example, $180^\circ=\pi$, and $90^\circ=\frac{\pi}2$.\\

\noindent\textbf{Axiom 2.6.} (i) \emph{If $\overset{\longrightarrow}{AB}$ is a ray, and $\alpha$ is a real number with $0^\circ<\alpha<180^\circ$, there exist two distinct rays $\overset{\longrightarrow}{AC},\overset{\longrightarrow}{AC'}$ such that $m\angle BAC=m\angle BAC'=\alpha$.}

(ii) \emph{If $\overset{\longrightarrow}{AB}\ne\overset{\longrightarrow}{AC}$ then $m\angle BAC>0$.}

(iii) \textsc{(Straight Angle)} \emph{If $A,B,C$ lie on one line with $A$ between $B$ and $C$, $m\angle BAC=180^\circ$.}

(iv) \textsc{(Angle Addition Postulate)} \emph{If either point $D$ is inside $\angle BAC$ or $A,B,C$ lie on one line with $A$ between $B$ and $C$, then $m\angle BAC=m\angle BAD+m\angle DAC$.}\\

\noindent Several remarks are in order.  Firstly, in part (i), the two rays can be thought of as being on either side of $\overset{\longrightarrow}{AB}$.  After all, if $C'$ were inside $\angle BAC$, we would have $m\angle BAC=m\angle BAC'+m\angle C'AC$ by (iv); i.e., $\alpha=\alpha+m\angle C'AC$, so that $m\angle C'AC=0$.  Therefore by (ii) $\overset{\longrightarrow}{AC}=\overset{\longrightarrow}{AC'}$, a contradiction.  Similarly, $C$ cannot be inside $\angle BAC'$.  Thus the insides of $\angle BAC$ and $\angle BAC'$ are non-overlapping, and are seen to be on ``opposite sides of $\overset{\longrightarrow}{AB}$.''

Secondly, if $A,B,C$ lie on one line with $A$ between $B$ and $C$ (as shown in the diagram below), $\angle BAD$ and $\angle DAC$ are said to form a \textbf{linear pair}.
\begin{center}\includegraphics[scale=.4]{LinearPair.png}\end{center}
In this case (iii) and (iv) jointly imply that $m\angle BAD+m\angle DAC=180^\circ$.  We say that two angles are \textbf{supplementary} if their measures add to $180^\circ$.  Thus, any two angles in a linear pair are supplementary.

Two angles $\angle BAC,\angle B'A'C'$ are said to be \textbf{congruent} if $m\angle BAC=m\angle B'A'C'$.  Again, this is an equivalence relation on the angles.

An angle $\angle BAC$ is said to be a \textbf{right angle} if $m\angle BAC=90^\circ$.  Note that this is equivalent to the angle being supplementary to itself.  Moreover, \emph{the two angles in a linear pair are congruent if and only if one of them is a right angle} (in which case the other one is as well).

Suppose $\ell_1$ and $\ell_2$ are intersecting lines with $\ell_1\cap\ell_2=A$.  Let $B$ and $C$ be points in $\ell_1$ with $A$ strictly between (they exist by (d) of Axiom 2.1(ii)).  Likewise, let $D$ and $E$ be points in $\ell_2$ with $A$ strictly between.  The angles $\angle DAC$ and $\angle BAE$, illustrated below, are called \textbf{vertical angles}.\footnote{The word ``vertical'' does not refer to a line being parallel to the $y$-axis in this context.  Rather, the angles are half of a full turn from one another, and ``vert-'' is a prefix for \emph{turn}.}
\begin{center}\includegraphics[scale=.4]{VerticalAngles.png}\end{center}
Moreover, since angles in a linear pair are supplementary, we have the following proposition\\

\noindent\textbf{Proposition 2.7.} \emph{Vertical angles are congruent.}\\
\begin{proof}
In the above diagram, $\angle BAD$ and $\angle DAC$ form a linear pair, and therefore $m\angle BAD+m\angle DAC=180^\circ$.  Similarly, since $\angle DAB$ and $\angle BAE$ form a linear pair (based on the line $\ell_2$), $m\angle DAB+m\angle BAE=180^\circ$.  Consequently
$$m\angle DAC=180^\circ-m\angle DAB=m\angle BAE$$
and $\angle DAC\cong\angle BAE$ as desired.
\end{proof}

\noindent If one (and hence all) of the angles in the diagram are right angles, the lines are said to be \textbf{perpendicular}, denoted $\ell_1\perp\ell_2$.  We pause now to give another postulate similar to Axiom 2.5.  This statement can actually be proved instead of being taken as an axiom, but the book will ignore this fact.\\ % The argument in the revisions would be great, except the triangle stuff hasn't been covered yet

\noindent\textbf{Proposition 2.8.} \textsc{(Perpendicular Postulate)} \emph{Given a line $\ell$ and a point $A$, there exists a unique line through $A$ perpendicular to $\ell$.}\\

\noindent Note that this is \emph{not} an immediate consequence of Axiom 2.6(i) because the point $A$ may not be on the line $\ell$.

If $D$ is inside $\angle BAC$ and $\angle BAD\cong\angle DAC$, the ray $\overset{\longrightarrow}{AD}$ is called an \textbf{angle bisector} of $\angle BAC$.  The existence and uniqueness of such a ray are left to the reader (Exercise 1).\\

\noindent\textbf{PARALLEL LINES AND TRANSVERSALS}\\

\noindent An important aspect of geometry revolves around a line going through two parallel lines.  They are illustrated below, with the horizontal lines parallel, and the lowercase Greek letters denoting the angles.
\begin{center}\includegraphics[scale=.4]{ParallelTransversal.png}\end{center}
There are several pieces of terminology associated with this diagram.s

* The slanted line is called a \textbf{transversal} going through the parallel lines.

* Angles $\alpha$ and $\zeta$ are called \textbf{corresponding angles}, since they are on the same side of both the transversal and the parallel lines, so they are intuitively pointing in the same direction in the plane.  ($\beta$ and $\theta$ are also corresponding angles.)

* Angles $\alpha$ and $\theta$ are called \textbf{corresponding exterior angles}, since they are both outside the parallel lines (``exterior''), yet they are on the same side of the transversal (hence ``corresponding'').

* Angles $\alpha$ and $\eta$ are called \textbf{alternate exterior angles}, since they are both outside the parallel lines, but on different sides of the transversal.

* Angles $\beta$ and $\zeta$ are called \textbf{corresponding interior angles}, and $\beta$ and $\gamma$ are called \textbf{alternate interior angles}, likewise.

A fundamental axiom which invovles parallel lines and their transversals is\\

\noindent\textbf{Axiom 2.9.} \emph{If $\ell_1\parallel\ell_2$ and $\ell_3$ is a transversal, then corresponding angles are congruent.}\\

\noindent The intuition is that $\ell_2$ has been obtained by translating line $\ell_1$ straight along $\ell_3$ without doing any bending or stretching.

Originally Euclid made Axiom 2.9 unnecessary, by giving the parallel postulate as: ``if a straight line falling on two straight lines make the interior angles on the same side less than two right angles, the two straight lines, if produced indefinitely, meet on that side on which the angles are less than two right angles.''  After all, if this axiom were placed and $\ell_1\parallel\ell_2$, the corresponding interior angles would have to sum to $\geqslant 180^\circ$ (otherwise $\ell_1$ and $\ell_2$ would intersect by the given axiom).  But the same is true for the corresponding interior angles on the \emph{other} side of the transversal.  Since linear pair angles are supplementary, all four of these angles together add to exactly $360^\circ$; hence each pair of corresponding interior angles adds to \emph{exactly} $180^\circ$; i.e., be supplementary.  It then follows that corresponding angles are congruent.

Going back to Axiom 2.9, let us prove:\\

\noindent\textbf{Proposition 2.10.} \emph{If $\ell_1,\ell_2$ are lines (not necessarily parallel) and $\ell_3$ is a transversal, the following statements are equivalent:}

(i) \emph{$\ell_1\parallel\ell_2$.}

(ii) \emph{Corresponding angles are congruent.}

(iii) \emph{Corresponding exterior angles are supplementary.}

(iv) \emph{Alternate exterior angles are congruent.}

(v) \emph{Corresponding interior angles are supplementary.}

(vi) \emph{Alternate interior angles are congruent.}
\begin{proof}
(i) $\implies$ (ii) is Axiom 2.9.

(ii) $\implies$ (i). Let $A=\ell_3\cap\ell_1$.  By Axiom 2.5, there is a unique line\----call it $\ell'$\----through $A$ parallel to $\ell_2$.  As $\ell'$ and $\ell_2$ are parallel lines with $\ell_3$ as a transversal, Axiom 2.9 entails that they have congruent corresponding angles.  Hence (by the hypothesis (ii)), the angle between $\ell'$ and $\ell_3$ facing in a particular direction is congruent to the angle between $\ell_1$ and $\ell_3$ facing in that direction (because they both correspond to the same angle between $\ell_2,\ell_3$).  Axiom 2.6(iv) implies that the angle between $\ell',\ell_1$ is zero, which yields $\ell'=\ell_1$ by Axiom 2.6(ii).  Therefore, $\ell_1$ (being $\ell'$) is parallel to $\ell_2$.

The equivalence of the five statements (ii) - (vi) easily follows from angles in linear pairs being supplementary.
\end{proof}

\noindent\textbf{TRIANGLES}\\

\noindent Triangles are a commonly discussed concept in plane geometry, as there are various interesting types of triangles and properties about them.  A \textbf{triangle} consists of three points $A,B,C$ that do not lie on the same line, as well as the three line segments $\overline{AB},\overline{BC},\overline{AC}$ connecting these points.  The points $A,B,C$ are called the \textbf{vertices} (singular \emph{vertex}), and the line segments are called the \textbf{edges} (or \textbf{sides}).
\begin{center}\includegraphics[scale=.4]{Triangle.png}\end{center}
Notice that each vertex has a canonical angle: for example, at vertex $A$ is the angle $\angle BAC$ (or just $\angle A$).  These three angles are called the \textbf{angles of the triangle}.  The sides $\overline{AB},\overline{AC}$ are said to be \textbf{adjacent} to vertex $A$ (because they touch it), and $\overline{BC}$ is said to be \textbf{opposite} $A$.  A point $D$ is said to be \textbf{inside} the triangle if it is inside all three of the angles.

A triangle with vertices $A,B,C$ is denoted $\triangle ABC$.\\

\noindent\textbf{Proposition 2.11.} \emph{The measures of the angles of a triangle add to $180^\circ$.}
\begin{proof}
Given any triangle, Axiom 2.5 guarantees that there is a line through a vertex of the triangle parallel to the opposite side (i.e., parallel to the line that contains the opposite side), as illustrated below.
\begin{center}\includegraphics[scale=.4]{Triangle180.png}\end{center}
Observe that angles $\alpha',\alpha$ are alternate interior angles of two parallel lines and a transversal (obtained by extending two sides of the triangle).  By Proposition 2.10, the angles are congruent, i.e., $\alpha'=\alpha$.  Similarly $\beta'=\beta$.  By Axiom 2.6 (iii) and (iv), $\alpha'+\gamma+\beta'=180^\circ$.  It follows from these results that $\alpha+\beta+\gamma=180^\circ$ as desired.
\end{proof}

With this, we can classify triangles in terms of their angles.  We recall that an angle is a \textbf{right angle} if its measure is equal to $90^\circ$.  We further define an angle to be \textbf{acute} if its measure is $<90^\circ$, and \textbf{obtuse} if its measure is $>90^\circ$.  Note that in an arbitrary triangle, if any angle is either obtuse or right, the other two angles must be acute (by Proposition 2.11).  Thus \emph{every triangle has at least two acute angles}.  We say that a triangle is:

* An \textbf{obtuse triangle} if it has an obtuse angle;

* A \textbf{right triangle} if it has a right angle;

* An \textbf{acute triangle} if all three angles are acute.

If $\triangle ABC$ is a right triangle with $\angle B$ the right angle, then sides $\overline{AB}$ and $\overline{BC}$ are called the \textbf{legs}, and side $\overline{AC}$ (the one opposite the right angle) is called the \textbf{hypotenuse}.  An important fact about right triangles will be covered at the end of the section.

Triangles can also be classified in terms of their side lengths, but before looking into this, let us first introduce the concept of congruence of triangles.  Two triangles $\triangle ABC,\triangle A'B'C'$ are said to be \textbf{congruent} (denoted $\triangle ABC\cong\triangle A'B'C'$) if the following six statements hold:
$$\overline{AB}\cong\overline{A'B'},~~~~~~\overline{BC}\cong\overline{B'C'},~~~~~~\overline{CA}\cong\overline{C'A'},$$
$$\angle BAC\cong\angle B'A'C',~~~~~~\angle ABC\cong\angle A'B'C',~~~~~~\angle ACB\cong\angle A'C'B'$$
In other words, congruent triangles have all their side lengths and angles matching up.  This is an equivalence relation on triangles.  However, exercise care in the fact though $\triangle BAC$ is the same triangle as $\triangle ABC$, the statement $\triangle BAC\cong\triangle A'B'C'$ is \emph{not} equivalent to $\triangle ABC\cong\triangle A'B'C'$.  Indeed, the former statement entails that $\overline{BC}\cong\overline{A'C'}$, unlike the latter.  This example shows that
\begin{center}
\textbf{When saying triangles are congruent, the order in which the vertices are listed is crucial: $\triangle ABC\cong\triangle A'B'C'$ specifically indicates that $\overline{AB}$ corresponds to $\overline{A'B'}$, etc.}
\end{center}
We will start by rewriting our definition of congruent triangles, because in geometry proofs, it is a statement which deserves a title.\\

\noindent\textbf{Definition 2.12.} \textsc{(Corresponding Parts of Congruent Triangles are Congruent / CPCTC)} \emph{If $\triangle ABC\cong\triangle A'B'C'$, then $\overline{AB}\cong\overline{A'B'}$, $\overline{BC}\cong\overline{B'C'}$, $\overline{CA}\cong\overline{C'A'}$, $\angle A\cong\angle A'$, $\angle B\cong\angle B'$ and $\angle C\cong\angle C'$.}\\

\noindent Knowing triangles to be congruent is thus quite a treasure, as it yields many consequences.  Thus it is useful to know two triangles are congruent without knowing in advance that each one of the six statements in Proposition 2.12 are true.  We set up some axioms for this.\\

\noindent\textbf{Axiom 2.13.} (i) \textsc{(Side-side-side / SSS)} \emph{If $\overline{AB}\cong\overline{A'B'}$, $\overline{BC}\cong\overline{B'C'}$, $\overline{CA}\cong\overline{C'A'}$, then $\triangle ABC\cong\triangle A'B'C'$.}

(ii) \textsc{(Side-angle-side / SAS)} \emph{If $\overline{AB}\cong\overline{A'B'}$, $\overline{AC}\cong\overline{A'C'}$, and $\angle A\cong\angle A'$, then $\triangle ABC\cong\triangle A'B'C'$.}

(iii) \textsc{(Hypotenuse-leg / HL / RHS)} \emph{If $\angle B,\angle B'$ are right angles, $\overline{AB}\cong\overline{A'B'}$ and $\overline{AC}\cong\overline{A'C'}$, then $\triangle ABC\cong\triangle A'B'C'$.}

(iv) \textsc{(Angle-side-angle / ASA)} \emph{If $\angle A\cong\angle A'$, $\overline{AB}\cong\overline{A'B'}$ and $\angle B\cong\angle B'$, then $\triangle ABC\cong\triangle A'B'C'$.}\\

\noindent You can realize the intuition behind each of these axioms by using pencil, paper, a ruler and a protractor.  Observe that ``SAS'' (with the A in the middle) indicates that both sides being compared between the triangles are adjacent to the angle.  Likewise, ``ASA'' indicates that both angles are adjacent to the side.  If there were ``SSA'' congruence, it would go like this:
\begin{center}
If $\overline{AB}\cong\overline{A'B'}$, $\overline{BC}\cong\overline{B'C'}$, and $\angle C\cong\angle C'$, then $\triangle ABC\cong\triangle A'B'C'$.\footnote{This time only \emph{one} side being compared is adjacent to the angle, just like there's only one S touching the A in ``SSA.''}
\end{center}
However, \emph{that statement may be false} \---- see Exercise 5.  Thus SSA is not generally a valid way to prove congruence.  However, SSA is valid if the angles being compared are right angles; this is hypotenuse-leg congruence (Axiom 2.13(iii)).

Also, AAS is valid, and actually follows from Axiom 2.13:\\

\noindent\textbf{Proposition 2.14.} \textsc{(Angle-angle-side / AAS)} \emph{If $\angle A\cong\angle A'$, $\angle B\cong\angle B'$ and $\overline{BC}\cong\overline{B'C'}$, then $\triangle ABC\cong\triangle A'B'C'$.}
\begin{proof}
By Proposition 2.11 we have
$$m\angle C=180^\circ-m\angle A-m\angle B=180^\circ-m\angle A'-m\angle B'=m\angle C'$$
and therefore $\angle C\cong\angle C'$.  Since $\angle B\cong\angle B'$ and $\overline{BC}\cong\overline{B'C'}$, Axiom 2.13(iv) then implies $\triangle ABC\cong\triangle A'B'C'$.
\end{proof}

\noindent With that, we give a new classification of triangles.  We say that a triangle is:

* \textbf{Equilateral} if all three sides are congruent;

* \textbf{Isosceles} if it has two congruent sides;

* \textbf{Scalene} if no two sides are congruent.

Note that under our convention, equilateral triangles are considered isosceles.  Hence a triangle is scalene if and only if it is not isosceles.  Observe that\\

\noindent\textbf{Proposition 2.15.} \textsc{(Isosceles Triangle Theorem)} \emph{If $\triangle ABC$ is a triangle, then $\overline{AB}\cong\overline{AC}$ if and only if $\angle B\cong\angle C$.}
\begin{proof}
By Proposition 2.12 and Axiom 2.13(i), we have that $\overline{AB}\cong\overline{AC}$ if and only if $\triangle ABC\cong\triangle ACB$.  (For, $\overline{AB}\cong\overline{AC}$ jointly implies $\overline{AB}\cong\overline{AC},\overline{BC}\cong\overline{CB},\overline{CA}\cong\overline{BA}$.)  By Proposition 2.12 and Axiom 2.13(iv), however, $\triangle ABC\cong\triangle ACB$ if and only if $\angle B\cong\angle C$.  Thus the statement in the theorem is proved.
\end{proof}

\noindent We conclude that a triangle is equilateral if and only if all three \emph{angles} are congruent (because congruence of angles is equivalent to congruence of the opposite sides by Proposition 2.15).  By Proposition 2.11, it follows that an equilateral triangle necessarily has $60^\circ$ angles.  Moreover, equilateral triangles are acute, but arbitrary isosceles triangles can be either acute, right, or obtuse (can you find examples?).  Similarly a triangle is scalene if and only if no two angles are congruent.\\

\noindent\textbf{SIMILARITY OF TRIANGLES}\\

\noindent Earlier we have described SSS, SAS, HL, ASA, AAS for proving congruence of triangles.  We may now observe that AAA is false in general: if $\angle A\cong\angle A',\angle B\cong\angle B',\angle C\cong\angle C'$, that does not imply $\triangle ABC\cong\triangle A'B'C'$.  Indeed, look at the diagram below, where a line parallel to $\overset{\longleftrightarrow}{BC}$ is constructed which intersects $\overline{AB}$ and $\overline{AC}$ in the respective points $D,E$.
\begin{center}\includegraphics[scale=.4]{SimilarTriangles.png}\end{center}
$\angle A\cong\angle A$ obviously, $\angle ADE\cong\angle B$ (because they are corresponding angles, see Proposition 2.10), and $\angle AED\cong\angle C$.  However, $\triangle ABC\not\cong\triangle ADE$ because $\overline{AB}\not\cong\overline{AD}$; the former segment is longer than the latter.

Your intuition probably tells you that $\triangle ABC$ and $\triangle ADE$ have corresponding side lengths in equal proportions, i.e., $\frac{AD}{AB}=\frac{DE}{BC}=\frac{AE}{AC}$.  We shall enforce this in a new concept.\\

\noindent\textbf{Definition.} \emph{Two triangles $\triangle ABC$ and $\triangle A'B'C'$ are said to be \textbf{similar}, denoted $\triangle ABC\sim\triangle A'B'C'$, if $\angle A\cong\angle A',\angle B\cong\angle B',\angle C\cong\angle C'$ and $\frac{AB}{A'B'}=\frac{BC}{B'C'}=\frac{CA}{C'A'}$.}\\

\noindent Just like congruence, the order in which the vertices are listed is crucial to the statement semantics.

Now we give the analogues of Proposition 2.12 and Axiom 2.13 for similarity of triangles.\\

\noindent\textbf{Proposition 2.16.} \emph{If $\triangle ABC\sim\triangle A'B'C'$, then $\angle A\cong\angle A'$, $\angle B\cong\angle B'$, $\angle C\cong\angle C'$, and $\frac{AB}{A'B'}=\frac{BC}{B'C'}=\frac{CA}{C'A'}$.}\\

\noindent\textbf{Axiom 2.17.} (i) \textsc{(Side-side-side / SSS)} \emph{If $\frac{AB}{A'B'}=\frac{BC}{B'C'}=\frac{CA}{C'A'}$, then $\triangle ABC\sim\triangle A'B'C'$.}

(ii) \textsc{(Side-angle-side / SAS)} \emph{If $\frac{AB}{A'B'}=\frac{AC}{A'C'}$ and $\angle A\cong\angle A'$, then $\triangle ABC\sim\triangle A'B'C'$.}

(iii) \textsc{(Hypotenuse-leg / HL / RHS)} \emph{If $\angle B,\angle B'$ are right angles and $\frac{AB}{A'B'}=\frac{AC}{A'C'}$, then $\triangle ABC\sim\triangle A'B'C'$.}

(iv) \textsc{(Angle-angle / AA)} \emph{If $\angle A\cong\angle A'$ and $\angle B\cong\angle B'$, then $\triangle ABC\sim\triangle A'B'C'$.}\\

\noindent Note that (iv) implies that any two triangles with identical angles are similar.  It is the most common way to prove triangles are similar, because ratios seldom arise in pure geometry without coming from similar triangles first.  Similarity of triangles will occur in many examples, such as Ceva's theorem (Exercise 16).\\

\noindent\textbf{CIRCLES}\\

\noindent Circles are a common geometrical concept, and also show up in many places in real life.  We thus introduce them here.

If $r$ is a positive real number and $O$ is a point, the set of points $A$ such that $AO=r$ is called a \textbf{circle}.  $O$ is called the \textbf{center} of the circle and $r$ is called its \textbf{radius}.
\begin{center}\includegraphics[scale=.4]{Circle.png}\end{center}
The center and radius are uniquely determined by that set of points (Exercise 12).

The technical name for the set of points on the circle is the \textbf{arc} of the circle.  If $A$ is on the arc, note that $\overline{OA}$ is a line segment whose length is the radius $r$ by definition.  $\overline{OA}$ is called a \textbf{radius} of the circle.\footnote{Understand the difference between \emph{a} radius and \emph{the} radius; the former is a line segment from the center to a point on the arc, and the latter is a real number which is the length of such a segment.}  A line segment whose endpoints are both on the arc of the circle is called a \textbf{chord}.  A chord which contains the center is called a \textbf{diameter}.

Our first observation is\\

\noindent\textbf{Proposition 2.18.} (i) \emph{Let $\omega$ be a circle with center $O$ and radius $r$.  If $\overline{AB}$ is a diameter of the circle, then $AB=2r$.}
\begin{proof}
$O\in\overline{AB}$ by definition.  Hence by the segment addition postulate, $AO+OB=AB$.  But $AO$ and $OB$ are both equal to the radius $r$, hence we conclude $AB=2r$.
\end{proof}
Thus, just like all radii have length $r$, every diameter has length $2r$.  We may as well refer to the real number $2r$ as \emph{the diameter} of the circle (just like $r$ is the radius).

Next, we see that two different circles have at most \emph{two} intersection points.  After all, let $\omega$ be the circle centered at $O$ with radius $r$, and $\omega'$ the circle centered at $O'$ with radius $r'$.  If $O=O'$, then the circles must be disjoint, for if $A$ were in their intersection, then we would have $r=OA=O'A=r'$ and hence the circles would be the same.  So assume $O\ne O'$.

If $A,A'$ are intersection points of the circles, then $OA=OA'=r$ and $O'A=O'A'=r'$, and therefore $\triangle OAO'\cong\triangle OA'O'$ by SSS congruence (Axiom 2.13(i)).  Hence by CPCTC, $\angle AOO'\cong\angle A'OO'$.  Therefore the radii $\overline{AO},\overline{A'O}$ meet $\overline{OO'}$ at the same angle; in other words, every radius of $\omega$ going to an intersection point has makes same angle with $\overline{OO'}$.  From Axiom 2.6 it follows that there are at most two such radii, hence at most two intersection points.

We now give a criterion for the intersection points to exist.  Using real analysis, it can be proven without being taken as an axiom, but that will take us too far afield.\\

\noindent\textbf{Proposition 2.19.} \emph{Let $\omega$ be a circle with center $O$ and radius $r$, and $\omega'$ a circle with center $O'$ and radius $r'$.  Set $s=OO'$.  If each of the numbers $r,r',s$ is less than the sum of the other two, then $\omega$ and $\omega'$ have two intersection points.}\\

\noindent The situation when one of the numbers $r,r',s$ is greater than or equal to the sum of the other two is covered in Exercise 15.

It is natural to ask how much a line $\ell$ can intersect a circle $\omega$.  Again the intersection consists of at most two points; this can be proven by constructing the line $\ell'$ through $O$ perpendicular to $\ell$, then seeing that every radius which meets a point in the intersection of $\ell$ and $\omega$ must make the same angle with $\ell$.  If $\ell\cap\omega$ consists of exactly one point, $\ell$ is said to be \textbf{tangent} to $\omega$; if $\ell\cap\omega$ consists of two points, $\ell$ is said to be a \textbf{secant}.

We observe that if $\ell$ is tangent to $\omega$ at a point $A$, then $\ell\perp\overline{OA}$: Let $B$ be a point in $\ell$ other than $A$.  If $m\angle OAB<90^\circ$, let $\alpha=180^\circ-2m\angle OAB$.  Now construct a ray with endpoint $O$ whose points are inside $\angle OAB$ and which makes an angle of $\alpha$ with $\overline{OA}$.  If it meets $\ell$ at a point $C$, then applying Proposition 2.11 to $\triangle OAC$ entails that $m\angle OCA=m\angle OAC[=m\angle OAB]$.  Hence $\overline{OA}\cong\overline{OC}$ by Proposition 2.15, so that $OC=r$ and $C$ is also an intersection point of $\ell$ and $\omega$; contradiction because $\ell$ is tangent to $\omega$.  The same contradiction arises if $m\angle OAB>90^\circ$; this time let $B'$ be a point in $\ell$ with $A$ in between $B,B'$, and then $m\angle OAB'<90^\circ$.  Therefore $m\angle OAB=90^\circ$, making $\angle OAB$ a right angle.

If $A,A'$ are two points on the arc of a circle, then the set of points on the arc that are inside $\angle AOA'$ is called a \textbf{minor arc} of the circle, and is denoted $\measuredangle AA'$; and its angle measure is defined to be $m\angle AOA'$.  [Note that the minor arc does not exist if $A,O,A'$ lie on one line.]  Similarly, the set of points that are outside the angle (along with $A$ and $A'$) is called a \textbf{major arc}.  The angle addition postulate then entails that if $B$ is in an arc $\measuredangle AA'$, we have $m\measuredangle AA'=m\measuredangle AB+m\measuredangle BA'$.

Now suppose $A,B,C$ are distinct points on the arc.  The set of points on the arc inside $\angle BAC$ is called the \textbf{intercepted arc}, and it measures twice the measure of $\angle BAC$, as we now prove.\\

\noindent\textbf{Proposition 2.20.} \emph{If $A,B,C$ are distinct points on the arc of a circle, then $m\angle BAC=\frac 12m\measuredangle BC$.}
\begin{proof}
We shall prove the case where the arc is minor and the center $O$ is inside $\angle BAC$.
\begin{center}\includegraphics[scale=.4]{IntArc.png}\end{center}
Since $\overline{OB}$ and $\overline{OA}$ are both radii, they have length $r$, hence $\overline{OB}\cong\overline{OA}$.  By Proposition 2.15 applied to $\triangle OAB$, $\angle OAB\cong\angle OBA$.  Similarly, $\angle OAC\cong\angle OCA$.  We also have that (by Proposition 2.11), 
$$m\angle BOA+m\angle OAB+m\angle OBA=180^\circ=m\angle BOA+2m\angle OAB$$
$$m\angle COA+m\angle OAC+m\angle OCA=180^\circ=m\angle COA+2m\angle OAC$$
and by extending line $\overset{\longleftrightarrow}{OA}$ and using the angle addition postulate, we get that $m\angle BOC+m\angle COA+m\angle BOA=360^\circ$.  Therefore it follows that
\begin{align*}
m\angle BAC & = m\angle BAO+m\angle OAC \\
& = \frac 12(180^\circ-m\angle BOA)+\frac 12(180^\circ-m\angle COA)\\
& = 180^\circ - \frac 12(m\angle BOA+m\angle COA)\\
& = 180^\circ - \frac 12(360^\circ - m\angle BOC)\\
& = 180^\circ - 180^\circ + \frac 12m\angle BOC\\
& = \frac 12m\angle BOC=\frac 12m\measuredangle BC.
\end{align*}
As for the case where the arc is major or $O$ is outside $\angle BAC$, the proof is similar and left to the reader.
\end{proof}

\noindent\textbf{THE PYTHAGOREAN THEOREM}\\

\noindent We conclude this section with an important theorem concerning the side lengths of a right triangle.\\

\noindent\textbf{Theorem 2.21.} \textsc{(The Pythagorean Theorem)} \emph{If $\triangle ABC$ is a right triangle with $\angle B$ the right angle, then $(AB)^2+(BC)^2=(AC)^2$.}
\begin{proof}
Construct the line through $B$ perpendicular to $\overline{AC}$, and let $D$ be its intersection point with $\overline{AC}$:
\begin{center}\includegraphics[scale=.4]{PythTheorem.png}\end{center}
We start by observing $\angle ACB\cong\angle BCD$ (they are the same angle) and $\angle ABC\cong\angle BDC$ (since both are right angles).  By AA similarity (Axiom 2.17(iv)), $\triangle ABC\sim\triangle BDC$.  Therefore by Axiom 2.16, $\frac{AC}{BC}=\frac{BC}{CD}$.  Multiplying both sides of this equation by $(BC)(CD)$, $(AC)(CD)=(BC)^2$.

A similar argument shows that $\triangle ABC\sim\triangle ADB$, and hence $\frac{AC}{AB}=\frac{AB}{AD}$ and $(AC)(AD)=(AB)^2$.  Therefore,
$$(AB)^2+(BC)^2=(AC)(AD)+(AC)(CD)=(AC)(AD+DC)=(AC)^2,$$
as desired.
\end{proof}

\subsection*{Exercises 2.1. (Axiomatic Plane Geometry)}
\begin{enumerate}
\item Prove that any angle (with measure $<180^\circ$) has a unique angle bisector.

\item Let $\overline{AB}$ be a line segment.  If $M$ is the midpoint of $\overline{AB}$, then a line is said to \textbf{bisect} the segment $\overline{AB}$ if it goes through $M$.  By Proposition 2.8, there is a unique line $\ell$ perpendicular to $\overline{AB}$ that bisects $\overline{AB}$; $\ell$ is called the \textbf{perpendicular bisector} of $\overline{AB}$.  If $C$ is any point, prove that $\overline{AC}\cong\overline{BC}$ if and only if $C\in\ell$.  Thus, the perpendicular bisector of a line segment is the locus of points equidistant from each of the endpoints of the line segment.

\item Let $A$ be a point and $\ell$ be a line.  By Proposition 2.8 there is a unique line $\ell'$ containing $A$ such that $\ell'\perp\ell$.  If $B=\ell\cap\ell'$, show that $AB$ is the minimum distance from $A$ to any point on $\ell$.  This number is called \textbf{the distance from the point $A$ to the line $\ell$.}

\item A \textbf{quadrilateral} consists of four points $A,B,C,D$ such that (a) no three of them lie on the same line, and (b) $\overline{AB}\cap\overline{CD}=\overline{BC}\cap\overline{DA}=\varnothing$; along with the four segments $\overline{AB},\overline{BC},\overline{CD},\overline{DA}$.  (\emph{Caution}: (b) does not mean $\overset{\longleftrightarrow}{AB}$ is parallel to $\overset{\longleftrightarrow}{CD}$!  It states that $\overline{AB}$ and $\overline{CD}$ are set theoretically disjoint as line \emph{segments}; the lines containing them could still intersect.)
\begin{center}\includegraphics[scale=.4]{Quadrilateral.png}\end{center}
The four points $A,B,C,D$ are called \textbf{vertices}, and the four line segments are called \textbf{edges} / \textbf{sides}.  The quadrilateral is denoted as \emph{quadrilateral $ABCD$}.\footnote{It is necessary for the letters to be spelled out in the order the line segments travel; for example, the quadrilateral can also be referred to as $BCDA$ or $DCBA$, but not as $ABDC$, because $B$ isn't connected to $D$ by a line segment.}  Just like in a triangle, each vertex has its canonical angle (e.g., $\angle BAD$ for $A$).

Show that the measures of the angles of a quadrilateral add to $360^\circ$.  [Constructing $\overline{BD}$ divides the quadrilateral into two triangles; now use Proposition 2.11.]

\item Construct a circle whose center is the point $A$, then let $C$ and $D$ be distinct points on the circle's arc.  Let $B$ be an arbitrary point on $\overset{\longrightarrow}{DC}$, with $C$ strictly between $B$ and $D$.  Construct segments $\overline{AB},\overline{AC},\overline{AD},\overline{BD}$ as shown below.
\begin{center}\includegraphics[scale=.4]{SSArefute.png}\end{center}
Then $\overline{AC}\cong\overline{AD}$, $\overline{AB}\cong\overline{AB}$ and $\angle ABC\cong\angle ABD$, but $\triangle ABC\not\cong\triangle ABD$.  These triangles thus constitute a counterexample to the assertion of SSA congruence.

\item Given a quadrilateral $ABCD$, prove that the following are equivalent.

(i) $\overset{\longleftrightarrow}{AB}\parallel\overset{\longleftrightarrow}{CD}$ and $\overset{\longleftrightarrow}{BC}\parallel\overset{\longleftrightarrow}{DA}$.

(ii) $\overline{AB}\cong\overline{CD}$ and $\overline{BC}\cong\overline{DA}$.

(iii) $\angle A\cong\angle C$ and $\angle B\cong\angle D$.

Such a quadrilateral is called a \textbf{parallelogram} (indeed, (i) states that opposite sides are parallel).  [Construct line segment $\overline{BD}$ to divide the quadrilateral into two triangles.]

\item If $ABCD$ is a quadrilateral, the segments $\overline{AC}$ and $\overline{BD}$ are called the \textbf{diagonals} of the quadrilateral.  Show that $ABCD$ is a parallelogram if and only if the diagonals bisect each other (i.e., each diagonal goes through the midpoint of the other one).

\item If $ABCD$ is a quadrilateral, let $A',B',C',D'$ be the midpoints of $\overline{AB},\overline{BC},\overline{CD},\overline{DA}$ respectively.  Prove that quadrilateral $A'B'C'D'$ is a parallelogram.  [First show that each side of this quadrilateral is parallel to one of the diagonals of quadrilateral $ABCD$.]

\item Suppose $ABCD$ is a quadrilateral.

(a) If $\overset{\longleftrightarrow}{AB}\parallel\overset{\longleftrightarrow}{CD}$ and $\angle A\cong\angle C$, prove that $ABCD$ is a parallelogram.

(b) If $\overset{\longleftrightarrow}{AB}\parallel\overset{\longleftrightarrow}{CD}$ and $\overline{AB}\cong\overline{CD}$, prove that $ABCD$ is a parallelogram.

(c) If $\overline{AB}\cong\overline{CD}$ and $\overset{\longleftrightarrow}{BC}\parallel\overset{\longleftrightarrow}{DA}$, is $ABCD$ necessarily a parallelogram?  [Think of Exercise 5.]

\item\emph{(Trigonometric functions.)} \---- Let $\triangle ABC$ be a right triangle with $\angle B$ the right angle.  Observe that the ratio $\frac{BC}{AC}$ is completely determined by the real number $m\angle A$ and not by the right triangle: for if $\triangle A'B'C'$ is another right triangle with $\angle B'$ the right angle and $m\angle A=m\angle A'$, AA similarity entails $\triangle ABC\sim\triangle A'B'C'$, and hence $\frac{BC}{AC}=\frac{B'C'}{A'C'}$.  Similarly, the ratios $\frac{AB}{AC}$ and $\frac{BC}{AB}$ are completely determined by $m\angle A$.  Define
$$\sin(m\angle A)=\frac{BC}{AC},~~~~\cos(m\angle A)=\frac{AB}{AC},~~~~\tan(m\angle A)=\frac{BC}{AB}$$
Then suppose $0^\circ<\alpha<90^\circ$.  Show that:

(a) $(\sin\alpha)^2+(\cos\alpha)^2=1$.  [Use Theorem 2.21.]

(b) $\frac{\sin\alpha}{\cos\alpha}=\tan\alpha$.

(c) $\sin(90^\circ-\alpha)=\cos\alpha$ and $\tan(90^\circ-\alpha)=\frac 1{\tan\alpha}$.  [If $\triangle ABC$ is a right triangle with $\angle B$ the right angle, then $m\angle A+m\angle C=90^\circ$ by Proposition 2.11.]

(d) $\tan 45^\circ=1$ and $\sin 30^\circ=\frac 12$.  [Construct a right triangle whose legs have equal length.  Then take an equilateral triangle and construct the perpendicular from one vertex to the opposite side, dividing it into two right triangles.]

(e) Use part (d) to find the $\sin\alpha,\cos\alpha,\tan\alpha$ for $\alpha=30^\circ,45^\circ,60^\circ$.

\item Here are some rather interesting \emph{inequalities} revolving around triangles.

(a) If $\triangle ABC$ is a triangle, prove that $AB<AC$ if and only if $m\angle C<m\angle B$.  Thus, the correspondence between sides and opposite angles is order-preserving.  [To prove $\Rightarrow$, let $D$ be the point on $\overset{\longrightarrow}{AC}$ such that $AD=AB$, then use Proposition 2.15 on $\triangle ABD$, as well as Proposition 2.11.  The direction $\Leftarrow$ then follows from Proposition 2.15 and the law of trichotomy.]

(b) \emph{(Triangle inequality.)} \----  If $\triangle ABC$ is a triangle, prove that $AB+BC>AC$.  [By Proposition 2.8, there is a unique line $\ell$ through $B$ perpendicular to $\overset{\longleftrightarrow}{AC}$.  Let $D=\ell\cap\overset{\longleftrightarrow}{AC}$.  If $D$ is between $A$ and $C$, then part (a) entails that $AB>AD$ and $BC>DC$; now use the segment addition postulate.  If $D$ is not between $A$ and $C$, the proof should be easier.]

(c) \emph{(Hinge theorem.)} \---- Let $\triangle ABC$ and $\triangle A'B'C'$ be triangles.  If $\overline{AB}\cong\overline{A'B'}$ and $\overline{AC}\cong\overline{A'C'}$, then $m\angle A<m\angle A'$ if and only if $BC<B'C'$.  [If $m\angle A<m\angle A'$, start by establishing a point $D$ such that $\triangle ADC\cong\triangle A'B'C'$.  With that $\overline{AB}\cong\overline{AD}$ (why?).  Now let the angle bisector of $\angle BAD$ meet $\overline{CD}$ at the point $E$.  By SAS, $\triangle ABE\cong\triangle ADE$.  Therefore $\overline{BE}\cong\overline{DE}$, so that $B'C'=DC=DE+EC=BE+EC>BC$ by part (b).  The converse follows from Axiom 2.13 and the law of trichotomy.]
\begin{center}\includegraphics[scale=.4]{HingeTheorem.png}\end{center}

\item Let $\omega$ be the circle centered at $O$ with radius $r$.  By Exercise 2, the perpendicular bisector of any chord passes through $O$.  Use this to show that $O$ can be recovered from the set of points in the arc of $\omega$.  Moreover, $r$ can also be recovered, because if $A$ is a point in the arc then $AO=r$.

\item A chord of a circle is a diameter if and only if it has the maximum possible length among all the chords of the circle.  [Exercise 11(b) may help.]

\item Let $\omega$ be the circle centered at $O$ with radius $r$, and let $A$ be a point on the circle's arc.  Then there is exactly one line that is tangent to $\omega$ at $A$.  [Consider the line through $A$ perpendicular to $\overline{OA}$.]

\item Let $\omega$ be the circle centered at $O$ with radius $r$, and $\omega'$ the circle centered at $O'$ with radius $r'$.  Set $s=OO'$.

(a) If one of $r,r',s$ is strictly greater than the sum of the other two, then $\omega$ and $\omega'$ do not intersect at all.

(b) If one of $r,r',s$ is equal to the sum of the other two, then $\omega$ and $\omega'$ intersect in exactly one point.  [That point must be on $\overset{\longleftrightarrow}{OO'}$.]

\item\emph{(Ceva's theorem.)} \---- Let $\triangle ABC$ be a triangle.  Suppose $A'\in\overline{BC},B'\in\overline{CA},C'\in\overline{AB}$ are points that are not any of the vertices of the triangle.  Then line segments $\overline{AA'},\overline{BB'},\overline{CC'}$ intersect in a common point if and only if $\frac{AB'}{B'C}\frac{CA'}{A'B}\frac{BC'}{C'A}=1$.
[To prove $\Rightarrow$, let $O$ be the common point of intersection of $\overline{AA'},\overline{BB'},\overline{CC'}$.  Construct line segments through $O$ parallel to the sides of the triangle.  Now use similarity of triangles, starting with the observation that the three triangles shaded below are similar to $\triangle ABC$.  To prove $\Leftarrow$, let $O=\overline{AA'}\cap\overline{BB'}$ and let $D$ be the point where $\overset{\longrightarrow}{CO}$ meets $\overline{AB}$.  Use the already proven $\Rightarrow$ direction to show that $\frac{BD}{DA}=\frac{BC'}{C'A}$.  Then $D=C'$ follows easily.]
\begin{center}\includegraphics[scale=.4]{Ceva.png}\end{center}

\item Let $\triangle ABC$ be a triangle.  We shall introduce four kinds of centers of the triangle.  Prove the following.

(a) The perpendicular bisectors of the sides of the triangle intersect in a common point.  This point is called the \textbf{circumcenter} of the triangle.  [First explain why any two of them must intersect.  Then use Exercise 2 to show that their intersection point is also contained in the third one.]

(b) If $O$ is the circumcenter, then $\overline{AO}\cong\overline{BO}\cong\overline{CO}$.  If $r$ is the length of these line segments, the circle centered at $O$ with radius $r$ is the unique circle passing through $A,B,C$.  [We know it is unique because distinct circles intersect in at most two points.]  This is called the \textbf{circumcircle} or \textbf{circumscribed circle} of the triangle.

(c) The angle bisector of an angle is the locus/set of points inside the angle whose distances to the two lines are equal (see Exercise 3).  Use this to show that the angle bisectors of the angles of the triangle intersect in a common point.  This point is called the \textbf{incenter} of the triangle.

(d) If $I$ is the incenter, then part (c) shows that the distance from $I$ to each of the three sides is equal.  Let $r$ be this common distance; then the circle centered at $I$ with radius $r$ is the unique circle tangent to the sides $\overline{AB},\overline{BC},\overline{CA}$.  This is called the \textbf{incircle} or \textbf{inscribed circle}.

(e) A \textbf{median} of the triangle is a line segment from a vertex to the midpoint of the opposite side.  By Ceva's theorem, the medians intersect in a common point.  This point is called the \textbf{centroid} of the triangle.

Let $A'$ be the midpoint of $\overline{BC}$, $B'$ the midpoint of $\overline{CA}$ and $C'$ the midpoint of $\overline{AB}$.  Let $O$ be the centroid (i.e., $\overline{AA'}\cap\overline{BB'}\cap\overline{CC'}$).  Then $\frac{AO}{OA'}=2$.  [First use SAS similarity (Axiom 2.17(ii)) to show that $\triangle B'A'C\sim\triangle ABC$ and $\frac{AB}{A'B'}=2$.  Moreover, $\angle CB'A'\cong\angle CAB$ by Proposition 2.16, and hence $\overline{A'B'}\parallel\overline{AB}$ by Proposition 2.10.  Now use AA similarity to show that $\triangle OA'B'\sim\triangle OAB$, and the rest is easy.]%[Construct $\triangle A'B'C'$ and use similarity of triangles, starting from the observation that $\triangle A'B'C'\sim\triangle ABC$ and the medians of $\triangle ABC$ where they are visible in $\triangle A'B'C'$ are medians of $\triangle A'B'C'$.]

(f) An \textbf{altitude} of the triangle is a line passing through a vertex perpendicular to the opposite side.  Show that the altitudes intersect in a common point.  [Through each vertex, construct the line parallel to the opposite side.  These three lines form a larger triangle, whose perpendicular bisectors are the altitudes of the original triangle.]  This point is called the \textbf{orthocenter}.

It turns out that the orthocenter, centroid and circumcenter always lie on one line: this fact will be covered in Section 2.2.

(g) If $\ell_1$ is the altitude from vertex $A$, $\ell_2$ is the median from vertex $A$, $\ell_3$ is the angle bisector from vertex $A$, and $\ell_4$ is the perpendicular bisector of $\overline{BC}$, then the following are equivalent:

~~~~(i) $\overline{AB}\cong\overline{AC}$.

~~~~(ii) $\ell_1,\ell_2,\ell_3,\ell_4$ are all the same line.

~~~~(iii) At least two of $\ell_1,\ell_2,\ell_3,\ell_4$ are the same line.

\item\emph{(Angle bisector theorem.)} \---- Let $\triangle ABC$ be a triangle.  Suppose the angle bisector of $\angle A$ meets $\overline{BC}$ at point $D$.  Show that $\frac{AB}{AC}=\frac{BD}{CD}$.  [Construct perpendicular line segments from $B$ and $C$ to $\overset{\longrightarrow}{AD}$, and use similar triangles.]

\item Let $ABCD$ be a quadrilateral.  Show that the following are equivalent:

~~~~(i) The four points $A,B,C,D$ all lie on one circle;

~~~~(ii) $\angle ACB\cong\angle ADB$;

~~~~(iii) $\angle ABC$ and $\angle CDA$ are supplementary angles.

Such a quadrilateral is said to be \textbf{cyclic}.  [(i) $\implies$ (ii), (iii): Use Proposition 2.20.  (ii) (resp., (iii)) $\implies$ (i): Let $\omega$ be the circumscribed circle of $\triangle ABC$, and suppose $\overset{\longrightarrow}{AD}$ meets $\omega$ at point $D'$.  Then the hypothesis and (i) $\implies$ (ii) (resp., (iii)) jointly imply that $\angle ADB\cong\angle AD'B$ (resp., $\angle CDA\cong\angle CD'A$); use Proposition 2.10 to conclude that $D=D'$.]

\item (a) If $B$ and $C$ are distinct points on a circle, let $\ell$ be the tangent line to the circle through point $B$, and let $A$ be a point of $\ell$ other than $B$.  Then $m\angle ABC=\frac 12m\measuredangle BC$, where the arc is taken as the set of points inside $\angle ABC$.  [Use Propositions 2.11 and 2.15 on $\triangle BOC$ (with $O$ the center of the circle), along with the observed fact that $\ell\perp\overline{OB}$.]

(b) In the diagram below, where $\overset{\longleftrightarrow}{AB}$ is tangent to the circle, show that $(AB)^2=(AC)(AD)$.
\begin{center}\includegraphics[scale=.3]{TangentSecant.png}\end{center}
[$m\angle ADB=\frac 12m\measuredangle BC$ by Proposition 2.20, and $m\angle ABC=\frac 12m\measuredangle BC$ by part (a).  Therefore AA similarity implies $\triangle ABC\sim\triangle ADB$.]

\item\emph{(Intersecting chords theorem.)} \---- Suppose $\overline{AC}$ and $\overline{BD}$ are two chords of circle $\omega$ which intersect in a point $E$.  Then $(AE)(EC)=(BE)(ED)$.  [$\angle ACB\cong\angle ADB$ by Exercise 19, and hence $\angle ECB\cong\angle ADE$.  Also $\angle AED\cong\angle BEC$ because they are vertical angles.  It follows from AA similarity that $\triangle AED\sim\triangle BEC$.]

\item\emph{(Ptolemy's theorem.)} \---- Suppose $ABCD$ is a cyclic quadrilateral.  Then $(AB)(CD)+(BC)(DA)=(AC)(BD)$.  [Start by establishing a point $K\in\overline{AC}$ such that $\angle ABK\cong\angle DBC$.  By Exercise 19, $\angle BAK=\angle BAC\cong\angle BDC$.  Therefore $\triangle ABK\sim\triangle DBC$ by AA similarity.  Now explain why $\angle KBC\cong\angle ABD$, and use this fact to show that $\triangle KBC\sim\triangle ABD$.  Therefore $\frac{AK}{AB}=\frac{DC}{DB}$ and $\frac{KC}{BC}=\frac{AD}{BD}$.  Cross-multiplying, $(AK)(BD)=(AB)(CD)$ and $(KC)(BD)=(BC)(AD)$.  Therefore, adding the two equations, $(AK)(BD)+(KC)(BD)=(AB)(CD)+(BC)(AD)$.  Yet $(AK)(BD)+(KC)(BD)=(AC)(BD)$ by the segment addition postulate.] %http://mypages.iit.edu/~maslanka/Ptolemy.pdf

\item What is wrong with the following ``proof'' that every triangle is isosceles?\footnote{Maxwell 1959, Chapter II, Section 1.}

Let $\triangle ABC$ be a triangle; we show that $\overline{AB}\cong\overline{AC}$.  Construct the angle bisector $\ell_1$ of $\angle A$ and the perpendicular bisector $\ell_2$ of $\overline{BC}$, letting $P$ be the midpoint of $\overline{BC}$.  If $\ell_1\parallel\ell_2$, then $\ell_1\perp\overset{\longleftrightarrow}{BC}$ which means the angle bisector of $\angle A$ coincides with the altitude, making $\overline{AB}\cong\overline{AC}$ by Exercise 17(g).
Thus we may assume $\ell_1\not\parallel\ell_2$.

Let $O$ be the point $\ell_1\cap\ell_2$.  Construct $\overline{OB}$ and $\overline{OC}$.  By Proposition 2.8 there is a line through $O$ perpendicular to $\overline{AB}$; let it intersect $\overline{AB}$ at point $Q$ and construct $\overline{OQ}$.  Similarly let $R$ be the point on $\overline{AC}$ such that $\overline{OR}\perp\overline{AC}$, as shown below.
\begin{center}\includegraphics[scale=.4]{ITFallacy.png}\end{center}
(The diagram is not drawn to scale, so as to not reveal the error.)

Now, $\angle AQO\cong\angle ARO$ (both are right angles), $\angle QAO\cong\angle RAO$ (by definition of an angle bisector) and $\overline{AO}\cong\overline{AO}$.  Hence by AAS, $\triangle AQO\cong\triangle ARO$.  Hence by CPCTC (Proposition 2.12), $\overline{AQ}\cong\overline{AR}$ and $\overline{OQ}\cong\overline{OR}$.

Observe now that $\angle OPB\cong\angle OPC$ (both are right angles), $\overline{OP}\cong\overline{OP}$ and $\overline{BP}\cong\overline{PC}$ (because $P$ is the midpoint of $\overline{BC}$).  By SAS, $\triangle OPB\cong\triangle OPC$.  By CPCTC, $\overline{OB}\cong\overline{OC}$.

Finally, consider triangles $\triangle OQB$ and $\triangle ORC$.  We have shown $\overline{OQ}\cong\overline{OR}$ and $\overline{OB}\cong\overline{OC}$ in the previous two paragraphs; also, $\angle OQB,\angle ORC$ are both right angles.  By HL, $\triangle OQB\cong\triangle ORC$.  By CPCTC, $\overline{QB}\cong\overline{RC}$ follows.

Since $\overline{AQ}\cong\overline{AR}$ and $\overline{QB}\cong\overline{RC}$, we get
$$AB=AQ+QB=AR+RC=AC$$
and therefore $\overline{AB}\cong\overline{AC}$ as desired.
\end{enumerate}

\subsection*{2.2. Giving the Plane Coordinates}
\addcontentsline{toc}{section}{2.2. Giving the Plane Coordinates}
In the previous section, we have gone through a basic course of axiomatic plane geometry.  In this section, we shall introduce coordinate axes, thereby turning the aforementioned constructions (line, line segment, ray, angle, triangle, circle) into algebraic figures.  We will then stick to the coordinate setting for the rest of the book.

We start by letting $\chi_1$ and $\chi_2$ be two perpendicular lines, which we call \emph{axes}.  Let $O=\chi_1\cap\chi_2$.  We will henceforth call $O$ the \textbf{origin} of the plane, or of our coordinate system.  Finally, let $A\in\chi_1$ and $B\in\chi_2$ be points distinct from $O$.  (We will use them to specify which semiaxes are positive.)

Now suppose $P$ is an arbitrary point in the plane.  We derive an ordered pair $(x,y)\in\mathbb R^2$ as follows.  By Proposition 2.8, there is a line through $P$ perpendicular to $\chi_1$; let it intersect $\chi_1$ at point $P'$.  Now $O,A,P'$ are all on $\chi_1$: either $O$ is between $A$ and $P'$ or it isn't.  If $O$ is between $A$ and $P'$, we set $x=-(OP')$; otherwise, we set $x=OP'$.  Likewise, suppose the line through $P$ perpendicular to $\chi_2$ meets $\chi_2$ at $P^*$.  If $O$ is between $B$ and $P^*$, then $y=-(OP^*)$, otherwise $y=OP^*$.
\begin{center}\includegraphics[scale=.4]{CoordPlane.png}

\emph{The coordinates in this example are $(-3,2)$.  Note that $x$ is negative because $O$ is between $A$ and $P'$.}\end{center}

\noindent This gives us a one-to-one correspondence between our Euclidean plane and $\mathbb R^2$:\\

\noindent\textbf{Lemma 2.22.} \emph{If $A,B,C$ are arbitrary points that lie on a line, then $\overline{BC}\subset\overline{AB}\cup\overline{AC}$.}
\begin{proof}
If $A$ is between $B$ and $C$, then $\overline{BC}=\overline{AB}\cup\overline{AC}$ by Axiom 2.1(ii), hence the lemma holds in this case.  If $B$ is between $A$ and $C$, then $\overline{BC}\subset\overline{AC}$ by Axiom 2.1(ii) so that $\overline{BC}\subset\overline{AB}\cup\overline{AC}$.  Similarly if $C$ is between $A$ and $B$.
\end{proof}
\noindent\textbf{Proposition 2.23.} \emph{The map $\varphi:P\mapsto(x,y)$ defined above is bijective.}
\begin{proof}
Suppose $\varphi(P)=\varphi(Q)=(x,y)$, say.  Then construct the line through $P$ perpendicular to $\chi_1$ and let $P'$ be its intersection point.  Similarly, let $Q'$ be the intersection point of the line through $Q$ perpendicular to $\chi_1$.  Then by definition, $OP'=|x|=OQ'$.

We claim that $P'=Q'$: by Axiom 2.4, this is immediate if $P'$ is between $O$ and $Q'$ (in this case $OQ'=OP'+P'Q'$ and hence $P'Q'=0$, implying $P'=Q'$), or if $Q'$ is between $O$ and $P'$.  So let us assume $O$ is between $P'$ and $Q'$.  If $x=0$, then $O=P'=Q'$ is clear so assume $x\ne 0$.  If $x$ is negative, then $O$ is between $A$ and $P'$ (that is the criterion for a negative coordinate value), and also $O$ is between $A$ and $Q'$; from this it follows that $O\in\overline{P'A}\cap\overline{AQ'}$.  If $A$ is between $P'$ and $Q'$ this implies $\overline{P'A}\cap\overline{AQ'}=\{A\}$ by the observations after Axiom 2.1, and hence $O=A$, a contradiction.  If $P'$ is between $A$ and $Q'$ then $O\in\overline{AP'}\cap\overline{P'Q'}=\{P'\}$, hence $O=P'$ and $x=0$ which doesn't hold water; similarly we cannot have $Q'$ between $A$ and $P'$.  Therefore the assumption that $x$ is negative leads to a contradiction; hence $x$ is positive.  Therefore $O$ is neither between $A$ and $P'$, nor between $A$ and $Q'$.  This implies $O\notin\overline{AP'}\cup\overline{AQ'}$.  Yet $O\in\overline{P'Q'}\subset\overline{AP'}\cup\overline{AQ'}$ by Lemma 2.22.  This is a contradiction, and hence it is impossible that $O$ is between $P'$ and $Q'$ and $x\ne 0$.  Since we've proven $P'=Q'$ in all other cases, we may conclude $P'=Q'$.

Furthermore this implies that $P$ and $Q$ lie on the same line perpendicular to $\chi_1$ (the perpendicular line going through $P'=Q'$); let $\ell_1$ be such a line.  The same argument shows that $P$ and $Q$ lie on the same line $\ell_2$ perpendicular to $\chi_2$.  Then $\ell_1\perp\ell_2$ (why?), and $P$ and $Q$ are both equal to the unique intersection point of $\ell_1$ and $\ell_2$.  Hence $P=Q$, and $\varphi$ is injective.

To show surjectivity, suppose $(x,y)\in\mathbb R^2$.  If $x\geqslant 0$, let $P'$ be the point on $\overset{\longrightarrow}{OA}$ such that $OP'=x$ (see Axiom 2.4).  In this case $O$ is not between $P'$ and $A$ if $O\ne P'$; indeed, if $O$ were between $P'$ and $A$, then (since $P'\in\overset{\longrightarrow}{OA}$), either $A$ is between $O$ and $P'$ (in which case $O=A$, contradiction), or else $P'$ is between $O$ and $A$ (in which case $O=P'$).  On the other hand, if $x<0$, let $A'$ be a point on $\overset{\longleftrightarrow}{OA}$ such that $O$ is strictly between $A'$ and $A$, then let $P'$ be the point on $\overset{\longrightarrow}{OA'}$ such that $OP'=|x|$.  With that, $O$ \emph{is} between $P'$ and $A$ \---- after all, $O\in\overline{AA'}\subset\overline{AP'}\cup\overline{A'P'}$ by Lemma 2.22, and if $O\in\overline{A'P'}$ then (since $P'\in\overset{\longrightarrow}{OA'}$), either $P'$ is between $O$ and $A'$ (which implies $O=P'$ and $|x|=OP'=0$, contradiction because $x<0$ strictly), or else $A'$ is between $O$ and $P'$ (which implies $O=A'$, another contradiction).  Thus $O$ is necessarily in $\overline{AP'}$.

Likewise, if $y\geqslant 0$, let $P^*$ be the point on $\overset{\longrightarrow}{OB}$ such that $OP^*=x$ (which implies, like in the last paragraph, that $O$ is not between $P^*$ and $B$ if $O\ne P^*$); and if $y<0$, let $B'$ be a point on $\overset{\longrightarrow}{OB}$ such that $O$ is strictly between $B'$ and $B$, then let $P^*$ be the point on $\overset{\longrightarrow}{OB'}$ such that $OP^*=|y|$ (which implies $O$ is between $P^*$ and $B$).  Now let $\ell_1$ be the line through $P'$ perpendicular to $\chi_1$; let $\ell_2$ be the line through $P^*$ perpendicular to $\chi_2$; and let $P=\ell_1\cap\ell_2$.  The reader can readily verify that $\varphi(P)=(x,y)$, and therefore $\varphi$ is surjective.
\end{proof} % Remark: Most of the hassle in this proof involves distinguishing positive from negative coordinates.  After all, length of a line segment is never negative

\noindent Now that we have identified our axiomatic plane with $\mathbb R^2$, we wish to find algebraic formulas for the constructions of the previous chapter.  Let us start with lines.  First note that if $P=(x,y)$, then $P\in\chi_1$ if and only if $y=0$, and $P\in\chi_2$ if and only if $x=0$.  (Verifications left to the reader.)  Hence, the origin $O=(0,0)$.\\

\noindent\textbf{Proposition 2.24.} \textsc{(Distance Formula)} \emph{Let $P_1$ and $P_2$ be points in the plane (where the plane is identified with $\mathbb R^2$).}

(i) \emph{If $P_1=(x_1,y)$ and $P_2=(x_2,y)$, then $P_1P_2=|x_1-x_2|$.}

(ii) \emph{If $P_1=(x,y_1)$ and $P_2=(x,y_2)$, then $P_1P_2=|y_1-y_2|$.}

(iii) \emph{If $P_1=(x_1,y_1)$ and $P_2=(x_2,y_2)$, then $P_1P_2=\sqrt{(x_1-x_2)^2+(y_1-y_2)^2}$.}\\

\noindent Formulas like Proposition 2.24 may seem familiar.  However, we will go through their proofs in detail because the methods will not work in the non-Euclidean geometries we will study in Chapters 3-5.  For example, the argument in the proof of (i) with $P_1P_2Q_2Q_1$ will \emph{not} work in the other geometries, because this quadrilateral will not have four right angles.
\begin{proof}
(i) Let $\ell$ be the line through $P_1$ perpendicular to $\chi_2$.  We first observe that $\ell\cap\chi_2=(0,y)$: the $x$-coordinate is zero because the point is on $\chi_2$ (see note before the proposition).  As for the $y$-coordinate, the trick is to observe in the definition of the coordinates of a point, that the $y$-coordinate of a point is completely determined by the particular line perpendicular to $\chi_2$ that goes through the point.  But this line is $\ell$ for both the points $P_1$ and $\ell\cap\chi_2$ (as they both lie on the line); hence, the two points have the same $y$-coordinate.  Since $P_1=(x_1,y)$, it follows that $y$ is the $y$-coordinate of $\ell\cap\chi_2$ and that $\ell\cap\chi_2=(0,y)$.

Likewise if $\ell'$ is the line through $P_2$ perpendicular to $\chi_2$, then $\ell'\cap\chi_2=(0,y)$.  Hence if $\ell$ and $\ell'$ are distinct lines, they would be parallel by Proposition 2.10 (both are perpendicular to $\chi_2$), which contradicts $(0,y)\in\ell\cap\ell'$.  Thus $\ell=\ell'$, and this line passes through both $P_1$ and $P_2$.

Now let $\ell_1$ be the line through $P_1$ perpendicular to $\chi_1$, and $\ell_2$ the line through $P_2$ perpendicular to $\chi_1$.  If $Q_k=\ell_k\cap\chi_1$ for $k=1,2$, then adapting the argument of the first paragraph of the proof shows that $Q_k=(x_k,0)$.  Observe that $\overset{\longleftrightarrow}{P_kQ_k}=\ell_k$, $\overset{\longleftrightarrow}{Q_1Q_2}=\chi_1$ and $\overset{\longleftrightarrow}{P_1P_2}=\ell$.  Repeated application of Exercise 4 of the previous section shows that quadrilateral $P_1P_2Q_2Q_1$ has four right angles.  Therefore $\overline{P_1P_2}\cong\overline{Q_1Q_2}$ by Exercise 6 of the previous section, and we need only show that $Q_1Q_2=|x_1-x_2|$.

If either $Q_1=O$ or $Q_2=O$, the statement is obvious, hence we may assume $Q_1,Q_2\ne O$.  If $Q_1$ is between $Q_2$ and $O$, then $Q_1Q_2=OQ_2-OQ_1=|x_2|-|x_1|$ by the segment addition postulate.  In this case, if $O$ is between $Q_1$ and $A$, then $O\in\overline{Q_1A}\subset\overline{Q_1Q_2}\cup\overline{Q_2A}$, and hence $O$ is between $Q_2$ and $A$ (because $O\in\overline{Q_1Q_2}$ implies $O$ is between $Q_1$ and $Q_2$, hence $O=Q_1$, contradiction); and conversely, if $O$ is between $Q_2$ and $A$, then $O\in\overline{Q_2A}\subset\overline{Q_2Q_1}\cup\overline{Q_1A}$ likewise implying $O$ is between $Q_1$ and $A$.  Thus we have proven that
$$x_2\leqslant 0\iff O\text{ between }Q_2\text{ and }A\iff O\text{ between }Q_1\text{ and }A\iff x_1\leqslant 0$$
with that, it is easy to see through casework on the signs of $x_1,x_2$ that $|x_2-x_1|$ is the absolute value of $|x_2|-|x_1|$.  Yet $|x_2|-|x_1|=Q_1Q_2\geqslant 0$ (as we showed earlier), which further entails $|x_2-x_1|=|x_2|-|x_1|$.  Hence $Q_1Q_2=|x_2-x_1|$ in this case.

If $Q_2$ is between $Q_1$ and $O$, repeat the same argument with the roles of $Q_1,Q_2$ exchanged.  So suppose $O$ is between $Q_1$ and $Q_2$.  Then $O\in\overline{Q_1Q_2}\subset\overline{Q_1A}\cup\overline{Q_2A}$, so either $O$ is between $Q_1$ and $A$, or else it is between $Q_2$ and $A$.  We claim that this is an \emph{exclusive} disjunction: for, if $O$ were both between $Q_1$ and $A$, and between $Q_2$ and $A$, then we would have $O\in\overline{Q_1A}\cap\overline{Q_2A}$.  If $A$ is between $Q_1$ and $Q_2$ this intersection is $\{A\}$, hence $O=A$, contradiction.  If $Q_1$ is between $A$ and $Q_2$, then $O\in\overline{Q_1Q_2}\cap\overline{Q_1A}=\{Q_1\}$, another contradiction; similarly if $Q_2$ is between $A$ and $Q_1$.  Thus, we have that:
\begin{center}
Either $O$ is between $Q_1$ and $A$, or $O$ is between $Q_2$ and $A$, but not both.
\end{center}
Therefore either $x_1\leqslant 0$ or $x_2\leqslant 0$, but not both.  Simple casework then shows that $|x_1-x_2|=|x_1|+|x_2|$.  Yet since $O$ is between $Q_1$ and $Q_2$, we have $Q_1Q_2=OQ_1+OQ_2=|x_1|+|x_2|=|x_1-x_2|$ as desired.

(ii) Repeat part (i) with the roles of $x$ and $y$ exchanged, along with the roles of $\chi_1$ and $\chi_2$.

(iii) Let $P^*$ be the point $(x_2,y_1)$.  Let $\ell_1$ be the line through $(0,y_1)$ perpendicular to $\chi_2$; the argument used in part (i) entails that $\ell_1$ contains any point of the form $(x,y_1)$; hence $P_1,P^*\in\ell_1$.  Similarly, if $\ell_2$ is the line through $(x_2,0)$ perpendicular to $\chi_1$, then $P^*,P_2\in\ell_2$.  Hence since $\ell_1\perp\ell_2$, $\triangle P_1P^*P_2$ is a right triangle with $\angle P^*$ the right angle.  By Theorem 2.21, $(P_1P^*)^2+(P^*P_2)^2=(P_1P_2)^2$.  Yet $P_1P^*=|x_1-x_2|$ (by part (i)), and $P^*P_2=|y_1-y_2|$.  It follows that $(P_1P^*)^2+(P^*P_2)^2=(x_1-x_2)^2+(y_1-y_2)^2$, and hence $P_1P_2=\sqrt{(x_1-x_2)^2+(y_1-y_2)^2}$ as desired.
\end{proof}
\noindent Now that we have the distance formula, we shall establish a fundamental inequality for verifying the equation of a line.\\

\noindent\textbf{Lemma 2.25.} \emph{If $a,b,c,d\in\mathbb R$ (with $c,d$ not both zero), then $\sqrt{a^2+b^2}+\sqrt{c^2+d^2}\geqslant\sqrt{(a+c)^2+(b+d)^2}$, and equality holds if and only if there is a constant $\lambda\geqslant 0$ such that $a=\lambda c,b=\lambda d$.}
\begin{proof}
Our goal is to reformulate this inequality into a more familiar one:
$$\sqrt{a^2+b^2}+\sqrt{c^2+d^2}\geqslant\sqrt{(a+c)^2+(b+d)^2}$$
Squaring both sides of the equation,
$$a^2+b^2+c^2+d^2+2\sqrt{(a^2+b^2)(c^2+d^2)}\geqslant(a+c)^2+(b+d)^2$$
Subtracting $a^2+b^2+c^2+d^2$, and using the binomial theorem on the right-hand side,
$$2\sqrt{(a^2+b^2)(c^2+d^2)}\geqslant 2ac+2bd$$
Dividing by $2$ and squaring,
$$(a^2+b^2)(c^2+d^2)\geqslant(ac+bd)^2$$
This statement is a well-known inequality in mathematics called the \emph{Cauchy-Schwarz inequality}.  It can be proven by noting that the quadratic polynomial $(cx-a)^2+(dx-b)^2$ sends real numbers to nonnegative real numbers, so its discriminant must be $\leqslant 0$ (otherwise it would have two real roots and its graph would cross the $x$-axis) \---- but straightforward computation shows the discriminant to be $4[(ac+bd)^2-(a^2+b^2)(c^2+d^2)]$.  Since our steps are reversible, we get the inequality in the lemma.

Now note that equality holds if and only if the discriminant is zero, which means that $(cx-a)^2+(dx-b)^2$ has a multiple root $\lambda$ on the real line.  Hence, we have $c\lambda-a=d\lambda-b=0$ (the only nonnegative real numbers that add to zero are zero themselves), and therefore $a=\lambda c$ and $b=\lambda d$.  If this holds and $\sqrt{a^2+b^2}+\sqrt{c^2+d^2}=\sqrt{(a+c)^2+(b+d)^2}$, then $|\lambda|\sqrt{c^2+d^2}+\sqrt{c^2+d^2}=|\lambda+1|\sqrt{c^2+d^2}$, hence $|\lambda|+1=|\lambda+1|$, which holds if and only if $\lambda\geqslant 0$.
\end{proof}
\noindent\textbf{Proposition 2.26.} \emph{Let $\ell$ be a line.}

(i) \emph{If $O\in\ell$, then there exist $a,b\in\mathbb R$, not both zero, such that $\ell=\{(x,y):ax+by=0\}$.}

(ii) \emph{In the general case, there exist $a,b,c\in\mathbb R$ such that $a$ and $b$ are not both zero and $\ell=\{(x,y):ax+by=c\}$.}

(iii) \emph{Conversely, if $\ell'=\{(x,y):ax+by=c\}$ with $a,b$ not both zero, then $\ell'$ is a line.}
\begin{proof}
(i) Let $P$ be a point of $\ell$ other than $O$.  Write $P=(b,-a)$ with $a,b\in\mathbb R$ \---- since $P\ne O$, $a$ and $b$ are not both zero.  We claim that $\ell=\{(x,y):ax+by=0\}$.

First suppose $Q=(x,y)$ is a point such that $ax+by=0$.  Then letting
$$u=\left\{\begin{array}{c l}-y/a\text{ if }a\ne 0\\x/b\text{ if }a=0\end{array}\right.$$
we have that $x=ub$ and $y=-ua$.  Note that by Proposition 2.24,
$$OQ=\sqrt{(ub)^2+(-ua)^2}=|u|\sqrt{a^2+b^2}$$
$$QP=\sqrt{(b-ub)^2+(-a+ua)^2}=|1-u|\sqrt{a^2+b^2}$$
$$OP=\sqrt{b^2+(-a)^2}=\sqrt{a^2+b^2}$$
Thus if $0\leqslant u\leqslant 1$ then $OQ+QP=OP$, and hence $Q$ must lie on the line $\ell$ containing $O$ and $P$ (otherwise we would have $OQ+QP>OP$ by Exercise 11(b) of the previous section).  If $u>1$ then $OQ=OP+PQ$, hence likewise $Q\in\ell$.  Similarly $u<0$ entails $PQ=PO+OQ$.

Conversely, suppose $Q=(x,y)\in\ell$.  Then Proposition 2.24 entails that
$$OQ=\sqrt{x^2+y^2}$$
$$OP=\sqrt{b^2+(-a)^2}=\sqrt{a^2+b^2}$$
$$QP=\sqrt{(b-x)^2+(-a-y)^2}=\sqrt{(x-b)^2+(y+a)^2}$$
Since $O,P,Q\in\ell$, at least one of the three points is between the other two.  If $Q$ is between $O$ and $P$, $OQ+QP=OP$ by the segment addition postulate, and hence
$$\sqrt{x^2+y^2}+\sqrt{(b-x)^2+(-a-y)^2}=\sqrt{a^2+b^2}=\sqrt{(x+b-x)^2+(y-a-y)^2};$$
from which Lemma 2.25 implies that either $b-x=-a-y=0$ (in which case $ax+by=0$ is clear), or else there is a constant $\lambda\geqslant 0$ such that $x=\lambda(b-x),y=\lambda(-a-y)$.  In the latter case $x=\frac{\lambda b}{\lambda+1}$ and $y=\frac{-\lambda a}{\lambda+1}$, and hence $ax+by=0$ is checked.

If $O$ is between $P$ and $Q$, or $Q$ is between $P$ and $O$, Lemma 2.25 can be likewise used to show $ax+by=0$.  We leave the verifications to the reader.

(ii) Suppose $P=(u,v)\in\ell$ is arbitrary.  Define $\mathcal T:\mathbb R^2\to\mathbb R^2$ via $\mathcal T(x,y)=(x-u,y-v)$.  By the distance formula (2.24) it is clear that whenever $A,B,A',B'$ are points such that $\mathcal T(A)=A'$ and $\mathcal T(B)=B'$, we have $A'B'=AB$.  Therefore $\mathcal T$ preserves lines by Exercise 2(a).  Hence $\ell'=\mathcal T(\ell)$ is a line that contains the origin (since $O=\mathcal T(P)$ and $P\in\ell$).  By part (i), there are $a,b\in\mathbb R$, not both zero, such that $\ell'=\{(x,y):ax+by=0\}$.  With that, it follows by definition of $\mathcal T$ and $\ell'$, that $\ell=\{(x,y):ax+by=c\}$, where $c$ is the real number $au+bv$; we leave the details to the reader.

(iii) It is clear that $\ell'$ contains at least two points: take
$$A=\left\{\begin{array}{c l}\left(0,\frac cb\right)\text{ if }b\ne 0\\\left(\frac ca,0\right)\text{ if }b=0\end{array}\right.$$
and $B$ to be the translation of $A$ by the vector $(b,-a)$; then $A$ and $B$ are both on $\ell'$.  By Axiom 2.1(i), there is a unique line $\ell$ containing $A$ and $B$.  By part (ii), $\ell$ is of the form $a_1x+b_1y=c_1$ with $a_1,b_1$ not both zero.  Now, the linear system
$$ax+by=c,~~~~a_1x+b_1y=c_1$$
has at least two solutions ($A$ and $B$), hence its coefficient matrix is singular.  Since both rows are nonzero, they are hence scalar multiples of one another; i.e., there exists $\lambda\in\mathbb R$ such that $a_1=\lambda a$ and $b_1=\lambda b$.  Multiplying the first equation by $\lambda$ then entails $c_1=\lambda c$.  Hence the equations are scalar multiples and have the same locus, so that $\ell'=\ell$ and $\ell'$ is a line.
\end{proof}

\noindent At this point it should be clear how line segments and rays are formulated.  For example, if $A=(1,0)$ and $B=(0,2)$, then the ray $\overset{\longrightarrow}{AB}$ is given by the equations
$$2x+y=2,y\geqslant 0.$$
With the distance formula in hand, it is easy to figure out angle measures.\\

\noindent\textbf{Proposition 2.27.} \emph{Suppose $A=(x_1,y_1)$ and $B=(x_2,y_2)$ are points $\ne O$.  Then $\cos(m\angle AOB)=\frac{x_1x_2+y_1y_2}{\sqrt{(x_1^2+y_1^2)(x_2^2+y_2^2)}}$.}\\

\noindent Incidentally this gives the formula for the cosine of the measure of \emph{any} angle $\angle ACB$, whether $C=O$ or not.  After all, one can adapt the proof of Proposition 2.26(b) specifically with $P=C$, and use Exercise 2(c).

Knowing the cosine of the measure, though, one can immediately find the measure (cosine is bijective from $[0^\circ,180^\circ]$ to $[-1,1]$)
\begin{proof} % Remark: This argument is applicable to any number of dimensions.  But in the general case \ell is a hyperplane.
Let $\ell$ be the line given by $x_1x+y_1y=x_1^2+y_1^2$ (i.e., $ax+by=c$ with $a=x_1,b=y_1,c=x_1^2+y_1^2$).  Then manifestly $A\in\ell$.  It turns out that $A$ is the point on $\ell$ closest to $O$, because if $P=(x,y)\in\ell$ then $x_1x+y_1y=x_1^2+y_1^2$, and therefore
$$(x^2+y^2)-(x_1^2+y_1^2)=(x^2+y^2)-2(x_1^2+y_1^2)+(x_1^2+y_1^2)$$
$$=(x^2+y^2)-2(x_1x+y_1y)+(x_1^2+y_1^2)=(x^2-2xx_1+x_1^2)+(y^2-2yy_1+y_1^2)$$
$$=(x-x_1)^2+(y-y_1)^2\geqslant 0,$$
from which it follows that $\sqrt{x^2+y^2}\geqslant\sqrt{x_1^2+y_1^2}$; i.e., $OA\leqslant OP$.  Therefore, $\overline{OA}\perp\ell$ by Exercise 3 of the previous section.

If $\ell$ is parallel to $\overset{\longleftrightarrow}{OB}$, then $\angle BOA$ is a right angle (why?)  Moreover, $\overset{\longleftrightarrow}{OB}$ is the line through $O$ parallel to $\ell$ and hence it is given by the equation $x_1x+y_1y=0$.  Since $B=(x_2,y_2)$ and is in this line, we get $x_1x_2+y_1y_2=0$, hence $\frac{x_1x_2+y_1y_2}{\sqrt{(x_1^2+y_1^2)(x_2^2+y_2^2)}}=0=\cos 90^\circ=\cos(m\angle AOB)$.

Now suppose $\ell$ is not parallel to $\overset{\longleftrightarrow}{OB}$.  Then $x_1x_2+y_1y_2\ne 0$ (if $x_1x_2+y_1y_2$ were zero, then the line $x_1x+y_1y=0$ \---- which is parallel to $\ell$ \---- would contain both $B$ and the origin, hence be $\overset{\longleftrightarrow}{OB}$).  Since it is clear from Proposition 2.26 that $\overset{\longleftrightarrow}{OB}=\{(kx_2,ky_2):k\in\mathbb R\}$, we have $Q=\overset{\longleftrightarrow}{OB}\cap\ell=(rx_2,ry_2)$ for some $r\in\mathbb R$.  Since this point is in $\ell$, $x_1(rx_2)+y_1(ry)_2=x_1^2+y_1^2$, from which $r=\frac{x_1^2+y_1^2}{x_1x_2+y_1y_2}$ follows.\footnote{At the beginning of the paragraph we ascertained that the division is defined, by showing $x_1x_2+y_1y_2\ne 0$.}

Now, $\overline{OA}\perp\ell$, and $\overline{AQ}\subset\ell$.  Therefore, $\triangle OAQ$ is a right triangle with $\angle A$ the right angle.  If $O$ is not between $Q$ and $B$, then $\overset{\longrightarrow}{OB}=\overset{\longrightarrow}{OQ}$, so we may conclude $\angle AOB=\angle AOQ$.  By definition of $\cos$ (Exercise 10 of Section 2.1), $\cos(m\angle AOQ)=\frac{AO}{OQ}$.  Now, by the distance formula,
$$AO=\sqrt{x_1^2+y_1^2}$$
$$OQ=\sqrt{(rx_2)^2+(ry_2)^2}=|r|\sqrt{x_2^2+y_2^2}$$
Now $r\geqslant 0$ (because $r<0$ would easily entail that $QB=QO+OB$, and therefore $O$ is between $Q$ and $B$), hence the $|r|$ may be changed to just $r$.  We conclude that
$$\cos(m\angle AOB)=\frac{\sqrt{x_1^2+y_1^2}}{r\sqrt{x_2^2+y_2^2}}=\frac{(x_1x_2+y_1y_2)\sqrt{x_1^2+y_1^2}}{(x_1^2+y_1^2)\sqrt{x_2^2+y_2^2}}=\frac{x_1x_2+y_1y_2}{\sqrt{(x_1^2+y_1^2)(x_2^2+y_2^2)}}$$
concluding this case.

If $O$ is between $Q$ and $B$, we get $\cos(m\angle AOB)=-\frac{AO}{OQ}$ this time around (because $\angle AOB$ and $\angle AOQ$ are supplementary angles).  We leave it to the reader to verify that in this case $r$ (and hence $x_1x_2+y_1y_2$) is negative, and so $OQ=-r\sqrt{x_2^2+y_2^2}$ and cancellation of minus signs yields the result.
\end{proof}

\noindent\emph{Remark.} When $\vec v$ and $\vec w$ are nonzero vectors, and $\theta$ is the angle between them (when they are touching tail-to-tail), then it is learned in precalculus that $\cos\theta=\frac{\vec v\cdot\vec w}{\|\vec v\|\|\vec w\|}$.  This is exactly what has been proven in Proposition 2.27 for two-dimensional vectors with $\vec v=x_1\hat i+y_1\hat j$ and $\vec w=x_2\hat i+y_2\hat j$.  The proof can be adapted to apply to two vectors in $d$-dimensional space for any positive integer $d$, but in this case $\ell$ is a hyperplane given by an equation.  $d$-dimensional space will be further studied in Section 2.5.\\

\noindent After our work on distances and anles, triangles are self-explanatory, and will be studied more in the next section anyway.  Thus, we shall go straight into circles.

We recall that if $P$ is a point and $r$ is a positive real number, the circle centered at $P$ with radius $r$ is the set of points $A$ such that $AP=r$.  If $P=(x_c,y_c)$ and $A=(x,y)$, then by the distance formula (Proposition 2.24(c)), $AP=\sqrt{(x-x_c)^2+(y-y_c)^2}$.  Hence, saying $AP=r$ is tantamount to saying that $(x-x_c)^2+(y-y_c)^2=r^2$.  Thus,
\begin{center}
The equation for the circle with radius $r$ centered at $P=(x_c,y_c)$ is $(x-x_c)^2+(y-y_c)^2=r^2$.
\end{center}
We can actually use this to prove that a circle and a line intersect in at most two points.  Let $\omega$ be the circle with center $(x_c,y_c)$ and radius $r$, and let $\ell$ be the line $ax+by=c$.  Then if $(x,y)\in\omega\cap\ell$ then
$$(x-x_c)^2+(y-y_c)^2=r^2$$
$$ax+by=c$$
If $b\ne 0$, we can derive $y=\frac{c-ax}b$ from the second equation.  Moreover, we can substitute this into the first equation:
$$(x-x_c)^2+\left(\frac{c-ax}b-y_c\right)^2=r^2$$
Basic algebra shows that this is a quadratic equation in $x$, where $a,b,c,x_c,y_c,r$ are regarded as constants.  Thus there are most two possible values of $x$ for the intersection point.  Given $x$, however, $y$ is determined (because $y=\frac{c-ax}b$), so that the circle and line intersect in at most two points.  If $b=0$, then $a\ne 0$ and the argument can be repeated with $x=\frac{c-by}a$.

Similarly, two distinct circles intersect in at most two points.  For, if $\omega$ is the circle with center $(x_c,y_c)$ and radius $r$, and $\omega'$ is the circle with center $(x_c',y_c')$ and radius $r'$, an intersection point $(x,y)\in\omega\cap\omega'$ must satisfy:
$$(x-x_c)^2+(y-y_c)^2=r^2$$
$$(x-x_c')^2+(y-y_c')^2=(r')^2$$
If $x_c=x_c'$ and $y_c=y_c'$, then $r\ne r'$ (because the circles are distinct); in this case the equations are incompatible and there are no intersection points.  If either $x_c\ne x_c'$ or $y_c\ne y_c'$, subtracting the equations gives
$$2(x_c'-x_c)x+(x_c^2-(x_c')^2)+2(y_c'-y_c)y+(y_c^2-(y_c')^2)=r^2-(r')^2$$
What is essential is that the $x^2$ and $y^2$ terms have both dropped out.  This is the equation for the line $ax+by=c$ where $a=2(x_c'-x_c)$, $b=2(y_c'-y_c)$ and $c=r^2-(r')^2-(x_c^2-(x_c')^2)-(y_c^2-(y_c')^2)$.  Thus any intersection point of the circles is also an intersection point of either circle and the line, so therefore there are at most two.

The reader should take the time to look through the axioms in Chapter 2.1 and try to prove that the concrete plane $\mathbb R^2$ satisfies each one.  Thus our axiomatic plane geometry can be realized in an algebraic setting.

\subsection*{Exercises 2.2. (Giving the Plane Coordinates)}
\begin{enumerate}
\item Let $P=(x_0,y_0)$, and let $\ell$ be the line given by $ax+by=c$ ($a,b$ not both zero).  Show that the distance from $P$ to $\ell$ (Exercise 3 of the previous section) is $\frac{|ax_0+by_0-c|}{\sqrt{a^2+b^2}}$.

\item Let $\mathcal T:\mathbb R^2\to\mathbb R^2$ be a function that preserves distances, i.e., whenever $\mathcal T(A)=A'$ and $\mathcal T(B)=B'$ we have $A'B'=AB$.

(a) $\mathcal T$ preserves lines.  [If $A$ and $B$ are distinct points, then for any $C$, the segment addition postulate and Exercise 11(b) of Section 2.1 jointly imply that $C\in\overset{\longleftrightarrow}{AB}$ if and only if one of $AB,BC,AC$ is equal to the sum of the other two.]

(b) $\mathcal T$ preserves line segments and rays.

(c) $\mathcal T$ preserves angles.  [If $\triangle ABC$ is a triangle, and $\mathcal T(A)=A',\mathcal T(B)=B',\mathcal T(C)=C'$, then $\triangle ABC\cong\triangle A'B'C'$ by SSS congruence.]

Such a $\mathcal T$ is called an \textbf{isometry}; these will be studied in more detail in Section 2.6.

\item\emph{(Some more trigonometry.)} \---- Recall the functions $\sin,\cos,\tan$ from Exercise 10 of Section 2.1.  Exercises (a)-(g) assume that $\alpha,\beta$ and any other angles that the trigonometric functions are applied to are strictly between $0^\circ$ and $90^\circ$.

(a) $\cos(\alpha-\beta)=\cos\alpha\cos\beta+\sin\alpha\sin\beta$. [Apply Proposition 2.27 to $A=(\cos\alpha,\sin\alpha)$ and $B=(\cos\beta,\sin\beta)$, first noting that $\overline{OA}$ is a line segment of length $1$ making an angle of $\alpha$ with the $x$-axis.]

(b) $\sin(\alpha+\beta)=\sin\alpha\cos\beta+\cos\alpha\sin\beta$. [By Exercise 10(c) of Section 2.1, $\sin(\alpha+\beta)=\cos((90^\circ-\alpha)-\beta)$; now use part (a).]

(c) $\sin(\alpha-\beta)=\sin\alpha\cos\beta-\cos\alpha\sin\beta$.  [Write $\sin\alpha=\sin((\alpha-\beta)+\beta)$.  Then expand using parts (a) and (b), and consider using Exercise 10(a) of Section 2.1.]

(d) $\cos(\alpha+\beta)=\cos\alpha\cos\beta-\sin\alpha\sin\beta$.

(e) $\sin(2\alpha)=2\sin\alpha\cos\alpha$ and $\cos(2\alpha)=\cos^2\alpha-\sin^2\alpha$.

(f) $\cos(\alpha/2)=\sqrt{\frac{1+\cos\alpha}2}$ and $\sin(\alpha/2)=\sqrt{\frac{1-\cos\alpha}2}$.  [Use (e) and Exercise 10(a) of Section 2.1.]

(g) $\tan(\alpha+\beta)=\frac{\tan\alpha+\tan\beta}{1-\tan\alpha\tan\beta}$ and $\tan(\alpha-\beta)=\frac{\tan\alpha-\tan\beta}{1+\tan\alpha\tan\beta}$.  Also, $\tan(2\alpha)=\frac{2\tan\alpha}{1-\tan^2\alpha}$ and $\tan(\alpha/2)=\sqrt{\frac{1-\cos\alpha}{1+\cos\alpha}}=\frac{\sin\alpha}{1+\cos\alpha}$.  [Use Exercise 10(b) of Section 2.1.]

(h) Using the results from Exercise 10(e) of Section 2.1, find $\sin\alpha$, $\cos\alpha$, $\tan\alpha$ for $\alpha=15^\circ,75^\circ$.

We extend the trigonometric functions to all real numbers through the following defining properties.  $\sin 0^\circ=0$, $\cos 0^\circ=1$, $\sin(\alpha+90^\circ)=\cos\alpha$, $\cos(\alpha+90^\circ)=-\sin\alpha$, $\tan\alpha=\frac{\sin\alpha}{\cos\alpha}$.  Observe that $\sin,\cos$ are periodic with a period of $360^\circ$ (in other words $\sin(\alpha+360^\circ)=\sin\alpha$ for all $\alpha$), and $\tan$ is periodic with a period of $180^\circ$.  However, $\tan$ is not defined at the numbers $(180k+90)^\circ$ for $k\in\mathbb Z$: the cosine of such numbers is zero.

(i) Do parts (a)-(c) of Exercise 10 of Section 2.1, and parts (a)-(g) of this problem, for these extended trigonometric functions.  [For (f) and (g) of this problem, the square roots could be either positive or negative.]

(j) Show that $\sin(-\alpha)=-\sin\alpha$, $\cos(-\alpha)=\cos\alpha$ and $\tan(-\alpha)=-\tan\alpha$.  Also show that $\tan(\alpha+90^\circ)=-\frac 1{\tan\alpha}$.

\item\emph{(Sine and cosine laws.)} \---- Suppose $\triangle ABC$ is a triangle.  Let $a=BC,b=CA,c=AB$, as shown below.
\begin{center}\includegraphics[scale=.4]{Triangle_abc.png}\end{center}
(Note that $a$ is the length of the side opposite vertex $A$, etc.)

(a) Show that $\frac a{\sin m\angle A}=\frac b{\sin m\angle B}=\frac c{\sin m\angle C}$.  This is known as the \textbf{law of sines}.  [If $\overline{AP}$ is the altitude from vertex $A$ (i.e., $P\in\overset{\longleftrightarrow}{BC}$ and $\overline{AP}\perp\overline{BC}$), explain why $AP=b\sin m\angle C=c\sin m\angle B$.]

(b) Let $\omega$ be the circumscribed circle of the triangle (Exercise 17(b) of Section 2.1), and let $O$ be the center and $r$ be the radius of $\omega$.  Then $\frac a{\sin m\angle A}=2r$.  [If $A'$ is the point $\ne B$ where $\overset{\longrightarrow}{BO}$ meets $\omega$, then $\angle BA'C\cong\angle BAC$ by Exercise 19 of Section 2.1.  Thus the problem reduces to showing that $\frac a{\sin m\angle BA'C}=2r$.  To do this, note that $\overline{A'B}$ is a diameter of the circle, and show that $\triangle A'BC$ is a right triangle with $\angle C$ the right angle.]

Since the same argument will show that $\frac b{\sin m\angle B}=2r$, etc., this actually proves the law of sines along with a stronger fact.

(c) Show that $c^2=a^2+b^2-2ab\cos m\angle C$.  This is known as the \textbf{law of cosines}.  [First suppose $m\angle B,m\angle C<90^\circ$.  If $\overline{AD}$ is the altitude from vertex $A$, let $m=CD$, $d=AD$ and $n=DB$.  Then by the Pythagorean Theorem and segment addition postulate, $d^2+m^2=b^2$, $d^2+n^2=c^2$ and $m+n=a$.  Hence $c^2=d^2+n^2=b^2-m^2+n^2=b^2+(n+m)(n-m)=b^2+a(a-2m)=a^2+b^2-2am$.  Yet $m=b\cos m\angle C$ (why?).  Similar arguments can be used for $m\angle B$ or $m\angle C\geqslant 90^\circ$.]  Note that since $\cos 90^\circ=0$, the Pythagorean Theorem is the special case where $\angle C$ is a right angle.

(d) Use these results to conclude that if $a,b,c$ are \emph{any} positive real numbers such that $a+b>c$, $b+c>a$ and $c+a>b$, there exists a triangle \---- unique up to congruence \---- with side lengths $a,b,c$.

\item\emph{(Euler's line.)} \---- Let $\triangle ABC$ be a triangle.  We shall show that the orthocenter, centroid and circumcenter are all on one line.  This line is called \textbf{Euler's line}.

Since the translations $(x,y)\mapsto(x+u,y+v)$ preserve distances and hence also (by Exercise 2) lines and angles, they preserve all the constructions we will deal with in this problem.  Thus, we may apply one to this triangle to assume its circumcenter is the origin $O$.  With that, $\overline{OA}\cong\overline{OB}\cong\overline{OC}$.  Thus, if $A=(x_1,y_1),B=(x_2,y_2),C=(x_3,y_3)$ then $x_1^2+y_1^2=x_2^2+y_2^2=x_3^2+y_3^2$.

(a) Show that the centroid of $\triangle ABC$ is $\left(\frac{x_1+x_2+x_3}3,\frac{y_1+y_2+y_3}3\right)$.  [First verify that the midpoint of $\overline{BC}$ is $\left(\frac{x_2+x_3}2,\frac{y_2+y_3}2\right)$, then form the line segment from this point to $A$ to get a median.  Show that this median contains $\left(\frac{x_1+x_2+x_3}3,\frac{y_1+y_2+y_3}3\right)$.  Then observe that the argument can be repeated with the roles of $A,B,C$ exchanged.]

(b) Show that the orthocenter of $\triangle ABC$ is $\left(x_1+x_2+x_3,y_1+y_2+y_3\right)$.  [Let $P=\left(x_1+x_2+x_3,y_1+y_2+y_3\right)$.  Then $\overset{\longleftrightarrow}{AP}$ is given by the equation $(y_2+y_3)x-(x_2+x_3)y=(y_2+y_3)x_1-(x_2+x_3)y_1$.  Also, verify that $\overset{\longleftrightarrow}{BC}$ is the line $(y_3-y_2)x-(x_3-x_2)y=y_3x_2-x_3y_2$.  Now let $\ell_1$ be the line $(y_2+y_3)x-(x_2+x_3)y=0$ and $\ell_2$ the line $(y_3-y_2)x-(x_3-x_2)y=0$; these are lines through the origin parallel to $\overset{\longleftrightarrow}{AP}$ and $\overset{\longleftrightarrow}{BC}$ respectively.  If $R=(x_2+x_3,y_2+y_3)$ and $S=(x_3-x_2,y_3-y_2)$, observe that $R\in\ell_1,S\in\ell_2$; then use Proposition 2.27 to show that $\angle ROS$ is a right angle.  Conclude that $\ell_1\perp\ell_2$, and hence $\overline{AP}\perp\overline{BC}$ and $P$ is on the altitude from $A$.]

(c) Conclude that if $O$ is the circumcenter, $E$ the centroid and $H$ the orthocenter, then $O,E,H$ all lie on one line, and $OE=\frac 12(EH)$.

\item\emph{(Feuerbach's 9-point circle.)} \---- Let $\triangle ABC$ be a triangle.  Consider the following nine points:

[1] The foot of each altitude (i.e., the point where the altitude from each vertex intersects the opposite side);

[2] The midpoint of each side (i.e., the foot of each median);

[3] The midpoint of each line segment connecting a vertex to the orthocenter.

Our goal is to prove that these nine points all lie on the arc of one circle.  This circle is called \textbf{Feuerbach's 9-point circle}.

By using a translation as in Exercise 2, we may assume that the orthocenter of $\triangle ABC$ is the origin $O$.  Now suppose $\overline{AA'},\overline{BB'},\overline{CC'}$ are the altitudes of the triangle.  Finally, let $\omega$ be the circumscribed circle of the triangle.

(a) Let $P$ be the point on $\overset{\longrightarrow}{OA'}$ such that $OP=2(OA')$; in this case $P\ne O$ and $A'P=A'O$.  Show that $P$ is on the arc of $\omega$.  [First explain why $\triangle BOA'\cong\triangle BPA'$, and use this to show that $\triangle BOC\cong\triangle BPC$.  Hence $\angle BPC\cong\angle BOC$ by CPCTC.  The fact that each altitude is perpendicular to the opposite side yields (please supply the details) that $\angle BOC$ and $\angle BAC$ are supplementary angles.  Hence $\angle BPC$ and $\angle BAC$ are supplementary angles; now use Exercise 19 of Section 2.1.]

(b) Now if $M$ is the midpoint of $\overline{BC}$, let $Q$ be the point on $\overset{\longrightarrow}{OM}$ such that $OQ=2(OM)$.  Show that $Q$ is on the arc of $\omega$.  [First show that $\triangle BOM\cong\triangle CQM$, then that $\triangle BOC\cong\triangle CQB$.  Then carry on as in part (a).]

(c) Now show that taking any of the nine points in [1]-[3], and multiplying both coordinates by $2$, yields a point on the arc of $\omega$.  [The foot $A'$ of altitude $\overline{AA'}$ yields the point $P$ of part (a).  The midpoint of $\overline{BC}$ yields the point $Q$ of part (b).  And the points in [3] yield the vertices of the triangle.]

(d) Conclude that if $\omega$ has center $(x_0,y_0)$ and radius $r$, then the nine points in [1]-[3] lie on the circle with center $(x_0/2,y_0/2)$ and radius $r/2$.

\item Let $\omega$ be a circle with radius $r$.  If $\overline{AB}$ is a chord, which intercepts an arc of angle $\alpha$, then $AB=r\sqrt{2-2\cos\alpha}$.  [If $M$ is the midpoint of $\overline{AB}$, then the line through $M$ perpendicular to $\overline{AB}$ \---- which is the perpendicular bisector \---- passes through $O$.  This implies $\triangle OMB$ is a right triangle with $\angle M$ the right angle.  Moreover, $\triangle OMB\cong\triangle OMA$ (why?); use this to show that $m\angle MOB=\alpha/2$.  Now use Exercise 3(f).]

\item\emph{(Stewart's theorem.)} \---- Let $\triangle ABC$ be a triangle, $D\in\overline{BC}$.  Then let $a=BC,b=CA,c=AB,d=AD,n=CD,m=DB$ as in the diagram below.  Note that $n+m=a$ by the segment addition postulate.
\begin{center}\includegraphics[scale=.4]{StewartsTheorem.png}\end{center}
Show that $a(mn+d^2)=b^2m+c^2n$.  [Let $\theta=m\angle ADC$ and $\varphi=m\angle ADB$.  By the law of cosines (Exercise 4(c)), $b^2=d^2+n^2-2dn\cos\theta$, and $c^2=d^2+m^2-2dm\cos\varphi$.  Yet since $\angle ADC,\angle ADB$ form a linear pair, they are supplementary angles, and therefore $\cos\varphi=-\cos\theta$.]

This useful statement in plane geometry has a clever mnemonic for memorization.  Indeed, it can be alternatively written as $man+dad=bmb+cnc$ \---- which can be remembered as ``a man and his dad put a bomb in the sink.''

\item\emph{(Conic sections.)} \---- (a) Let $a>b>0$ be fixed real numbers, and let $\gamma$ be the set of points $(x,y)$ such that $\frac{x^2}{a^2}+\frac{y^2}{b^2}=1$.  Then $\gamma$ is called an \textbf{ellipse}.  Observe that it has $(\pm a,0)$ and $(0,\pm b)$ as its leftmost, rightmost, highest and lowest points.  (Under the assumption that $a>b$), the line segment from $(-a,0)$ to $(a,0)$ is called the \textbf{major axis}, and the line segment from $(b,0)$ to $(-b,0)$ is called the \textbf{minor axis}.

Let $c=\sqrt{a^2-b^2}$.  Let $P$ be the point $(c,0)$ and $\ell$ the line $x=\frac{a^2}c$.  Show that a point $Q$ is on the ellipse if and only if the distance from $Q$ to $P$ is exactly $c/a$ times the distance from $Q$ to $\ell$.  [$P$ is called a \textbf{focus} (plural ``foci'') of the ellipse, and $\ell$ is called a \textbf{directrix} (plural ``directrices'').  The factor $c/a$ \---- which is a real number strictly between $0$ and $1$ \---- is called the \textbf{eccentricity} of the ellipse.]

(b) If $P'=(-c,0)$ and $\ell'$ is the line $x=-\frac{a^2}c$, it likewise follows that $Q$ is on the ellipse if and only if $QP'$ is $c/a$ times the distance from $Q$ to $\ell'$.  Use this to show that $PQ+P'Q=2a$ for every point $Q$ on the ellipse.

(c) Suppose $p>0$ is a fixed real number.  Let $\zeta$ be the set of points $(x,y)$ such that $x^2=4py$.  Then $\zeta$ is called a \textbf{parabola}.  Observe that $\zeta$ has $O=(0,0)$ as its lowest point, but it has no points farthest to the left, right or above.  (Indeed, for any real number $x$ whatsoever, there is a unique $y$ \---- namely $\frac{x^2}{4p}$ \---- such that $(x,y)\in\zeta$.  Moreover, these values of $y$ have no upper bound.)

If $P$ is the point $(0,p)$ and $\ell$ is the line $y=-p$, a point $Q$ is on the parabola if and only if the distance from $Q$ to $P$ is equal to the distance from $Q$ to $\ell$.  [$P$ is called the \textbf{focus} of the parabola, and $\ell$ is called its \textbf{directrix}.  Observe that the factor which relates the distances \---- which was $c/a$ in part (a) \---- is equal to $1$ this time.  Thus the eccentricity of any parabola is $1$.]

(d) Now let $a,b>0$ be fixed real numbers, and let $\eta$ be the set of points $(x,y)$ such that $\frac{y^2}{b^2}-\frac{x^2}{a^2}=1$.  $\eta$ is called a \textbf{hyperbola}.  It does not have ``farthest'' points in any direction, but it does have $(0,\pm b)$ as two \emph{local} highest and lowest points (because there are no points in $\eta$ with $-b<y<b$).  It is composed of two separate curves, one with each of these points.  The line going through the points $(0,\pm b)$ (which is the $y$-axis in this case) is called the \textbf{transverse axis}.

Note that $\frac{x^2}{a^2}-\frac{y^2}{b^2}=1$ also gives a hyperbola, but this time it has $(\pm a,0)$ as its local leftmost and rightmost points, and therefore the $x$-axis as its transverse axis.  However, in this problem, we shall specifically deal with the hyperbola $\eta$ (given by $\frac{y^2}{b^2}-\frac{x^2}{a^2}=1$).

Let $c=\sqrt{a^2+b^2}$.  Let $P$ be the point $(0,c)$ and $\ell$ the line $y=\frac{b^2}c$.  Show that a point $Q$ is on the hyperbola if and only if the distance from $Q$ to $P$ is exactly $c/b$ times the distance from $Q$ to $\ell$.  [$P$ is called a \textbf{focus} of the hyperbola and $\ell$ is called a \textbf{directrix}, just like the case for the ellipse.  The factor $c/b$ \---- which is a real number $>1$ this time \---- is called the \textbf{eccentricity}.]

(e) Adapt part (b) to show that if $P'=(0,-c)$, then for every point $Q$ on the hyperbola, $|PQ-P'Q|=2b$.
\end{enumerate}

\subsection*{2.3. Triangles, Polygons and Tilings}
\addcontentsline{toc}{section}{2.3. Triangles, Polygons and Tilings}
We recall triangles from the previous two sections.  They are structures obtained by taking three points and joining them with line segments in all possible ways.  They can be either acute, right or obtuse.  They can be either equilateral, isosceles or scalene.

First, we will study something that we have not considered in Sections 2.1 or 2.2 \---- namely, how much ``space'' is inside the triangle.  This is captured in a notion called \emph{area}, which is quite important throughout mathematics.  We will avoid delving into calculus in this section, and only deal with areas of basic figures.\\

\noindent\textbf{Definition.} \emph{Let $\triangle ABC$ be a triangle.  If the altitude from vertex $A$ meets $\overset{\longleftrightarrow}{BC}$ at point $P$, then the \textbf{area} of $\triangle ABC$ is defined to be $\frac 12(AP)(BC)$.}\\

\noindent Intuitively, $\overline{AP}$ is called the \textbf{height} of the triangle and $\overline{BC}$ is called the \textbf{base}.  The area can thus be remembered as ``half the base times the height.''

You may have noticed that this notion of area appears to be ambiguous, because it appears to depend on which vertex we take the altitude from.  However, we claim that this is not the case:\\

\noindent\textbf{Lemma 2.28.} \emph{Suppose $\triangle ABC$ is a triangle.  If the altitude from vertex $B$ meets $\overset{\longleftrightarrow}{AC}$ at point $D$, and the altitude from vertex $C$ meets $\overset{\longleftrightarrow}{AB}$ at point $E$, then $(BD)(AC)=(CE)(AB)$.  Therefore, the area of a triangle is well-defined.}
\begin{center}\includegraphics[scale=.3]{TriangleArea.png}\end{center}
\begin{proof}
$\angle BAD\cong\angle CAE$ (because they're the same angle) and $\angle ADB\cong\angle AEC$ (both are right angles).  By AA similarity, $\triangle ADB\sim\triangle AEC$.  By Proposition 2.16, $\frac{AB}{BD}=\frac{AC}{CE}$, hence $(BD)(AC)=(CE)(AB)$ as desired.
\end{proof}

\noindent Moreover, we need the following lemma to define areas of quadrilaterals and other kinds of figures.\\

\noindent\textbf{Lemma 2.29.} \emph{Suppose $\triangle ABC$ is a triangle.  If $D\in\overline{BC}$, as shown below, then $\operatorname{Area}(\triangle ABD)+\operatorname{Area}(\triangle ADC)=\operatorname{Area}(\triangle ABC)$.}
\begin{center}\includegraphics[scale=.3]{AreaAdding.png}\end{center}
\begin{proof}
If $\ell$ is the line through $A$ perpendicular to $\overset{\longleftrightarrow}{BC}$, and $P=\ell\cap\overset{\longleftrightarrow}{BC}$, we have, by definition,
$$\operatorname{Area}(\triangle ABD)=\frac 12(AP)(BD)$$
$$\operatorname{Area}(\triangle ADC)=\frac 12(AP)(DC)$$
$$\operatorname{Area}(\triangle ABC)=\frac 12(AP)(BC)$$
Yet $\frac 12(AP)(BD)+\frac 12(AP)(DC)=\frac 12(AP)(BD+DC)=\frac 12(AP)(BC)$ by the segment addition postulate; hence the statement in the lemma.
\end{proof}

\noindent\textbf{Definition}. \emph{If $n\geqslant 3$ is an integer, an \textbf{$n$-gon} (or \textbf{polygon} when $n$ is not specified) is defined to be an ordered $n$-tuple of points $A_1,A_2,\dots,A_n$ (called \textbf{vertices}) equipped with the $n$ line segments $\overline{A_1A_2},\dots,\overline{A_{n-1}A_n},\overline{A_nA_1}$ (called \textbf{edges} or \textbf{sides}), such that no two of the line segments intersect each other unless they share a vertex.}\\

\noindent Note that two line segments not intersecting does \emph{not} imply that they are parallel; the \emph{lines} that contain the segments could still intersect one another.

Some basic examples have already been covered in the chapter; e.g., if $n=3$, an $n$-gon is a triangle.  (In this case the last condition is superfluous because any two sides share a vertex.)  If $n=4$, an $n$-gon is a quadrilateral (Exercise 4 of Section 2.1).  The reader is assumed to be familiar with the names of polygons for other values of $n$ (\textbf{pentagon} for $n=5$, \textbf{hexagon} for $n=6$, \textbf{heptagon} for $n=7$, etc.)

Each vertex of an $n$-gon has its canonical angle (if $1<j<n$, the angle at $A_j$ is $\angle A_{j-1}A_jA_{j+1}$; the angle at $A_1$ is $\angle A_nA_1A_2$ and the angle at $A_n$ is $\angle A_{n-1}A_nA_1$).  A polygon is said to be \textbf{convex} if there exists a point $P$, which is not on any of the sides (nor equal to any of the vertices), and is simultaneously inside all $n$ angles.  The difference is illustrated below.
\begin{center}\includegraphics[scale=.5]{Convex.png}\end{center}
Throughout this book we will be restricting ourselves to convex polygons.  If $A_1A_2\dots A_n$ is a convex $n$-gon ($n\geqslant 4$), we define an \textbf{edge triangle} to be any of the following triangles:
$$\triangle A_nA_1A_2,~~~~\triangle A_{n-1}A_nA_1,~~~~\triangle A_{j-1}A_jA_{j+1}\text{ for }1<j<n.$$
Thus an edge triangle is the triangle obtained from three consecutive vertices.  Clearly it shares two sides with the $n$-gon; but the third side is new.  Moreover, we get a convex $(n-1)$-gon upon forgetting the middle vertex (for example, if $\triangle A_{j-1}A_jA_{j+1}$ is constructed, we get the $(n-1)$-gon $A_1\dots A_{j-1}A_{j+1}\dots A_n$).  The proof that this is really a convex $(n-1)$-gon is left to the reader.

We recall (Proposition 2.11) that the measures of the angles of a triangle add to $180^\circ$.  We are now in a position where we can add the measures of the angles of \emph{any} polygon:\\

\noindent\textbf{Proposition 2.30.} \emph{If $n\geqslant 3$, the measures of the angles of an $n$-gon add to $180(n-2)^\circ$.}\\

\noindent For instance, the measures of the angles of a pentagon add to $540^\circ$ (special case where $n=5$); the measures of the angles of a hexagon add to $720^\circ$.
\begin{proof}
We shall use induction on $n$.  The case $n=3$ has been covered in Proposition 2.11.

Now suppose inductively that $n\geqslant 4$, and the measures of the angles of an $(n-1)$-gon add to $180(n-3)^\circ$.  Let $A_1A_2\dots A_n$ be an $n$-gon.  Construct the edge triangle $\triangle A_1A_2A_3$ and the corresponding $(n-1)$-gon $A_1A_3\dots A_n$.  By Proposition 2.11, the angles of the triangle add to $180^\circ$, and by the induction hypothesis, the angles of the $(n-1)$-gon add to $180(n-3)^\circ$.  However, observe that the set of all angles of these two polygons altogether is obtained by taking the set of angles of the $n$-gon, and then splitting $\angle A_1$ and $\angle A_3$ each into two adjacent angles.  By the angle addition postulate, the sum of each pair of adjacent angles is equal to the original angle of the $n$-gon; thus the sum of the $n$ angles is also the sum of the $n+2$ angles of the triangle and $(n-1)$-gon.  Hence this sum is $180^\circ+180(n-3)^\circ=180(n-2)^\circ$.
\end{proof}

\noindent If $A_1A_2\dots A_n$ is an $n$-gon with $n\geqslant 4$, we define its \textbf{area} \---- through recursion on $n$ \---- via taking an edge triangle and adding its area to the area of the induced $(n-1)$-gon:
$$\operatorname{Area}(A_1A_2\dots A_n)=\operatorname{Area}(\triangle A_1A_2A_3)+\operatorname{Area}(A_1A_3\dots A_n).$$
As in Lemma 2.28, we must show that this is well-defined, and independent of the particular edge triangle chosen.\\

\noindent\textbf{Lemma 2.31.} \emph{If $n\geqslant 3$, then the area of an $n$-gon is well-defined.}
\begin{proof}
The case $n=3$ was proved in Lemma 2.28.

We first tackle the case $n=4$.  Suppose $ABCD$ is a quadrilateral.  Then the recursive definition has only two possible choices:\footnote{We use the notation ``$:=$''\----which indicates setting a variable to a value in computer science\----so that the reader does not assume $\operatorname{Area}(\triangle ABC)+\operatorname{Area}(\triangle CDA)=\operatorname{Area}(\triangle BCD)+\operatorname{Area}(\triangle DAB)$ a priori.}
$$\operatorname{Area}(ABCD):=\operatorname{Area}(\triangle ABC)+\operatorname{Area}(\triangle CDA);$$
$$\operatorname{Area}(ABCD):=\operatorname{Area}(\triangle BCD)+\operatorname{Area}(\triangle DAB);$$
and we must show that they give the same value.  Let $P=\overline{AC}\cap\overline{BD}$ (this point exists by the convexity).  Then $\operatorname{Area}(\triangle ABC)=\operatorname{Area}(\triangle ABP)+\operatorname{Area}(\triangle PBC)$ by Lemma 2.29.  Similar summations can be derived for the other three triangles on the right-hand sides of the above equations.  We wind up concluding that both sums are equal to $\operatorname{Area}(\triangle APB)+\operatorname{Area}(\triangle BPC)+\operatorname{Area}(\triangle CPD)+\operatorname{Area}(\triangle DPA)$, thus they are equal.

Now (using complete induction), suppose $n\geqslant 5$ and the area of a $k$-gon is well-defined for all $3\leqslant k<n$.  Suppose the area of an $n$-gon $A_1\dots A_n$ is defined using the edge triangle $\triangle A_1A_2A_3$.  Then it suffices to show that the area of $A_1\dots A_n$ is invariant under changing the edge triangle to the adjacent one $\triangle A_2A_3A_4$; as every edge triangle can be obtained by starting at $\triangle A_1A_2A_3$ and switching to an adjacent edge triangle a finite number of times.  In other words, we need only show
$$\operatorname{Area}(\triangle A_1A_2A_3)+\operatorname{Area}(A_1A_3\dots A_n)=\operatorname{Area}(\triangle A_2A_3A_4)+\operatorname{Area}(A_1A_2A_4\dots A_n).$$
The trick is to consider quadrilateral $A_1A_2A_3A_4$.  By the previous case where $n=4$, we get
\begin{equation}\tag{*}
\operatorname{Area}(\triangle A_1A_2A_3)+\operatorname{Area}(\triangle A_3A_4A_1)=\operatorname{Area}(\triangle A_2A_3A_4)+\operatorname{Area}(\triangle A_4A_1A_2).
\end{equation}
Also, by the definition of area,
$$\operatorname{Area}(A_1A_3\dots A_n)=\operatorname{Area}(\triangle A_1A_3A_4)+\operatorname{Area}(A_1A_4\dots A_n)$$
$$\operatorname{Area}(A_1A_2A_4\dots A_n)=\operatorname{Area}(\triangle A_1A_2A_4)+\operatorname{Area}(A_1A_4\dots A_n)$$
(the area of the $(n-2)$-gon $A_1A_4\dots A_n$ can be taken, since we are using \emph{complete} induction).  Thus, adding $\operatorname{Area}(A_1A_4\dots A_n)$ to both sides of (*) gives the desired statement.
\end{proof}
\noindent At this point, the reader should be able to find certain areas easily.  For example, if $ABCD$ is a quadrilateral with four right angles (which implies $AB=CD$ and $BC=DA$ by Exercise 6 of Section 2.1), then the area of $ABCD$ is $(AB)(BC)$.  [This follows because the area of a right triangle is one-half the product of the lengths of its legs.]  Such a quadrilateral is called a \textbf{rectangle}.

Notice that the angle measures of an arbitrary polygon do \emph{not} determine the side lengths up to ratios (unless $n=3$).  For example, the rectangle can have its side lengths in any ratio whatsoever \---- just let $a,b>0$ be fixed real numbers and set $A=(0,0),B=(0,b),C=(a,b),D=(a,0)$.

Likewise the side lengths do not determine the angle measures.  If $0^\circ<\alpha<180^\circ$, then the points $A=(0,0),B=(1,0),C=(1+\cos\alpha,\sin\alpha),D=(\cos\alpha,\sin\alpha)$ determine a quadrilateral $ABCD$ where all four side lengths equal $1$.  (This quadrilateral is called a \textbf{rhombus}.)  However, the fact that the side lengths equal $1$ does not determine the measure $\alpha$ (which can be taken arbitrarily).  We thus introduce several new notions.\\

\noindent\textbf{Definition.} \emph{An $n$-gon $A_1A_2\dots A_n$ is said to be:}

(i) \emph{\textbf{Equilateral} if $\overline{A_1A_2}\cong\dots\cong\overline{A_{n-1}A_n}\cong\overline{A_nA_1}$;}

(ii) \emph{\textbf{Equiangular} if $\angle A_1\cong\angle A_2\cong\dots\cong\angle A_n$;}

(iii) \emph{\textbf{Regular} if it is both equilateral and equiangular.}
\begin{center}\includegraphics[scale=.4]{EquiRegular.png}\end{center}

\noindent Note by the way, that if $n=3$ then (i) $\iff$ (ii) by Proposition 2.15.  Since (iii) is the conjunction of (i) and (ii), we conclude that every equilateral triangle is regular, and also that every equiangular triangle is regular.  However, this is not true for polygons with more sides: for instance, a rhombus satisfies (i) but not (ii), and a rectangle satisfies (ii) but not (i).

It turns out that if $A_1A_2\dots A_n$ is an equiangular $n$-gon, then we can determine the (common) angle measure.  Let $\alpha=m\angle A_1=\dots=m\angle A_n$.  Then Proposition 2.30 entails $n\alpha=180(n-2)^\circ$.  Therefore, $\alpha=\left(\frac{180(n-2)}n\right)^\circ$.  Thus we've proven that
\begin{center}
\textbf{If $A_1A_2\dots A_n$ is an equiangular $n$-gon, then each of its angles measures $\left(\frac{180(n-2)}n\right)^\circ$.}
\end{center}
In particular, a regular $n$-gon has an interior angle measure of $\left(\frac{180(n-2)}n\right)^\circ$.  We can take some sample values of $n$ to see that an equilateral triangle has angle measure $60^\circ$, a regular quadrilateral (henceforth called a \textbf{square}) has angle measure $90^\circ$, a regular pentagon has angle measure $108^\circ$, and a regular hexagon has angle measure $120^\circ$.\\

\noindent\textbf{CONGRUENCE AND SIMILARITY}\\

\noindent As in the case for triangles, $n$-gons $A_1\dots A_n$ and $A'_1\dots A'_n$ are said to be \textbf{congruent} (denoted $A_1\dots A_n\cong A'_1\dots A'_n$) if $\overline{A_jA_{j+1}}\cong\overline{A'_jA'_{j+1}}$ for $1\leqslant j<n$, $\overline{A_nA_1}\cong\overline{A'_nA'_1}$, and $\angle A_j\cong\angle A'_j$ for $1\leqslant j<n$.  And $A_1\dots A_n$ and $A'_1\dots A'_n$ are said to be \textbf{similar} (denoted $A_1\dots A_n\sim A'_1\dots A'_n$) if $\frac{A_1A_2}{A'_1A'_2}=\frac{A_2A_3}{A'_2A'_3}=\dots=\frac{A_{n-1}A_n}{A'_{n-1}A'_n}=\frac{A_nA_1}{A'_nA'_1}$, and $\angle A_j\cong\angle A'_j$ for $1\leqslant j<n$.

Most of the congruence and similarity theorems for triangles do not hold for arbitrary polygons.  However, since a regular $n$-gon has angle measure $\left(\frac{180(n-2)}n\right)^\circ$ \---- which depends only on $n$ \---- it follows that any two regular $n$-gons are similar.\\

\noindent\textbf{CENTER OF A REGULAR POLYGON AND CIRCLES}\\

\noindent Now let $A_1\dots A_n$ be a regular $n$-gon.  Let $\alpha=\left(\frac{180(n-2)}n\right)^\circ$, the angle measure of the polygon.

Suppose, for each $1\leqslant j\leqslant n$, $\ell_j$ is the angle bisector of $\angle A_j$.  We claim these angle bisectors all meet at one common point: let $O$ be $\ell_1\cap\ell_2$, the intersection of the angle bisectors of $\angle A_1$ and $\angle A_2$.  Then $\angle OA_1A_2$ and $\angle OA_2A_1$ both have measure $\alpha/2$ (why?).  Hence, $\angle OA_1A_2\cong\angle OA_2A_1$, thus $\overline{OA_1}\cong\overline{OA_2}$ by Proposition 2.15.  But $m\angle OA_2A_3=\alpha/2$ as well, and therefore $\angle OA_1A_2\cong\angle OA_2A_3$.  Finally, $\overline{A_1A_2}\cong\overline{A_2A_3}$ since this is a regular polygon.  By SAS congruence, $\triangle OA_1A_2\cong\triangle OA_2A_3$; by CPCTC $\angle OA_3A_2\cong\angle OA_2A_1$ and therefore $m\angle OA_3A_2=\alpha/2$ and $\overset{\longleftrightarrow}{OA_3}$ is the angle bisector of $\angle A_3$ (hence $O\in\ell_3$).

Now repeat this argument: $m\angle OA_2A_3=m\angle OA_3A_2=m\angle OA_3A_4$, and therefore $\overline{OA_2}\cong\overline{OA_3}$, and $\overline{A_2A_3}\cong\overline{A_3A_4}$, therefore $\triangle OA_2A_3\cong\triangle OA_3A_4$ and $m\angle OA_4A_3=\alpha/2$, making $\overset{\longleftrightarrow}{OA_4}$ the angle bisector of $\angle A_4$, so that $O\in\ell_4$.  The argument repeats successively and shows that $O\in\ell_j$ for all $j$.  This point $O$ is unique and is called the \textbf{center} of the regular polygon.

Since the argument above shows $\overline{OA_1}\cong\overline{OA_2}\cong\dots\cong\overline{OA_n}$, we can let $r$ be the common value of these line segments.  Then the circle centered at $O$ with radius $r$ \emph{contains all the vertices of the polygon} (why?).  This circle is called the \textbf{circumscribed circle} of the regular polygon.

On the other hand, for each $1\leqslant j<n$ let $A'_j$ be the midpoint of $\overline{A_1A_2}$; then let $A'_n$ be the midpoint of $\overline{A_nA_1}$.  Then by SSS congruence, $\triangle OA'_1A_1\cong\triangle OA'_1A_2$, so that $\angle OA'_1A_1\cong\angle OA'_1A_2$.  Since these angles form a linear pair, they are right angles.  Moreover, it can be shown that $\overline{OA'_1}\cong\overline{OA'_2}\cong\dots\cong\overline{OA'_n}$: for starters, since $\angle OA_1A_2\cong\angle OA_2A_3$, we have $\angle OA_1A'_1\cong\angle OA_2A'_2$.  Also $\overline{OA_1}\cong\overline{OA_2}$ and $\overline{A_1A'_1}\cong\overline{A_2A'_2}$ (both have length one-half the side length of the polygon).  By SAS congruence $\triangle OA_1A'_1\cong\triangle OA_2A'_2$; hence $\overline{OA'_1}\cong\overline{OA'_2}$ follows from CPCTC.  Repeating this argument shows the rest of the $\overline{OA'_j}$ to be congruent as well.

If $r'$ is the common value of the segments $\overline{OA'_j}$, then the circle centered at $O$ with radius $r'$ is tangent to every side of the polygon (because $\overline{OA'_j}\perp\overline{A_jA_{j+1}}$).  This is called the \textbf{inscribed circle} of the regular polygon.

Thus, every regular polygon (like arbitrary triangles) has a circumscribed circle and an inscribed circle.  However, arbitrary polygons need not have them.  For instance, a quadrilateral has a circumscribed circle if and only if it is cyclic (Exercise 19 of Section 2.1).

Going in the other direction, let $n$ be an integer $\geqslant 3$ and $\omega$ a circle with center $O$.  We construct $n$ radii of $\omega$ as follows: Let $\overline{OA_1}$ be an arbitrary radius.  If $\alpha=(360/n)^\circ$, then by Axiom 2.6(i) there exists a ray with endpoint $O$ making an angle of $\alpha$ with $\overset{\longrightarrow}{OA_1}$.  Suppose it meets $\omega$ at $A_2$.  Now for $\overset{\longrightarrow}{OA_2}$ there are \emph{two} rays coming from $O$ making an angle of $\alpha$ \---- one of them is $\overset{\longrightarrow}{OA_1}$, so let the other one meet $\omega$ at $A_3$.  Likewise, let the ray making an angle of $\alpha$ from $\overset{\longrightarrow}{OA_3}$ (other than $\overset{\longrightarrow}{OA_2}$) meet $\omega$ at $A_4$.  Continue this process to define $A_5,\dots,A_n$.

Since $\alpha=m\angle A_1A_2=m\angle A_2A_3=\dots=m\angle A_{n-1}A_n$, but also $m\angle A_1A_2+\dots+m\angle A_{n-1}A_n+m\angle A_nA_1=360^\circ$ (the angles close around a point), we have that $m\angle A_nA_1=\alpha$ as well.  The case $n=8$ is illustrated down below.
\begin{center}\includegraphics[scale=.4]{WheelSpokes.png}\end{center}
With that, it follows that $A_1\dots A_n$ is a regular $n$-gon: Since $\angle A_1OA_2\cong\angle A_2OA_3$ (both have measure $\alpha$) and $\overline{OA_1}\cong\overline{OA_2}\cong\overline{OA_3}$ (they are radii), we have $\triangle A_1OA_2\cong\triangle A_2OA_3$ by SAS, and therefore $\overline{A_1A_2}\cong\overline{A_2A_3}$.  Similar arguments show all the triangles $\triangle A_jOA_{j+1}$ ($1\leqslant j<n$) and $\triangle A_nOA_1$ are congruent, and hence all sides of the $n$-gon to be congruent.  Moreover, the congruence $\triangle A_nOA_1\cong\triangle A_1OA_2\cong\triangle A_2OA_3$ shows that, by CPCTC,
\begin{align*}
m\angle A_nA_1A_2 & =m\angle A_nA_1O+\angle OA_1A_2 \\
& =m\angle A_1A_2O+\angle OA_2A_3 = m\angle A_1A_2A_3
\end{align*}
so that $\angle A_nA_1A_2\cong\angle A_1A_2A_3$; and similar arguments show all $n$ angles of the $n$-gon to be congruent.

Since $A_1,\dots,A_n$ are all in $\omega$, $\omega$ is the circumscribed circle of the $n$-gon and $O$ is the $n$-gon's center.  Similarly, we get a regular polygon for which $\omega$ is the \emph{inscribed} circle by taking the line tangent to $\omega$ at each $A_j$.  We leave the verifications to the reader.\\

\noindent\textbf{TILINGS}\\

\noindent We conclude this section by showing how polygons can tile the plane.  Polygonal tilings are seen many places, such as on bathroom floors, or in bee's honeycomb.  By a \emph{tiling} we mean that the polygons along with their interiors\footnote{The interior, of course, means the set of points inside all angles of the polygon.  These points exist as long as the polygon is convex.} unite to form the plane, and that the intersection of any two polygons is either (i) empty, (ii) a single vertex of each one or (iii) an entire edge of each one.  (For example, we do not allow the vertex of any polygon to be in the middle of an edge of another, nor do we allow any two polygons to have overlapping interior.)  The polygons are called the \textbf{faces} of the tiling, and the edges (resp., vertices) of the polygons are called the \textbf{edges} (resp., \textbf{vertices}) of the tiling.

The first question is, if $n$ is an integer $\geqslant 3$, when can the plane be tiled by just regular $n$-gons?  For starters, the $n$-gons are all congruent (as we will see).  Every vertex must have at least three faces around it (because the angle measures are strictly $<180^\circ$).  Let $\alpha=\left(\frac{180(n-2)}n\right)^\circ$, the interior angle measure of an $n$-gon.  Let $k$ be the number of faces ($n$-gons) around a certain vertex; then the angles around that vertex \---- each of which measures $\alpha$ \---- must add to a total of $360^\circ$ to circulate the point.  This entails $k\alpha=360^\circ$.  Thus the tiling will only exist if $\alpha$ is an exact divisor of $360^\circ$.

If $n=3$ then $\alpha=60^\circ$; in this case $360^\circ=6\alpha$, so the tiling will work with $6$ equilateral triangles around each vertex; see below.  If $n=4$ then $\alpha=90^\circ$, so that $360^\circ=4\alpha$ and one can fit $4$ squares around each vertex.  If $n=5$ then $\alpha=108^\circ$: in this case the tiling does not exist, because $k\alpha=360^\circ$ implies $3<k<4$, which means that three regular pentagons will not be enough to circulate the vertex, and a fourth pentagon will cause an overlap (of interiors).  If $n=6$ then $\alpha=120^\circ$, so $3$ regular hexagons can be fit around each vertex.

Finally, there is never a tiling for $n\geqslant 7$, because in this case $\alpha>120^\circ$, making it impossible to have three faces around a vertex.  Thus the plane can be tiled by regular $n$-gons alone if and only if $n=3$, $4$ or $6$:
\begin{center}\includegraphics[scale=.4]{Tilings1.png}\end{center}
Though these are the only tilings involving just \emph{one} type of regular polygon; there are many other tilings involving two or more types of regular polygons; the first two tilings below are examples.
\begin{center}\includegraphics[scale=.4]{Tilings2.png}\end{center}
The first of these tilings is made up of two distinct types of regular polygons, a square and an octagon; and the second is made up of three distinct types of regular polygons, a triangle, a square and a hexagon.  The third tiling, however, is made up of pentagons \---- not \emph{regular} pentagons; rather, congruent copies of the pentagon with vertices $(-1,0),(1,0),(2,2),(0,3),(-2,2)$.  It is worth noting that the first two tilings have the same pattern of faces around each vertex (e.g., the first tiling has two octagons and one square at each vertex); and each vertex of the third tiling has either four right angles or three angles, and every three-angle vertex has the same angle measures. %% About the bibliography thing, I don't see myself needing to add a source to the biblio, if I didn't use it to write this nor was I aware of it.  The Giant Golden Book of Mathematics is the main book where I remember learning the regular tilings, if that is worth citing.

\subsection*{Exercises 2.3. (Triangles, Polygons and Tilings)}
\begin{enumerate}
\item Let $\triangle ABC$ be a triangle.

(a) Show that its area is equal to $\frac 12(AB)(AC)\sin m\angle A$.  [Let the altitude from vertex $B$ meet $\overset{\longleftrightarrow}{AC}$ at point $P$.  Then explain why $BP=AB\sin m\angle A$.]

(b) If $A=(0,0)$, $B=(x_1,y_1)$ and $C=(x_2,y_2)$, show that the area is equal to $\frac 12\big|x_1y_2-x_2y_1\big|$.  [Use part (a) and Exercise 3(c) of Section 2.2.]

\item If $\triangle ABC$ is a triangle with $A=(x_1,y_1),B=(x_2,y_2),C=(x_3,y_3)$, show that the area of $\triangle ABC$ is
$$\frac 12\big|x_1(y_2-y_3)+x_2(y_3-y_1)+x_3(y_1-y_2)\big|$$
[Verify that the above expression is invariant under translations (i.e., $(x,y)\mapsto(x+u,y+v)$ for $u,v$ fixed).  Use this to assume $A=(0,0)$.]

\item\emph{(Sine sum formula.)} \---- Here is another proof that if $\alpha,\beta$ are angles, $\sin(\alpha+\beta)=\sin\alpha\cos\beta+\cos\alpha\sin\beta$.
\begin{center}\includegraphics[scale=.4]{SineSum2.png}\end{center}
Let $\triangle ADB$ and $\triangle ADC$ be right triangles, adjacent as above, with $m\angle CAD=\alpha$ and $m\angle DAB=\beta$.

(a) Show that the area of $\triangle ADC$ is $\frac 12ab\sin\alpha\cos\beta$.  [First explain why the area of a right triangle is one-half the product of the lengths of its legs.]

(b) Show that the area of $\triangle ADB$ is $\frac 12ab\cos\alpha\sin\beta$.

(c) By Lemma 2.29, the area of $\triangle ABC$ is $\frac 12ab\sin\alpha\cos\beta+\frac 12ab\cos\alpha\sin\beta=\frac 12ab(\sin\alpha\cos\beta+\cos\alpha\sin\beta)$.  Now use Exercise 1(a) to conclude.

\item (a) The area of an equilateral triangle with side length $s$ is equal to $\frac{\sqrt 3}4s^2$.

(b) The area of a square with side length $s$ is equal to $s^2$.

\item Suppose $A_1\dots A_n\sim A'_1\dots A'_n$ are similar $n$-gons with $r=\frac{A_1A_2}{A'_1A'_2}$.  [$r$ is called the \textbf{similarity ratio of $A_1\dots A_n$ to $A'_1\dots A'_n$}.]  Show that $\frac{\operatorname{Area}(A_1\dots A_n)}{\operatorname{Area}(A'_1\dots A'_n)}=r^2$.

\item Let $\triangle ABC$ be a triangle, $P$ a point inside the triangle.  Let $\overset{\longrightarrow}{AP}$ meet side $\overline{BC}$ at point $A'$, let $\overset{\longrightarrow}{BP}$ meet $\overline{AC}$ at $B'$, and let $\overset{\longrightarrow}{CP}$ meet $\overline{AB}$ at $C'$.  Then $\frac{PA'}{AA'}+\frac{PB'}{BB'}+\frac{PC'}{CC'}=1$.  [First show that $\frac{PA'}{AA'}=\frac{\operatorname{Area}(\triangle BPC)}{\operatorname{Area}(\triangle BAC)}$.]

\item A \textbf{diagonal} of an $n$-gon is defined to be a line segment between two distinct vertices which is not a side.  Show that an $n$-gon has $\frac{n(n-3)}2$ diagonals.  [In particular, a triangle has no diagonals; a quadrilateral has $2$ diagonals; a pentagon has $5$ diagonals; a hexagon has $9$ diagonals.]

If the $n$-gon is regular, how many diagonals have the same lengths?  What are these lengths?

\item Let $ABCD$ be a square of side length $s$.  Show that the length of a diagonal of the square is $s\sqrt 2$.

\item Let $ABCDE$ be a regular pentagon of side length $s$, and let $\omega$ be its circumscribed circle.

(a) Show that the five diagonals of the pentagon are all congruent.

(b) Show that the length of a diagonal is equal to $s\frac{1+\sqrt 5}2$.  [Use Ptolemy's theorem (Exercise 22 of Section 2.1) on the quadrilateral $ABCD$.  Note that if $\phi=\frac{1+\sqrt 5}2$ then $\phi$ has the special property that $\phi^2=\phi+1$ \---- this number is called the \textbf{golden ratio}.]

\item\emph{(Area of a circle.)} \---- Let $\omega$ be a circle with center $O$, radius $r$.  Though we have not given a definition of the area of a circle, this exercise uses approximations togive a reasonable formula for (or bounds on) the area.

(a) If $n\geqslant 3$, let $A_1\dots A_n$ be a regular $n$-gon having $\omega$ as a circumscribed circle.  Then $A_1\dots A_n$ has area $\frac{nr^2}2\sin((360/n)^\circ)$.  [Split the $n$-gon into the triangles $\triangle A_jOA_{j+1}$ ($1\leqslant j<n$), $\triangle A_nOA_1$; then use Exercise 1(a).]

(b) Since making $n$ larger and larger makes the region inside the polygon closer and closer to that of the circle (why?), we now take the limit as $n\to\infty$ to obtain the area of the circle.  To do this we first convert the angle to a radian measure \---- i.e., we think of the area as $\frac{nr^2}2\sin(2\pi/n)$.  Show that $\lim_{n\to\infty}\frac{nr^2}2\sin(2\pi/n)=\pi r^2$, and hence the circle has area $\pi r^2$.  [Substitute $h=1/n$ in the expression, and use L'H\^opital's Rule to find the limit as $h\to 0$.]

\item\emph{(Pythagorean Theorem.)} \---- Here is an alternative proof of the Pythagorean Theorem (2.21) using areas.  Suppose $\triangle ABC$ is a right triangle with $\angle B$ the right angle, $AB=a,BC=b,AC=c$.  Take a square of side length $a+b$, and construct four line segments creating four congruent copies of $\triangle ABC$ as below.
\begin{center}\includegraphics[scale=.4]{PythTheoremAlt.png}\end{center}
(a) Explain why the shaded area has area $c^2$.

(b) $\operatorname{Area}(\triangle ABC)=\frac 12ab$, and the area of the big square of side length $a+b$ is $(a+b)^2=a^2+2ab+b^2$.  Use this to show that the shaded area has area $a^2+b^2$.

\item This exercise gives many examples of tilings made up of two or more types of regular polygons.  Try a hand at constructing each of these tilings, either using pencil and paper or computer graphics software.  [All the polygons have the same side length in each tiling.]

(a) Octagons and squares, with two octagons and one square at each vertex

(b) Hexagons and triangles, with one hexagon and one triangle at each edge

(c) Dodecagons\footnote{A dodecagon is a $12$-sided polygon.} and triangles, with two dodecagons and one triangle at each vertex

(d) Triangles, squares and hexagons, with one triangle, two squares and one hexagon at each vertex (the squares do not share any edges)

(e) Dodecagons, hexagons and squares, with one of each kind of polygon at each vertex

(f) Hexagons and triangles, with one hexagon and four triangles at each edge [This tiling has no orientation-reversing symmetry.]

(g) Triangles and squares, with three triangles and two squares at each vertex (the squares do not share any edges)

(h) Triangles and squares, with three triangles and two squares at each vertex (but this time, the two squares at each vertex share an edge)
\end{enumerate}

\subsection*{2.4. Straightedge and Compass Constructions}
\addcontentsline{toc}{section}{2.4. Straightedge and Compass Constructions}
One of the most beautiful things about plane geometry is the ability to draw mathematically accurate figures.  Many tools can be used to draw these figures, ranging from binder edges, compasses, rulers, or even a loop of string.\footnote{An ellipse can be drawn by placing tacks at the foci and putting a loop of string around them \---- see Exercise 9 of Section 2.2.}  In this section, we will particularly focus on the most basic kind of geometric tools: the straightedge and compass.

We will assume we are given an infinite-sized sheet of paper, a writing utensil with unlimited ink or graphite, a \textbf{straightedge} (a hard object which can be used to construct lines, such as the edge of a binder), and a \textbf{compass} (which can be used to construct perfect circles).  [In real life, situations are obviously different, but we would like to stick to the mathematical setting where supplies are unlimited.]

Our ambition in this section is to establish what kinds of things can be constructed, purely with the use of a straightedge and compass.  For starters, there are only three kinds of \emph{deterministic} constructions we can do with these tools.

(A) Given two distinct points, drawing the line between them (using the straightedge);

(B) Given distinct points $O$ and $P$, drawing the circle centered at $O$ with radius $OP$ (so that $P$ is on the arc \---- this is done by setting the compass point to $O$ and the graphite to $P$ then twirling it around);

(C) Taking intersection points of lines/circles with other lines/circles.

In addition, we would like to do a few \emph{nondeterministic} constructions:

(D) Given a point, one can construct a line through it going in an arbitrary direction.  However, without the use of other constructions, it cannot be controlled to a particular direction.  [Remember, you only have a straightedge and compass.] % Yes, you can pick any Q \ne P and draw \overset{\longleftrightarrow}{PQ}.  It's just that without other constructions, you can't put it in an exact/precise direction you described with words.

(E) One can draw a circle with a possibly arbitrary center or radius.  But the center/radius cannot be controlled without the use of the other constructions.  [Note that one might be able to control a \emph{relation} between the center and radius without controlling either one separately, if they start with the arc (compass' graphite) on a constructed point.]

(F) A point can be taken on a line or circle.  However, without other constructions, you cannot control its exact position with your mind \---- you can only declare it to be on the line or circle.

We also assume that we can copy a basic construction accurately on a different spot of the paper:

(G) Given a line segment $\overline{AB}$, the compass radius can be adjusted to the length $AB$.  Thus, one can construct other line segments with the same length, given a point on a line (or possibly not given a line).  In other words, given $C$, and any line $\ell$ through $C$, one can construct $D$ on $\ell$ so that $CD=\ell$.

(H) Given an angle $\angle BAC$, one can construct angles of measure $m\angle BAC$ anywhere on the paper, given any line/segment/ray (or possibly not given any).

It is a bit tricky to grasp the exact concepts.  Any segment length can come ``randomly'' (for example, if you just draw a line and mark two points on it), but there are only certain segment lengths that you can think of, and do some of the aforementioned valid steps, to \emph{guarantee} a line segment of that length.  To delve into this concept, you will first need to (for normalization purposes) draw a nondegenerate line segment, and declare its length to be $1$.  [This is called the \textbf{unit segment}.]

We identify $\mathbb R^2$ with $\mathbb C$ by the correspondence $(a,b)\leftrightarrow a+bi$.  We now define\\

\noindent\textbf{Definition.} \emph{A real number $r$ is said to be \textbf{constructible} if one can apply steps (A)-(H) in such a way that will guarantee them a line segment of length $|r|$.  A complex number $z$ is said to be \textbf{constructible} if its real and imaginary parts are both constructible.}\\

\noindent The importance of constructible complex numbers will be revealed later on.

As a basic example of the definition, $1$ is constructible (because we declared it to be the length of the unit segment).  $0$ is clearly constructible too (because if $A$ is any point then $0=AA$).  Moreover, $2$ is constructible \---- if you let $\ell$ be a line and $A,B,C\in\ell$ be points such that $AB=BC=1$ (possible by constructibility of $1$ by (G)), we have $AC=2$ by the segment addition postulate.

The best way to classify constructible numbers is to digress and introduce an algebraic concept called a field.\\

\noindent\textbf{FIELDS}\\

\noindent A field is essentially a ``restricted'' number system where the four basic arithmetic operations (addition, subtraction, multiplication and division) are all possible.  It is assumed to consist of complex numbers; this includes real numbers and rational numbers.  We now give a formal definition.\\

\noindent\textbf{Definition.} \emph{A \textbf{subfield of $\mathbb C$} (or a \textbf{field} in this section) is a subset $F$ of the set of complex numbers $\mathbb C$ such that:}

(i) \emph{Whenever $a,b\in F$, $a+b\in F$ and $ab\in F$.}

(ii) \emph{$0$ and $1$ are in $F$.}

(iii) \emph{Whenever $a\in F$, we have $-a\in F$, and if $a\ne 0$ then $1/a\in F$.}\\

\noindent Effectively, this states that $F$ is a subgroup of the additive group $\mathbb C$, and that $F-\{0\}$ is a subgroup of the multiplicative group $\mathbb C_{\ne 0}$.

For example, $\mathbb Q,\mathbb R,\mathbb C$ are all fields.  $\mathbb Q(\sqrt 2)=\{x+y\sqrt 2:x,y\in\mathbb Q\}$ is a field as well; here are the main things to check to show this:
\begin{center}
$(x+y\sqrt 2)(x'+y'\sqrt 2)=(xx'+2yy')+(xy'+yx')\sqrt 2$;

If $x+y\sqrt 2\ne 0$ then $x^2-2y^2\ne 0$,\footnote{If $x^2-2y^2=0$, then $y$ must be zero, otherwise we would have $x/y=\pm\sqrt 2$, contradicting the irrationality of $\sqrt 2$.  With $y=0$, we obviously have $x=0$ as well, so $x+y\sqrt 2=0$.} and $\frac 1{x+y\sqrt 2}=\frac x{x^2-2y^2}-\frac y{x^2-2y^2}\sqrt 2$.
\end{center}
In $\mathbb Q(\sqrt 2)$, every element can uniquely be written as $x+y\sqrt 2$ with $x,y\in\mathbb Q$.  (The uniqueness follows from the fact that if $x+y\sqrt 2=0$, we must have $x=y=0$.)  This may sound like a familiar concept: in linear algebra, we learn that if vectors $\vec v_1,\vec v_2,\dots,\vec v_n\in\mathbb R^n$ form a basis, then every element of $\mathbb R^n$ is uniquely expressible as $c_1\vec v_1+\dots+c_n\vec v_n$, $c_1,\dots,c_n\in\mathbb R$ (the spanning of the vectors implies existence, and we have uniqueness due to linear independence).  We now generalize this to various fields.\\

\noindent\textbf{Definition.} \emph{Let $F\subset K$ be fields, and let $u_1,\dots,u_n\in K$.  Then:}

(i) \emph{$\{u_1,\dots,u_n\}$ is said to \textbf{span $K$} over $F$ if every element of $K$ can be written as $c_1u_1+\dots+c_nu_n$ with the $c_j\in F$.}

(ii) \emph{$\{u_1,\dots,u_n\}$ is said to be \textbf{linearly independent} over $F$ provided that whenever $c_1u_1+\dots+c_nu_n=0$ with the $c_j\in F$, we have $c_1=c_2=\dots=c_n=0$.  The set is said to be \textbf{linearly dependent} if it is not linearly independent.}

(iii) \emph{$\{u_1,\dots,u_n\}$ is said to be a \textbf{basis} (plural ``bases'') of $K$ over $F$ if it both spans $K$ and is linearly independent over $F$.}

\emph{If $F=\mathbb Q$, then $K$ is said to be a \textbf{number field} if it has a finite basis over $\mathbb Q$.}\\

\noindent\textbf{Examples.}

(1) The set $\{1,\sqrt 2\}$ is a basis of $\mathbb Q(\sqrt 2)$ over $\mathbb Q$, as we have previously established.  On the other hand, consider the set $\{2,7\sqrt 2,1+\sqrt 2\}$.  This set spans $\mathbb Q(\sqrt 2)$ over $\mathbb Q$, because any $x+y\sqrt 2,x,y\in\mathbb Q$ can be written as $\frac x2(2)+\frac y7(7\sqrt 2)$.  However, it is \emph{not} linearly independent, because there exist linear combinations that equal zero without the coefficients being zero, for example:
$$7(2)+2(7\sqrt 2)-14(1+\sqrt 2)=0$$
(2) Consider the subset $\{\sqrt 2,\sqrt 5\}$ of $\mathbb R$.  If $x\sqrt 2+y\sqrt 5=0$ for $x,y\in\mathbb Q$, then $y$ must be zero (otherwise, we would have $-x/y=\sqrt{5/2}$, and elementary number theory shows that $\sqrt{5/2}$ is irrational).  Therefore $0=x\sqrt 2+0\sqrt 5=x\sqrt 2$ from which $x=0$ follows.  This proves $x=y=0$, and therefore, $\{\sqrt 2,\sqrt 5\}$ is linearly independent over $\mathbb Q$.  However, $\{\sqrt 2,\sqrt 5\}$ does not span $\mathbb R$ over $\mathbb Q$ because, for example, there is no way to write $\sqrt 3\in\mathbb R$ as $x\sqrt 2+y\sqrt 5$ with $x,y\in\mathbb Q$; see Exercise 1.\\

\noindent Now we introduce the concept of dimension.  The statements for Lemma 2.31 and Propositions 2.32-2.33 are slightly unorthodox, since they are usually for a vector space $V$ over $F$,\footnote{This means that $(V,+,0)$ is an abelian group, and there is a ``scalar multiplication'' $(a,x)\mapsto ax$ from $F\times V\to V$ such that $a(x+y)=ax+ay$, $(a+b)x=ax+bx$, $(ab)x=a(bx)$ and $1x=x$.  If $F\subset K$ are fields, then $K$ is clearly an example.} rather than a field.  But since fields are the only vector spaces involved in this section, we will omit the details.\\

\noindent\textbf{Lemma 2.31.} \emph{Let $F\subset K$ be fields, and $u_1,\dots,u_n\in K$.}

(i) \emph{If $v\in K$ is a linear combination of $u_1,\dots,u_n$, then $\{v,u_1,\dots,u_n\}$ is linearly dependent over $F$.}

(ii) \emph{$\{u_1,\dots,u_n\}$ is linearly dependent if and only if there exists $1\leqslant k\leqslant n$ and $c_1,\dots,c_{k-1}\in F$ such that $u_k=c_1u_1+\dots+c_{k-1}u_{k-1}$.  In other words, an ordered list of elements of $K$ is linearly dependent if and only if some element is a linear combination of the preceding ones.}
\begin{proof}
(i) because $v=c_1u_1+\dots+c_nu_n$ implies that $1v-c_1u_1-\dots-c_nu_n=0$, and the coefficient $1$ is nonzero.

(ii) If $u_k=c_1u_1+\dots+c_{k-1}u_{k-1}$, then we clearly have linear dependence, because
$$c_1u_1+\dots+c_{k-1}u_{k-1}+(-1)u_k+0u_{k+1}+\dots+0u_n=0$$
and the coefficient $-1$ is nonzero.  Conversely, suppose $u_1,\dots,u_n$ are linearly dependent.  Then there exist $a_1,\dots,a_n\in F$, such that $a_1u_1+\dots+a_nu_n=0$ but the $a_j$ are not all zero.  Let $k$ be the largest integer such that $a_k\ne 0$.  Then we have $a_{k+1}=\dots=a_n=0$, hence $a_1u_1+\dots+a_ku_k=0$ and
$$u_k=-\frac{a_1}{a_k}u_1-\dots-\frac{a_{k-1}}{a_k}u_{k-1}$$
as desired.
\end{proof}

\noindent In Example (1) above, notice that the spanning set has more elements than the linearly independent set first mentioned.  This is no accident, as we show that if $F\subset K$ are fields, any linearly independent subset of $K$ cannot contain any more elements than a spanning subset.\\

\noindent\textbf{Proposition 2.32.} \emph{Suppose $F\subset K$ are fields.  If $\{v_1,\dots,v_n\}\subset K$ spans $K$ over $F$, and $\{u_1,\dots,u_m\}\subset K$ is linearly independent over $F$, then $m\leqslant n$.}
\begin{proof}
First we shall use induction on $k$ to show that if $1\leqslant k\leqslant\min(m,n)$, there exist $n-k$ elements of $\{v_1,\dots,v_n\}$, say $v_{j_1},\dots,v_{j_{n-k}}$, such that $\{u_1,\dots,u_k,v_{j_1},\dots,v_{j_{n-k}}\}$ spans $K$ over $F$.

If $k=1$, then note that $u_1$ \---- like every element of $K$ \---- is a linear combination of $v_1,\dots,v_n$.  By Lemma 2.31(i), $\{u_1,v_1,\dots,v_n\}$ is linearly dependent.  By Lemma 2.31(ii) one of the elements \---- say $v_j$ \---- is a linear combination of the preceding ones, say $v_j=au_1+c_1v_1+\dots+c_{j-1}v_{j-1}$.  With that, the set $\{u_1,v_1,\dots,v_{j-1},v_{j+1},\dots,v_n\}$ obtained by removing $v_j$ still spans $K$ over $F$, because every element of $K$ is a linear combination of $v_1,\dots,v_n$, and then in such a linear combination, we can replace $v_j$ with $au_1+c_1v_1+\dots+c_{j-1}v_{j-1}$ and expand the result.  This concludes the case where $k=1$.

Now suppose inductively that $k<\min(m,n)$ and there are $n-k$ elements $v_{j_1},\dots,v_{j_{n-k}}$, such that $\{u_1,\dots,u_k,v_{j_1},\dots,v_{j_{n-k}}\}$ spans $K$.  Then $u_{k+1}$ is a linear combination of those elements, and hence $\{u_1,\dots,u_k,u_{k+1},v_{j_1},\dots,v_{j_{n-k}}\}$ is linearly dependent.  Therefore, some element is a linear combination of the preceding ones.  This element cannot be one of the $u$'s (because that would imply that $\{u_1,\dots,u_m\}$ is linearly dependent), hence it must be one of the $v$'s, and we can, as before, remove that $v$ and still have a spanning set.  This spanning set consists of $u_1,\dots,u_{k+1}$ and $n-(k+1)$ $v$'s, completing this inductive step.

If $m>n$, then the statement in the first paragraph holds for $k=n$, which means $\{u_1,\dots,u_n\}$ spans $K$ over $F$.  In this case, $u_m$ is a linear combination of $u_1,\dots,u_n$: contradiction, because $\{u_1,\dots,u_m\}$ is linearly independent.
\end{proof}
\noindent\textbf{Proposition 2.33.} \emph{Suppose $F\subset K$ are fields.  Then any two (finite) bases have the same number of elements.}
\begin{proof}
If $\{v_1,\dots,v_n\}$ and $\{u_1,\dots,u_m\}$ are bases, then two applications of Proposition 2.32 yield $m\leqslant n$ and $n\leqslant m$.  Hence, $m=n$.
\end{proof}
\noindent Thus, the number of elements of a basis of $K$ over $F$ is independent of the particular basis.  We hereby have the following definition.\\

\noindent\textbf{Definition.} \emph{Suppose $F\subset K$ are fields.  If there exists a (finite) basis of $K$ over $F$, then the number of elements in a basis of $K$ is denoted $[K:F]$ and called the \textbf{dimension} of $K$ over $F$.  Such a field is said to be \textbf{finite-dimensional}.  If no (finite) basis exists, then $K$ is said to be \textbf{infinite-dimensinal} over $F$.}\\

\noindent For example, $[\mathbb Q(\sqrt 2):\mathbb Q]=2$ because $\{1,\sqrt 2\}$ is a basis of $\mathbb Q(\sqrt 2)$ over $\mathbb Q$.  However, $\mathbb R$ is infinite-dimensional over $\mathbb Q$ \---- Exercise 2 provides a strategy for proving this.  Also, the number fields are precisely the fields which are finite-dimensional over $\mathbb Q$.

Now for a few basic facts about dimension.\\

\noindent\textbf{Proposition 2.34.} \emph{Suppose $F\subset K$ are fields.  Then $[K:F]=1$ if and only if $K=F$.}
\begin{proof}
Suppose $[K:F]=1$.  Then $K$ has some single-element basis, say $\{u\}$.  Since $\{u\}$ spans $K$ over $F$, we can write, for instance $1=cu$ with $c\in F$.  This entails that $u=1/c\in F$.  With that, clearly $au\in F$ for all $a\in F$, which means $K=F$.

Conversely, if $K=F$, then $\{1\}$ is a single-element basis of $K$ over $F$.
\end{proof}
\noindent\textbf{Proposition 2.35.} \emph{Suppose $F\subset K\subset L$ are fields with $L$ finite-dimensional over $K$, and $K$ finite-dimensional over $F$.  Then $L$ is finite-dimensional over $F$ and $[L:F]=[L:K][K:F]$.}
\begin{proof}
Suppose $[L:K]=n$ and $[K:F]=m$; we show that $[L:F]=nm$.

Let $\{v_1,\dots,v_n\}$ be a basis of $L$ over $K$, and $\{u_1,\dots,u_m\}$ is a basis of $K$ over $F$.  Then consider the set $S=\{v_ju_k:1\leqslant j\leqslant n,1\leqslant k\leqslant m\}\subset L$.  Observe that it contains $nm$ elements (they are all distinct because if $v_ju_k=v_{j'}u_{k'}$ with $(j,k)\ne(j',k')$, then $u_kv_j-u_{k'}v_{j'}=0$ \---- yet $u_k,u_{k'},u_k-u_{k'}$ are nonzero elements of $K$, so this contradicts the linear independence of the $v$'s over $K$).

Our aim is to show that $S$ is a basis of $L$ over $F$.  Firstly, take any $x\in L$.  Since $v_1,\dots,v_n$ span $L$ over $K$, we can write $x=c_1v_1+\dots+c_nv_n$ with the $c_j\in K$.  Moreover, since $u_1,\dots,u_m$ span $K$ over $F$, every $c_j$ can be written as a linear combination, say $c_j=a_{j1}u_1+\dots+a_{jm}u_m$.  With that,
$$x=\sum_{j=1}^n\sum_{k=1}^m a_{jk}v_ju_k$$
hence $x$ is a linear combination of the elements of $S$ with coefficients in $F$, so that $S$ spans $L$ over $F$.

Now suppose $\sum_{j=1}^n\sum_{k=1}^m a_{jk}v_ju_k=0$ with the $a_{jk}\in F$.  Then
$$\sum_{j=1}^n\left(\sum_{k=1}^m a_{jk}u_k\right)v_j=0,$$
which means that every $\sum_{k=1}^m a_{jk}u_k=0$ by the linear independence of the $v$'s over $K$.  Moreover, for each $j$, $a_{j1}u_1+\dots+a_{jm}u_m=0$, which means that (since the $u$'s are linearly independent over $F$), $a_{j1}=\dots=a_{jm}=0$.  Therefore all the $a_{jk}$ (for any $j,k$) are zero, hence $S$ is linearly independent over $F$, and is therefore a basis of $L$ over $F$.
\end{proof}
\noindent If $F$ is a subfield of $\mathbb C$ and $u\in\mathbb C$, then we define $F(u)$ to be the intersection of all fields containing both $F$ and $u$.  This is clearly a field, and it is the smallest field containing those things.  The aforementioned field $\mathbb Q(\sqrt 2)$ is the special case with $F=\mathbb Q,u=\sqrt 2$.

The next proposition covers the essential kinds of fields built upon one another in geometric constructions.\\

\noindent\textbf{Proposition 2.36.} \emph{Let $F\subset K$ be fields.  Then $[K:F]=2$ if and only if $K=F(\sqrt u)$ for some $u\in F,\sqrt u\notin F$.}
\begin{proof}
Suppose $[K:F]=2$.  Then $F\subsetneq K$ by Proposition 2.34; let $x$ be an element of $K-F$.  Then no three-element subset of $K$ can be linearly independent over $F$ (since there is a two-element basis, so that would contradict Proposition 2.32).  In particular, $\{x^2,x,1\}$ is not linearly independent, which means we have an equation
$$ax^2+bx+c1=0$$
with $a,b,c\in F$, not all zero.  If $a=b=0$ then $c=0$ clearly follows, a contradiction; and if $a=0$ and $b\ne 0$, then $x=-c/b$, another contradiction since $x\notin F$.  Therefore, we must have $a\ne 0$.  By dividing the above equation by $a$, we may write (with $p=b/a,q=c/a$),
$$x^2+px+q=0$$
Now let $u=(x+p/2)^2$.  Then $\sqrt u=\pm(x+p/2)$, which is not in $F$ (because $\sqrt u\in F$ implies $x\in F$ easily).  However, $u=x^2+px+\frac{p^2}4=\frac{p^2}4-q\in F$.  We claim that $K=F(\sqrt u)$; indeed, if we let $K'=F(\sqrt u)$ then $F\subsetneq K'\subset K$.  As $[K:F]=2$, a prime, Exercise 4 implies that we must have $K'=K$.  This proves $K=F(\sqrt u),u\in F,\sqrt u\notin F$.

Conversely, let $K=F(\sqrt u)$ for $u\in F,\sqrt u\notin F$.  Then $\{1,\sqrt u\}$ is a basis of $K$ over $F$: indeed, one can check that for $x,y,x',y'\in F$:
\begin{center}
$(x+y\sqrt u)(x'+y'\sqrt u)=(xy+ux'y')+(xy'+yx')\sqrt u$;
If $x+y\sqrt u\ne 0$ then $x^2-uy^2\ne 0$, and $\frac 1{x+y\sqrt u}=\frac x{x^2-uy^2}-\frac y{x^2-uy^2}\sqrt u$.
\end{center}
With this aid, it is easily proven that $K'=\{x+y\sqrt u:x,y\in F\}$ is a field containing $F$ and $\sqrt u$: hence, since $K$ is the \emph{smallest} such field, we have $K\subset K'$.  Yet also $K'\subset K$ because $K$ manifestly contains elements of the form $x+y\sqrt u$ with $x,y\in F$.  Hence, $K=K'$, and $\{1,\sqrt u\}$ spans $K'$ (i.e., $K$) over $F$.  Linear independence over $F$ is a straightforward exercise, using the fact that $\sqrt u\notin F$.  Therefore $[K:F]=2$.
\end{proof}

\noindent\textbf{FINDING CONSTRUCTIBLE NUMBERS}\\

\noindent We have finally developed enough material to classify constructible numbers.  We have previously established the constructibility of $0$ and $1$ \---- as well as $2$ by putting two line segments together.  Thus our intuition tells us that the sum of two numbers is constructible.  This is indeed the case:\\

\noindent\textbf{Proposition 2.37.} \emph{If $a$ and $b$ are constructible numbers, so are $a+b$ and $a-b$.}
\begin{proof}
It suffices to assume $a,b$ are real numbers.  The statement will then follow for complex numbers by considering their real and imaginary parts separately.

We will prove the case where $a,b\geqslant 0$; the case where either $a$ or $b$ is negative should be worked out by the reader.  In this case, let $\ell$ be a line, $A$ a point on $\ell$, $B$ a point on $\ell$ such that $AB=a$ [constructible by rule (G)], and $C$ a point on $\ell$ such that $BC=b$ and $B$ is between $A$ and $C$.  Then $AC=a+b$, hence $a+b$ is constructible.
\begin{center}\includegraphics[scale=.4]{AddConstr.png}\end{center}
As for $a-b$, arrange for $AB=a$ and $BC=b$ again, but this time $B$ should \emph{not} be between $A$ and $C$.  In other words, either $A$ is between $B,C$ or $C$ is between $A,B$.  In this case $AC=|a-b|$, which implies $a-b$ is constructible.
\end{proof}
\noindent Similarly, multiplication and division can be worked out:\\

\noindent\textbf{Lemma 2.38.} (i) \emph{If $\ell$ is a line and $A$ is a point, then the line through $A$ perpendicular to $\ell$ can be constructed.}

(ii) \emph{If $\ell$ is a line and $A$ is a point not on $\ell$, then the line through $A$ parallel to $\ell$ can be constructed.}
\begin{proof}
(i) Using the compass, construct a circle centered at $A$ whose radius is sufficiently large for the arc to intersect $\ell$ in two points $B,C$.  Then $\overline{AB}\cong\overline{AC}$ (both are radii of the circle).  Now fix a compass radius $r$ and construct circles $\omega_1,\omega_2$ centered at $B$ and $C$ respectively, with radius $r$: this radius must be large enough for $\omega_1,\omega_2$ to intersect in two points $P,Q$.

Since $BP=CP=BQ=CQ=r$, Exercise 2 of Section 2.1 implies that $P$ and $Q$ are on the perpendicular bisector of $\overline{BC}$, whence $\overset{\longleftrightarrow}{PQ}$ is the perpendicular bisector of $BC$.  But since $\overline{AB}\cong\overline{AC}$, the perpendicular bisector is also the altitude of $\triangle ABC$ from vertex $A$, by Exercise 17(g) of Section 2.1, hence contains $A$.  Thus, the (constructible) line $\overset{\longleftrightarrow}{PQ}$ is perpendicular to $\overline{BC}$ (hence to $\ell$) and passes through $A$, completing this part.

(ii) By part (i), one can construct the line $\ell_1$ through $A$ perpendicular to $\ell$.  By part (i) again, one can construct the line $\ell_2$ through $A$ perpendicular to $\ell_1$.  With that, $\ell_2$ goes through $A$ and $\ell_2\parallel\ell$ by Proposition 2.10.
\end{proof}
\noindent\textbf{Proposition 2.39.} \emph{If $a$ and $b$ are constructible numbers, then $ab$ is constructible, and if $b\ne 0$ then $a/b$ is constructible.}
\begin{proof}
Again we may assume $a,b$ are real numbers, in view of the statements
$$(x+yi)(x'+y'i)=(xx'-yy')+(xy'+yx')i$$
$$x'+y'i\ne 0\implies\frac{x+yi}{x'+y'i}=\frac{xx'+yy'}{(x')^2+(y')^2}+\frac{yx'-xy'}{(x')^2+(y')^2}i$$
and the fact that we have already proven sums and differences of constructible numbers to be constructible.

Moreover, since $a$ is constructible if and only if $|a|$ is, we may assume $a$ and $b$ are both $\geqslant 0$.

Let $\overline{AB}$ be a line segment of length $a$, and consider the ray $\overset{\longrightarrow}{AB}$.  Let $P$ be the point on this ray such that $AP=1$.  (We already know that $1$ is constructible.)  Now, let $C$ be a point such that $PC=b$ and $A,C,P$ don't lie on one line; then construct the line through $B$ parallel to $\overset{\longleftrightarrow}{PC}$ (Lemma 2.38).  Finally, let $D$ be the intersection of this line with $\overset{\longleftrightarrow}{AC}$, as shown below.
\begin{center}\includegraphics[scale=.4]{MultConstr.png}\end{center}
Since $\overset{\longleftrightarrow}{PC}\parallel\overset{\longleftrightarrow}{BD}$, $\angle ACP\cong\angle ADB$ by Proposition 2.10.  Since $\angle CAP\cong\angle DAB$ clearly, AA similarity entails $\triangle ACP\sim\triangle ADB$.  Hence $\frac{CP}{AP}=\frac{DB}{AB}$, i.e., $b=\frac{DB}a$.  Therefore $DB=ab$, and $ab$ is constructible.

To show that $a/b$ is constructible for $b\ne 0$, the argument is similar; this time let $AB=b$, let $P$ be the point on $\overset{\longrightarrow}{AB}$ with $AP=1$, then pick $D$ first so that $DB=a$, and let $C$ be the intersection of $\overset{\longleftrightarrow}{AD}$ and the line through $P$ parallel to $\overset{\longleftrightarrow}{DB}$.  Then $CP=a/b$.
\end{proof}

\noindent From Propositions 2.37 and 2.39 (along with the constructibility of $0$ and $1$), we conclude that \textbf{the constructible numbers form a field}.  This field we shall denote as $\mathscr C$.  $\mathscr C$ is actually more than just a field, as the square root of a constructible number is constructible:\\

\noindent\textbf{Proposition 2.40.} \emph{If $a$ is a constructible number, then so is $\sqrt a$.}
\begin{proof}
Again we may assume $a$ is a real number, in view of the formula
$$\sqrt{x+yi}=\sqrt{\frac{\sqrt{x^2+y^2}+x}2}\pm i\sqrt{\frac{\sqrt{x^2+y^2}-x}2}$$
for $x,y\in\mathbb R$.

Moreover, since $a\geqslant 0\implies\sqrt{-a}=i\sqrt a$, we may assume $a\geqslant 0$.

Let $\ell$ be a line, $P$ a point on $\ell$.  Construct the circle $\omega$ with center $P$ and radius $\frac{1+a}2$ (which is constructible).  Let $A$ and $B$ be the intersection points of $\omega$ and $\ell$.  Observe that $\overline{AB}$ is a diameter (because it contains $P$, the center of the circle), hence has length $1+a$.  Now, construct the point $C\in\ell$ such that $AC=1$ and $C$ is between $A$ and $B$.  By the segment addition postulate, $BC=a$.  Finally, construct the line through $C$ perpendicular to $\ell$ (Lemma 2.38) and let $D$ be its intersection point with $\omega$:
\begin{center}\includegraphics[scale=.4]{SqrtConstr.png}\end{center}
(To avoid confusion, the point $P$ has not been marked in the diagram.)

We claim that $DC=\sqrt a$, from which the constructibility of $\sqrt a$ will follow.  Since $\overline{AB}$ is a diameter, $m\measuredangle AB=180^\circ$, and therefore $m\angle ADB=90^\circ$ by Proposition 2.20.  Hence, $\angle ADB\cong\angle ACD$ and obviously $\angle DAC\cong\angle BAD$, which implies $\triangle ADC\sim\triangle ABD$ by AA similarity.  A similar argument shows that $\triangle ABD\sim\triangle DBC$.  Therefore, it easily follows that $\triangle ADC\sim\triangle DBC$.  Moreover, $\frac{DC}{AC}=\frac{BC}{DC}$; in other words $DC=\frac a{DC}$, which equates to $DC=\sqrt a$.
\end{proof}
\noindent In view of this, we define a field $F$ to be \textbf{quadratically closed} if it is closed under taking square roots; i.e., $a\in F$ implies $\sqrt a\in F$.\footnote{``Quadratically closed'' more directly means that whenever $b,c\in F$, the polynomial $x^2+bx+c$ has roots in $F$.  Using the quadratic formula, it is clear that these definitions are equivalent.}  We have just shown that the field $\mathscr C$ of constructible numbers is quadratically closed.  The surprising result of the matter is that $\mathscr C$ is actually the \emph{smallest} quadratically closed field containing $\mathbb Q$: in other words, if $F$ is a quadratically closed field containing $\mathbb Q$, then $\mathscr C\subset F$.  To prove this, we shall start with the following observation:
\begin{center}
\textbf{A real number $r$ is constructible if and only if steps (A)-(C) alone (the determinstic steps) can be used to guarantee a line segment of length $|r|$.}
\end{center}
For example, in the proof of Proposition 2.39, one can specifically arrange for $\overline{AC}$ to have a certain length, thus using purely deterministic steps.

The intuition behind this observation is that any nondeterministic constructions will lead to segments where you cannot tell the length unless the segment is made in ``deterministic ways.''  It is a rather tricky statement.

In the coordinate plane, if points $(0,0)$ and $(1,0)$ are marked, then Exercise 6 shows that a point is constructible if and only if it is a constructible number when considered as a point in the complex plane, (where $(0,0)=0$ and $(1,0)=1$).  More generally, if $F$ is a field, we define a point $(x,y)$ to be \textbf{constructible with respect to $F$} if $x+yi\in F$.  (Thus the constructible points are precisely the points constructible with respect to $\mathscr C$.)

We define a line to be \textbf{constructible with respect to $F$} if it contains at least two points in $F$.  We define a circle to be \textbf{constructible with respect to $F$} if its center is in $F$, and at least one point on its arc is in $F$.  Our first lemma is\\

\noindent\textbf{Lemma 2.41.} \emph{Suppose $F$ is a subfield of $\mathbb C$ containing the imaginary unit $i$, and there is a geometric construction in the complex plane made up of points, lines and circles which are constructible with respect to $F$.  If one of steps (A)-(C) is applied, the resulting line/circle/point is constructible with respect either to $F$, or to $F(\sqrt u)$ for some $u\in F$.}
\begin{proof}
(A) is easy: if you draw a line through two points in $F$, this line is constructible with respect to $F$.  (B) is likewise clear.

As for (C), we shall temporarily think of the elements of $F$ as ordered pairs rather than complex numbers.  If $P$ is the intersection of two lines
$$ax+by=c,~~~~a'x+b'y=c',~~~~a,b,c,a',b',c'\in F$$
then we must have $ab'-ba'\ne 0$ (otherwise the lines would either be parallel or coincide).  With that, elementary linear algebra shows that $P=\left(\frac{cb'-bc'}{ab'-ba'},\frac{ac'-ca'}{ab'-ba'}\right)$, and clearly both its coordinates are in $F$.  Therefore, the point $P$ is in $F$ (since $i\in F$).

Now suppose $P=(x,y)$ is an intersection point of a line and a circle:
$$ax+by=c,~~~~(x-x_c)^2+(y-y_c)^2=r^2,~~~~a,b,c,x_c,y_c,r\in F$$
If $b\ne 0$ then $y=\frac{c-ax}b$, which implies
$$(x-x_c)^2+\left(\frac{c-ax}b-y_c\right)^2=r^2$$
$$\left(1+\frac{a^2}{b^2}\right)x^2+2\left(\frac{ay_c}b-x_c-\frac{ac}{b^2}\right)x+\left(x_c^2+y_c^2+\frac{c^2}{b^2}-\frac{2cy_c}b-r^2\right)=0$$
This is a quadratic equation in $x$, where the leading coefficient is nonzero (because $a,b\in\mathbb R$, hence $1+a^2/b^2\geqslant 1$).  Let $\mathfrak a,\mathfrak b,\mathfrak c$ be the respective coefficients of $x^2,x,1$; then we have $\mathfrak ax^2+\mathfrak bx+\mathfrak c=0$.  With that, the well-known quadratic formula entails $x=\frac{-\mathfrak b\pm\sqrt{\mathfrak b^2-4\mathfrak a\mathfrak c}}{2\mathfrak a}$, and either way we have $x\in F(\sqrt u)$ where $u=\mathfrak b^2-4\mathfrak a\mathfrak c\in F$.  Since $y=\frac{c-ax}b$, $y\in F(\sqrt u)$ follows at once.  Hence $x+yi\in F(\sqrt u)$, and $P$ is constructible with respect to $F(\sqrt u)$.

If $b=0$, then $a\ne 0$; repeat the above argument with the roles of $x,y$ swapped, the roles of $x_c,y_c$ swapped and the roles of $a,b$ swapped.

Finally, if $P$ is an intersection of two circles,
$$(x-x_c)^2+(y-y_c)^2=r^2,~~~~x_c,y_c,r\in F$$
$$(x-x_c')^2+(y-y_c')^2=(r')^2,~~~~x_c',y_c',r'\in F$$
then subtracting the equations yields a line that must also contain $P$:
$$2(x_c'-x_c)x+2(y_c'-y_c)y=r^2-(r')^2-(x_c^2-(x_c')^2)-(y_c^2-(y_c')^2)$$
($2(x_c'-x_c)$ and $2(y_c'-y_c)$ are not both zero, because $x_c'=x_c$ and $y_c'=y_c$ implies that $r^2=(r')^2\implies r=r'$ from the first two equations, hence the circles coincide.)

Thus, by what we have proven before $P$ is constructible with respect to some $F(\sqrt u)$.
\end{proof}

\noindent We thus have our desired result:\\

\noindent\textbf{Theorem 2.42.} \emph{If $F$ is a quadratically closed subfield of $\mathbb C$, then $\mathscr C\subset F$.}
\begin{proof}
First note that $i\in F$ because $i=\sqrt{-1}$, so Lemma 2.41 can be applied to $F$.

Since $F$ is quadratically closed, $F(\sqrt u)=F$ for all $u\in F$.  Hence Lemma 2.41 implies that whenever points, lines and circles are constructible with respect to $F$, applying any of steps (A)-(C) results in figures still constructible with respect to $F$.  Yet the first two marked points ($(0,0),(1,0)$) are manifestly constructible with respect to $F$.  It follows from induction that all constructible points, lines and angles are constructible with respect to $F$.  Since $\mathscr C$ is the set of constructible points (in the complex plane), we conclude that $\mathscr C\subset F$.
\end{proof}

\noindent From Theorem 2.42, the reader can readily verify that if $x\in\mathscr C$, then there exists a chain of (number) fields
$$\mathbb Q=F_0\subset F_1\subset\dots\subset F_n$$
such that $x\in F_n$, and for each $1\leqslant k\leqslant n$, $F_k=F_{k-1}(\sqrt u)$ for some $u\in F_{k-1}$.  (The aim is to show that the set of $x\in\mathbb C$ satisfying this condition is a quadratically closed field.)

Moreover, $[F_n:\mathbb Q]=[F_n:F_{n-1}]\dots[F_2:F_1][F_1:F_0]$ by Proposition 2.35.  Since $F_k=F_{k-1}(\sqrt u)$, $[F_k:F_{k-1}]$ is either equal to $1$ or $2$.  Moreover, it follows that \textbf{$[F_n:\mathbb Q]$ is a power of $2$}, because it is the product of numbers, each of which is either $1$ or $2$.  Also, since $x\in F_n$, $\mathbb Q(x)\subset F_n$, and hence $[F_n:\mathbb Q]=[F_n:\mathbb Q(x)][\mathbb Q(x):\mathbb Q]$.  Therefore, $[\mathbb Q(x):\mathbb Q]$ is a divisor of $[F_n:\mathbb Q(x)]$.  Since any divisor of a power of $2$ is also a power of $2$, we conclude
\begin{center}
\textbf{If $x$ is a constructible number then $[\mathbb Q(x):\mathbb Q]$ is a power of $2$.} %% Yes, \mathbb Q(x) has been formally defined in this context, because $F(u)$ has been defined.
\end{center}
However, the converse is false: it is possible for $[\mathbb Q(x):\mathbb Q]$ to be a power of $2$ without $x$ being constructible \---- see Exercise 8.\\

\noindent We conclude this section by describing some possible/impossible straightedge and compass constructions.
\begin{itemize}
\item One can construct the perpendicular bisector of a line segment $\overline{AB}$.  The idea imitates the proof of Lemma 2.38(i): if $r$ is a fixed sufficiently-large radius, then draw circles of radius $r$ centered at $A$ and $B$.  The line through their two intersection points is the perpendicular bisector of $\overline{AB}$.  Moreover, since the perpendicular bisector intersects $\overline{AB}$ in its midpoint, the midpoint of any line segment can also be constructed.

\item The angle bisector of an angle is also constructible; see Exercise 7.

\item For this and the next few examples, we will use the fact that if $\zeta\in\mathbb C$ is the root of a polynomial $p(x)=x^d+a_{d-1}x^{d-1}+\dots+a_1x+a_0$ which is irreducible over $\mathbb Q$, then $\{1,\zeta,\dots,\zeta^{d-1}\}$ is a basis of $\mathbb Q(\zeta)$ over $\mathbb Q$, and hence $[\mathbb Q(\zeta):\mathbb Q]=d$.  This follows from algebra and proving it here would take us too far afield.

 Suppose $n\geqslant 0$ is an integer such that $p=2^{2^n}+1$ is prime.  [Such a $p$ is called a \textbf{Fermat prime}.]  Then a regular $p$-gon is constructible.  This fact follows once we can prove that $\zeta=e^{2\pi i/p}$\----a primitive $p$-th root of unity\----is a constructible complex number.  (For then $1,\zeta,\zeta^2,\dots,\zeta^{p-1}$ are constructible, and their vertices form a regular $p$-gon.)

Here, we will use some results from Galois theory, without proving them.  Firstly, $x^{p-1}+\dots+x^2+x+1$ is an irreducible polynomial over $\mathbb Q$ with $\zeta$ as a root, which means that $[\mathbb Q(\zeta):\mathbb Q]=p-1=2^{2^n}$.  Moreover, $\mathbb Q(\zeta)$ is a Galois extension of $\mathbb Q$ (this means it is generated by \emph{all} roots of that polynomial), so that the Galois group\footnote{If $F\subset K$ are fields, then the set of permutations $\sigma\in S(K)$ such that $\sigma(a+b)=\sigma(a)+\sigma(b),\sigma(ab)=\sigma(a)\sigma(b)$ for all $a,b\in K$, and $\sigma(c)=c$ for all $c\in F$, is a group under function composition, called the \textbf{Galois group} of $K$ over $F$.} has order $2^{2^n}$.  The Sylow theorems (Exercise 12 of Section 1.6) imply that there is a chain of subgroups whose orders are $2^k,k=0,\dots,2^n$.  By the Fundamental Theorem of Galois Theory, the subgroups of the Galois group are in an order-reversing bijection with the fields lying between $K$ and $F$, hence there exist fields
$$\mathbb Q=F_0\subset F_1\subset\dots\subset F_{2^n-1}\subset F_{2^n}=K$$
such that every $[F_k:F_{k-1}]=2$.  Using Proposition 2.36, it follows that every element of $K$, in particular $\zeta$, is constructible.

As special cases, it follows that the regular pentagon, $17$-gon, $257$-gon and $65537$-gon are constructible.  It is not known whether there are any Fermat primes larger than $65537$.

\item The regular heptagon ($7$ sides), on the other hand, is \emph{not} constructible.  If it were constructible, then it would be possible to construct an angle of $(360/7)^\circ$, and therefore, the complex number $\zeta=e^{2\pi i/7}$ would be constructible (verify!).  Since $x^6+\dots+x+1$ is an irreducible polynomial with $\zeta$ as a root, $[\mathbb Q(\zeta):\mathbb Q]=6$.  Since $6$ is not a power of $2$, we conclude that $\zeta$ can't be constructible.

%http://mathworld.wolfram.com/CubeDuplication.html
\item\emph{(Duplicating the Cube / Delian Problem.)} \---- There is a curious Greek story involving a cubical altar and a plague.\footnote{Weisstein, Eric W.~``Cube Duplication.''~From \emph{MathWorld}\----A Wolfram Web Resource.}  While Athenians had the plague, an Athenian leader asked the Oracle at Delos for advice.  The god Apollo told him, through the Oracle, to double the size of the altar in attempt to end the disease.  By ``size,'' of course, he meant the \emph{volume} of the cube.  He was also particularly picky and did not want the altar's size to be larger than he specified.  This led to the following geometric problem: Given a cube, construct (with straightedge and compass) another cube with exactly twice the volume.

Of course, this question is ill-posed with what we have gone through during the chapter, because we are dealing with a coordinate \emph{plane}, not $3$-space.  However, one can construct a cube by constructing its net (see below), then cutting it out, folding along the edges and taping.  Thus, we will assume a cube of side length $s>0$ is constructible if and only if $s$ is a constructible number.
\begin{center}\includegraphics[scale=.4]{CubeNet.png}\end{center}
Now, take a cube of side length $s$.  If we were able to construct another cube with exactly twice the volume, the volume of the second cube would be $2s^3$, hence its side length would be $\sqrt[3]{2s^3}=s\sqrt[3]2$.  With that, $s$ and $s\sqrt[3]2$ are both constructible, and therefore so is $\sqrt[3]2$.  But this does not hold water, as $\sqrt[3]2$ is a root of the polynomial $x^3-2$, and this polynomial is irreducible over $\mathbb Q$.  This makes $[\mathbb Q(\sqrt[3]2):\mathbb Q]=3$, not a power of $2$.

The confirmation that duplicating the cube is impossible took around 2000 years after the aforementioned Greek story.  It was first proven by Descartes in 1637.

\item\emph{(Trisecting the angle.)} \---- Given an angle $\angle BAC$ with measure $\alpha$, Axiom 2.6 implies that there exist points $P,Q$ inside $\angle BAC$, such that $m\angle BAP=m\angle QAC=\alpha/3$, and therefore (by the angle addition postulate) $m\angle PAQ=\alpha/3$ as well.  $\overset{\longrightarrow}{AP}$ and $\overset{\longrightarrow}{AQ}$ are said to \textbf{trisect} the angle.

An ancient problem concerning the straightedge and compass is: given an arbitrary angle, can you construct the rays that trisect it?  We know we can bisect the angle (and this had already been known to the geometers); however, with the material in this chapter, we can answer our question with ``no'': straightedge and compass cannot trisect an arbitrary angle.  [Note, however, that trisection of an angle is possible with \emph{ruler} and compass: see Exercise 9.]

Suppose trisection of an angle \emph{were} possible.  It is clear that an equilateral triangle is constructible (let $\overline{AB}$ be a line segment of length $1$, then construct circles of radius $1$ centered at $A$ and $B$ and see where they intersect); hence, a $60^\circ$ angle is constructible.  Under the assumption that angles can be trisected, it follows that a $20^\circ$ angle can be constructed.  Hence by Exercise 6, $\cos 20^\circ$ is a constructible real number.  However, if $\alpha$ is real number, then the results of Exercise 3 of Section 2.2 can be used to show
$$\cos(3\alpha)=4(\cos\alpha)^3-3\cos\alpha$$
Moreover if $\alpha=20^\circ$ and $x=\cos\alpha$, then $4x^3-3x=\cos(3\alpha)=\cos(60^\circ)=\frac 12$.  Hence, $8x^3-6x-1=0$.  That polynomial can be seen to be irreducible over $\mathbb Q$ (because it has degree $3$ and it has no rational roots).  Consequently $[\mathbb Q(x):\mathbb Q]=3$, hence $x$ is not constructible.  This proves that $\cos 20^\circ$ is not constructible, and we have a contradiction.

\item\emph{(Squaring the circle.)} \---- Another ancient problem is this: given a circle, can a square be constructed with exactly the same area as the circle?  There is an interesting piece of history that Babylonian mathematicians were trying to use squares to approximate the area of a circle.  Later, Anaxagoras and Hippocrates of Chios tried tackling the problem of squaring the circle.  They were not successful.  The answer to whether squaring the circle was possible was not found until 1882, when the Lindemann-Weierstrass Theorem confirmed $\pi$ to be transcendental.\footnote{The Lindemann-Weierstrass Theorem states that if $x_1,\dots,x_n$ are distinct algebraic numbers, then $e^{x_1},\dots,e^{x_n}$ are linearly independent over the field of algebraic numbers.  Taking $x_1=0,x_2=x$, it follows that $e^x$ is transcendental whenever $x$ is a nonzero algebraic number.  In particular, if $\pi$ were algebraic then $e^{i\pi}$ would be transcendental: absurd, because Euler discovered that $e^{i\pi}=-1$.}

Let $\omega$ be a circle of radius $r$.  Then $\omega$ has area $\pi r^2$ (Exercise 10 of Section 2.3).  If it were possible to construct a square of the same area, the square would have side length $\sqrt{\pi r^2}=r\sqrt{\pi}$.  Moreover, line segments of length $r$ and $r\sqrt{\pi}$ would both be present, entailing constructibility of $\sqrt{\pi}$.  However, since $\pi$ is transcendental, $\mathbb Q(\sqrt{\pi})$ is infinite-dimensional over $\mathbb Q$ (since for any $n>0$ whatsoever, $1,\pi,\pi^2,\dots,\pi^n$ are linearly independent over $\mathbb Q$).  Therefore $\sqrt{\pi}$ is certainly not constructible, and we have a contradiction.
\end{itemize}
\subsection*{Exercises 2.4. (Straightedge and Compass Constructions)}
Unless otherwise specified, $F$ and $K$ denote subfields of $\mathbb C$ with $F\subset K$.
\begin{enumerate}
\item Show that $\sqrt 3$ cannot be written as $x\sqrt 2+y\sqrt 5$ with $x,y\in\mathbb Q$.  [Suppose $\sqrt 3=x\sqrt 2+y\sqrt 5$ with $x,y\in\mathbb Q$.  By squaring both sides and noting that $\sqrt{10}$ is irrational, conclude that either $x=0$ or $y=0$.  Now there is an easy contradiction.]

\item Show that the following statements are equivalent:

~~~~(i) $K$ is infinite-dimensional over $F$.

~~~~(ii) For every positive integer $n$, there exist linearly independent elements $v_1,\dots,v_n\in K$.

~~~~(iii) There is no finite set which spans $K$ over $F$.

\item Suppose $F\subset K\subset L$ are fields with $L$ finite-dimensional over $F$.  Show that $L$ is finite-dimensional over $K$ and $K$ is finite-dimensional over $F$ (from which Proposition 2.35 entails $[L:F]=[L:K][K:F]$).  [A basis of $L$ over $F$ spans $L$ over $K$; therefore $L$ is finite-dimensional over $K$ by Exercise 2.  If $K$ were infinite-dimensional over $F$, then Exercise 2 would imply that there are arbitrary large subsets of $K$ which are linearly independent over $F$ \---- but these are also subsets of $L$ \---- use Proposition 2.32 to get a contradiction.]

\item If $[K:F]$ is prime, show that there is no field $E$ such that $F\subsetneq E\subsetneq K$.

\item If $[K:F]=n$ and $u_1,\dots,u_n\in K$, show that $u_1,\dots,u_n$ span $K$ if and only if they are linearly independent.

\item\emph{(Constructibility of points, lines and angles.)} \---- A point $P$ is said to be \textbf{constructible} if there is a way to apply steps (A)-(H) to guarantee that you will land on exactly that point.

In the following exercises, assume you are dealing with a coordinate plane on the sheet of paper, and (for normalization purposes), only the points $(0,0)$ and $(1,0)$ are marked.

(a) Show that a point $(x,y)$ is constructible if and only if $x,y$ are both constructible real numbers.  Conclude that if the coordinate plane is the complex plane, then the constructible points are precisely the constructible numbers.

A line $\ell$ is said to be \textbf{constructible} if it contains at least two distinct constructible points.

(b) Show that $\ell$ is constructible if and only if $\ell$ is given by an equation $ax+by=c$ with $a,b,c\in\mathscr C$, $a,b$ not both zero.  Conclude that if $\ell$ is constructible and $A$ is a constructible point, the lines through $A$ parallel and perpendicular to $\ell$ are both constructible.

An angle $\angle BAC$ is said to be \textbf{constructible} if $\overset{\longleftrightarrow}{AB}$ and $\overset{\longleftrightarrow}{AC}$ are constructible lines (i.e., both rays of the angle determine constructible lines).

(c) If $0^\circ\leqslant\alpha\leqslant 180^\circ$ is a real number, show that the following statements are equivalent:

~~~~(i) There exists a constructible angle whose measure is $\alpha$.

~~~~(ii) $\sin\alpha$ is constructible (as a real number).

~~~~(iii) $\cos\alpha$ is constructible.

~~~~(iv) Either $\tan\alpha$ is constructible, or else $\alpha=90^\circ$ (making the tangent undefined).

In this case we say that $\alpha$ is a \textbf{constructible angle measure}.  This is very different from being a constructible real number: for instance, $\pi/4=45^\circ$ is a constructible angle (its tangent is $1$), but $\pi/4$ is transcendental, hence not a constructible real number.

\item If $\angle BAC$ is an angle, then its angle bisector can be constructed with straightedge and compass.  [Use the compass to draw a circular arc $\omega$, centered at $A$, which goes through both $\overset{\longrightarrow}{AB}$ and $\overset{\longrightarrow}{AC}$.  Let $B'$ and $C'$ be the points where $\omega$ meets those rays respectively.  Then $\overline{AB'}\cong\overline{AC'}$; both have length the radius of $\omega$.  Now, fix a particular radius $r$ for the compass, and draw circular arcs of radius $r$ centered at $B'$ and $C'$: make sure they intersect.  If $P$ is their intersection point, then $\overline{B'P}\cong\overline{C'P}$ because both have length $r$.  Now show that $\overset{\longrightarrow}{AP}$ is the angle bisector of $\angle BAC$.]

\item Let $\zeta$ be a root in $\mathbb C$ of the polynomial $x^4-4x+2$.  This polynomial is irreducible over $\mathbb Q$,\footnote{This follows from Eisenstein's criterion; we will not go into details here.} and hence, $[\mathbb Q(\zeta):\mathbb Q]=4$, which is a power of $2$.  However, we claim that $\zeta$ is \emph{not} constructible.

(a) If $a,b,c,d,r\in\mathbb Q$ and $(ax^3+bx^2+cx+d)^2\equiv r\pmod{x^4-4x+2}$, show that $a=b=c=0$.  [If $a\ne 0$ or $b\ne 0$, expand the left-hand side, then use the polynomial Division Algorithm to derive a contradiction.  If $a=b=0$, it should be clear that $c=0$.]

(b) If $u\in\mathbb Q$ such that $v=\sqrt u\in\mathbb Q(\zeta)$, show that $v\in\mathbb Q$.  [$\{1,\zeta,\zeta^2,\zeta^3\}$ is a basis of $\mathbb Q(\zeta)$ over $\mathbb Q$, so one may write $v=a\zeta^3+b\zeta^2+c\zeta+d$ with $a,b,c,d\in\mathbb Q$.  Explain why $(ax^3+bx^2+cx+d)^2\equiv u\pmod{x^4-4x+2}$, and use part (a).]

(c) Conclude that there is no field $\mathbb Q\subset F\subset\mathbb Q(\zeta)$ such that $[F:\mathbb Q]=2$, and therefore $\zeta$ is not constructible.

\item\emph{(Trisection of an angle with ruler and compass.)} \---- In this problem, instead of a straightedge, you are given a ruler.  This time you are allowed to mark points on the ruler's edge for later usage.  This gives the ability to do more constructions than just with the straightedge and compass.  For example, you can trisect the angle; the following construction is due to Archimedes.

Let $\angle BAC$ be an angle.  Construct a circle $\omega$ centered at $A$, as shown below, and assume $B,C$ are on the circle's arc.  Extend line $\overset{\longleftrightarrow}{AC}$.  Place the ruler against $\overset{\longleftrightarrow}{AB}$, and mark the locations of $A$ and $B$ on the ruler.  Now the ruler has two marks, at a distance of $AB$.

Adjust the ruler so that one mark is on the arc of $\omega$, the other mark is on $\overset{\longleftrightarrow}{AC}$, and the ruler meets point $B$ (away from the marks).  [Your intuition should tell you that this can be done.]  Then, trace the ruler, resulting in a line through $B$ as shown below.  Let it meet the arc at $D$ and let it meet $\overset{\longleftrightarrow}{AC}$ at $E$.  We have $DE=AB$ because $D,E$ are at the marks on the ruler.
\begin{center}\includegraphics[scale=.4]{AngleTrisection.png}\end{center}
Show that $m\angle DEC=\frac 13m\angle BAC$.  [Use the angle addition postulate and Propositions 2.11 and 2.15.]  Thus, by copying $\angle DEC$ into $\angle BAC$ at both rays, the angle is trisected.

\item This exercise introduces the concept of an (abstract) field.  A \textbf{field} is a set $F$ equipped with two binary operations $+,\cdot$ (called \emph{addition} and \emph{multiplication} respectively), such that:

~~~~(i) $F$ is an abelian group under $+$, with identity element $0$;

~~~~(ii) $F-\{0\}$ is an abelian group under $\cdot$, with identity element $1$;

~~~~(iii) For all $a,b,c\in F$, $a(b+c)=ab+ac$ [distributivity of multiplication over addition].

There are examples of fields which are not subfields of $\mathbb C$.  For instance, if $p$ is prime, $\mathbb Z/p\mathbb Z$ under modular arithmetic is a field, ((ii) follows from a statement mentioned in Exercise 12(c) of Section 1.2).  Note that the definitions imply that $0$ and $1$ denote distinct elements of $F$ (distinct since $1\in F-\{0\}$ by (ii)).  Now for a few exercises.

(a) If $F$ is a field and $a,b\in F$, then $a\cdot 0=0$, and $a(-b)=-(ab)$.  [(iii) implies that $x\mapsto ax$ is a homomorphism from the additive group $F$ to itself; now use Proposition 1.10.]  Conclude that $a(b-c)=ab-ac$ for $a,b,c\in F$.

(b) If $K$ is a field, a \textbf{subfield} of $K$ is a subset $F$ such that $F$ is a subgroup of the additive group $K$, and $F-\{0\}$ is a subgroup of the multiplicative group $K-\{0\}$.  In this case, $F$ is itself a field under the restrictions of the operations.

If $F$ is a subfield of $K$, imitate the material in this section to define what it means for a subset of $K$ to span/be linearly independent/be a basis over $F$.  Then prove Propositions 2.31-2.35 in this setting, giving a definition of $[K:F]$ after proving Proposition 2.33.

(c) Suppose $K$ is a field and $F$ is a subfield such that $[K:F]=2$.  If $1+1\ne 0$ in $F$, imitate Proposition 2.36 to prove that $K=F(v)$ for some $v\in K$ such that $v^2\in F$.

(d) Show by example that part (c) may be false if $1+1=0$ in $F$.  [There is a unique field with $4$ elements; find it, and then take it to be $K$.]

A \textbf{polynomial} in $F$ is a formal linear combination of the symbols $1,x,x^2,x^3,\dots$ with coefficients in $F$.  For example, $ax^2+bx+c$ is a polynomial when $a,b,c\in F$, and so is $ax^2+(c-a)$.  They can be added and multiplied the normal way you would add/multiply polynomials:
$$(a_nx^n+\dots+a_1x+a_0)+(b_nx^n+\dots+b_1x+b_0)=(a_n+b_n)x^n+\dots+(a_1+b_1)x+(a_0+b_0)$$
$$(a_nx^n+\dots+a_1x+a_0)(b_mx^m+\dots+b_1x+b_0)=$$
$$(a_nb_m)x^{n+m}+(a_nb_{m-1}+a_{n-1}b_m)x^{n+m-1}+\dots+(a_2b_0+a_1b_1+a_0b_2)x^2+(a_1b_0+a_0b_1)x+(a_0b_0)$$
and essentially the same rules we are familiar with (such as associativity of multiplication) hold for the polynomials.  We shall admit only \emph{finite} linear combinations (infinite linear combinations of $1,x,x^2,x^3,\dots$ are called \emph{power series}).

The set of polynomials in $F$ is denoted $F[x]$.  [Note that this is not a field; for example, $x$ has no multiplicative inverse, so (ii) fails.]

(e) Let $K$ be a field and $F$ a subfield.  Then every polynomial $f\in F[x]$ induces a function $\varphi_f:K\to K$, which applies a polynomial with $x$ as the input.  For instance, if $f=ax^2+bx+c$ with $a,b,c\in F$, then for every $u\in K$, $\varphi_f(u)=au^2+bu+c$.  The element $u\in K$ is said to be a \textbf{root} of $f$ if $\varphi_f(u)=0$.

If $u\in K$, show that $f(x)$, regarded as a polynomial in $K$, can be written as $g(x)(x-u)+r$ with $g(x)\in K[x],r\in K$.  [Imitate the polynomial Division Algorithm.]  Furthermore, show that $r=\varphi_f(u)$.

(f) Show that if $f(x)=a_nx^n+\dots+a_1x+a_0$ with $a_n\ne 0$ (in this case we say $n$ is the \textbf{degree} of $f$), then $f$ has at most $n$ roots in $K$.  [If $u$ is a root of $f$ in $K$, then $f(x)=g(x)(x-u)$ with $g(x)\in K[x]$ by part (e); now note that $g(x)$ has degree $n-1$ and use induction.]

(g) If $F$ is a field and $G$ is a finite subgroup of the multiplicative group $F-\{0\}$, show that $G$ is cyclic.  [Let $n=|G|$, and let $a\in G$ be an element of maximal order out of all elements of $G$, say $|a|=d$.  By Exercise 14(b) of Section 1.2, the order of every element of $G$ divides $d$.  Hence all elements of $G$ are roots of the polynomial $x^d-1\in F[x]$ (why?); use part (f).]
\end{enumerate}

\subsection*{2.5. Euclidean $n$-space} % Sections 2.1-2.4 are 65 pages. XD I find that rather heavy for half of a chapter.  Guess it doesn't matter, I've seen lots of books that'll obviously be bigger than this one.  This *is* a first draft, after all; if judging mathematicians have a better idea of a chapter layout, I'm willing to go with it.
\addcontentsline{toc}{section}{2.5. Euclidean $n$-space}
In the first half of the chapter, we have studied plane geometry.  In other words, we have dealt with points and lines in $\mathbb R^2$.  However, in this section, we shall introduce higher dimensions of Euclidean space.  Some of the concepts to be introduced will be important for the next few chapters.

If $n>1$ is an integer, we consider $\mathbb R^n$ to be \textbf{Euclidean $n$-space}.  We do not regard it as an $n$-dimensional vector space: right now, its origin does not have any more importance than the other points.  (This is sometimes referred to as ``affine space.'')

If $a,b\in\mathbb R^n$, we define the \textbf{distance} from $a$ to $b$, denoted $\rho(a,b)$ or $\|a-b\|$, to be the magnitude of the vector from $a$ to $b$.  This still makes sense even though $\mathbb R^n$ is not regarded as a vector space.  If $a=(u_1,\dots,u_n)$ and $b=(v_1,\dots,v_n)$, this is $\sqrt{(u_1-v_1)^2+\dots+(u_n-v_n)^2}$.

We now introduce the concept of a $k$-plane and a $k$-sphere:\\

\noindent\textbf{Definition.} \emph{Let $0\leqslant k<n$.}

(i) \emph{A \textbf{$k$-plane} refers to a translation of a $k$-dimensional vector subspace of $\mathbb R^n$.  In other words, $P$ is a $k$-plane if there exist linearly independent vectors $\vec v_1,\dots,\vec v_k\in\mathbb R^n$ and a vector $\vec w\in\mathbb R^n$ such that}
$$P=\{\vec w+c_1\vec v_1+\dots+c_k\vec v_k:c_1,\dots,c_k\in\mathbb R\}.$$
\emph{A $0$-plane is called a \textbf{point}; a $1$-plane is called a \textbf{line}; a $2$-plane is called a \textbf{plane}; and an $(n-1)$-plane in $\mathbb R^n$ is called a \textbf{hyperplane}.}

(ii) \emph{A \textbf{hypersphere} (or $(n-1)$-sphere) in $\mathbb R^n$ is the set of points with a given positive distance from a given point; what is the same thing, a set given by an equation}
$$(x_1-x_{c_1})^2+(x_2-x_{c_2})^2+\dots+(x_n-x_{c_n})^2=r^2$$
\emph{where $x_{c_1},x_{c_2},\dots,x_{c_n},r\in\mathbb R$ and $r>0$.  $(x_{c_1},\dots,x_{c_n})\in\mathbb R^n$ is called the \textbf{center} of the hypersphere, and $r$ is called its \textbf{radius}.}

\emph{If $k<n-1$, then a \textbf{$k$-sphere} is defined to be the intersection of a hypersphere with a $(k+1)$-plane, provided that this intersection contains more than one point.  A $1$-sphere is called a \textbf{circle}, and a $2$-sphere is called a \textbf{sphere}.}\\

\noindent A $3$-sphere is alternatively called a ``glome''\footnote{Weisstein, Eric W.~``Glome.''~From \emph{MathWorld}\----A Wolfram Web Resource.} %http://mathworld.wolfram.com/Glome.html
\---- however, we shall avoid this terminology so that the reader is not befuddled.

We recall from Section 2.1 that two points determine a line.  They do so also in $\mathbb R^n$: indeed, if $P\ne Q$ are points in $\mathbb R^n$, the unique line going through them is $\{tP+(1-t)Q:t\in\mathbb R\}$, where $P$ and $Q$ are regarded as vectors.

A set of points is said to be \textbf{collinear} if there exists one line containing them.  For example, two points are always collinear (two distinct points determine a line); yet three points are not in general collinear.  In $\mathbb R^2$, three points form a triangle if and only if they are not collinear.  Similarly, a set of points is said to be \textbf{coplanar} if there exists one plane containing them, and $k$-\textbf{coplanar} if there exists one $k$-plane containing them.

Now for some basic facts:\\

\noindent\textbf{Proposition 2.43.} \emph{In $\mathbb R^3$:}

(i) \emph{Any two distinct planes are either disjoint (in which case we say they are \textbf{parallel}) or intersect in a line.}

(ii) \emph{Three points that are not collinear determine a plane.}

\begin{proof}
(i) follows from elementary results in linear algebra.  If $P_1$ and $P_2$ are planes such that $P_1\cap P_2\ne\varnothing$, we can apply a translation of $\mathbb R^3$ to assume $0\in P_1\cap P_2$.  With that, $P_1$ and $P_2$ are distinct $2$-dimensional vector subspaces of $\mathbb R^3$, which means that $P_1\cap P_2\subsetneq P_1$.  Hence, $\dim_{\mathbb R}(P_1\cap P_2)<\dim_{\mathbb R}(P_1)=2$.  If $\dim_{\mathbb R}(P_1\cap P_2)=0$ then $P_1\cap P_2=0$, hence we have $\mathbb R^3\supset P_1\oplus P_2$; taking the dimension of both sides gives $3\geqslant 2+2=4$, contradiction.  Therefore $\dim_{\mathbb R}(P_1\cap P_2)=1$ and $P_1\cap P_2$ is a line.

(ii) Suppose $a=(x_0,y_0,z_0),b=(x_1,y_1,z_1),c=(x_2,y_2,z_2)$ are noncollinear points in $\mathbb R^3$.  Then the vectors $\vec v=(x_1-x_0,y_1-y_0,z_1-z_0)$ and $\vec w=(x_2-x_0,y_2-y_0,z_2-z_0)$ are linearly independent (if, for instance, $\vec w=k\vec v$ for $k\in\mathbb R$ then $a,b,c$ lie in the line through $a$ parallel to $\vec v$).  Moreover, one can let $P$ be the plane spanned by $\vec v,\vec w$; more specifically, let $\vec u$ be the cross product $\vec v\times\vec w$ and then $P$ is given by $\vec u\cdot\vec x=0$.  With that, the plane through $a$ parallel to $P$ contains $a,b,c$.

To show that this plane is unique, suppose $P'$ is a different plane containing $a,b,c$.  Then $P\cap P'$ is either empty or a line by (i), and it contains $a,b,c$: contradiction, because those three points are noncollinear.
\end{proof}

\noindent Similarly, it can be shown that in $\mathbb R^4$, four points that are not coplanar determine a hyperplane.  Also, in $\mathbb R^3$, the intersection of a plane with a sphere is either empty, a point or a circle (Exercise 1); when the intersection is a point, the plane is said to be \textbf{tangent} to the sphere.  Various other basic facts, such as the possibilities for the intersection of a $k$-sphere with a $c$-sphere in $\mathbb R^n$, or the determination of a hypersphere from points (Exercise 2), can also be shown.  However, we will not list them here and prove them; we will leave the verifications to the reader so that he or she can get a good glimpse of the concept.

\subsection*{Exercises 2.5. (Euclidean $n$-space)}
\begin{enumerate}
\item In $\mathbb R^3$, the intersection of a plane with a sphere is either empty, a point, or a circle.  Furthermore, if it is a point $P$, then $\overset{\longleftrightarrow}{OP}$ is perpendicular to the plane.

\item (a) If $0<k<n$, then every $(k-1)$-sphere is contained in a unique $k$-plane.

(b) Suppose $0<k<n$; let $\omega$ be a $(k-1)$-sphere and $p$ a point outside the $k$-plane containing $\omega$.  Then there is a unique $k$-sphere containing $\omega$ and $p$.

(c) In $\mathbb R^n$, let $p_1,\dots,p_{n+1}$ be points that do not lie in one hyperplane.  Show that there is a unique hypersphere containing all of the points.  [First show that for any $0<k<n$, no $k+1$ of the points can lie in one $(k-1)$-plane; in particular, e.g., no three of the points can be collinear.]

\item\emph{(Sylvester–Gallai theorem.)} \---- Suppose $k\geqslant 3$ and $x_1,\dots,x_k\in\mathbb R^2$ are distinct points, not all collinear.  Show that there exists a line (going infinitely in both directions) which goes through exactly two of those points.  [Since the points are not all collinear, there exist triangles which use these points as vertices.  Now consider the altitudes of these triangles.  One of the altitudes has minimal length, in other words there is a triangle $\triangle ABC$ with $A,B,C\in\{x_1,\dots,x_k\}$ such that the altitude from vertex $A$ has the smallest length out of all the altitudes of all the triangles (why?).  Show that $B$ and $C$ are the only points among $x_1,\dots,x_k$ that $\overset{\longleftrightarrow}{BC}$ goes through.]

\item Let $\vec v,\vec w\in\mathbb R^n$ be vectors, say $\vec v=(x_1,\dots,x_n)$ and $\vec w=(y_1,\dots,y_n)$.  Then the \textbf{dot product} $\vec v\cdot\vec w$ is defined to be $x_1y_1+\dots+x_ny_n$.

(a) $\vec v\cdot\vec v=\|\vec v\|^2$, where $\|\vec v\|$ is the magnitude of the vector $\vec v$.

(b) We define $\vec v,\vec w$ to be \textbf{perpendicular} / \textbf{orthogonal}, denoted $\vec v\perp\vec w$, provided that $\vec v\cdot\vec w=0$.  If $\vec v\perp\vec w$, show the Pythagorean identity $\|\vec v+\vec w\|^2=\|\vec v\|^2+\|\vec w\|^2$.  [Use part (a) and the bilinearity of the dot product.]

(c) More generally, we define the angle between $\vec v$ and $\vec w$ to be $\theta=\cos^{-1}\frac{\vec v\cdot\vec w}{\|\vec v\|\|\vec w\|}$, which is between $0^\circ$ and $180^\circ$.  [This makes sense in view of Exercise 4(c) of Section 2.2.]  Show that $\|\vec v-\vec w\|^2=\|\vec v\|^2+\|\vec w\|^2-2\|\vec v\|\|vec w\|\cos\theta$; this is the new version of the law of cosines from Exercise 8 of Section 2.2.

(d) If $\vec w\ne\vec 0$, define $\operatorname{proj}_{\vec w}\vec v=\frac{\vec v\cdot\vec w}{\vec w\cdot\vec w}\vec w$.  Then $\operatorname{proj}_{\vec w}\vec v$ is the unique vector $\vec u$ such that $\vec u=k\vec w$ for some $k\in\mathbb R$, and $(\vec v-\vec u)\perp\vec w$.  $\vec u$ is called the \textbf{orthogonal projection of $\vec v$ onto $\vec w$}.

Now suppose that $\vec v,\vec w\in\mathbb R^3$, say $\vec v=(x_1,y_1,z_1)$ and $\vec w=(x_2,y_2,z_2)$.  Then we define the \textbf{cross product} $\vec v\times\vec w$ as $(y_1z_2-y_2z_1,z_1x_2-z_2x_1,x_1y_2-x_2y_1)$.

(e) The cross product is bilinear, and satisfies $\vec v\times\vec v=\vec 0$ and $\vec v\times\vec w=-\vec w\times\vec v$.

(f) $\vec u\cdot(\vec v\times\vec w)=\vec v\cdot(\vec w\times\vec u)=\vec w\cdot(\vec u\times\vec v)$.  [Show that $\vec u\cdot(\vec v\times\vec w)$ is the determinant of the matrix $\begin{bmatrix}\leftarrow\vec u\rightarrow\\\leftarrow\vec v\rightarrow\\\leftarrow\vec w\rightarrow\end{bmatrix}$.]  Conclude that $\vec v$ and $\vec w$ are both perpendicular to $\vec v\times\vec w$.

(g) $\vec u\times(\vec v\times\vec w)+\vec v\times(\vec w\times\vec u)+\vec w\times(\vec u\times\vec v)=\vec 0$.  [Since the left-hand side is linear in each of the three vectors, you need only show this for $\vec u,\vec v,\vec w\in\{\vec e_1,\vec e_2,\vec e_3\}$.]  This is called the \textbf{Jacobi identity}, and incidentally proves that $\mathbb R^3$ is a Lie algebra over $\mathbb R$ with the cross product.

(h) If $\theta$ is the angle between $\vec v$ and $\vec w$ when they are touched tail-to-tail, $\|\vec v\times\vec w\|=\|\vec v\|\|\vec w\||\sin\theta|$.  [Show that $\|\vec v\|^2\|\vec w\|^2=(\vec v\cdot\vec w)^2+\|\vec v\times\vec w\|^2$.  Then use part (c) and Exercise 10(a) of Section 2.1.]

(i) Now show that $\vec u=\vec v\times\vec w$ is the unique vector such that (1) $\vec u$ is perpendicular to both $\vec v$ and $\vec w$; (2) $\|\vec u\|$ is equal to the area of the parallelogram with sides $\vec v$ and $\vec w$; (3) when you place your right index finger along $\vec v$ and your right middle finger along $\vec w$ (without stretching them apart $>180^\circ$), $\vec u$ points away from your right palm.  [Show that the area of a parallelogram $ABCD$ (in the plane) is equal to $(AB)(AC)\sin m\angle A$.  Then use part (h).  Note that (3) is sometimes called the \emph{right-hand rule}.]
\end{enumerate}

\subsection*{2.6. Isometries of Euclidean $n$-space}
\addcontentsline{toc}{section}{2.6. Isometries of Euclidean $n$-space}
In the previous section, Euclidean $n$-space was introduced, along with $k$-spheres and $k$-planes.  We shall now cover the important concept of its symmetry; this will enable us to find a wide variety of other sorts of geometric figures in space.  An isometry is a distance-preserving function.  As we will eventually see, such a function must necessarily preserve lines, angles and circles.  We shall study them as they relate to other branches of mathematics, like linear algebra.

At this time we shall think of elements of $\mathbb R^n$ as vectors.\\

\noindent\textbf{Definition}. \emph{A function $T:\mathbb R^n\to\mathbb R^n$ is called an \textbf{isometry} if $\|T(\vec x)-T(\vec y)\|=\|\vec x-\vec y\|$ for all $\vec x,\vec y\in\mathbb R^n$.}\\ % Distance has been defined in Section 2.5 now ^^

\noindent The main step to pinning down isometries is to introduce a distance-preserving matrix as follows. An $n\times n$ matrix $A$ is said to be \textbf{orthogonal} if $A^TA=I_n$, where $A^T$ is the transpose of $A$.  The set of all orthogonal $n\times n$ matrices is denoted $O(n)$.

We shall start by noting the relation of the transpose to the dot product in linear algebra: if $\vec v,\vec w\in\mathbb R^n$ are viewed as $n\times 1$ matrices (column vectors), then $\vec v\cdot\vec w=\vec v^T\vec w$.  This is straightforward from the definition of the dot product.  Moreover, if $A$ is an $n\times n$ matrix,
$$A\vec v\cdot\vec w=\vec v\cdot A^T\vec w$$
because $A\vec v\cdot\vec w=(A\vec v)^T\vec w=(\vec v^TA^T)\vec w=\vec v^T(A^T\vec w)=\vec v\cdot A^T\vec w$.  Now we can show\\

\noindent\textbf{Lemma 2.44.} \emph{If $A$ is an $n\times n$ matrix with real entries, the following are equivalent:}

(i) \emph{$A\in O(n)$.}

(ii) \emph{$A\vec v\cdot A\vec w=\vec v\cdot\vec w$ for all $\vec v,\vec w\in\mathbb R^n$.}

(iii) \emph{$\|A\vec v\|=\|\vec v\|$ for all $\vec v\in\mathbb R^n$.}

(iv) \emph{$A\vec e_j\cdot A\vec e_j=1$ and $A\vec e_j\cdot A\vec e_k=0$ whenever $j\ne k$.}\footnote{By $\vec e_j$, of course, we mean the vector $(0,0,\dots,\overset{\begin{matrix}j\\\downarrow\end{matrix}}1,\dots,0,0)$ where the $j$-th component is $1$ and the rest are $0$.}
\begin{proof}
(i) $\implies$ (ii). Suppose $A\in O(n)$.  Then for every $\vec v,\vec w\in\mathbb R^n$,
$$A\vec v\cdot A\vec w=\vec v\cdot A^TA\vec w=\vec v\cdot I_n\vec w=\vec v\cdot\vec w$$
by the orthogonality of $A$ and the relations noted before this lemma was stated.

(ii) $\implies$ (i). For any $\vec v,\vec w\in\mathbb R^n$, $\vec v\cdot\vec w=A\vec v\cdot A\vec w=\vec v\cdot A^TA\vec w$.  Hence since the dot product is linear on the right, $\vec v\cdot(\vec w-A^TA\vec w)=0$; thus, $\vec v\cdot(I_n-A^TA)\vec w=0$.  Since this holds for \emph{any} $\vec v,\vec w$, we can take $\vec v=(I_n-A^TA)\vec w$ to conclude that $\|(I_n-A^TA)\vec w\|=0$, and hence $(I_n-A^TA)\vec w=0$.  With this holding for all $\vec w\in\mathbb R^n$, $I_n-A^TA=0$, hence $A^TA=I_n$ and $A\in O(n)$.

(ii) $\implies$ (iii) because $\|\vec v\|=\sqrt{\vec v\cdot\vec v}$ for any $\vec v\in\mathbb R^n$.

(ii) $\implies$ (iv) is clear.

(iii) $\implies$ (ii). It can be verified that $\vec v\cdot\vec w=\frac 12(\|\vec v+\vec w\|^2-\|\vec v\|^2-\|\vec w\|^2)$: use the fact that $\|\vec v\|^2=\vec v\cdot\vec v$, and the symmetry and bilinearity of the dot product, to express the right-hand side in terms of dot products.  Thus we also have $A\vec v\cdot A\vec w=\frac 12(\|A\vec v+A\vec w\|^2-\|A\vec v\|^2-\|A\vec w\|^2)=\frac 12(\|A(\vec v+\vec w)\|^2-\|A\vec v\|^2-\|A\vec w\|^2)$.  (ii) follows at once from these statements and the hypothesis (iii).

(iv) $\implies$ (ii). Let $\vec v,\vec w\in\mathbb R^n$.  Since $\{\vec e_1,\dots,\vec e_n\}$ is a basis of $\mathbb R^n$, we can write $\vec v=a_1\vec e_1+\dots+a_n\vec e_n$ and $\vec w=b_1\vec e_1+\dots+b_n\vec e_n$.  Thus, since $A$ is linear, $A\vec v=a_1(A\vec e_1)+\dots+a_n(A\vec e_n)$ and $A\vec w=b_1(A\vec e_1)+\dots+b_n(A\vec e_n)$.  With that, the hypothesis (iv) and the bilinearity of the dot product entail
$$A\vec v\cdot A\vec w=\sum_{j=1}^n\sum_{k=1}^n a_jb_k(A\vec e_j\cdot A\vec e_k)=a_1b_1+\dots+a_nb_n;$$
and similarly, $\vec v\cdot\vec w=a_1b_1+\dots+a_nb_n$.  Thus (ii) follows.
\end{proof}

\noindent $O(n)$ is a subgroup of the multiplicative group $GL_n(\mathbb R)$ of $n\times n$ nonsingular matrices.  This is easy to prove from either criterion (ii) or (iii) of Lemma 2.44.  It can also be proven directly from the definition of an orthogonal matrix:
\begin{itemize}
\item If $A$ and $B$ are orthogonal $n\times n$ matrices, then so is $AB$; this follows from the fact that $(AB)^T=B^TA^T$.

\item $I_n$ is orthogonal.

\item An $n\times n$ matrix $A$ is orthogonal if and only if $A$ is nonsingular and $A^{-1}=A^T$; in this case, $A^{-1}$ is also orthogonal (this follows from $(A^{-1})^T=(A^T)^{-1}$).
\end{itemize}
Thus, we call $O(n)$ the \textbf{orthogonal group of dimension $n$}.  Now we claim that every isometry is the composition of an orthogonal matrix and a translation; for that, we start with a lemma:\\

\noindent\textbf{Lemma 2.45.} \emph{If $S:\mathbb R^n\to\mathbb R^n$ is an isometry such that $S(\vec e_j)=\vec e_j$ for all $1\leqslant j\leqslant n$, and $S(\vec 0)=\vec 0$, then $S$ is the identity map on $\mathbb R^n$.}
\begin{proof}
For any $(a_1,\dots,a_n)\in\mathbb R^n$, let $(x_1,\dots,x_n)$ denote $S(a_1,\dots,a_n)$.  We show that $a_j=x_j$ for all $j=1,\dots,n$.  The statement in the lemma will follow.

Since $S$ is an isometry, we have, for each $j$,
$$\|(x_1,\dots,x_n)-\vec 0\|=\|S(a_1,\dots,a_n)-S(\vec 0)\|=\|(a_1,\dots,a_n)-\vec 0\|$$
$$\|(x_1,\dots,x_n)-\vec e_j\|=\|S(a_1,\dots,a_n)-S(\vec e_j)\|=\|(a_1,\dots,a_n)-\vec e_j\|$$
By squaring both sides of each equation, we get:
\begin{equation}\tag{1}
x_1^2+\dots+x_n^2=a_1^2+\dots+a_n^2
\end{equation}
\begin{equation}\tag{2}
x_1^2+\dots+(x_j-1)^2+\dots+x_n^2=a_1^2+\dots+(a_j-1)^2+\dots+a_n^2
\end{equation}
Subtracting equation (2) from equation (1), $2x_j-1=2a_j-1$, and therefore $x_j=a_j$.
\end{proof}
\noindent\textbf{Proposition 2.46.} (i) \emph{If $A\in O(n)$ and $\vec v\in\mathbb R^n$, then $T:\mathbb R^n\to\mathbb R^n$ defined by $T(\vec x)=A\vec x+\vec v$ is an isometry.}

(ii) \emph{Every isometry $T$ of $\mathbb R^n$ is uniquely of the form $\vec x\mapsto A\vec x+\vec v$, with $A\in O(n)$ and $\vec v\in\mathbb R^n$.}
\begin{proof}
(i) This follows from Lemma 2.44: for $\vec x,\vec y\in\mathbb R^n$,
$$\|T(\vec x)-T(\vec y)\|=\|(A\vec x+\vec v)-(A\vec y+\vec v)\|=\|A\vec x-A\vec y\|=\|A(\vec x-\vec y)\|=\|\vec x-\vec y\|.$$
(ii) Let $A:\mathbb R^n\to\mathbb R^n$ be the linear transformation, sending each standard basis vector $\vec e_j$ to the vector $T(\vec e_j)-T(\vec 0)$.  Then we have, for every $j$, %% You ask why A must be linear.  Any function from \{\vec e_1,\dots,\vec e_n\}\to\mathbb R^n extends to a unique linear map, and this is what I did here for \vec e_j\mapsto T(\vec e_j)-T(\vec 0).  I did not say \vec v\mapsto T(\vec v)-T(\vec 0).  But, I guess I'll reword this anyway.
$$\|A\vec e_j\|=\|T(\vec e_j)-T(\vec 0)\|=\|\vec e_j-\vec 0\|=1$$
(because $T$ is an isometry); thus, $A\vec e_j\cdot A\vec e_j=1$.  Furthermore, whenever $j\ne k$,
$$\|A\vec e_j-A\vec e_k\|=\|[T(\vec e_j)-T(\vec 0)]-[T(\vec e_k)-T(\vec 0)]\|=\|T(\vec e_j)-T(\vec e_k)\|$$
$$=\|\vec e_j-\vec e_k\|=\sqrt 2$$
because $(\vec e_j-\vec e_k)\cdot(\vec e_j-\vec e_k)=\vec e_j\cdot\vec e_j-2\vec e_j\cdot\vec e_k+\vec e_k\cdot\vec e_k=1-2\cdot 0+1=2$.  Since $\|A\vec e_j-A\vec e_k\|=\sqrt 2$, we conclude
$$2=\|A\vec e_j-A\vec e_k\|^2=(A\vec e_j-A\vec e_k)\cdot(A\vec e_j-A\vec e_k)=A\vec e_j\cdot A\vec e_j-2A\vec e_j\cdot A\vec e_k+A\vec e_k\cdot A\vec e_k$$
$$=2-2A\vec e_j\cdot A\vec e_k;$$
therefore $A\vec e_j\cdot A\vec e_k=0$.  Hence $A$ satisfies (iv) in Lemma 2.44, and therefore $A\in O(n)$.

Now let $\vec v=T(\vec 0)$; with that, $T(\vec x)=A\vec x+\vec v$.  This follows from Lemma 2.45 with $S$ defined via $S(\vec x)=A^{-1}(T(\vec x)-\vec v)$.

To show that $A$ and $\vec v$ are unique, suppose also that $T(\vec x)=A'\vec x+\vec v'$.  Then for all $\vec x\in\mathbb R^n$, $A\vec x+\vec v=A'\vec x+\vec v'$.  Taking $\vec x=\vec 0$ gives $\vec v=\vec v'$.  Hence $A\vec x=A'\vec x$ for all $\vec x$, which means $(A-A')\vec x=\vec 0$: hence $A-A'=0$ and $A=A'$.
\end{proof}

\noindent From Proposition 2.46, it is clear that every isometry of $\mathbb R^n$ is bijective, and its inverse is also an isometry.  Therefore the isometries form a group under function composition.  This group is denoted $\operatorname{Isom}(\mathbb R^n)$ and called the \textbf{isometry group of $\mathbb R^n$}.

It is also clear that isometries preserve $k$-planes, angles and $k$-spheres; for instance, isometries of $\mathbb R^2$ preserve lines, angles and circles.

The isometry given by $A\in O(n)$ and $\vec v\in\mathbb R^n$ via Proposition 2.46 is denoted $[A,\vec v]$.  Since $[A,\vec v]$ is defined by $\vec x\mapsto A\vec x+\vec v$, the following formulas are readily verified:
\begin{equation}\tag{*1}[A,\vec v]\circ[A',\vec v']=[AA',A\vec v'+\vec v];\end{equation}
\begin{center}
The identity isometry on $\mathbb R^n$ is $[I_n,\vec 0]$;
\end{center}
$$[A,\vec v]^{-1}=[A^{-1},-A^{-1}\vec v].$$
In particular, we have a homomorphism from $\operatorname{Isom}(\mathbb R^n)\to O(n)$ given by $[A,\vec v]\mapsto A$; it is well defined by Proposition 2.46(ii), and a homomorphism by (*1).

The next thing we shall observe is that if $A\in O(n)$, $\det A$ is either $1$ or $-1$: after all, $\det(A^T)=\det A$ [a well-known fact about determinants], and hence $A\in O(n)\implies(\det A)^2=\det(A^T)\det A=\det(A^TA)=\det I_n=1$.  If $\det A=1$ we say that $A$ is \textbf{orientation-preserving}; if $\det A=-1$ we say that $A$ is \textbf{orientation-reversing}.  We define $SO(n)=\{A\in O(n):\det A=1\}$; this is a subgroup of $O(n)$ of index $2$, which we call the \textbf{special orthogonal group of dimension $n$}.

Likewise, if $T=[A,\vec v]\in\operatorname{Isom}(\mathbb R^n)$, we say that $T$ is \textbf{orientation-preserving} if $\det A=1$ and \textbf{orientation-reversing} if $\det A=-1$.  Then the product of two orientation-preserving isometries is orientation-preserving, the product of two orientation-reversing isometries is orientation-preserving, and the product of an orientation-preserving and an orientation-reversing isometry is orientation-reversing.  The group of orientation-preserving isometries is denoted $\operatorname{Isom}^+(\mathbb R^n)$ and called the \textbf{orientation-preserving isometry group of $\mathbb R^n$}.

Since isometries preserve $k$-planes and $k$-spheres, we get many group actions (Section 1.6).  For instance, $\operatorname{Isom}(\mathbb R^n)$ acts on the set of all $k$-planes, by application of an isometry.  Also, $\operatorname{Isom}(\mathbb R^n)$ acts on the set of $k$-spheres of radius $r$, when $r$ is fixed.  We claim that these actions are actually transitive, thus accounting for the ``homogeneity'' of the space:\\

\noindent\textbf{Proposition 2.47.} \emph{$\operatorname{Isom}(\mathbb R^n)$ acts transitively on the following sets:} % You say "by application of an isometry" is redundant.  But using the Axiom of Choice or otherwise, one can certainly define group actions of Isom(\mathbb R^n) on \mathbb R^n,\mathcal P_k,\mathcal S_{k,r} which are not transitive!  In this retrospect, the new version of the statement is inaccurate.

(i) $\mathbb R^n$.

(ii) \emph{The set $\mathcal P_k$ of $k$-planes in $\mathbb R^n$, with $1\leqslant k\leqslant n$ fixed.}

(iii) \emph{The set $\mathcal S_{k,r}$ of $k$-spheres of radius $r$, with $k$ and $r>0$ fixed.}
\begin{proof}
(i) We show that the orbit $\mathcal O_{\mathbb R^n}(\vec 0)$ equals $\mathbb R^n$.  This follows because whenever $\vec v\in\mathbb R^n$, the translation $[I_n,\vec v]:\vec x\mapsto\vec x+\vec v$ maps $\vec 0$ to $\vec v$, and therefore $\vec v\in\mathcal O_{\mathbb R^n}(\vec 0)$.

(ii) If $E$ is the $k$-plane spanned by $\vec e_1,\dots,\vec e_k$, we show that $\mathcal O_{\mathcal P_k}(E)=\mathcal P_k$.  Let $P$ be a plane in $\mathcal P_k$.  Let $V$ be the plane through the origin parallel to $P$; this is a $k$-dimensional vector subspace of $\mathbb R^n$.  Let $\{\vec v_1,\dots,\vec v_k\}$ be a basis of $V$; extend this set to a basis $\{\vec v_1,\dots,\vec v_n\}$ of $\mathbb R^n$.  Then by recursively setting
$$\vec u_1=\frac{\vec v_1}{\|\vec v_1\|};$$
$$\vec u_k=\frac{\vec v_k-\sum_{j=1}^{k-1}(\vec v_k\cdot\vec u_j)\vec u_j}{\|\vec v_k-\sum_{j=1}^{k-1}(\vec v_k\cdot\vec u_j)\vec u_j\|}$$
we get an orthonormal basis $\{\vec u_1,\dots,\vec u_n\}$ of $\mathbb R^n$, and moreover $V$ is also the plane spanned by $\vec u_1,\dots,\vec u_k$.  [This is known as the \emph{Gram-Schmidt process}.]  Thus, if $A:\mathbb R^n\to\mathbb R^n$ is the linear map given by $\vec e_j\mapsto\vec u_j$, then $A$ is orthogonal (since the $\vec u$'s are orthonormal).  Finally, if $\vec v$ is any point in $P$, we get that $[A,\vec v]\in\operatorname{Isom}(\mathbb R^n)$ sends $E$ to $P$ as desired.

(iii) If $k=n-1$, then a $k$-sphere is a hypersphere; i.e., a $k$-sphere of radius $r$ is of the form $(x_1-c_1)^2+\dots+(x_n-c_n)^2=r^2$ where $(c_1,\dots,c_n)$ is the center.  Now let $W$ be the $k$-sphere of radius $r$ centered at the origin; it is straightforward to show that our $k$-sphere of radius $r$ is obtained by applying to $W$ the translation that carries $\vec 0$ to $(c_1,\dots,c_n)$.  Hence the isometries (the translations for that matter) are transitive on the $k$-spheres of radius $r$.

The case $k<n-1$ we leave as an exercise to the reader.
\end{proof}

\noindent\textbf{CATEGORIZING ISOMETRIES OF $\mathbb R^2$ AND $\mathbb R^3$}\\

\noindent We conclude this section by identifying different kinds of isometries of the plane and $3$-space, in terms of what our geometric point of view tells us.  First, note that if $A\in O(2)$ and $\det A=-1$, then $A^2=I_2$: after all, we must have $A=\begin{bmatrix}u&v\\v&-u\end{bmatrix}$ with $u^2+v^2=1$ (by (iv) of Lemma 2.44), and then $A^2=I_2$ is a straightforward calculation.

Now, what isometries of $\mathbb R^2$ are there?  For orientation-preserving isometries, there is obviously the identity (``doing nothing''), translation by a vector, and rotation around a point, as shown below.
\begin{center}\includegraphics[scale=.4]{TransRot.png}\end{center}
Simple casework on $[A,\vec v]$ with $A\in SO(2),\vec v\in\mathbb R^2$ shows that these are the only orientation-preserving isometries:
\begin{itemize}
\item If $A=I_2$ and $\vec v=\vec 0$, then $[A,\vec v]$ is the identity.

\item If $A=I_2$ and $\vec v\ne\vec 0$, then $[A,\vec v]$ translates along the vector $\vec v$.

\item If $A\ne I_2$, then we must have $A=\begin{bmatrix}u&-v\\v&u\end{bmatrix}$ with $u^2+v^2=1$.  We may write $u=\cos\theta,v=\sin\theta$ for some $0^\circ<\theta<360^\circ$.  Then $A=\begin{bmatrix}\cos\theta&-\sin\theta\\\sin\theta&\cos\theta\end{bmatrix}$, the matrix which rotates counterclockwise at an angle of $\theta$.

In this case, $I_2-A$ is nonsingular, because $\det(I_2-A)=\det\begin{bmatrix}1-u&v\\-v&1-u\end{bmatrix}=(1-u)^2+v^2=u^2-2u+1+v^2=2-2u$ (since $u^2+v^2=1$); moreover $u\ne 1$ because $A\ne I_2$, and therefore $2-2u\ne 0$.  It is then a straightforward calculation that $\vec p=(I_2-A)^{-1}\vec v$ is the unique vector fixed by $[A,\vec v]$ (because $A\vec x+\vec v=\vec x\iff(I_2-A)\vec x=\vec v$).

If $T$ is the translation $[I_2,\vec p]$, then $T^{-1}\circ[A,\vec v]\circ T=[A,\vec 0]$, which rotates counterclockwise around the origin through an angle of $\theta$.  From this it is clear that $[A,\vec v]$ rotates counterclockwise around $\vec p$ through an angle of $\theta$. %% I'm not going to insert a new proposition.  That would require me to change the numberings all over the book where the later propositions are referenced, and I guarantee if I do this, it won't be perfect.  (Besides, in what way did I assert that [A,\vec v] for \vec v\ne\vec 0 does not rotate around \vec 0 ?)
\end{itemize}
\noindent Hence, $\operatorname{Isom}^+(\mathbb R^2)$ consists only of the identity, translations and rotations.  It is worth thinking about the composition of any two of these things.  The composition of two translations is manifestly a translation.  The composition of a translation and a rotation is a rotation (though that's not so obvious without the results we just proved); and the composition of two rotations (around possibly different centers) could be a translation or a rotation.

What about orientation-reversing isometries of $\mathbb R^2$?  Well, an obvious example is a reflection around a line.  [Note in this case it sends an uppercase R to a Cyrillic Ya (seen by rotating this page), rather than an R.  This is because the isometry is orientation-reversing.]  However, there are other orientation-reversing isometries which are not reflections; they are \emph{glide reflections}, obtained by reflecting over a line then translating along a vector parallel to the line.  See below:
\begin{center}\includegraphics[scale=.4]{RefGlRef.png}\end{center}
Since a reflection fixes every point on the line of reflection, but a glide reflection does not fix any points, they are essentially distinct.

Every orientation-reversing isometry $[A,\vec v]$ is either a reflection or a glide reflection.  Recall that $A=\begin{bmatrix}u&v\\v&-u\end{bmatrix},u^2+v^2=1$, and that $A^2=I_2$.  We also note that $\operatorname{tr}A=0$, and hence the characteristic polynomial of $A$ is $x^2-(\operatorname{tr}A)x+\det A=x^2-1$.  Therefore, $A$ has $\pm 1$ as its eigenvalues.
\begin{itemize}
\item If $(I_2+A)\vec v=\vec 0$, then it is readily verified that $[A,\vec v]$ fixes $\frac 12\vec v$.  Also, since the eigenvalues of $A$ are $\pm 1$, there exists $\vec w\ne\vec 0$ such that $A\vec w=\vec w$.  If $\ell=\{\frac 12\vec v+t\vec w:t\in\mathbb R\}$, then $\ell$ is a line and $[A,\vec v]$ is reflection over $\ell$, as the reader can readily verify.  The reflection fixes each point of $\ell$.

\item Now suppose $(I_2+A)\vec v\ne\vec 0$.  In this case, let $\vec w_1=\frac 12(I_2-A)\vec v$ and $\vec w_2=\frac 12(I_2+A)\vec v$.  With that, $\vec v=\vec w_1+\vec w_2$, and hence $[A,\vec v]=[I_2,\vec w_2]\circ[A,\vec w_1]$.  Since $(I_2-A)\vec w_2=\frac 12(I_2-A^2)\vec v=\vec 0$, $A\vec w_2=\vec w_2$, which means $A$ reflects across the line spanned by $\vec w_2$.  Since $(I_2+A)\vec w_1=\vec 0$ as well, the previous paragraph entails that $[A,\vec w_1]$ is reflection over a line $\ell$ parallel to $\vec w_2$.  Since $[I_2,\vec w_2]$ is a translation along $\vec w_2$, which is parallel to $\ell$, the composition $[A,\vec v]$ is reflection over a line followed by a translation parallel to the line; in other words, $[A,\vec v]$ is a glide reflection.
\end{itemize}
\noindent We have just shown that every isometry of $\mathbb R^2$ is either (i) the identity, (ii) a translation, (iii) a rotation, (iv) a reflection, or (v) a glide reflection.  (iv) and (v) are orientation-reversing; the other three are orientation-preserving.  (ii) and (v) have no fixed points, (iii) has one fixed point, and (iv) has infinitely many fixed points (on a line).

Similarly, the isometries of $\mathbb R^3$ can be classified.  This requires sophisticated linear algebra, in terms of finding eigenvalues of $A$, so we merely state the results.  Here are all the kinds of isometries of $\mathbb R^3$:

(i) The identity $[I_3,\vec 0]$.  Orientation-preserving; fixes all points.

(ii) Translations: $[I_3,\vec v]$ for $\vec v\ne\vec 0$.  Orientation-preserving; no fixed points.

(iii) Rotations: $[A,\vec v]$ with $A\in SO(3),A\ne I_3,\vec v\in\operatorname{im}(I_3-A)$.  This rotates around a line which is its axis.  Orientation-preserving; fixes all points on the axis.

(iv) Screw translations: $[A,\vec v]$ with $A\in SO(3),A\ne I_3,\vec v\notin\operatorname{im}(I_3-A)$.  This rotates around a line and then translates along the same line.  Orientation-preserving; no fixed points.

(v) Reflections: $[A,\vec v]$ with $A\in O(3)-SO(3),A^2=I_3,A\ne -I_3,(I_3+A)\vec v=\vec 0$.  This reflects across a plane.  Orientation-reversing; fixes all points in the plane.

(vi) Glide reflections: $[A,\vec v]$ with $A\in O(3)-SO(3),A^2=I_3,(I_3+A)\vec v\ne\vec 0$.  This reflects across a plane then translates along a vector parallel to the plane.  Orientation-reversing; no fixed points.

(vii) Rotatory reflections: $[A,\vec v]$ with $A\in O(3)-SO(3)$, and either $A=-I_3$ or $A^2\ne I_3$.  This reflects across a plane, then rotates around a line perpendicular to the plane.  Orientation-reversing; fixes exactly one point.

\subsection*{Exercises 2.6. (Isometries of Euclidean $n$-space)}
\begin{enumerate}
\item (a) If $T\in\operatorname{Isom}(\mathbb R^n)$, show that the set $\{\vec x\in\mathbb R^n:T(\vec x)=\vec x\}$ is either empty, a $k$-plane for some $0\leqslant k<n$, or all of $\mathbb R^n$.

(b) If $S_1,S_2\in\operatorname{Isom}(\mathbb R^n)$, then $\{\vec x\in\mathbb R^n:S_1(\vec x)=S_2(\vec x)\}$ is either empty, a $k$-plane for some $0\leqslant k<n$, or all of $\mathbb R^n$.  [Apply part (a) to $T=S_1^{-1}\circ S_2$.]

\item (a) For each $A\in O(n)$, let $T_A:\mathbb R^n\to\mathbb R^n$ be the linear transformation induced by $A$; this is an isomorphism of the additive group $\mathbb R^n$, i.e., $T_A\in\operatorname{Aut}(\mathbb R^n)$.  Define $\varphi:O(n)\to\operatorname{Aut}(\mathbb R^n)$ via $\varphi(A)=T_A$.  Then $\varphi$ is a homomorphism.

(b) Show that $\operatorname{Isom}(\mathbb R^n)$ is isomorphic to the semidirect product $\mathbb R^n\rtimes_\varphi O(n)$.  [See Exercises 16-17 of Section 1.4.]

(c) Likewise, show that $\operatorname{Isom}^+(\mathbb R^n)\cong\mathbb R^n\rtimes_\psi SO(n)$, where $\psi=\varphi|_{SO(n)}:SO(n)\to\operatorname{Aut}(\mathbb R^n)$.

\item Let $A$ be an $n\times n$ matrix with complex entries.  We define the \textbf{Hermitian transpose} $A^*$ of $A$ as the complex conjugate of the transpose of $A$.  In other words, $(A^*)_{jk}=\overline{A_{kj}}$ for $1\leqslant j,k\leqslant n$.  We say that $A$ is \textbf{unitary} if $A^*A=I_n$.

(a) If $A$ has real entries, then $A$ is unitary if and only if it is orthogonal.

(b) The $n\times n$ unitary matrices form a subgroup of the multiplicative group $GL_n(\mathbb C)$ of $n\times n$ nonsingular matrices with complex entries.  This group is denoted $U(n)$ and called a \textbf{unitary group}.

(c) If $A$ is a unitary matrix, $|\det A|=1$.  [First show that $\det(A^*)=\overline{\det A}$ for any $n\times n$ matrix $A$.]

(d) $SU(n)=\{A\in U(n):\det A=1\}$ is a normal subgroup of $U(n)$, and $U(n)/SU(n)$ is isomorphic to the multiplicative group $\mathbb T$ of complex numbers with absolute value $1$.  $SU(n)$ is called a \textbf{special unitary group}.

(e) If $\vec v=(x_1,\dots,x_n),\vec w=(y_1,\dots,y_n)\in\mathbb C^n$, we define the \textbf{(Hermitian) inner product} $\left<\vec v,\vec w\right>$ to be $\sum_{j=1}^n\overline{x_j}y_j$.  Show that for $\vec u,\vec v,\vec w\in\mathbb C^n$ and $\alpha\in\mathbb C$:
$$\left<\vec u,\vec v+\vec w\right>=\left<\vec u,\vec v\right>+\left<\vec u,\vec w\right>;$$
$$\left<\vec u,\alpha\vec v\right>=\alpha\left<\vec u,\vec v\right>;$$
$$\left<\vec w,\vec v\right>=\overline{\left<\vec v,\vec w\right>};$$
$$\vec u\ne\vec 0\implies\left<\vec u,\vec u\right>>0.$$
Note that as a consequence of these conditions, we have that $\left<\vec v+\vec w,\vec u\right>=\left<\vec v,\vec u\right>+\left<\vec w,\vec u\right>$ and $\left<\alpha\vec v,\vec u\right>=\overline{\alpha}\left<\vec v,\vec u\right>$.  Thus the inner product is \emph{linear} on the right and \emph{antilinear} on the left.

(f) If $A$ is an $n\times n$ complex matrix and $\vec v,\vec w\in\mathbb C^n$, then $\left<A\vec v,\vec w\right>=\left<\vec v,A^*\vec w\right>$.  [First explain why $\left<\vec v,\vec w\right>=\vec v^*\vec w$ when they are considered as column vectors.]  Conclude that if $A$ is unitary, then $\left<A\vec v,A\vec w\right>=\left<\vec v,\vec w\right>$.

(g) Use part (f) to show that every eigenvalue of a unitary matrix $A$ has absolute value $1$.  [In particular, if $A$ is a (real) orthogonal matrix, every eigenvalue of $A$ has absolute value $1$.]

\item If $n\geqslant 3$ and $A_1\dots A_n$ is a regular $n$-gon, show that the subgroup of $\operatorname{Isom}(\mathbb R^2)$ consisting of isometries which fix the $n$-gon is isomorphic to $D_n$.

\item (a) Show that an orientation-preserving isometry of $\mathbb R^n$ (with $n$ arbitrary) is a composition of rotations and translations.  It need not itself be a rotation or a translation; e.g., screw translations of $\mathbb R^3$.

(b) Use part (a) to show that every isometry of $\mathbb R^n$ is a composition of reflections.

\item Every nonidentity element of $SO(3)$ is a rotation around a uniquely determined axis.

\item This exercise studies $SO(4)$, rotations in $4$-dimensional Euclidean space.

(a) Let $T\in SO(4)$.  Show that there exist orthogonally complementary $2$-dimensional subspaces $V,W\subset\mathbb R^4$, and real numbers $-180^\circ\leqslant\alpha,\beta\leqslant 180^\circ$, such that $V$ and $W$ are invariant under $T$, $T|_V$ is rotation by an angle of $\alpha$ and $T|_W$ is rotation by an angle of $\beta$.  [Note that $\alpha$ or $\beta$ could be zero, in which case $T$ is rotation around a plane axis.]

(b) The planes $V$ and $W$ are determined by this property if and only if $\alpha\ne\pm\beta$.

If $\alpha=\pm\beta$, there are infinitely many possible pairs of orthogonally complementary planes.  In this case, we say that $T$ is an \textbf{isoclinic} rotation.

Now suppose $\vec u_1,\vec u_2,\vec u_3,\vec u_4$ is a positive\footnote{This means that $\det\begin{bmatrix}\uparrow&\uparrow&\uparrow&\uparrow\\\vec u_1&\vec u_2&\vec u_3&\vec u_4\\\downarrow&\downarrow&\downarrow&\downarrow\end{bmatrix}$ is positive.} orthonormal basis of $\mathbb R^4$ such that $\vec u_1,\vec u_2\in V$ and $\vec u_3,\vec u_4\in W$; with respect to said basis an isoclinic rotation looks like one of these:
$$\begin{bmatrix}\cos\alpha&-\sin\alpha\\\sin\alpha&\cos\alpha\\&&\cos\alpha&-\sin\alpha\\&&\sin\alpha&\cos\alpha\end{bmatrix}~~~~~~\begin{bmatrix}\cos\alpha&-\sin\alpha\\\sin\alpha&\cos\alpha\\&&\cos\alpha&\sin\alpha\\&&-\sin\alpha&\cos\alpha\end{bmatrix}$$
In the former case, $T$ is said to be \textbf{left isoclinic}; in the latter case $T$ is \textbf{right isoclinic}.

(c) Show that whether an isoclinic rotation $\ne\pm I_4$ is left or right is well defined, i.e., independent of the particular invariant planes $V,W$ and positive orthonormal basis.  Then show that any left isoclinic rotations with a particular angle $\alpha$ are conjugate in $SO(4)$, and likewise for right-isoclinic rotations.

(d) Let $S^3_R=\left\{\begin{bmatrix}a&-b&-c&d\\b&a&-d&-c\\c&d&a&b\\-d&c&-b&a\end{bmatrix}:a,b,c,d\in\mathbb R,a^2+b^2+c^2+d^2=1\right\}$.  Then $S^3_R$ is a subgroup of $SO(4)$. % Fair point.  When I mentioned quaternions, I meant it as an "elective" statement which readers have the right to ignore.  But I'll remove it anyway.

(e) $SO(4)$ is generated by the following matrices for all $\alpha\in\mathbb R$:
$$\begin{bmatrix}\cos\alpha&-\sin\alpha\\\sin\alpha&\cos\alpha\\&&1\\&&&1\end{bmatrix},\begin{bmatrix}1\\&\cos\alpha&-\sin\alpha\\&\sin\alpha&\cos\alpha\\&&&1\end{bmatrix},\begin{bmatrix}1\\&1\\&&\cos\alpha&-\sin\alpha\\&&\sin\alpha&\cos\alpha\end{bmatrix}$$
[Given $A\in SO(4)$, consider the unit vector $A(\vec e_1)=(x,y,z,w)$.  There exists $\alpha$ such that $z\sin\alpha+w\cos\alpha=0$ (why?), now apply the above matrix on the right to $(x,y,z,w)$ and the last component becomes zero.  After that, the middle matrix can make the third component zero, and then the left matrix can make the vector equal to $\vec e_1$.  We thus get an element $A'$ of $SO(4)$ which fixes $\vec e_1$, by left-multiplying $A$ by the above matrices.  Now $A'(\vec e_2)$ is of the form $(0,y',z',w')$ by orthogonality; use only the two matrices on the right to transfer this vector to $\vec e_2$.  Similar arguments eventually land you at the identity element of $SO(4)$.]

(f) $S^3_R$ is a normal subgroup of $SO(4)$.  [Show that the normalizer $N(S^3_R)$ contains all of the matrices stated in part (e).]

(g) If $a,b,c,d\in\mathbb R$, the determinant of $\begin{bmatrix}a&-b&-c&d\\b&a&-d&-c\\c&d&a&b\\-d&c&-b&a\end{bmatrix}$ is equal to $(a^2+b^2+c^2+d^2)^2$.  (Note that we are \emph{not} assuming $a^2+b^2+c^2+d^2=1$ here.)  [Use the formula $\det(kA)=k^n\det A$ for $n\times n$ matrices $A$.]  Conclude that the characteristic polynomial of that matrix is $[(x-a)^2+b^2+c^2+d^2]^2$.

(h) Show that $S^3_R$ is the set of all right isoclinic rotations.  [Use part (f) and the second half of part (c) to show that $S^3_R$ contains all right isoclinic rotations.  Conversely, given an element of $S^3_R$, use part (g) to show that it is a right isoclinic rotation.]

(i) For $g\in O(4)-SO(4)$ [i.e., $g$ is an orientation-reversing isometry of $\mathbb R^4$], $S^3_L=\{gag^{-1}:a\in S^3_R\}$ is a normal subgroup of $SO(4)$, and is the set of all left isoclinic rotations (hence is independent of which $g$ is picked).

(j) $S^3_L\cap S^3_R=\{I_4,-I_4\}$.

(k) If $A\in S^3_L$ and $B\in S^3_R$, then $AB=BA$.  [Part (j) and the normality of the subgroups entail $A^{-1}B^{-1}AB=\pm I_4$.  Now, $A^{-1}B^{-1}AB=I_4$ if $A=B=I_4$, and the expression varies continuously in the entries of $A$ and $B$.]

\item\emph{(Frieze groups.)} \---- We let $S=\{(x,y)\in\mathbb R^2:|y|\leqslant 1\}$.  Observe that $\{T\in\operatorname{Isom}(\mathbb R^2):T(S)=S\}$ is a subgroup of $\operatorname{Isom}(\mathbb R^2)$, which we denote as $\operatorname{Isom}(S)$.

(a) $[A,\vec v]\in\operatorname{Isom}(S)$ if and only if $A$ is diagonal with $\pm 1$ as its entries and $\vec v$ is parallel to the $x$-axis.

We set $\operatorname{Isom}^+(S)=\operatorname{Isom}(S)\cap\operatorname{Isom}^+(\mathbb R^2)$, and $\operatorname{Tra}(S)$ the subgroup of $\operatorname{Isom}(S)$ consisting of the translations; i.e.,
$$\operatorname{Tra}(S)=\{[I_n,\vec v]:\vec v\text{ parallel to the }x\text{-axis}\}.$$
(b) $\operatorname{Tra}(S)\subset\operatorname{Isom}^+(S)\subset\operatorname{Isom}(S)$ are subgroups with $[\operatorname{Isom}(S):\operatorname{Isom}^+(S)]=2=[\operatorname{Isom}^+(S):\operatorname{Tra}(S)]$.  Furthermore, $\operatorname{Tra}(S)\cong\mathbb R$.

(c) Let $V=\left\{\begin{bmatrix}a&0\\0&b\end{bmatrix}:a,b=\pm 1\right\}$; this is an abelian group under multiplication.  There is a surjective homomorphism $\psi:\operatorname{Isom}(S)\to V$ given by $[A,\vec v]\mapsto A$, and the kernel of $\psi$ is $\operatorname{Tra}(S)$.

A \textbf{Frieze group} is a subgroup $G\subset\operatorname{Isom}(S)$ such that $G\cap\operatorname{Tra}(S)\cong\mathbb Z$.  Thus, $G$ has a nontrivial translation which generates all translations in $G$.  We will assume Frieze groups to be the same if, for some matrix $A=\begin{bmatrix}a&0\\0&\pm 1\end{bmatrix}$ and horizontal vector $\vec v=(b,0)$, the map $\mathbf x\mapsto A\vec x+\vec v$ conjugates one subgroup to the other.  (Thus, for example, the exact distance of the translations will not be important to us.)

Our ambition in this exercise is to show that there are exactly $7$ Frieze groups.

If $G$ is a Frieze group, let $\varphi=\psi|_G:G\to V$.  Then $\ker\varphi$ is a subgroup of $G$ of index $1$, $2$ or $4$, and $\ker\varphi\cong\mathbb Z$ (why?).

(d) If $\operatorname{im}\varphi=\{I_2\}$ (i.e., $\varphi$ is trivial), we have $G\cong\mathbb Z$; in other words, $G$ is cyclic, generated by a translation.

(e) Suppose $\operatorname{im}\varphi=\left\{I_2,\begin{bmatrix}1&0\\0&-1\end{bmatrix}\right\}$, which means that $G$ consists of translations and reflections/glide reflections over the $x$-axis.  Then either $G$ is generated by a translation and a reflection over the $x$-axis (in which case $G\cong\mathbb Z\times\mathbb Z/2\mathbb Z$), or else $G$ is cyclic and generated by a glide reflection, which moves half the distance of the shortest translation in $G$ (in which case $G\cong\mathbb Z$).  [If $\ker\varphi=\left<a\right>$ and $g\in G$ is an orientation-reversing isometry, then $g^2=a^n$ for some $n\in\mathbb Z$; depending on whether $n$ is even or odd, there exists an orientation-reversing isometry $g\in G$ such that either $g^2=1$ or $g^2=a$.]

(f) Suppose $\operatorname{im}\varphi=\left\{I_2,\begin{bmatrix}-1&0\\0&1\end{bmatrix}\right\}$, which means that $G$ consists of translations and reflections over lines parallel to the $y$-axis.  Show that $G$ is isomorphic to $D(\mathbb Z)$, the subgroup of $S(\mathbb Z)$ generated by the translation $x\mapsto x+1$ from $\mathbb Z\to\mathbb Z$ and the negation $x\mapsto -x$.  [$D(\mathbb Z)$ is called the \textbf{infinite dihedral group}.]  Note that the lines of reflection are spaced apart by half the distance of the shortest translation.

(g) Suppose $\operatorname{im}\varphi=\left\{I_2,\begin{bmatrix}-1&0\\0&-1\end{bmatrix}\right\}$, which means that $G$ consists of translations and $180^\circ$ rotations around points on the $x$-axis.  Show that $G\cong D(\mathbb Z)$ in this case as well.  [Thus this group is isomorphic to the one in part (f); yet, they are still different Frieze groups because this one has no orientation-reversing symmetry but the one in (f) does.]

(h) The final case to consider is when $\operatorname{im}\varphi=V$; i.e., $\varphi$ is surjective.  In this case, suppose $\ker\varphi=\left<a\right>$, and show that there exists $g\in\varphi^{-1}\left(\begin{bmatrix}1&0\\0&-1\end{bmatrix}\right)$ such that either $g^2=1$ or $g^2=a$.  Show that if $g^2=1$ then $G\cong D(\mathbb Z)\times\mathbb Z/2\mathbb Z$, and if $g^2=a$ then $G\cong D(\mathbb Z)$.  These are the remaining two Frieze groups.

Below are Frieze patterns whose isometry groups are the Frieze groups established in parts (d)-(h).
\begin{center}\includegraphics[scale=.4]{FriezePatterns.png}\end{center}
\item\emph{(Conic sections revisited.)} \---- A \textbf{quadratic curve} in $\mathbb R^2$ is given by an equation $Ax^2+Bxy+Cy^2+Dx+Ey+F=0$, where $A,B,C,D,E,F\in\mathbb R$ and $A,B,C$ are not all zero.

The number $B^2-4AC$ is called the \textbf{discriminant} of the curve.  The curve is:
\begin{itemize}
\item A \textbf{circle} if $A=C$ and $B=0$;

\item An \textbf{ellipse} if $B^2-4AC<0$;

\item A \textbf{parabola} if $B^2-4AC=0$;

\item A \textbf{hyperbola} if $B^2-4AC>0$.
\end{itemize}
Note that the curve can be given by many equations, but such equations are scalar multiples of one another (if $k\ne 0$, $kAx^2+kBxy+kCy^2+kDx+kEy+kF=0$ is an equation for the same curve).

Also, in this problem, a circle is considered an ellipse, since $B=0$ implies $B^2-4AC<0$.

(a) Isometries of $\mathbb R^2$ preserve circles, ellipses, parabolas and hyperbolas.  [It suffices to show that translations, rotations around the origin, and reflection over the $x$-axis preserve them.]

Show that the ellipse, parabola and hyperbola of Exercise 9 of Section 2.1 satisfy this exercise's definitions of their respective names.  Conclude that the notions of the conics coincide with Exercise 9 of Section 2.1.

(b) More generally, let $G$ be the subgroup of $S(\mathbb R^2)$ generated by $GL_2(\mathbb R)$ and the translations.  [Elements of $G$ are called \textbf{affine transformations} of $\mathbb R^2$.]  Then $G$ acts transitivitely on each of the following sets: (i) the ellipses; (ii) the parabolas; (iii) the hyperbolas.

Now consider the cone $z^2=x^2+y^2$ in $\mathbb R^3$.  (Technically, this is two antipodal cones with their points meeting at the origin.)  We claim that each of the quadratic curves can be obtained by intersecting this cone with a plane in $\mathbb R^3$ that does not go through the origin (hence the name ``conic sections'').  The intersection of such a plane with the cone is called a \textbf{cross section} of the cone.  It can be identified with a subset of $\mathbb R^2$, when one establishes an isometry from $\mathbb R^2$ to the plane (by taking a $3\times 2$ matrix with orthonormal columns whose image is parallel to the plane, then left-multiplying by a translation).

Thus, let $P$ be a plane that does not go through the origin.  By rotating around the $z$-axis, we may assume that the plane has equation $ax+bz=c$ ($a,b$ not both zero, $c\ne 0$).

(c) If $a=0$, the cross section is a circle.

(d) If $|a|<|b|$, the cross section is an ellipse.

(e) If $|a|=|b|$, the cross section is a parabola.

(f) If $|a|>|b|$, the cross section is a hyperbola.
\end{enumerate}

\subsection*{2.7. Finite Groups of Isometries}
\addcontentsline{toc}{section}{2.7. Finite Groups of Isometries}
In the remainder of the chapter, we shall focus on bounded figures with only finitely many isometries preserving them.  First, let us introduce the concept of isometries of various figures in $\mathbb R^n$.

In general, if $X\subset\mathbb R^n$ is a subset, we call $X$ a \textbf{figure} in $n$-space, and define $\operatorname{Isom}(X)=\{f\in\operatorname{Isom}(\mathbb R^n):f(X)=X\}$.  $\operatorname{Isom}(X)$ is a subgroup of $\operatorname{Isom}(\mathbb R^n)$, called the \textbf{isometry group} of $X$.  [See Exercise 8 of the previous section for an example.]

Now for some important notes:\\

\noindent(1) If $f\in\operatorname{Isom}(\mathbb R^n)$ maps $X$ into $X$, it is \emph{not} necessarily an isometry of $X$.  For example, suppose $X$ is the upper half plane $\{(x,y)\in\mathbb R^2:y>0\}$, and $f$ is the isometry $(x,y)\mapsto(x,y+1)$ of $\mathbb R^n$.  Then $f(X)\subset X$; however, $f(X)\ne X$ because, for example, $(0,1/2)$ is in $X$ but not in the image $f(X)$.

The example above shows that if $X$ is a figure, there may be distance-preserving maps on $X$ that are not bijective.  However, a distance-preserving map $g:X\to X$ must be \emph{injective}; for any $x,y\in X$, $g(x)=g(y)\iff\|g(x)-g(y)\|=0\iff\|x-y\|=0\iff x=y$.  Since the map need not be surjective, the distance-preserving maps need not form a group under function composition (due to lack of inverses).

A bijective distance-preserving map $X\to X$ (not assumed to extend to an isometry of $\mathbb R^n$) is called a \textbf{direct isometry} of $X$.  It is readily verified that a direct isometry has an inverse which is also a direct isometry, hence the direct isometries form a group under function composition.  The group of direct isometries of $X$ is denoted $\widetilde{\operatorname{Isom}}(X)$.\\

\noindent(2) A direct isometry $f:X\to X$ always extends to an isometry of $\mathbb R^n$, but that isometry need not be unique.  The proof that the extension exists requires a lot of details from analysis, so it will be omitted here.\footnote{For a proof, see Theorem 11.4 in \emph{Embeddings and Extensions in Analysis} by J.H.~Wells and L.R.~Williams.} %https://math.stackexchange.com/questions/1687436/every-partially-defined-isometry-can-be-extended-to-a-isometry

However, suppose $X$ is the $xy$-plane $\{(x,y,0):x,y\in\mathbb R\}$ in $\mathbb R^3$, and $f:X\to X$ is the identity map.  Then there are two isometries of $\mathbb R^3$ that extend $f$: the identity, and the map $(x,y,z)\mapsto(x,y,-z)$ which reflects over the $xy$-plane; in this case the extension is not unique.  [Note that uniqueness fails if and only if $X$ is contained in a hyperplane.]

We thus get a surjective homomorphism $f\mapsto f|_X$ from $\operatorname{Isom}(X)\to\widetilde{\operatorname{Isom}}(X)$.\\

\noindent If $X$ is a figure, we define the \textbf{orientation-preserving isometry group} of $X$, denoted $\operatorname{Isom}^+(X)$, to be $\operatorname{Isom}(X)\cap\operatorname{Isom}^+(\mathbb R^2)$.  This is the group of isometries which fix $X$ \emph{and} preserve orientation.

Note, however, that we cannot define a notion of orientation for direct isometries in $\widetilde{\operatorname{Isom}}(X)$.  Whenever a direct isometry of $X$ extends to more than one isometry of $\mathbb R^n$, some extensions preserve orientation and others reverse it (e.g., if $X$ is the $xy$-plane in $\mathbb R^3$, then the identity on $X$ extends to both the identity on $\mathbb R^3$ and the reflection $(x,y,z)\mapsto(x,y,-z)$ over the plane).  However, if $X$ is a $k$-plane, we can \textbf{orient} it, thus giving a notion of when a direct isometry is orientation-preserving.  We will not delve into details here.

We now start with a fundamental proposition about finite isometry groups; see Exercise 1 for a followup.\\

\noindent\textbf{Lemma 2.48.} \emph{If $T\in\operatorname{Isom}(\mathbb R^n)$, $\vec x_1,\dots,\vec x_n\in\mathbb R^n$ and $c_1,\dots,c_n\in\mathbb R$ with $c_1+\dots+c_n=1$, $T(c_1\vec x_1+\dots+c_n\vec x_n)=c_1T(\vec x_1)+\dots+c_nT(\vec x_n)$.  In other words, $T$ preserves affine combinations of vectors.}
\begin{proof}
Write $T=[A,\vec v]$ with $A\in O(n)$ and $\vec v\in\mathbb R^n$.  Then this is a straightforward computation:
$$T(c_1\vec x_1+\dots+c_n\vec x_n)=A(c_1\vec x_1+\dots+c_n\vec x_n)+\vec v=c_1A\vec x_1+\dots+c_nA\vec x_n+\vec v$$
$$=c_1A\vec x_1+\dots+c_nA\vec x_n+1\vec v=c_1A\vec x_1+\dots+c_nA\vec x_n+(c_1+\dots+c_n)\vec v$$
$$=c_1A\vec x_1+\dots+c_nA\vec x_n+c_1\vec v+\dots+c_n\vec v=c_1(A\vec x_1+\vec v)+\dots+c_n(A\vec x_n+\vec v)$$
$$=c_1T(\vec x_1)+\dots+c_nT(\vec x_n).$$
\end{proof}
\noindent\textbf{Proposition 2.49.} \emph{Let $G$ be a finite subgroup of $\operatorname{Isom}(\mathbb R^n)$.  Then there exists $\vec p\in\mathbb R^n$ such that $g(\vec p)=\vec p$ for all $g\in G$.}
\begin{proof}
Let $\vec p=\frac 1{|G|}\sum_{g\in G}g(\vec 0)$.  We claim that this does the trick.  After all, if $T\in\operatorname{Isom}(\mathbb R^n)$, then applying Lemma 2.48 with $n=|G|$ and $c_1=\dots=c_n=1/|G|$ gives
$$T(\vec p)=T\left(\frac 1{|G|}\sum_{g\in G}g(\vec 0)\right)=\frac 1{|G|}\sum_{g\in G}T(g(\vec 0)).$$
In particular, for $h\in G$, we have
$$h(\vec p)=\frac 1{|G|}\sum_{g\in G}h(g(\vec 0))=\frac 1{|G|}\sum_{g\in G}(h\circ g)(\vec 0).$$
Yet, as $g$ runs through all elements of $G$, since $G$ is a group, $h\circ g$ also runs through every element of $G$ exactly once, just in a different order.  Hence $\frac 1{|G|}\sum_{g\in G}(h\circ g)(\vec 0)$ is equal to $\frac 1{|G|}\sum_{g\in G}g(\vec 0)=\vec p$.  We conclude that $h(\vec p)=\vec p$ for all $h\in G$ as desired.
\end{proof}

\noindent Proposition 2.49 states that any finite group of isometries fixes some point entirely.  Such a group of isometries is called a \textbf{point group}.  If $G$ is a point group, then $G$ is conjugate to a subgroup of $O(n)$: namely, if $\vec p$ is fixed by $G$ and $\tau:\vec x\mapsto\vec x+\vec p$ is the translation, $\tau^{-1}\circ G\circ\tau$ is a subgroup of $\operatorname{Isom}(\mathbb R^n)$ which fixes $\vec 0$ entirely.  Hence, the conjugate is contained in $O(n)$ (the set of $[A,\vec v]$ with $\vec v=\vec 0$).

In particular, every finite group of isometries is conjugate to a subgroup of $O(n)$, which means that we may restrict ourselves to finite subgroups of $O(n)$.  Here are some examples:\\

(1) Fix an integer $n\geqslant 1$.  In $O(2)$, let $\theta=\begin{bmatrix}\cos(2\pi/n)&-\sin(2\pi/n)\\\sin(2\pi/n)&\cos(2\pi/n)\end{bmatrix}$; this rotates counterclockwise around the origin by $\frac{2\pi}n$, or an $n$-th of a full turn.  Moreover, $\theta^n=1$, as one can verify, and $\{1,\theta,\dots,\theta^{n-1}\}$ is a subgroup of $O(2)$; for that matter, of $SO(2)$.  It is the orientation-preserving symmetry group of a regular $n$-gon centered at the origin.\\

(2) Take $\theta\in O(2)$ as before, and let $\varphi=\begin{bmatrix}1&0\\0&-1\end{bmatrix}$.  Then $\varphi^2=1$ and $\varphi\theta\varphi=\theta^{-1}$.  Moreover, $\{1,\theta,\dots,\theta^{n-1},\varphi,\varphi\theta,\dots,\varphi\theta^{n-1}\}$ is a subgroup of $O(2)$ isomorphic to the dihedral group $D_n$; but this time, it is not contained in $SO(2)$.\\

(3) If $G$ is any finite subgroup of $O(2)$, then $G_1=\left\{\begin{bmatrix}1&0\\0&a\end{bmatrix}:a\in G\right\}$ is a finite subgroup of $O(3)$, where $\begin{bmatrix}1&0\\0&a\end{bmatrix}$ is viewed as a block matrix with $a$ as a $2\times 2$ block in the lower right corner.  Note that $G_1$ fixes every point on the line spanned by $\vec e_1$.\\

The remaining examples cover the five Platonic solids in Euclidean $3$-space.  
\begin{center}
\begin{tabular}{ccccc}
\includegraphics[scale=.1]{Tetrahedron.png}&
\includegraphics[scale=.1]{Octahedron.png}&
\includegraphics[scale=.1]{Icosahedron.png}&
\includegraphics[scale=.1]{Cube.png}&
\includegraphics[scale=.1]{Dodecahedron.png}\\
Tetrahedron&
Octahedron&
Icosahedron&
Cube&
Dodecahedron
\end{tabular}
\end{center}
A Platonic solid is obtained by picking one type of regular polygon, making copies of it, and hinging copies together by the edges, as shown above.  The polygons are called \textbf{faces}, and the edges (resp., vertices) of the polygon are called \textbf{edges} (resp., \textbf{vertices}) of the solid.  It is additionally required that every vertex of the solid have the same number of faces surrounding it.

Which Platonic solids exist?  We can use casework on the type of regular polygon used, and the number of faces surrounding each vertex. % Basic graph theory shows that the solid is uniquely determined by these parameters. (I don't understand what you meant by the revision)

Let us first consider Platonic solids whose faces are equilateral triangles.  If there are three equilateral triangles around a vertex, the tetrahedron (above) is obtained by gluing one more equilateral triangle as a base.  If there are four triangles around a vertex, you get the octahedron, by making another vertex with four triangles around it, then pasting them together.  If there are five triangles around a vertex, you get the icosahedron ($20$ faces)\----this time you need to make a copy of the vertex; then a ring of ten triangles, alternating between pointing up and down; then sandwich the ring between the caps.  If there are six triangles around a vertex, then the vertex lies flat in the plane\----continuing to add more triangles will merely get the triangular tiling from Section 2.3; it will never close up upon itself into a solid body.  The reader can see that it is impossible to construct a Platonic solid with 7 or more triangles to a vertex, because they will inevitably buckle with creases in different directions.\footnote{As a foretaste of Chapter 4, we remark that in hyperbolic geometry, any number $n\geqslant 7$ of equilateral triangles will be able to fit around a given vertex without buckling.  The same holds for any number $n\geqslant 5$ of squares, $n\geqslant 4$ of regular pentagons, etc.}

Hence, the only Platonic solids made from triangles are the tetrahedron, octahedron and icosahedron.

What about squares?  If there are three squares around a vertex, the cube (or hexahedron) is obtained by gluing on another triple of squares around another vertex.  If there are four squares around a vertex, then they lie flat in the plane, and you get a tiling from Section 2.3; and 5 or more squares will never work at all.  Thus, the only Platonic solid made from squares is the cube.

Now for pentagons.  If there are three pentagons around a vertex, the dodecahedron is obtained by gluing together four copies of that pentagon triple.  However, since a pentagon has an angle of $108^\circ>90^\circ$, trying to fit four pentagons to a vertex will not even work lying flat in the plane; it will buckle with creases in different directions.  Therefore, the only Platonic solid made from pentagons is the dodecahedron.

We may try to carry on with hexagons: if there are three hexagons to a vertex, they lie flat already.  And for $n\geqslant 7$, you can never fit three regular $n$-gons to a vertex.  This means that there are no more Platonic solids, and we have classified them all.

The following constructions in $\mathbb R^3$ yield the Platonic solids and their isometry groups.\\

(4) One can form a regular tetrahedron, $\mathbf{Tet}$, by taking for vertices the points
$$(1,1,1),(1,-1,-1),(-1,1,-1),(-1,-1,1)\in\mathbb R^3.$$
Any two of those points have a distance equal to $2\sqrt 2$ between them, and hence any three of those points will form an equilateral triangle with side length $2\sqrt 2$.  Since $\mathbf{Tet}$ is the convex hull of its vertices, an isometry $T\in\operatorname{Isom}(\mathbb R^3)$ will be in $\operatorname{Isom}(\mathbf{Tet})$ if and only if $T$ permutes the vertices.

What kinds of isometries $T$ do this?  The first important observation is that since the sum of the vertices is zero (verify), we have $T(\vec 0)=\vec 0$ by Lemma 2.48 (with $n=4$ and $c_1=\dots=c_n=1/4$).  Hence $T\in O(3)$, and we may think of $T$ as a matrix.

It is readily verified that the six points $\pm\vec e_1,\pm\vec e_2,\pm\vec e_3$ are precisely the midpoints of the six edges (unordered pairs of distinct vertices), hence are permuted by $T$.  Hence $T$ is a monomial matrix\footnote{This means $T$ has exactly one nonzero entry in each row and in each column.} with $\pm 1$ as its nonzero entries.  Moreover, since each vertex has its components multiplying to (positive) $1$, the nonzero entries of $T$ must multiply to $1$, in order for $T$ to permute the vertices.  The reader can then verify that under these conditions, we have isometries of the tetrahedron:
$$\begin{bmatrix}1&0&0\\0&-1&0\\0&0&-1\end{bmatrix},~~~~\begin{bmatrix}0&1&0\\1&0&0\\0&0&1\end{bmatrix},~~~~\begin{bmatrix}0&-1&0\\0&0&1\\-1&0&0\end{bmatrix},~~~~\text{etc.}$$
There are $24$ isometries of the tetrahedron: $6$ for the permutation matrix used to construct $T$, times $2^3=8$ for whether each nonzero entry is $1$ or $-1$, divided by $2$ due to the constraint that the nonzero entries must multiply to $1$.

Note that if $V$ is the set of vertices of $\textbf{Tet}$, we get an injective homomorphism $\varphi:\operatorname{Isom}(\textbf{Tet})\to S(V)$.  [It is injective because an affine map of the convex hull of $V$ is determined by its action on $V$.]  Since $|\operatorname{Isom}(\textbf{Tet})|=|S(V)|=24$, $\varphi$ is an isomorphism; hence, by Exercise 7(b) of Section 1.3, $\operatorname{Isom}(\textbf{Tet})\cong S_4$.  Thus we have concretely represented the isometry group of the tetrahedron.

An isometry $T\in\operatorname{Isom}(\textbf{Tet})$ is orientation-preserving if and only if its determinant is $1$.  Since the determinant of a monomial matrix is the sign of the permutation times the product of the nonzero entries, but the product of the nonzero entries of $T$ is necessarily $1$, we get that $\det T=1$ if and only if $T$ is an even permutation.  There are $3$ even permutations on the three dimensions, and hence $12$ orientation-preserving isometries of the tetrahedron.  In other words, $|\operatorname{Isom}^+(\mathbf{Tet})|=12$.  The isomorphism $\varphi$ corresponds $\operatorname{Isom}^+(\mathbf{Tet})$ with $A_4$, the only subgroup of $S_4$ of order $12$ (by Exercise 13(h) of Section 1.6).  Hence $\operatorname{Isom}^+(\mathbf{Tet})\cong A_4$.\\

(5) One can form a regular octahedron $\mathbf{Oct}$ by taking the six points
$$\pm\vec e_1,\pm\vec e_2,\pm\vec e_3\in\mathbb R^3$$
for the vertices.  A pair of vertices $\pm\vec e_j,\pm\vec e_k$ where $j\ne k$ and the two signs are independent of one another, is at a distance of $\sqrt 2$; there are $12$ such unordered pairs of vertices, and they give the edges.  (The vertices $\vec e_1$ and $-\vec e_1$ don't form an edge; they form a line segment of length $2$ going through the origin, a body diagonal.)  Finally, there are eight ways to assign either $+$ or $-$ to each of the three expressions $\pm\vec e_1,\pm\vec e_2,\pm\vec e_3$; each case gives a face of the octahedron upon taking the convex hull.

If $T\in\operatorname{Isom}(\mathbf{Oct})$, then since $T$ permutes the $\pm\vec e_j$, again $T(\vec 0)=\vec 0$ by Lemma 2.48, hence $T\in O(3)$.  Moreover $T$ must be a monomial matrix with $\pm 1$'s as its nonzero entries.  But this time, the nonzero entries need not multiply to $1$.  In fact, it is clear that \emph{any} monomial matrix with $\pm 1$'s as the nonzero entries is an isometry of $\mathbf{Oct}$.  There are $48$ such matrices; therefore $|\operatorname{Isom}(\mathbf{Oct})|=48$.

The reader can readily verify that $\operatorname{Isom}^+(\mathbf{Oct})$ has order $24$ [see Exercise 4(a)].  Exercise 6 shows that $\operatorname{Isom}^+(\mathbf{Oct})\cong S_4$, and $\operatorname{Isom}(\mathbf{Oct})\cong S_4\times\mathbb Z/2\mathbb Z$.\\

(6) One can form a regular cube $\mathbf{Cube}$ by taking the eight points $(\pm 1,\pm 1,\pm 1)$ as the vertices.  Two vertices are joined by an edge if and only if they differ in exactly one component (i.e., $(a,b,c)$ with $a,b,c\in\{-1,1\}$ is connected to $(-a,b,c)$, $(a,-b,c)$ and $(a,b,-c$)).  There are six possible ways to pick one of the three components of the vector and set it equal to $1$ or $-1$; in each case, you get a face of the cube.

It turns out that $\operatorname{Isom}(\mathbf{Cube})=\operatorname{Isom}(\mathbf{Oct})$ as subgroups of $\operatorname{Isom}(\mathbb R^3)$: both consist of monomial matrices with $\pm 1$'s as the nonzero entries.  This is because the cube and octahedron are \emph{dual polyhedra}; the faces of each one correspond bijectively to the vertices of the other, and the solids were arranged in these examples so that the octahedron's vertices are the centers of the cube's faces.  They can be rescaled so the cube's vertices are the centers of the octahedron's faces.  As an immediate consequence as well, $\operatorname{Isom}^+(\mathbf{Cube})=\operatorname{Isom}^+(\mathbf{Oct})$.

Since the cube and octahedron have the same symmetry, we may refer to either one interchangeably when dealing with the isometry group.  For definiteness, we shall normally refer to the octahedron; this is the standard convention, the isometry group being referred to as the ``octahedral symmetry group'' by mathematicians.\\

(7) Throughout this example and the next, we let $\phi=\frac{1+\sqrt 5}2=1.6180339\dots$.  This is the \textbf{golden ratio}, and it satisfies $\phi^2=\phi+1$, as one can readily verify.

There are $12$ points in $\mathbb R^3$ of the form
$$(\pm\phi,\pm 1,0),~~~~(0,\pm\phi,\pm 1),~~~~(\pm 1,0,\pm\phi)$$
where the signs are independent of one another (e.g., in $(\pm\phi,\pm 1,0)$, the first two coordinates need not have the same sign as each other).  The convex hull of the 12 points is a regular icosahedron $\mathbf{Icos}$.  There are $30$ edges (of length $2$), and $20$ (triangular) faces: the reader should take the time to work out which vertices they connect in terms of the above formulas.

Now let $T\in\operatorname{Isom}(\mathbf{Icos})$; again $T$ permutes the vertices, so as in the previous three examples we have $T\in O(3)$.

The midpoints of the edges are points of the form $(\pm\phi,0,0)$, $\frac 12(\pm\phi^2,\pm 1,\pm\phi)$, along with even permutations of them; $T$ must permute those points.  Since $T$ sends $(\phi,0,0)=\phi\vec e_1$ to one of those points, the left column of $T$ must look like one of these:
$$\begin{bmatrix}\pm 1\\0\\0\end{bmatrix},~~~~\begin{bmatrix}\pm\frac 12\phi\\\pm\frac 12\phi^{-1}\\\pm\frac 12\end{bmatrix},~~~~\text{or even permutations of those.}$$ % The even permutations could change the signs, but since \frac 12(\pm\phi^2,\pm 1,\pm\phi) already has complete choice of signs, just the permutations alone are enough...
The same manifestly holds for the other two columns of $T$.

If $T$ contains any $\pm 1$ entry, the other entries in the same row and column must be zero (by orthogonality).  Hence since the columns must be of the form written above, $T$ is a monomial matrix with $\pm 1$ as its nonzero entries.  In this case, \emph{$T$ must be an even permutation}, even though its nonzero entries need not multiply to $1$: indeed, if $T$ were an odd permutation, then $T$ would map $(\pm\phi,\pm 1,0)$ to an odd permutation of the vector, and such a vector is not a vertex of $\mathbf{Icos}$.  There are $24$ isometries $T$ of this form: $3$ for the even permutation times $2^3=8$ for the sign of each nonzero entry. % Here, T is a monomial matrix with \pm 1 as its nonzero entries, not necessarily a permutation matrix.  So it can change the signs.

Now suppose that $T$ does not contain $\pm 1$.  Then it must contain even permutations of $\begin{bmatrix}\pm\frac 12\phi\\\pm\frac 12\phi^{-1}\\\pm\frac 12\end{bmatrix}$ as its columns.  Yet since $T$ is orthogonal, the dot product of any two columns is zero.  The columns thus must have \emph{different} permutations of that vector, because if two columns have the same permutation with various signs on the entries, the following formulas should convince you that it is impossible for their dot product to be zero:
$$\left(\frac 12\phi\right)^2=\frac{\phi+1}4,~~~~\left(\frac 12\phi^{-1}\right)^2=\frac{2-\phi}4,~~~~\left(\frac 12\right)^2=\frac 14$$
Hence, each even permutation of the vector is involved in exactly one column; in other words, $T$ looks like this when the columns are permuted:
$$\begin{bmatrix}\pm\frac 12\phi&\pm\frac 12\phi^{-1}&\pm\frac 12\\\pm\frac 12\phi^{-1}&\pm\frac 12&\pm\frac 12\phi\\\pm\frac 12&\pm\frac 12\phi&\pm\frac 12\phi^{-1}\end{bmatrix}$$
Now we can use the fact that the dot product of two columns is zero to determine some of the signs from other ones.  From the formulas:
$$\left(\frac 12\phi\right)\left(\frac 12\phi^{-1}\right)=\frac 14,~~~~\left(\frac 12\phi^{-1}\right)\left(\frac 12\right)=\frac{\phi-1}4,~~~~\left(\frac 12\right)\left(\frac 12\phi\right)=\frac{\phi}4$$
we see that the dot product of the first two columns above is zero if and only if either the bottom two entries of the first two columns have a positive product, and each of the other two rows of the first two columns has a negative product; or else the bottom two entries have a negative product and each of the other two rows of the first two columns has a positive product.  Hence, once the first column is written out, there are only two possibilities for the signs in the second column.  The same holds for the third column.

It turns out that we will successfully get an isometry, \emph{if and only if the main diagonal of $T$ has distinct entries up to sign}.  The above matrix, for example, has
$$\begin{bmatrix}\frac 12\phi&-\frac 12\phi^{-1}&\frac 12\\\frac 12\phi^{-1}&-\frac 12&-\frac 12\phi\\\frac 12&\frac 12\phi&-\frac 12\phi^{-1}\end{bmatrix}$$
as a possible way to assign the signs.  We leave it to the reader to verify that this is really an isometry.

By multiplying by various monomial matrices in $\operatorname{Isom}(\mathbf{Icos})$, we see that we still have an isometry whenever we negate any column or row, or apply an \emph{even} permutation to the columns (odd permutations will not work).  There are $96$ isometries $T$ of this form: $3$ for the even permutation, times $2^3=8$ for the signs of the entries of the first column, times $2$ for the signs in the second column (according to the previous paragraph), times $2$ for the signs in the third column.

Overall, there are $24+96=120$ isometries of the regular icosahedron.  Since there are obviously some which reverse orientation, Exercise 4(a) shows that there are $60$ orientation-preserving isometries of the regular icosahedron.  Exercise 6 shows that $\operatorname{Isom}^+(\mathbf{Icos})\cong A_5$, and $\operatorname{Isom}(\mathbf{Icos})\cong A_5\times\mathbb Z/2\mathbb Z$.\\

(8) Consider the $20$ points in $\mathbb R^3$ of the form
$$(\pm 1,\pm 1,\pm 1),~~~~(\pm\phi^{-1},\pm\phi,0),~~~~(\pm\phi,0,\pm\phi^{-1}),~~~~(0,\pm\phi^{-1},\pm\phi)$$
where the signs are independent of one another.  Their convex hull is a regular dodecahedron $\mathbf{Dodec}$.  There are $30$ edges (of length $2\phi^{-1}$), and $12$ faces that are regular pentagons.  The reader should take the time to work out which vertices they connect.

As in Example (6) one can show that this dodecahedron is dual to the icosahedron: the center of a face of either of these solids is a scalar multiple of a vertex of the other solid.  Thus, we get $\operatorname{Isom}(\mathbf{Dodec})=\operatorname{Isom}(\mathbf{Icos})$ as subgroups of $\operatorname{Isom}(\mathbb R^3)$.  Therefore also $\operatorname{Isom}^+(\mathbf{Dodec})=\operatorname{Isom}^+(\mathbf{Icos})$.  Since the dodecahedron and icosahedron have the same symmetry, we may refer to either one interchangeably when dealing with the isometry group.  We shall usually refer to the icosahedron, as this isometry group is called the ``icosahedral symmetry group.''\\ % Far as I can tell, neither solid's vertices are inscribed at the face-centers of the other. Why?

\noindent Now that we have constructed sophisticated examples of finite groups of isometries of $\mathbb R^3$, let us conclude this chapter by showing that finite groups of isometries of $\mathbb R^2$ can never be that complicated.  In fact, we can easily classify them right now.

First recall from Section 2.6 that every element of $SO(2)$ is of the form $R_\theta=\begin{bmatrix}\cos\theta&-\sin\theta\\\sin\theta&\cos\theta\end{bmatrix}$ with $\theta\in\mathbb R$, and this matrix is rotation by an angle of $\theta$ around the origin: hence $SO(2)$ consists of only rotations.\\

\noindent\textbf{Theorem 2.50.} \textsc{(Classification of finite subgroups of $SO(2)$ and $O(2)$)}

(i) \emph{Every finite subgroup of $SO(2)$ is cyclic of order $n$, generated by a rotation of angle $2\pi/n$.}

(ii) \emph{Every finite subgroup of $O(2)$ is either cyclic of order $n$ as in \text{(i)}, or else it is isomorphic to $D_n$, generated by a rotation of angle $2\pi/n$ and a reflection.}\\

\noindent The theorem effectively states that Examples (1) and (2) above give all finite subgroups of $SO(2)$ and $O(2)$.
\begin{proof}
(i) Let $G\subset SO(2)$ be a finite subgroup.  We may assume $G\ne\{1\}$.  Write $G=\{1,R_{\alpha_1},\dots,R_{\alpha_k}\}$ where $0<\alpha_1<\alpha_2<\dots<\alpha_k<2\pi$.  [This is possible because $R_\theta=R_{\theta+2\pi}$ for any $\theta$, hence the subscript can be arranged to be between $0$ and $2\pi$; and then the subscripts can be sorted in ascending order.]  Let $\lambda=\alpha_1$.  Then $\lambda>0$ and $R_\lambda\in G$, but $R_\mu\notin G$ for any $0<\mu<\lambda$.

We now show that for $\gamma\in\mathbb R$,
\begin{equation}\tag{*}
R_\gamma\in G\iff\gamma/\lambda\in\mathbb Z.
\end{equation}
After all, if $\gamma/\lambda\in\mathbb Z$, then $\gamma=m\lambda$ for some $m\in\mathbb Z$; moreover, since $G$ is a subgroup, $R_\gamma=R_{m\lambda}=R_\lambda^m\in G$.  Conversely, suppose $R_\gamma\in G$.  Then let $m=\lfloor\gamma/\lambda\rfloor$; then $m$ is an integer with $m\leqslant\gamma/\lambda<m+1$.  With that, multiplying by $\lambda$ gives $m\lambda\leqslant\gamma<(m+1)\lambda$, then subtracting $m\lambda$ gives $0\leqslant\gamma-m\lambda<\lambda$.  Yet also $R_{\gamma-m\lambda}=R_\gamma R_\lambda^{-m}\in G$ since $G$ is a subgroup.  Since no $\mu$ strictly between $0$ and $\lambda$ satisfies $R_\mu\in G$, we must have $\gamma-m\lambda=0$.  Therefore $\gamma=m\lambda$ and $\gamma/\lambda=m\in\mathbb Z$.

From (*) it immediately follows that $2\pi/\lambda\in\mathbb Z$, because $R_{2\pi}=1\in G$.  Let $n=2\pi/\lambda$.  Then $\lambda=2\pi/n$ and therefore $R_\lambda$ is a rotation of order $n$.  Moreover, (*) entails at once that $G=\{R_{c\lambda}:c\in\mathbb Z\}=\left<R_\lambda\right>$.  Thus we also conclude $|G|=n$.

(ii) Let $G\subset O(2)$ be a finite subgroup.  If $G\subset SO(2)$, then $G$ is cyclic of order $n$ by (i).  So assume $G\not\subset SO(2)$.  Let $N=G\cap SO(2)$; this is a normal subgroup of $G$ of index $2$, because it is the kernel of the surjective homomorphism $\det:G\to\{-1,1\}$.

Since $N$ is a finite subgroup of $SO(2)$, we have $N=\left<R_{2\pi/n}\right>$ where $n=|N|$ by (i).  Now let $g\in G-N$; $g$ is an orientation-reversing operator in $O(2)-SO(2)$.  As we saw in Section 2.6, $g$ must be a reflection around a line through the origin.  One readily verifies $g^2=1$ and $gR_{2\pi/n}g=R_{-2\pi/n}$; one way to see the latter statement is that $gR_{2\pi/n}\in O(2)-SO(2)$, hence is also a reflection, so it squares to the identity.  Furthermore, every element of $G-N$ is of the form $gn$ with $n\in N$ (because $a\in G-N\implies g^{-1}a\in N$).  This implies that if $\lambda=2\pi/n$, then
$$G=\{1,R_\lambda,R_{2\lambda},\dots,R_{(n-1)\lambda},g,gR_\lambda,\dots,gR_{(n-1)\lambda}\}$$
and the reader can verify in this case that $G\cong D_n$, generated by a rotation of angle $\lambda$ and a reflection.
\end{proof}

\noindent\emph{Remark.} If $G_1=\left<R_\pi\right>$ and $G_2$ is cyclic generated by reflection over the $x$-axis, then $G_1\cong G_2~[\cong\mathbb Z/2\mathbb Z]$ as abstract groups.  However, they are essentially distinct as subgroups of $O(2)$, because $G_1\subset SO(2)$ but $G_2\not\subset SO(2)$.  In this context, we distinguish them by saying that $G_1$ is ``cyclic of order $2$'' and $G_2$ is ``dihedral of degree $1$.''  Different symmetry type, same group!

\subsection*{Exercises 2.7. (Finite Groups of Isometries)}
\begin{enumerate}
\item A \textbf{convex combination} of vectors $\vec v_1,\dots,\vec v_n$ is a linear combination $c_1\vec v_1+\dots+c_n\vec v_n$ where the $c_j$ are nonnegative real numbers and $c_1+\dots+c_n=1$.  If $X\subset\mathbb R^n$, the \textbf{convex hull} of $X$, denoted $\hat X$, is defined to be the set of all (finite) convex combinations of elements of $X$.

(a) Every isometry of $X$ is an isometry of $\hat X$.

(b) Show by example that an isometry of $\hat X$ need not be an isometry of $X$.

\item Let $G$ be a subgroup of $\operatorname{Isom}(\mathbb R^2)$.  If $G$ contains two rotations $\ne 1$ around distinct points, show that $G$ is infinite.

\item If $P_1$ and $P_2$ are planes which intersect in a line $\ell$, let $p$ be a point on $\ell$.  Construct lines through $p$ contained in $P_1$ and $P_2$ perpendicular to $\ell$.  The angle between these lines is called the \textbf{dihedral angle} between the planes $P_1$ and $P_2$.

(a) If $q$ is a point outside the planes, then the dihedral angle between $P_1$ and $P_2$ is supplementary to the angle between the lines through $q$ perpendicular to $P_1$ and $P_2$.

(b) Use part (a) to find the dihedral angles of the edges of the five Platonic solids.

(c) A \textbf{regular $4$-polytope} is obtained by taking a Platonic solid and hinging copies of it (called ``cells'') together in $\mathbb R^4$, so that faces meet faces, edges meet edges and vertices meet vertices, and every edge has the same number of cells \---- exactly how polygons were hinged together to get the Platonic solids.  For a regular $4$-polytope, explain why the number of cells around an edge must be less than $360^\circ$ divided by the dihedral angle of an edge.

(d) Use this to classify all possible regular $4$-polytopes.  [There are six of them, discovered by Ludwig Schl\"afli in the 19th century.  You do not need to prove that each one really exists.]

\item Let $X\subset\mathbb R^n$.

(a) If $\operatorname{Isom}(X)$ contains an orientation-reversing isometry, show that $\operatorname{Isom}^+(X)$ is a subgroup of $\operatorname{Isom}(X)$ of index $2$.

(b) Give an example of a figure with no orientation-reversing isometry.

\item\emph{(Simplices.)} \---- (a) The points
$$\left(0,0,1\right),\left(0,\frac{2\sqrt 2}3,-\frac 13\right),\left(\frac{\sqrt 6}3,-\frac{\sqrt 2}3,-\frac 13\right),\left(-\frac{\sqrt 6}3,-\frac{\sqrt 2}3,-\frac 13\right)\in\mathbb R^3$$
form a regular tetrahedron with its vertices on the unit sphere.

(b) The side length of this tetrahedron is $\frac{2\sqrt 6}3$.

(c) More generally, suppose for each positive integer $k$ that $\alpha_k=1/k$, and $\beta_k=\sqrt{1-\alpha_k^2}$.  Consider the following $n+1$ points in $\mathbb R^n$:
$$\begin{bmatrix}0\\\vdots\\0\\0\\0\\1\end{bmatrix},\begin{bmatrix}0\\\vdots\\0\\0\\\beta_n\\-\alpha_n\end{bmatrix},\begin{bmatrix}0\\\vdots\\0\\\beta_{n-1}\beta_n\\-\alpha_{n-1}\beta_n\\-\alpha_n\end{bmatrix},\begin{bmatrix}0\\\vdots\\\beta_{n-2}\beta_{n-1}\beta_n\\-\alpha_{n-2}\beta_{n-1}\beta_n\\-\alpha_{n-1}\beta_n\\-\alpha_n\end{bmatrix},\dots,\begin{bmatrix}\beta_2\dots\beta_n\\\vdots\\-\alpha_{n-2}\beta_{n-1}\beta_n\\-\alpha_{n-1}\beta_n\\-\alpha_n\end{bmatrix},\begin{bmatrix}-\beta_2\dots\beta_n\\\vdots\\-\alpha_{n-2}\beta_{n-1}\beta_n\\-\alpha_{n-1}\beta_n\\-\alpha_n\end{bmatrix}$$
Show that each point has a distance of $1$ from the origin, and that the distance between any two of these points is equal to $\sqrt{\frac{2n+2}n}$.  The convex hull of these points is a \textbf{regular simplex} centered at the origin, in $n$ dimensions.  In terms of the number of vertices, it is the smallest possible nondegenerate $n$-polytope.  [This is the line segment for $n=1$, the triangle for $n=2$, the tetrahedron for $n=3$ and the 5-cell for $n=4$.]

\item The aim of this exercise is to show that $\operatorname{Isom}^+(\mathbf{Oct})\cong S_4$, $\operatorname{Isom}(\mathbf{Oct})\cong S_4\times\mathbb Z/2\mathbb Z$, $\operatorname{Isom}^+(\mathbf{Icos})\cong A_5$, and $\operatorname{Isom}(\mathbf{Icos})\cong A_5\times\mathbb Z/2\mathbb Z$.

(a) If $\vec v$ is any vertex of the cube with vertices $(\pm 1,\pm 1,\pm 1)$, the line segment from $\vec v$ to $-\vec v$ is called a \textbf{body diagonal} of the cube.  The cube has four body diagonals, one for each pair of antipodal vertices.

Let $B$ be the set of body diagonals of the cube.  Show that $\operatorname{Isom}^+(\mathbf{Oct})$ acts on $B$ by application of an isometry.  [Recall that $\operatorname{Isom}^+(\mathbf{Oct})=\operatorname{Isom}^+(\mathbf{Cube})$.]

(b) The aforementioned group action induces a homomorphism $\varphi:\operatorname{Isom}^+(\mathbf{Oct})\to S(B)$.  Show that $\varphi$ is an isomorphism.  [The only orientation-preserving isometries that fix one particular body diagonal are rotations around that body diagonal.  This leads to a proof of injectivity.  As for surjectivity, what are the orders of both groups?]  Conclude that $\operatorname{Isom}^+(\mathbf{Oct})\cong S_4$.

(c) Now show that $\operatorname{Isom}(\mathbf{Oct})\cong S_4\times\mathbb Z/2\mathbb Z$.  [Using Exercise 13 of Section 1.4, show that $\operatorname{Isom}(\mathbf{Oct})$ is the internal direct product of the subgroups $\operatorname{Isom}^+(\mathbf{Oct})$ and $\{I_3,-I_3\}$.]

(d) $\operatorname{Isom}^+(\mathbf{Icos})$ acts on $\mathcal P(\mathbb R^3)$, the set of subsets of $\mathbb R^3$.  The stabilizer of $\mathbf{Tet}$ (as an element of $\mathcal P(\mathbb R^3)$) is the entire group $\operatorname{Isom}^+(\mathbf{Tet})$.  [Observe that it suffices to show that $\operatorname{Isom}^+(\mathbf{Tet})$ is a subgroup of $\operatorname{Isom}^+(\mathbf{Icos})$.]

(e) Let $X$ be the orbit of $\mathbf{Tet}$.  Then $|X|=5$ by the Orbit-Stabilizer Theorem (since $|\operatorname{Isom}^+(\mathbf{Icos})|=60$ and $|\operatorname{Isom}^+(\mathbf{Tet})|=12$).  The union of the five tetrahedra in $X$ looks like this:
\begin{center}
\includegraphics[scale=.35]{CompoundFiveTetrahedra.png}
\end{center}

(f) The action then induces a homomorphism $\varphi:\operatorname{Isom}^+(\mathbf{Icos})\to S(X)$.  Show that $\varphi$ is injective.  [The kernel of $\varphi$ is a normal subgroup of $\operatorname{Isom}^+(\mathbf{Icos})$ contained in $\operatorname{Isom}^+(\mathbf{Tet})$.]

(g) Use part (f) to show $\operatorname{Isom}^+(\mathbf{Icos})\cong A_5$.  [Use Exercise 13(h) of Section 1.6.]

(h) Now show that $\operatorname{Isom}(\mathbf{Icos})\cong A_5\times\mathbb Z/2\mathbb Z$.

\item Use the Orbit-Stabilizer Theorem to show that for any Platonic solid:

(a) if there are $k$ faces to a vertex, rotation by an angle $2\pi/k$ around the axis through that vertex is an isometry;

(b) the action of the orientation-preserving isometry group on the solid's vertices is transitive;

(c) the action of the orientation-preserving isometry group on the solid's faces is transitive;

(d) if a face is a regular $n$-gon, rotation by an angle $2\pi/n$ around the axis through the center of the face is an isometry;

(e) the action of the orientation-preserving isometry group on the solid's edges is transitive;

(f) a $180^\circ$ rotation around the axis through the midpoint of an edge is an isometry.

\item (a) Suppose there are $n$ available colors, where $n$ is a positive integer.  Each face of a cube is to be colored with one of these colors.  Two colorings are considered to be the same if one can be obtained from the other through rotating the cube (however, ``flipping'' the cube by an orientation-reversing transformation is not allowed; that can't be done with a cube in real life).  How many different ways are there to color the faces?  (The answer is a polynomial in $n$.)

[Let $X$ be the set of all colorings of the faces with the cube in standard position.  Then $\operatorname{Isom}^+(\mathbf{Cube})$ acts on $X$, and the different ways to color the faces (according to the problem) are the orbits of this action.  Now use Burnside's Counting Theorem (1.23).  To avoid having to do $24$ separate additions, partition the group into its conjugacy classes.]

(b) List the answers to part (a) for $n=1,2,\dots,7$.  Then (using number theory), show that for any $n\in\mathbb Z$ whatsoever, the expression obtained in part (a) is an integer.  (We know it is an integer by another method\----it comes from counting.)

(c) Now suppose there are $n$ available colors.  Each face of a regular icosahedron is to be colored with one of these colorings.  Two colorings are considered to be the same if (and only if) one can be obtained from the other through rotating the icosahedron.  How many different ways are there to color the faces?

\item\emph{(Wallpaper groups.)} \---- The aim of this exercise is to establish the $17$ wallpaper groups, established by Evgraf Fedorov and George P\'olya.

First let $T(\mathbb R^2)$ be the subgroup of $\operatorname{Isom}(\mathbb R^2)$ consisting of the translations.  Then $T(\mathbb R^2)\cong\mathbb R^2$ for obvious reasons (where the latter group is under addition).  Also, the map $\psi:\operatorname{Isom}(\mathbb R^2)\to O(2)$ sending $[A,\vec v]\mapsto A$ is a surjective homomorphism of groups with kernel $T(\mathbb R^2)$; and hence, $T(\mathbb R^2)$ is normal.

A \textbf{wallpaper group} is defined to be a subgroup $G$ of $\operatorname{Isom}(\mathbb R^2)$ such that $G\cap T(\mathbb R^2)\cong\mathbb Z^2$.  Such a group $G$ is the symmetry group of some plane design (i.e., a function $\mathbb R^2\to\mathcal C$ where $\mathcal C$ is the ``set of colors''), which repeats itself in a tiling pattern in two nonparallel directions.  We assume two wallpaper groups to be the same if conjugating one subgroup by an affine transformation (Exercise 9(b) of Section 2.6) yields the other.  Indeed, in this case, given any pattern whose symmetry group is the first group, one can simply apply the affine transformation to take it to the other. % Actually, that's not the completely accurate definition of a wallpaper group.  A subgroup of $\mathbb R^2$ isomorphic to $\mathbb Z^2$ could be generated by parallel vectors $\vec v,c\vec v$ with $c$ irrational.

Let $G$ be a wallpaper group and $\varphi=\psi|_G:G\to O(2)$.  We classify all possibilities for $G$.

(a) First suppose $G\subset T(\mathbb R^2)$.  Then $G\cong\mathbb Z\times\mathbb Z$ for obvious reasons; this is one wallpaper group.

Now we assume $G\not\subset T(\mathbb R^2)$, but also $G\subset\operatorname{Isom}^+(\mathbb R^2)$.  In this case, $G$ consists of only orientation-preserving isometries.  Let $H=\varphi(G)$; note that $H\subset SO(2)$.

(b) Let $L$ be the lattice $\{\vec v\in\mathbb R^2:[I_2,\vec v]\in G\}$.  Show that every $h\in H$ maps $L$ to itself.  [Take an element $g\in G$ such that $\varphi(g)=h$.  Conjugation by that element induces an automorphism of the normal subgroup $G\cap\operatorname{Tra}(\mathbb R^n)$.]

(c) Every $h\in H$ is conjugate in $GL_2(\mathbb R)$ to a matrix with integer coefficients (which necessarily has determinant $1$).  Conclude that $h$ has finite order, and its eigenvalues are either linear or quadratic over $\mathbb Q$.  Hence, the Euler totient of $|h|$ is either $1$ or $2$, so that $|h|\in\{1,2,3,4,6\}$.

(d) Use part (c) to conclude that $H$ is one of the subgroups
$$\left<R_\pi\right>,\left<R_{2\pi/3}\right>,\left<R_{\pi/2}\right>,\left<R_{\pi/3}\right>$$
of $SO(2)$.  [The assumption $G\not\subset T(\mathbb R^2)$ implies $H\ne\{1\}$.]

(e) If $H=\left<R_\lambda\right>$ with $\lambda\in\{\pi,2\pi/3,\pi/2,\pi/3\}$, show that $G$ is generated by its translations and a rotation by $\lambda$ around some point.  In each of these four cases, there is a unique wallpaper group.  [Note that if $|H|=4$, the lattice $L$ in part (b) must be a square lattice; if $|H|=3$ or $6$, it must be an equilateral triangle lattice; but if $|H|=2$, the lattice can be made up of any kind of parallelogram, thus giving more possibilities.]  Adding these to the wallpaper group in part (a), we have 5 wallpaper groups so far.

The remaining case, which is noticeably hardest, is where $G\not\subset\operatorname{Isom}^+(\mathbb R^2)$; i.e., $G$ contains orientation-reversing isometries.  Let $H=\varphi(G)$.  Then $H$ is a subgroup of $O(2)$ which is not contained in $SO(2)$; by Theorem 2.50, it is dihedral generated by a rotation of an angle $\lambda=2\pi/n$ and a reflection.  By conjugating $G$ via an isometry of $\mathbb R^2$, we may assume:

~~~~(i) $H$ contains a reflection over the $x$-axis;

~~~~(ii) $G$ contains a rotation of an angle $\lambda$ around the origin.

(f) Show that $n\in\{1,2,3,4,6\}$.  [Adapt the argument for $G\subset\operatorname{Isom}^+(\mathbb R^2)$.]

Define $L=\{\vec v\in\mathbb R^2:[I_2,\vec v]\in G\}$ as in part (b).  We say that a lattice point $\vec v\in L$ is \textbf{primitive} if $\vec v\ne\vec 0$ and the line segment from $\vec 0$ to $\vec v$ does not meet any other lattice point in $L$.  In particular, a lattice point $(a,b)\in\mathbb Z^2$ is primitive $\iff\gcd(a,b)=1$.

Again (b) applies to show that each element of $H$ maps $L$ to itself.

(g) Suppose $n=1$.  Then $H$ consists solely of the identity and reflection over the $x$-axis.  Show that either $G$ has a reflection, (in which case there are two wallpaper groups); or else $G$ has a glide reflection whose square is translation by a primitive point in $L$, (in which case, if $G$ has no reflection at all, there is only one wallpaper group).  This adds 3 more wallpaper groups, for a total of 8.

(h) Suppose $n=2$.  Then $H$ contains the identity, the $180^\circ$ rotation, and the horizontal and vertical flips.  Show that either $L$ is built out of rectangles with sides parallel to the $x$ and $y$ axes (in which case, there are 2 wallpaper groups, depending on whether there are reflections), or else $L$ is built out of rhombi with diagonals parallel to the $x$ and $y$ axes (in which case there are 2 wallpaper groups again).  This adds 4 more, for a total of 12.

(i) Suppose $n=3$.  Then $L$ is an equilateral-triangle lattice.  Then $G$ \emph{must} have a reflection, and there are 2 wallpaper groups in this case.  We have gathered 14 to this point.

(j) Suppose $n=4$.  Then $L$ is a square lattice.  In this case $G$ must have a reflection over a line making a $45^\circ$-angle with the $x$-axis.  Moreover there are 2 wallpaper groups in this case; thus we have a total of 16.

(k) Finally suppose $n=6$.  Then $L$ is an equilateral-triangle lattice.  Show that $G$ must have a reflection, and that there is only one wallpaper group.  This completes the classification at $17$ wallpaper groups, as desired.

(l) If $G$ is a wallpaper group, show that there exists a polygon $A_1\dots A_k$ in $\mathbb R^2$ such that the union of the polygons $g(A_1)\dots g(A_k),g\in G$ along with their interiors is $\mathbb R^2$, and any two of those polygons have disjoint interiors.  Then one can get a wallpattern for $G$ by merely drawing something fairly asymmetrical in that polygon, and then transforming it via every element of $G$.  [$A_1\dots A_k$ is called a \textbf{basic unit} for the group $G$.]
\end{enumerate}

\subsection*{2.8. Classification of Finite Subgroups of $SO(3)$ and $O(3)$}
\addcontentsline{toc}{section}{2.8. Classification of Finite Subgroups of $SO(3)$ and $O(3)$}
In the previous section, we have classified finite subgroups of $SO(2)$ and $O(2)$.  In this section, we shall tackle the isometries of the orthogonal group for $3$-space.  They are slightly more complicated, and we have gathered some in the previous section.  Once we know all possible finite subgroups of $O(3)$, we will effectively know every possible discrete pattern style on a finite figure in $3$-space.

Let us start with a few lemmas.  We let $S^2=\{\vec v\in\mathbb R^3:\|\vec v\|=1\}$ be the unit sphere centered at the origin.\\

\noindent\textbf{Lemma 2.51.} \emph{Every $g\ne 1$ in $SO(3)$ is rotation around a uniquely determined axis.}
\begin{proof}
We first show that the operator $g$ must have $1$ as an eigenvalue.

The characteristic polynomial of $g$ is a cubic polynomial with real coefficients.  Since any polynomial with real coefficients of odd degree has a real root, the characteristic polynomial of $g$ must have a real root; in other words, $g$ must have a real eigenvalue $\alpha$.  By Exercise 3(g) of Section 2.6, $\alpha=\pm 1$.

If the other two eigenvalues $\beta,\gamma$ of $g$ are real, then $g$ has $\pm 1$ for its eigenvalues counted with algebraic multiplicity; the product of these eigenvalues is $1$ (since $g\in SO(3)$, we have $\det g=1$), which means that among the three of them, there are an even number of $-1$'s.  Thus $g$ has an eigenvalue of $1$ in this case.  On the other hand, if $\beta$ and $\gamma$ are not real, they must be complex conjugates of each other; hence $\beta\gamma=|\beta|^2=1$ (by Exercise 3(g) of Section 2.6), which implies $\alpha=1$ (because $\det g=\alpha\beta\gamma=1$).  Therefore, $g$ has an eigenvalue of $1$ in this case as well.

Since $g$ must have an eigenvalue of $1$, there is a unit vector $\vec p\in S^2$ such that $g(\vec p)=\vec p$ (take an eigenvector for $1$ and multiply by the inverse of its magnitude).  Let $a$ be an element of $SO(3)$ such that $a(\vec e_3)=\vec p$; a basic argument using the Gram-Schmidt process shows that $a$ exists.  Now set $h=a^{-1}ga$.  Then $h\in SO(3)$ and $h(\vec e_3)=\vec e_3$.  Thus, the orthogonality of $h$ entails that $h(\vec e_1),h(\vec e_2)$ are perpendicular to $\vec e_3$, and hence $h$ looks like this:
$$h=\begin{bmatrix}x&y&0\\z&w&0\\0&0&1\end{bmatrix}$$
Since $\det h=1$ and its columns are orthonormal, we get that $\begin{bmatrix}x&y\\z&w\end{bmatrix}\in SO(2)$ by straightforward calculation.  Hence this $2\times 2$ submatrix must be of the form $R_\theta$ with $\theta\in\mathbb R$; in other words, $x=w=\cos\theta$ and $z=-y=\sin\theta$.  It is then clear that $h$ rotates by an angle of $\theta$ around the $z$-axis (parallel to $\vec e_3$).  Since $g=aha^{-1}$, $g$ rotates by an angle of $\theta$ around the axis parallel to $\vec p$, as desired.

To show that the axis is uniquely determined, we claim that $\pm\vec p$ are the \emph{only} unit vectors fixed by $g$; it will follow that any axis around which $g$ rotates must be parallel to the vectors $\pm\vec p$.  Well, suppose $\vec x\in S^2$ is fixed by $g$.  Then $h(a^{-1}(\vec x))=[a^{-1}ga](a^{-1}(\vec x))=a^{-1}gaa^{-1}(\vec x)=a^{-1}g(\vec x)=a^{-1}(\vec x)$; hence $\vec y=a^{-1}(\vec x)$ is fixed by $h$.  Moreover, $\vec y$ is in the kernel of the linear operator
$$h-I_3=\begin{bmatrix}\cos\theta-1&-\sin\theta&0\\\sin\theta&\cos\theta-1&0\\0&0&0\end{bmatrix},$$
whose upper-left $2\times 2$ submatrix is nonsingular (its determinant is $(\cos\theta-1)^2+(\sin\theta)^2=2-2\cos\theta\ne 0$, since $g\ne 1$ and hence $\cos\theta\ne 1$).  Consequently $\vec y$ must be of the form $(0,0,r)$ with $r\in\mathbb R$; since $\vec y\in S^2$ we conclude $\vec y=\pm\vec e_3$, and hence $\vec x=a(\vec y)=\pm\vec p$.
\end{proof}
\noindent\textbf{Lemma 2.52.} (i) \emph{Let $\vec p\in S^2$, and let $G=\{g\in SO(3):g(\vec p)=\vec p\}$.  Then there is an isomorphism $SO(2)\cong G$, sending every rotation of an angle $\theta$ to a rotation of the same angle $\theta$ around the axis parallel to $\vec p$.}

(ii) \emph{Let $\vec p\in S^2$, and let $G=\{g\in SO(3):g(\vec p)=\pm\vec p\}$.  Then there is an isomorphism $O(2)\cong G$, sending every rotation of an angle $\theta$ to a rotation of the same angle around the axis parallel to $\vec p$, and sending every reflection to an element of $G$ sending $\vec p\mapsto-\vec p$.}
\begin{proof}
In each case, by conjugating $G$ by an element of $SO(3)$ sending $\vec e_3\mapsto\vec p$, we may assume that $\vec p=\vec e_3$.

(i) Define $\varphi:SO(2)\to G$ via $\varphi(a)=\begin{bmatrix}a&0\\0&1\end{bmatrix}$, where the latter matrix is regarded as a block matrix.  In other words, $\varphi\left(\begin{bmatrix}x&y\\z&w\end{bmatrix}\right)=\begin{bmatrix}x&y&0\\z&w&0\\0&0&1\end{bmatrix}$.

$\varphi$ is well-defined, because a matrix of the form $\varphi(a)$ maps $\vec e_3\mapsto\vec e_3$.  It is clear that $\varphi$ is an injective homomorphism of groups.  Moreover, if $b\in G$ is any element, then $b(\vec e_3)=\vec e_3$, hence we must have $b=\begin{bmatrix}\cos\theta&-\sin\theta&0\\\sin\theta&\cos\theta&0\\0&0&1\end{bmatrix}$ as in the proof of Lemma 2.51.  Moreover, $b=\begin{bmatrix}R_\theta&0\\0&1\end{bmatrix}=\varphi(R_\theta)$, hence $\varphi$ is surjective, and an isomorphism.

The reader can readily verify that $\varphi$ sends rotation by an angle of $\theta$ to rotation by an angle of $\theta$.

(ii) Define $\psi:O(2)\to G$ via $\psi(a)=\begin{bmatrix}a&0\\0&\det a\end{bmatrix}$; i.e., we have $\psi\left(\begin{bmatrix}x&y\\z&w\end{bmatrix}\right)=\begin{bmatrix}x&y&0\\z&w&0\\0&0&xw-yz\end{bmatrix}$.  One can readily verify that every $\psi(a)$ is an element of $SO(3)$ mapping $\vec e_3$ to either $\vec e_3$ or $-\vec e_3$; hence $\psi$ is well-defined.  Also, $\psi$ is an injective homomorphism of groups.

To show that $\psi$ is surjective, adapt the proof of part (a); noting that $b\in G$ satisfies $b(\vec e_3)=-\vec e_3$ if and only if the upper-left $2\times 2$ submatrix is orientation-reversing.  Therefore $\psi$ is an isomorphism.  Again the reader can readily verify that $\psi$ sends rotation by an angle of $\theta$ to rotation by an angle of $\theta$, and that $\psi(a)$ sends $\vec e_3\mapsto-\vec e_3$ when $a\in O(2)$ is a reflection.
\end{proof}

\noindent According to the previous lemma, every subgroup of $O(2)$ can be identified with a subgroup of $SO(3)$ which fixes a certain axis [with possibly group elements swapping the endpoints].  We therefore get all the cyclic and dihedral groups as subgroups of $SO(3)$.  We also know three other finite subgroups: $\operatorname{Isom}^+(\mathbf{Tet})$, the orientation-preserving isometry group of the regular tetrahedron; $\operatorname{Isom}^+(\mathbf{Oct})$, that of the octahedron; and $\operatorname{Isom}^+(\mathbf{Icos})$, that of the icosahedron.  We claim that these are the only ones:\\

\noindent\textbf{Theorem 2.53.} \textsc{(Classification of finite subgroups of $SO(3)$)}

\emph{Every finite subgroup of $SO(3)$ is isomorphic to either a cyclic group, a dihedral group, $\operatorname{Isom}^+(\mathbf{Tet})$, $\operatorname{Isom}^+(\mathbf{Oct})$ or $\operatorname{Isom}^+(\mathbf{Icos})$.}
\begin{proof} % This proof is due to Prof. Claire Burrin (RU), I have memorized it from her
Let $G$ be a finite subgroup of $SO(3)$.  We may assume $G\ne\{1\}$ for obvious reasons.

We start by letting $P=\{\vec p\in S^2:g(\vec p)=\vec p\text{ for some }g\ne 1\text{ in }G\}$.  By Lemma 2.51, every $g\ne 1$ in $G$ is a rotation around a uniquely determined axis; moreover, $P$ is the set of unit vectors on these axes, which implies that $P$ is finite.

Moreover, $G$ acts on $P$; i.e., $g\cdot\vec p=g(\vec p)$.  Indeed, if $g\in G$ and $\vec p\in P$, then by definition of $P$, we have $h(\vec p)=\vec p$ for some $h\ne 1$ in $G$.  With that, $ghg^{-1}\ne 1$, and $ghg^{-1}(g(\vec p))=ghg^{-1}g(\vec p)=gh(\vec p)=g(\vec p)$.  Therefore $g(\vec p)\in P$.

We shall first establish an equation which will guide us swiftly through the rest of the proof.  To do this, we let
$$S=\{(g,\vec p)\in G\times P:g\ne 1,g\cdot\vec p=\vec p\},$$
and compute $|S|$ in two ways.  On the one hand, partitioning $S$ by the left component shows $|S|=\sum_{g\in G,g\ne 1}|\{\vec p\in P:g\cdot\vec p=\vec p\}|$.  Yet for $g\ne 1$, there are exactly two points fixed by $g$ (the endpoints of the rotation axis), so that $|\{\vec p\in P:g\cdot\vec p=\vec p\}|=2$.  Therefore $|S|=\sum_{g\in G,g\ne 1}2=2(|G|-1)$.

On the other hand, we may partition $S$ by the \emph{right} component, to get $|S|=\sum_{\vec p\in P}|\{g\in G:g\ne 1,g\cdot\vec p=\vec p\}|$.  Now, the set of $g\in G$ such that $g\cdot\vec p=\vec p$ is precisely the stabilizer $\operatorname{Stab}(\vec p)$.  Yet this subgroup always contains the identity, which we are deliberately removing from the set, thus decrementing its size by $1$.  We thus get $|\{g\in G:g\ne 1,g\cdot\vec p=\vec p\}|=|\operatorname{Stab}(\vec p)|-1$, and hence $|S|=\sum_{\vec p\in P}(|\operatorname{Stab}(\vec p)|-1)$.  Equating the two expressions for $|S|$,
$$2(|G|-1)=\sum_{\vec p\in P}(|\operatorname{Stab}(\vec p)|-1)$$
We may write $P=\bigsqcup_{j=1}^k\mathcal O_j$, where the $\mathcal O_j$ are the distinct orbits of the group action.  The summation above can then be interpreted as a sum over the orbits, where each summand sums over the elements of the orbit:
$$2(|G|-1)=\sum_{j=1}^k\sum_{\vec p\in\mathcal O_j}(|\operatorname{Stab}(\vec p)|-1)$$
By the Orbit-Stabilizer Theorem, we have
$$\sum_{\vec p\in\mathcal O_j}(|\operatorname{Stab}(\vec p)|-1)=\sum_{\vec p\in\mathcal O_j}\left(\frac{|G|}{|\mathcal O_j|}-1\right)=|\mathcal O_j|\left(\frac{|G|}{|\mathcal O_j|}-1\right)=|G|-|\mathcal O_j|$$
And hence
$$2(|G|-1)=\sum_{j=1}^k(|G|-|\mathcal O_j|)$$
For each $j$, let $\vec p_j$ be an element of $\mathcal O_j$.  Dividing both sides of the above equation by $|G|$, and using the Orbit-Stabilizer Theorem again, we arrive at
\begin{equation}\tag{1}
2\left(1-\frac 1{|G|}\right)=\sum_{j=1}^k\left(1-\frac 1{|\operatorname{Stab}(\vec p_j)|}\right)
\end{equation}
First observe that in (1), the left-hand side is $\geqslant 1$ (we are assuming $G\ne\{1\}$, hence $|G|\geqslant 2$, from which $1-\frac 1{|G|}\geqslant\frac 12$ follows), and every summand on the right-hand side is $<1$, which implies that $k>1$ (in other words, there must be more than one summand on the right-hand side).  Also, the left-hand side is $<2$, and every summand on the right-hand side is $\geqslant\frac 12$ (because by definition of $P$, $\vec p_j\in P$ implies $\operatorname{Stab}(\vec p_j)$ is a nontrivial subgroup, i.e., has order $\geqslant 2$).  This implies $k<4$ (because $k\geqslant 4$ would make the right-hand side $\geqslant 2$, a contradiction).  Therefore $k$ is either $2$ or $3$.\\

\noindent\textbf{Case 1}: $k=2$.  Subtracting both sides of equation (1) from $2$ entails
\begin{equation}\tag{2}
\frac 2{|G|}=\frac 1{|\operatorname{Stab}(\vec p_1)|}+\frac 1{|\operatorname{Stab}(\vec p_2)|}
\end{equation}
Multiplying by $|G|$ and using the Orbit-Stabilizer Theorem, $2=|\mathcal O_1|+|\mathcal O_2|$.  This implies that the orbits $\mathcal O_1$ and $\mathcal O_2$ are single-element sets, hence $\mathcal O_1=\{\vec p_1\}$ and $\mathcal O_2=\{\vec p_2\}$.  Moreover, since $\{\vec p_1\}$ is a single-element orbit, every $g\in G$ satisfies $g(\vec p_1)=\vec p_1$.  Thus $G$ is identified with a finite subgroup of $SO(2)$ via the isomorphism in Lemma 2.52(i).  According to Theorem 2.50(i), $G$ is cyclic of order $n$, generated by rotation of $2\pi/n$ around the axis parallel to $\vec p_1$.

Note in this case that $\vec p_2=-\vec p_1$.\\

\noindent\textbf{Case 2}: $k=3$.  Subtracting both sides of equation (1) from $3$ gives
\begin{equation}\tag{3}
1+\frac 2{|G|}=\frac 1{|\operatorname{Stab}(\vec p_1)|}+\frac 1{|\operatorname{Stab}(\vec p_2)|}+\frac 1{|\operatorname{Stab}(\vec p_3)|}
\end{equation}
Since the left-hand side is $>1$, we must have $|\operatorname{Stab}(\vec p_j)|=2$ for one of $j=1,2,3$ (for if every $|\operatorname{Stab}(\vec p_j)|\geqslant 3$, the right-hand side would be $\leqslant 1$, a contradiction).  By reordering and relabeling we may assume $|\operatorname{Stab}(\vec p_1)|=2$.  In this case, we subtract $1/2$ from equation (3) to get
\begin{equation}\tag{4}
\frac 12+\frac 2{|G|}=\frac 1{|\operatorname{Stab}(\vec p_2)|}+\frac 1{|\operatorname{Stab}(\vec p_3)|}
\end{equation}
Since the left-hand side is $>\frac 12$, we must have $|\operatorname{Stab}(\vec p_j)|<4$ for some $j=2,3$ (otherwise the right-hand side would be $\leqslant\frac 12$).  Without loss of generality we assume $|\operatorname{Stab}(\vec p_2)|\leqslant|\operatorname{Stab}(\vec p_3)|$.  In this case $|\operatorname{Stab}(\vec p_2)|<4$, hence is either $2$ or $3$.

If $|\operatorname{Stab}(\vec p_2)|=2$, then subtracting $1/2$ from both sides of (4) gives
\begin{equation}\tag{5}
\frac 2{|G|}=\frac 1{|\operatorname{Stab}(\vec p_3)|},
\end{equation}
and hence multiplying by $|G|$ and using the Orbit-Stabilizer Theorem gives $|\mathcal O_3|=2$.  This means that $\mathcal O_3=\{\vec p_3,\vec q\}$ for some $\vec q\in S^2$.  Since $\vec p_3\in P$, there exists $g\ne 1$ in $G$ such that $g(\vec p_3)=\vec p_3$ and (since $\mathcal O_3$ is an orbit), this implies $g(\vec q)=\vec q$.  Therefore $\vec q=-\vec p_3$ and $g$ rotates around the axis parallel to $\vec p_3$.  Hence, $\mathcal O_3=\{\vec p,-\vec p\}$.  This implies that every element of $G$ maps $\vec p\mapsto\pm\vec p$, and hence $G$ identifies with a finite subgroup of $O(2)$ via the isomorphism in Lemma 2.52(ii).  Moreover, there exists $g\in G$ such that $g(\vec p)=-\vec p$ (otherwise $\vec p,-\vec p$ would be in separate orbits), so the aforementioned subgroup of $O(2)$ contains orientation-reversing elements.  By Theorem 2.50(ii), this subgroup is generated by a rotation of angle $2\pi/n,n\in\mathbb Z$ and a reflection, and is moreover isomorphic to $D_n$; it follows that $G\cong D_n$ in this case and $G$ is dihedral.

The final case to consider (when $k=3$) is that $|\operatorname{Stab}(\vec p_2)|=3$.  With that, subtracting $1/3$ from both sides of (4) gives
\begin{equation}\tag{6}
\frac 16+\frac 2{|G|}=\frac 1{|\operatorname{Stab}(\vec p_3)|}.
\end{equation}
Since the left-hand side is $>\frac 16$, $|\operatorname{Stab}(\vec p_3)|<6$.  But also $|\operatorname{Stab}(\vec p_3)|\geqslant|\operatorname{Stab}(\vec p_2)|=3$, and hence $|\operatorname{Stab}(\vec p_3)|$ is either $3$, $4$ or $5$.  From (6) it follows that if $|\operatorname{Stab}(\vec p_3)|=3$ then $|G|=12$; if $|\operatorname{Stab}(\vec p_3)|=4$ then $|G|=24$; and if $|\operatorname{Stab}(\vec p_3)|=5$ then $|G|=60$.  We summarize the three cases in the following chart:
\begin{center}
\begin{tabular}{cc|c|c||c}
& $|\operatorname{Stab}(\vec p_1)|$ & $|\operatorname{Stab}(\vec p_2)|$ & $|\operatorname{Stab}(\vec p_3)|$ & $|G|$ \\\hline
(i) & $2$ & $3$ & $3$ & $12$ \\
(ii) & $2$ & $3$ & $4$ & $24$ \\
(iii) & $2$ & $3$ & $5$ & $60$
\end{tabular}
\end{center}
Observe in each case that $\operatorname{Stab}(\vec p_3)$ is identified with a subgroup of $SO(2)$ via the isomorphism in Lemma 2.52(i), with $\vec p=\vec p_3$.  Hence by Theorem 2.50(i), $\operatorname{Stab}(\vec p_3)$ is cyclic generated by a rotation around $\vec p_3$'s axis, of an angle $2\pi/n$, where $n=|\operatorname{Stab}(\vec p_3)|$.

We start by tackling case (i): By the Orbit-Stabilizer Theorem, $|\mathcal O_3|=12/3=4$.  Suppose $\mathcal O_3=\{\vec v_1,\vec v_2,\vec v_3,\vec v_4\}$ with $\vec v_1=\vec p_3$.  Since $|\operatorname{Stab}(\vec p_3)|=3$, the preceding paragraph implies $\operatorname{Stab}(\vec p_3)$ is cyclic generated by a rotation $R\in G$ of angle $2\pi/3$ (or $120^\circ$).  As $R^3=1$, and $R$ fixes $\vec p_3=\vec v_1$, $R$ must permute the vectors $\vec v_2,\vec v_3,\vec v_4$ in a $3$-cycle ($R$ cannot fix them all because it only fixes two points on the sphere by Lemma 2.51).  Therefore $\vec v_2,\vec v_3,\vec v_4$ form an equilateral triangle whose planar interior is perpendicular to $\vec p_3$ (why?).  Yet the same argument can be repeated with \emph{any} element of $\mathcal O_3$ in place of $\vec p_3$: that element's stabilizer will be cyclic of order $3$, hence generated by a $120^\circ$-rotation, which implies that the other three elements of $\mathcal O_3$ form an equilateral triangle.  Since any three points of $\mathcal O_3$ form an equilateral triangle, the four points form a regular tetrahedron via convex hull.  Moreover, every element of $G$ permutes the four points, and is therefore an isometry of the tetrahedron.  We thus get an injective homomorphism $G\to\operatorname{Isom}^+(\mathbf{Tet})$; this is an isomorphism because both groups have order $12$.  Hence $G\cong\operatorname{Isom}^+(\mathbf{Tet})$ in this case.

In case (ii), $|\mathcal O_3|=6$.  Moreover, $\operatorname{Stab}(\vec p_3)$ is cyclic generated by a $90^\circ$ rotation $R\in G$.  This rotation fixes $\vec p_3$ and $-\vec p_3$ and orbits every other element of the sphere in a $4$-cycle (there are no $2$-cycles in this rotation).  Thus, $R$ must fix one point $\vec v_2$ of $\mathcal O_3$ other than $\vec p_3$ and permute the remaining four elements $\vec v_3,\dots,\vec v_6$ in a $4$-cycle (because it cannot fix more than two points), from which we conclude that $\vec v_2=-\vec p_3$ and $\vec v_3,\dots,\vec v_6$ form a square whose planar interior is perpendicular to $\vec p_3$.  Thus,
$\mathcal O_3$ consists of $\vec p_3$, $-\vec p_3$ and four points forming a square perpendicular to $\vec p_3$.
However, the above argument works with \emph{any} element of $\mathcal O_3$ in place of $\vec p_3$; it particularly follows that the antipode of any point in $\mathcal O_3$ is also in $\mathcal O_3$.  This, along with the above sentence, implies that $\mathcal O_3$ forms a regular octahedron via convex hull, and $G$ consists only of isometries of this octahedron.  Since $|G|=|\operatorname{Isom}^+(\mathbf{Oct})|=24$, we conclude $G\cong\operatorname{Isom}^+(\mathbf{Oct})$.

We leave it to the reader to show that case (iii) implies $|\mathcal O_3|=12$, and that $\mathcal O_3$ forms a regular icosahedron via convex hull, leading to $G\cong\operatorname{Isom}^+(\mathbf{Icos})$.
\end{proof}

\noindent It is worth remarking that by Exercise 6 of Section 2.7, Proposition 2.53 shows that every finite subgroup of $SO(3)$ is isomorphic to either $\mathbb Z/n\mathbb Z$ for some $n$, $D_n$ for some $n$, $A_4$, $S_4$ or $A_5$.

Once finite subgroups of $SO(3)$ are classified, it is easy to classify finite subgroups of $O(3)$.  However, much casework needs to be considered; thus this classification deserves a new proposition.\\
% AUTHOR'S NOTES: Suppose G is a finite subgroup of O(3), not contained in SO(3).  Then let N = G\cap SO(3).
% N is either cyclic, dihedral, Isom+(Tet), Isom+(Oct) or Isom+(Icos) by the previous theorem.
% Let P = {(g,p)\in N\times P:g\ne 1,g\cdot p=p} as before.  Now all of G (including OR isometries) acts on P by application of an isometry.
% If N is cyclic, either G is dihedral, or G\cong Z_n\times Z_2 (if the reflection goes across the plane with the n-gon), or G\cong Z_{2n} (cyclic generated by a rotatory reflection, rotating half the angle of the shortest rotation).  In particular if N = {1}, then G is cyclic of order 2, generated by either a reflection or the central inversion.
% If N=D_1, then N is cyclic generated by a 180-degree rotation, reducing to previous case
% If N=D_2, then N is the Klein four-group.  There are 2 possibilities for G in this case, Z_2\times Z_2\times Z_2 and "strange-looking" D_4
% If N=D_n with n >= 3, then there are 2 possibilities for G in this case, depending on whether the mirror planes containing the tall axis contain the 2n equatorial points in P
% If N=Isom+(Oct), by consideration of each of the three orbits unioning to P, each orbit has the same "rotatory order" at every point and it's different for each orbit, so they must still be orbits for all of G.  Thus G must consist of isometries of the octahedron and therefore be Isom(Oct).
% If N=Isom+(Icos), G = Isom(Icos) by the same token.
% If N=Isom+(Tet), this time two orbits have the same rotatory order so the situation is different.  In fact, P consists of the vertices of a cube along with the centers of its faces centrally projected onto S^2.  Consideration of the two distinct rotatory axes shows that G is contained in Isom(Cube).  Hence there are two possibilities for G: Isom(Tet)\cong S_4, and the group generated by Isom+(Tet) and orientation-reversing isometries that send Tet to its dual image, isomorphic to A_4\times Z_2.
% *** And here's a fun little exercise: for each finite subgroup of O(3), come up with a sphere design for which it is the isometry group.

\noindent\textbf{Theorem 2.54.} \textsc{(Classification of finite subgroups of $O(3)$)}

\emph{Every finite subgroup of $O(3)$ is isomorphic to one of the following:}

(i) \emph{The cyclic group of order $n$, generated by a rotation of angle $2\pi/n$;}

(ii) \emph{The dihedral group of order $2n$, generated by a rotation of angle $2\pi/n$ around an axis and a reflection over a plane containing said axis;}

(iii) \emph{The abelian group $\mathbb Z/n\mathbb Z\times\mathbb Z/2\mathbb Z$, generated by a rotation of angle $2\pi/n$ around an axis and a reflection over the plane perpendicular to said axis;}

(iv) \emph{The cyclic group of order $2n$, generated by a rotatory reflection of angle $\pi/n$;}

(v) \emph{The dihedral group of order $2n$, generated by a rotation of angle $2\pi/n$ around an axis and a $180$-degree rotation around a perpendicular axis;}

(vi) \emph{The group $D_n\times\mathbb Z/2\mathbb Z$, generated by a rotation of angle $2\pi/n$ around an axis, reflection over the plane perpendicular to said axis, and a $180$-degree rotation around an axis perpendicular to that axis;}

(vii) \emph{The dihedral group of order $4n$, generated by a rotatory reflection of angle $\pi/n$ around an axis and a reflection over a plane containing said axis;}

(viii) \emph{$\operatorname{Isom}^+(\mathbf{Tet})$;}

(ix) \emph{$\operatorname{Isom}(\mathbf{Tet})$;}

(x) \emph{$\operatorname{Isom}^+(\mathbf{Oct})$;}

(xi) \emph{$\operatorname{Isom}(\mathbf{Oct})$;}

(xii) \emph{$\operatorname{Isom}^+(\mathbf{Icos})$;}

(xiii) \emph{$\operatorname{Isom}(\mathbf{Icos})$;}

(xiv) \emph{The group $A_4\times\mathbb Z/2\mathbb Z$, generated by $\operatorname{Isom}^+(\mathbf{Tet})$ and the central inversion $-I_3$.}\\

\noindent We remark that of the five Platonic solids, the tetrahedron is the only one for which $-I_3$ is not an isometry.  This is why (xiv) is separate from (viii)-(ix), and no other Platonic solid has this separate case.

\begin{proof}
Let $G$ be a finite subgroup of $O(3)$.  If $G\subset SO(3)$, then $G$ is isomorphic to either (i), (v), (viii), (x) or (xii) by Theorem 2.53.  Thus, we assume $G\not\subset SO(3)$.  With that, $N=G\cap SO(3)$ is a normal subgroup of $G$ of index $2$ (because $\det:G\to\{-1,1\}$ is a surjective homomorphism with kernel $N$).  Since $N$ is a finite subgroup of $SO(3)$, we have, by Theorem 2.53, that $N$ is either cyclic, dihedral, $\operatorname{Isom}^+(\mathbf{Tet})$, $\operatorname{Isom}^+(\mathbf{Oct})$ or $\operatorname{Isom}^+(\mathbf{Icos})$.

Also, if $P=\{\vec p\in S^2:g(\vec p)=\vec p\text{ for some }g\ne 1\text{ in }N\}$ as in Theorem 2.53, then all of $G$ (not just $N$) acts on $P$ by application of an isometry.  The reader can readily verify this by imitating the corresponding part of the proof Theorem 2.53.

We shall use casework on $N$.\\

\noindent\textbf{Case 1}: \emph{$N$ is cyclic of order $n$, generated by a rotation of angle $2\pi/n$.}

If $n=1$, then $N$ is trivial, and hence (since $[G:N]=2$), $G$ is cyclic of order $2$, generated by an orientation-reversing isometry $a$.  Since $|a|=2$, $a$ must be either a reflection or the central inversion $-I_3$ (since a rotatory reflection by $\theta$ squares to a rotation by $2\theta$, and a rotatory reflection by $180^\circ$ is the central inversion).  If $a$ is a reflection then $G$ is isomorphic to (iii) (with $n=1$), and if $a$ is the central inversion then $G$ is isomorphic to (iv).  We henceforth assume $n>1$.

In this case, $P$ consists of two antipodal points $\vec p,-\vec p$ (the endpoints of the axis of the generating rotation in $N$).  Let $a\in G-N$.  By the remarks preceding the casework, the isometry $a$ must permute $P$.  If $a$ fixes both $\vec p,-\vec p$, $a$ must be a reflection across a plane containing $\vec p$'s axis (rotatory reflections only fix the center of the sphere).  By composing $a$ with various elements of $N$, it is clear that every element of $G-N$ is a reflection across a plane containing $\vec p$'s axis, and $G$ is isomorphic to (ii).

On the other hand, what if $a$ swaps $\vec p,-\vec p$ (i.e., $a(\vec p)=-\vec p$)?  In this case, (since $a$ is orthogonal), $a$ must fix the plane $\Pi$ through the origin perpendicular to $\vec p$, though not necessarily pointwise.  In this case, the reader can see that $a$ is either a reflection or a rotatory reflection across $\Pi$. Let $R$ be the rotation of angle $2\pi/n$ that generates $N$; then $a^2\in N$ and hence $a^2=R^k$ for some $0\leqslant k<n$.  In this case, $a$ is a reflection / rotatory reflection that rotates at either an angle of $k\pi/n$ or $\pi+k\pi/n$; either way the angle of rotation for $a$ is of the form $c\pi/n,c\in\mathbb Z$.  If $c=2\ell$ is even, then $aR^{-\ell}$ is a reflection across $\Pi$, and $G$ is isomorphic to (iii).  If $c=2m+1$ is odd, then $aR^{-m}$ is a rotatory reflection which rotates at an angle of $\pi/n$, and we are in the situation of (iv).\\

\noindent\textbf{Case 2}: \emph{$N$ is dihedral of order $2n$, generated by a rotation of angle $2\pi/n$ around an axis and a $180$-degree rotation around a perpendicular axis.}

If $n=1$, then $N$ is cyclic generated by a $180^\circ$ rotation, thus this case reduces to Case 1 ($N$ cyclic of order $2$).

If $n=2$, then $N$ is generated by two perpendicular $180$-degree rotations, hence is conjugate to the subgroup consisting of:
$$\begin{bmatrix}1&0&0\\0&1&0\\0&0&1\end{bmatrix},~~~~\begin{bmatrix}1&0&0\\0&-1&0\\0&0&-1\end{bmatrix},~~~~\begin{bmatrix}-1&0&0\\0&1&0\\0&0&-1\end{bmatrix},~~~~\begin{bmatrix}-1&0&0\\0&-1&0\\0&0&1\end{bmatrix}$$
(why?); without loss of generality, we may assume $N$ is that subgroup.  This means that $P$ consists of the six points $\{\pm\vec e_1,\pm\vec e_2,\pm\vec e_3\}$.  Since earlier we noted that $G$ acts on $P$ by application of an isometry, every element of $G$ must therefore be a monomial matrix with $\pm 1$'s as the nonzero entries.  Now consider the four elements of $G-N$: taking one of them and multiplying it by the elements of $N$ gives the rest of them.  Since every element of $N$ is a diagonal matrix (i.e., a monomial matrix for the identity permutation), we conclude that every element of $G-N$ must be a monomial matrix \emph{for the same permutation}.  This permutation must square to the identity (since elements of $G-N$ square to elements of $N$), hence either be the identity or a transposition.  In either case, there are exactly four ways to assign $\pm 1$ to the nonzero entries that make the matrix orientation-reversing, as required of $G-N$; and these four matrices must be the elements of $G-N$.

If $G-N$ consists of monomial matrices for the identity permutation, then $G$ consists of all diagonal matrices with $\pm 1$'s as the nonzero entries.  With that it is clear that $G\cong\mathbb Z/2\mathbb Z\times\mathbb Z/2\mathbb Z\times\mathbb Z/2\mathbb Z$.  This satisfies (vi) with $n=2$, because $D_2\cong\mathbb Z/2\mathbb Z\times\mathbb Z/2\mathbb Z$.

If $G-N$ consists of monomial matrices for a transposition, then we may assume (by conjugating by $G$ by a permutation matrix), that the transposition swaps the first two components.  In this case the matrices are
$$\begin{bmatrix}0&1&0\\1&0&0\\0&0&1\end{bmatrix},~~~~\begin{bmatrix}0&1&0\\-1&0&0\\0&0&-1\end{bmatrix},~~~~\begin{bmatrix}0&-1&0\\1&0&0\\0&0&-1\end{bmatrix},~~~~\begin{bmatrix}0&-1&0\\-1&0&0\\0&0&1\end{bmatrix}$$
If $r=\begin{bmatrix}0&1&0\\-1&0&0\\0&0&-1\end{bmatrix}$ and $d=\begin{bmatrix}1&0&0\\0&-1&0\\0&0&-1\end{bmatrix}$, the reader can verify that $r,d\in G$, $|r|=4$, $|d|=2$ and $dr=r^{-1}d$.  From this it follows that $G\cong D_4$ and we have (vii) with $n=2$.

This concludes the case $n=2$; now we assume $n\geqslant 3$.  With that, $P$ consists of two antipodes $\pm\vec p$ (the endpoints of the rotation by $2\pi/n$), and $2n$ equally spaced points on the plane $\Pi$ through the origin perpendicular to $\vec p$.  If we define the ``rotatory order'' of a point in $P$ to be the number of rotations around that point that will map $P$ to itself, it is clear that elements of $G$ preserve points in $P$ of a particular rotatory order, and that $\pm\vec p$ have rotatory order $2n$ and the other $2n$ points have rotatory order $2$.  Therefore, an element $a\in G-N$ must permute $\pm\vec p$ again.

If $a(\vec p)=\vec p$, then $a$ is a reflection across a plane containing $\vec p$.  Since this reflection preserves the $2n$ points in $\Pi$, the mirror plane must either go through two of these points, or through the midpoints of edges of the regular $2n$-gon they span (one or the other, not both, because $2n$ is even).  If the mirror plane goes through equatorial points $\pm\vec q\in P$, composing the reflection with the $180$-degree rotation around $\pm\vec q$ results in a reflection across $\Pi$, thus $G$ contains a reflection across $\Pi$ and is isomorphic to (vi).  If the mirror plane instead goes through the midpoints of edges of the regular $2n$-gon, let $\pm\vec q$ be antipodal vertices of the $2n$-gon that meet the edges the mirror plane goes through.  The reflection is then the reflection whose mirror plane goes through $\pm\vec q$, composed with a $\pi/n$-angle rotation.  Consequently, when the reflection is composed with the $180$-degree rotation around $\pm\vec q$, we have in our hands a rotatory reflection across $\Pi$ with an angle of $\pi/n$.  This lands us in (vii) as one can readily verify.

If $a(\vec p)=-\vec p$, then (unlike Case 1), we may reassign $a$ to $ah$, where $h$ is a $180$-degree rotation in $N$ which swaps $\pm\vec p$, thus boiling this case to the one settled in the previous paragraph.\\

\noindent\textbf{Case 3}: \emph{$N\cong\operatorname{Isom}^+(\mathbf{Oct})$.}

In this case, $P$ consists of $6$ points $(\pm\vec e_1,\pm\vec e_2,\pm\vec e_3)$ with a rotatory order of $4$, $12$ points with a rotatory order of $2$, and $8$ points with a rotatory order of $3$.  (The reader should take the time to verify this.)  Hence, since elements of $G$ preserve points in $P$ of a particular rotatory order, every element of $G$ must permute $\pm\vec e_1,\pm\vec e_2,\pm\vec e_3$, and therefore $G\subset\operatorname{Isom}(\mathbf{Oct})$.  Yet $G$ has order $2|N|=48$, making the aforementioned inclusion an equality.  Therefore $G=\operatorname{Isom}(\mathbf{Oct})$ and we have (xi).\\

\noindent\textbf{Case 4}: \emph{$N\cong\operatorname{Isom}^+(\mathbf{Icos})$.}

In this case, $P$ consists of $12$ points with a rotatory order of $5$; $30$ points with a rotatory order of $2$; and $20$ points with a rotatory order of $3$.  As in the previous case, $G\subset\operatorname{Isom}(\mathbf{Icos})$, and therefore $G=\operatorname{Isom}(\mathbf{Icos})$ because their orders are equal; this is (xiii).\\

\noindent\textbf{Case 5}: \emph{$N\cong\operatorname{Isom}^+(\mathbf{Tet})$.}

In this case, we \emph{cannot} simply adapt the previous two cases to conclude $G=\operatorname{Isom}(\mathbf{Tet})$.  After all, $P$ consists of \emph{eight} points with a rotatory order of $3$ (the vertices and the centers of the faces of the tetrahedron), and $6$ points with a rotatory order of $2$.  Thus an element of $G$ need not permute the four vertices of the tetrahedron.

Now $N$ consists of all the monomial matrices for even permutations, with $\pm 1$'s as nonzero entries, and with determinant $1$.  Every $180^\circ$ rotation in $\operatorname{Isom}^+(\mathbf{Tet})$ must be a diagonal matrix (being a monomial matrix for an even permutation which squares to the identity); it follows that the six points of $P$ with a rotatory order of $2$ are $\pm\vec e_1,\pm\vec e_2,\pm\vec e_3$.  Hence, every element of $G$, permuting those six points, is a monomial matrix with $\pm 1$'s as nonzero entries.  The elements of $G-N$ have determinant $-1$ by definition; they could be either even or odd permutations, but since $N$ has monomial matrices for even permutations, we get that \emph{$G-N$ has monomial matrices for permutations of the same sign}, i.e., all even or all odd.

If $G-N$ consists of monomial matrices for odd permutations, there are exactly $12$ of them with determinant $-1$, and moreover they must be the elements of $G-N$.  In this case $G$ consists of all monomial matrices with $\pm 1$'s as its nonzero entries, such that the nonzero entries multiply to $1$ [as the permutation is odd if and only if the determinant is $-1$].  Hence, $G=\operatorname{Isom}(\mathbf{Tet})$ [look back at Example (4) in Section 2.7] which is (ix).

But what if $G-N$ consists of monomial matrices for even permutations?  Then again, exactly $12$ have determinant $-1$, and those twelve must be the elements of $G-N$.  With that, $G$ consists of all monomial matrices for even permutations with $\pm 1$'s as its nonzero entries.  In particular $-I_3\in G$, and it is clear that $N=\operatorname{Isom}^+(\mathbf{Tet})$ and $\{I_3,-I_3\}$ are trivially-intersecting subgroups which commute with each other elementwise; hence $G$ is the internal direct product of $\operatorname{Isom}^+(\mathbf{Tet})$ and $\{I_3,-I_3\}$, therefore $G\cong A_4\times\mathbb Z/2\mathbb Z$.  This is (xiv).
\end{proof}

Some of the subgroups listed above have clever names; e.g., (vi) is called \textbf{prismatic symmetry} and (vii) is called \textbf{antiprismatic symmetry}.  The orientation-preserving isometry groups (viii), (x), (xii) are called \textbf{chiral symmetry groups}\footnote{``Chirality'' refers to the lack of orientation-reversing symmetry.  Though the Platonic solids have orientation-reversing symmetry, the notion of chirality counts that symmetry out.}, and (xiv) is called \textbf{pyritohedral symmetry}.  Note that (i)-(vii) are the only ones which fix an axis entirely; they are called the \textbf{axial groups}.  These families also fix an equatorial strip bounded by planes perpendicular to the axis (such as $\{(x,y,z)\in S^2:|z|<\frac 12\}$ if the axis is the $z$-axis); they correspond to the Frieze groups [Exercise 8 of Section 2.6], and are ``circular'' versions of them.

A fun little exercise is to take any of the finite subgroups above, and come up with a design on a sphere for which the subgroup is its isometry group.  For example, (vi) is the symmetry group of the design of $n$ equally spaced circles centered on the equator.  (iii) is the symmetry group of the design of $n$ equally spaced arrows centered on the equator, pointing clockwise from the North Pole's point of view \---- note the inability to reflect vertically.  [In addition, if certain small values of $n$ are excluded, (vi) is the isometry group of the regular $n$-gonal prism, and (vii) is the isometry group of the regular $n$-gonal antiprism.] % No, I meant the first picture.  The second one doesn't have the reflection over the xy-plane as an isometry

This theorem leads to a classification of Archimedean (and Catalan) solids, which will be carried out in Chapter 5.

\subsection*{Exercises 2.8. (Classification of Finite Subgroups of $SO(3)$ and $O(3)$)}
\begin{enumerate}
\item Give an example of a design on the sphere for which the isometry group is the group (xiv) stated in Theorem 2.54.

\item Look back at Exercise 9(l) of Section 2.7.  Give a reasonable definition for a ``basic unit'' of a finite subgroup of $O(3)$; then show that every finite subgroup of $O(3)$ has one.

\item The aim of this exercise is to classify finite subgroups of $SO(4)$.  Recall from Exercise 7 of Section 2.6 that $SO(4)$ has two normal subgroups, $S^3_L$ and $S^3_R$.

(a) Show that $S^3_L\cong S^3_R\cong SU(2)$.  [$SU(n)$ is defined in Exercise 3 of Section 2.6.  Show that a typical element of $SU(2)$ is of the form $\begin{bmatrix}a+bi&-c-di\\c-di&a-bi\end{bmatrix}$, with $a,b,c,d\in\mathbb R$, $a^2+b^2+c^2+d^2=1$.]

(b) Every element of $SO(4)$ is a product of a left isoclinic rotation and a right isoclinic rotation.  [Think of the element as a double-rotation, which rotates two orthogonally complementary planes by angles $\alpha,\beta$ (which are allowed to be zero).  Consider isoclinic rotations via $\frac{\alpha+\beta}2$ and $\frac{\alpha-\beta}2$.]

(c) Now show that there is a surjective homomorphism $\varphi:SU(2)\times SU(2)\to SO(4)$, whose kernel is cyclic generated by $(-I_2,-I_2)$.  [There is a function $S^3_L\times S^3_R\to SO(4)$ sending $(a,b)\mapsto ab$.  This is surjective by part (b).  Use Exercise 7(k) of Section 2.6 to show that it is a homomorphism.  Then use Exercise 7(j) of Section 2.6 to find its kernel.  Finally, use part (a) of this exercise.]

By the First Isomorphism Theorem (1.17), it follows that $SO(4)\cong\frac{SU(2)\times SU(2)}{\left<(-I_2,-I_2)\right>}$.  Hence, by Exercise 4(d) of Section 1.5, there is a one-to-one correspondence between subgroups of $SO(4)$ and subgroups of $SU(2)\times SU(2)$ containing $(-I_2,-I_2)$.  Clearly this correspondence matches finite subgroups with one another.  Thus the problem boils down to finding finite subgroups of $SU(2)\times SU(2)$ containing $(-I_2,-I_2)$.

(d) Let $S$ be the subset of $SU(2)$ consisting of matrices with trace zero.  [By part (a), $S=\left\{\begin{bmatrix}bi&-c-di\\c-di&-bi\end{bmatrix}:b,c,d\in\mathbb R,b^2+c^2+d^2=1\right\}$.]  Though $S$ is not a subgroup, it is closed under conjugation, due to the linear algebra fact that similar matrices have the same trace.  Thus for each $a\in SU(2)$, we have the permutation $g\mapsto aga^{-1}$ of $S$.  Use this to establish a surjective group homomorphism $SU(2)\to SO(3)$ whose kernel is $\left<-I_2\right>$.  [Observe that $S$ is identified with a sphere in $\mathbb R^3$.]

(e) Use Theorem 2.53 and part (d) to classify all finite subgroups of $SU(2)$.

(f) Now classify all finite subgroups of $SU(2)\times SU(2)$.  [Let $\pi_1:(a,b)\mapsto a$ and $\pi_2:(a,b)\mapsto b$ be the projection maps from $SU(2)\times SU(2)\to SU(2)$.  Then if $G$ is a finite subgroup of $SU(2)\times SU(2)$, $\pi_1(G),\pi_2(G)$ are finite subgroups of $SU(2)$.  Use part (e) and casework on $\pi_1(G),\pi_2(G)$.]  By listing only those which contain $(-I_2,-I_2)$, the finite subgroups of $SO(4)$ will be in your hands.
\end{enumerate}

\end{document}